\documentclass[
 a4paper,
 12pt,
 parskip=half
 ]{scrartcl}

\usepackage{../.tex/settings}

%\usepackage{tabularx}
\usepackage{ltablex}
% X-Spalte wird vertikal zentriert ausgerichtet
\def\tabularxcolumn#1{m{#1}}

\usepackage{../.tex/mathpkgs}
\usepackage{../.tex/mathcmds}

\theoremstyle{plain}
\newtheorem{thm}{Satz}[section] % reset theorem numbering for each chapter
\newtheorem{lem}[thm]{Lemma}

\theoremstyle{definition}
\newtheorem{defn}[thm]{Definition} % definition numbers are dependent on theorem numbers
\newtheorem{folg}[thm]{Folgerung} % definition numbers are dependent on theorem numbers
\newtheorem{exmp}[thm]{} % same for example numbers

\numberwithin{equation}{section}

%opening
\title{Vorlesung\\Geometrie}
\subtitle{Wintersemester 2016/2017}
\author{Vorlesung: Prof. Dr. Ulrich Brehm\\Mitschrift: Jonas Hippold}

\begin{document}

\maketitle

\tableofcontents

\setcounter{secnumdepth}{0}
\section{Einführung}
\subsection{Begriffe der Geometrie}
\begin{itemize}
 \item Felix Klein: Erlanger Programm (1872)
 \item Euklidischer Raum $\real^n$ (mit Standardskalarprodukt)
 \item Isometriegruppe
  \[ f(x) = Ax + t, A \in O(n), t \in \real^n. \]
 \item Kongruenz
  \begin{itemize} 
   \item Alle Geraden sind einander kongruent, ebenso alle Kreise vom selben Radius.
   \item Kongruente Objekte sind durch Invarianten verbunden.
  \end{itemize}
 \item In der Geometrie verwendete Begriffe: Längen, Flächeninhalte, Volumina, Winkel
 \item Winkel sind kongruent, wenn sie die selbe Größe haben.
 \item Kongruenzsätze
 \item Kreis, Invariante Radius
 \item Affine Geometrie, Gruppe: affine Gruppe des $\real^n$
  \[ \{ f: \real^n \to \real^n | f(x) = Ax + t, A \in \GL(n), A \in \realmat{n}{n} \text{ invertierbar}, t \in \real^n. \} \]
 \item Begriffe der affinen Geometrie:
  \begin{itemize}
   \item Geraden
   \item $k$-dimensionale affine Unterräume
   \item Ellipsen
   \item Mittelpunkte von Strecken 
    \[ A \left( \frac{x+y}{2} \right) + t = \frac{A(x) + t}{2} + \frac{A(y) + t}{2} \]
   \item Mittelpunkte von Ellipsen
  \end{itemize}
 \item \emph{Nicht} in der affinen Geometrie: 
  \begin{itemize}
   \item Längen, Winkel
   \item Kreise
   \item Winkelhalbierende
  \end{itemize}
 \item Euklidische Geometrie
 \item Flächeninhalt: Äquiaffine Abbildung
  \[ f(x) = Ax + t, A \in \GL(n), |\det A| = 1 \]
 \item ``Axiomatischer Zugang'' $\longleftrightarrow$ ``Modelle''
\end{itemize}

\subsection{Literatur}
Horst Knörrer: Geometrie (nur eingeschränkt hilfreich)

\subsection{Themen}
\begin{itemize}
 \item Isometrien des $\real^n$, eventuell normale Endomorphismen
 \item Projektive Geometrie
 \item Inversion (Spiegelung) an Sphären und die Möbiusgruppe
 \item Sphärische und hyperbolische (nicht-euklidische) Geometrie
 \item Quadriken
\end{itemize}

\subsection{Bezeichnungen und Konventionen}
\begin{itemize}
 \item Vektoren werden ohne Pfeile geschrieben.
 \item Skalare sind in der Regel griechische Buchstaben
 \item Matrizen werden durch Großbuchstaben bezeichnet.
 \item Wir betrachten oft den $\real^n$, $\complex^n$ mit 
  \begin{itemize}
   \item dem Standardskalarprodukt
    \[ \langle x, y \rangle := \sum_{i=1}^n x_i \bar{y}_i, \]
    wobei $x = (x_1, \ldots, x_n)^T, y = (y_1, \ldots, y_n)^T \in \real^n$ oder $\complex^n$, 
   \item der euklidischen bzw. unitären Norm für $\real^n$ bzw. $\complex^n$
    \[ \| x \| := \sqrt{ \langle x,x \rangle }. \]
  \end{itemize}
 \item Standardbasis des $K^n$ (Körper $K$):
  \[ e_i = \begin{pmatrix} 0 \\ \vdots \\ 0 \\ 1 \\ 0 \\ \vdots \\ 0 \end{pmatrix} \ldots i\text{-te Stelle} \quad (e_1, e_2, \ldots, e_n). \]
 \item Einheitsmatrix des $\real^n$: $E_n$ bzw. $E$, wenn $n$ aus dem Kontext klar ist.
\end{itemize}

\setcounter{secnumdepth}{1}
\section{Isometrien des \texorpdfstring{$\real^n$}{Rn}}
Eine Isometrie eines metrischen Raumes $(X, d)$ ist eine bijektive Abbildung $f: X \to X$ mit
\[ d( f(x), f(y) ) = d(x,y) \text{ für alle } x, y \in X. \]

Wir wollen die Isometriegruppe des $(\real^n, d)$ untersuchen mit $d(x,y) := \| x - y \|$.

\begin{thm}
 Sei $V$ ein euklidischer Vektorraum und $g: V \to V$ eine Isometrie. Dann ist $f: V \to V$ mit $f(x) := g(x) - g(0)$ eine orthogonale Abbildung. Insbesondere ist $g$ eine affine Abbildung.
\end{thm}

\begin{proof}
 Sei $f: V \to V$ definiert durch $f(x) := g(x) - g(0)$. Dann folgt $\| f(x) \| = \| g(x) - g(0) \| = \| x - 0 \| = \| x \|$, also
 \begin{align*}
\| f(x_1) - f(x_2) \|^2 
    &= \| f(x_1) \|^2 + \| f(x_2) \|^2 - 2 \langle f(x_1), f(x_2) \rangle \\
    &= \| x_1 \|^2 + \| x_2 \|^2 - 2 \langle f(x_1), f(x_2) \rangle.
 \end{align*}
 Es gilt auch
 \[ \| g(x_1) - g(x_2) \|^2 = \| x_1 - x_2 \| = \| x_1 \|^2 + \| x_2 \|^2 - 2 \langle g(x_1), g(x_2) \rangle, \]
 also bewahrt $f$ auch das Skalarprodukt.
 
 $f$ ist linear:
 \begin{align*} \| f(x_1 + x_2) - f(x_1) - f(x_2) \|^2 
    =\, &\| f(x_1 + x_2) \|^2 
      - 2 \langle f(x_1 + x_2), f(x_1) \rangle \\
    & - 2 \langle f(x_1 + x_2), f(x_2) \rangle
      - 2 \langle f(x_1 ), f(x_2) \rangle \\
    & + \| f(x_1) \|^2 + \| f(x_2) \|^2 \\
    =\, &\| x_1 + x_2 \|^2 
      - 2 \langle x_1 + x_2, x_1 \rangle
      - 2 \langle x_1 + x_2, x_2 \rangle \\
    & - 2 \langle x_1, x_2 \rangle
      + \| x_1 \|^2 + \| x_2 \|^2 \\
    =\, &\| x_1 + x_2 - x_1 - x_2 \|^2 \\
    =\, &0,
 \end{align*}
 also $f(x_1 + x_2) = f(x_1) + f(x_2)$. Es gilt auch
 \begin{align*} \| f(\lambda x) - \lambda f(x) \|^2 &= \| f( \lambda x ) \|^2 
      - 2 \lambda \langle f( \lambda x ), f( x ) \rangle 
      + \lambda^2 \| f(x) \|^2 \\
    &= \| \lambda x \|^2
      - 2 \lambda \langle \lambda x, x \rangle
      + \lambda^2 \| x \|^2 \\
    &= 0,
 \end{align*}
 also $f(\lambda x) = \lambda f(x)$.
 
 Damit ist $f$ eine (bijektive) Abbildung, die das Skalarprodukt bewahrt, also eine orthogonale Abbildung.
\end{proof}

\begin{mydef}[Ähnlichkeitsbegriff]
 Seien $A, B \in \realmat{n}{n}$. $A$ heißt \emph{ähnlich} zu $B$, falls es $S \in \realmat{n}{n}$ gibt, $S$ invertierbar und $B = S^{-1} A S$. Wir schreiben $A \approx B$.
\end{mydef}

\begin{thm}
 Sei $g: V \to V$ eine affine Abbildung mit $\dim V < \infty$ ($V$ euklidischer Vektorraum) und $f: V \to V$ mit $f(x) = g(x) - t$ mit $t := g(0)$ die zugehörige lineare Abbildung.
 
 Falls 1 \emph{kein} Eigenwert von $f$ ist, dann hat $g$ hat genau einen Fixpunkt $x_0$ und lässt sich schreiben als
 \[ g(x) = f(x-x_0) + x_0.\footnote{$f$ ist also im obigen Sinn ähnlich zu $g$.} \]
\end{thm}

\begin{proof}
 Für einen Fixpunkt von $g$ gilt
 \[ g(x_0) = x_0 \,\Leftrightarrow\, f(x_0) + t = x_0 \,\Leftrightarrow\, (\id - f)(x_0) = t \,\Leftrightarrow\, x_0 = (\id - f)^{-1}(t). \]
 Beachte $(\id -f)$ ist bijektiv, da 1 kein Eigenwert von $f$ ist (und $\dim V < \infty$). Also
 \[ f(x - x_0) + x_0 = f(x) + \underbrace{(\id-f)(x_0)}_{t} = f(x) + t = g(x). \]
\end{proof}

Sei $f:\real^n \to \real^n$ und $f(x) = Ax + t$ eine Isometrie, also $A \in O(n)$. Falls 1 Eigenwert von $A$ ist, dann zerlegen wir $t$ in eine Komponente $t_1 \in \ker(A-E)$ und $t_2 \in ker(A-E)^\bot$, $t = t_1 + t_2$. Das heißt, $t_1$ ist die orthogonale Projektion von $t$ auf $\ker(A-E)$ (Fixraum von $A$), $A_2 = t - t_1$.

\begin{bem}
 Falls $(b_1, \ldots, b_n)$ eine Orthonormalbasis von $ker(A-E)$ ist, dann ist also $t_1 = \sum_{i=1}^k \langle t, b_i \rangle b_i$.
\end{bem}

\begin{thm}
 Falls $f: \real^n \to \real^n$ mit $f(x) = Ax + t$, $A \in O(n)$ ist und $t = t_1 + t_2$ mit $t_1 \in \ker(A-E)$, $t_2 \in \ker(A-E)^\bot$, dann ist $f(x) = A(x-x_0) + x_0 + t_1$ für genau ein $x_0 \in \ker(A-E)^\bot$.
\end{thm}

\begin{proof}
 $A$ bildet $\ker(A-E)$ punktweise auf sich selbst ab. Da ferner $A \in O(n)$, bildet $A$ auch $\ker(A-E)^\bot$ auf sich selbst ab, das heißt die Abbilung $\tilde{f}: \ker(A-E)^\bot \to \ker(A-E)^\bot$ mit $\tilde{f}(x) = Ax + t_2 = f(x) - t_1$ ist eine Isometrie von $\ker(A-E)^\bot$, für die die eingeschränkte Abbildung auf diesen Unterraum nicht den Eigenwert 1 hat.
 
 Nach Satz 1.2 hat also $\tilde{f}$ genau einen Fixpunkt in diesem Unterraum und damit gilt
 \[ \tilde{f}(x) = A(x-x_0) + x_0. \]
\end{proof}

\subsection*{Klassifikation der Isometrien des $\real^2$}
Sei $A \in O(2)$, also $\left(\begin{pmatrix} a_{11} \\ a_{21} \end{pmatrix}, \begin{pmatrix} a_{12} \\ a_{22} \end{pmatrix}\right)$ ist eine Orthonormalbasis des $\real^2$ und $a_{11}^2 + a_{21}^2 = 1$. Dann gibt es $\varphi \in [0,2 \pi)$ mit $a_{11} = \cos \varphi$, $a_{21} = \sin \varphi$ und es ist entweder $\begin{pmatrix} a_{12} \\ a_{22} \end{pmatrix} = \begin{pmatrix} - \sin \varphi \\ \cos \varphi \end{pmatrix}$, falls $\det A = 1$ oder $\begin{pmatrix} a_{12} \\ a_{22} \end{pmatrix} = \begin{pmatrix} \sin \varphi \\ - \cos \varphi \end{pmatrix}$, falls $\det A = -1$.

\begin{enumerate}[1)]
 \item Falls $\det A = 1$ beschreibt $A = \begin{pmatrix} \cos \varphi & - \sin \varphi \\ \sin \varphi & \cos \varphi \end{pmatrix}$ eine Drehung um den Winkel $\varphi$. Die Eigenwerte sind komplex:
 \[ \det \begin{pmatrix} \cos \varphi - \lambda & - \sin \varphi \\ \sin \varphi & \cos \varphi - \lambda \end{pmatrix} = \lambda^2 - 2 \lambda \cos \varphi + 1, \]
 also ist $\lambda_{1,2} = \cos \varphi \pm i \sin \varphi = e^{\pm i \varphi}$.
 
 \emph{Spezialfälle:}
 \begin{enumerate}[a)]
  \item $\varphi = 0$, $A = E$. Die Abbildung $f: \real^2 \to \real^2$ mit $f(x) = x + t$ ist also für $t=0$ die \emph{Identität}, für $t \ne 0$ eine \emph{Translation} um $t$ ($t = t_1$).
  \item $\varphi = \pi$, $A = -E$. Hier ist $f$ eine \emph{Punktspiegelung} (Drehung um $180^\circ = \pi$).
 \end{enumerate}
 Mit $\varphi \ne 0$ beschreibt $f$ nach Satz 1.2 also eine \emph{Drehung}.
 \item Falls $\det A = -1$ hat $A = \begin{pmatrix} \cos \varphi & \sin \varphi \\ \sin \varphi & - \cos \varphi \end{pmatrix}$ reelle Eigenwerte:
 \[ \det \begin{pmatrix} \cos \varphi - \lambda & \sin \varphi \\ \sin \varphi & - \cos \varphi - \lambda \end{pmatrix} = \lambda^2 - 1, \]
 also gilt $\lambda_{1,2} = \pm 1$.
 
 Die affine Abbildung $f(x) = Ax + t_1 + t_2$ mit $t_1 \in \ker(A-E)$, $t_2 \in \ker(A-E)^\bot$ beschreibt für $t_1 = 0$ eine \emph{Spiegelung} an einer Geraden und für $t_1 \ne 0$ eine \emph{Schubspiegelung} an einer Geraden.
\end{enumerate}

%\begin{table}[ht]
 {\center \small
 \begin{tabularx}{.95\textwidth}{|>{\center}m{3.5cm}|>{\center}m{3cm}|>{\center}m{3cm}|X|}
  \hline
  Zugehörige lineare Abbildung in Normalform & Isometrie des $\real^2$ & Menge der Fixpunkte & Welche Geraden werden auf sich abgebildet? \\
  \hline
  $\begin{pmatrix} 1 & 0 \\ 0 & 1 \end{pmatrix}$ &
  Identität, falls $t = t_1 = 0$, &
  $\real^2$ &
  Alle Geraden \\
   & Translation, falls $t = t_1 \ne 0$
   & $\emptyset$
   & Alle Geraden parallel zu $\real t$ \\  
  \hline
  $\begin{pmatrix} -1 & 0 \\ 0 & -1 \end{pmatrix}$ &
  Punktspiegelung an $p = \frac{t}{2}$ (Drehung in $p$ um den Winkel $\pi$) &
  $\{p\}$ &
  Alle Geraden durch $p$ \\
  \hline
  $\begin{pmatrix} \cos \varphi & - \sin \varphi \\ \sin \varphi & \cos \varphi \end{pmatrix}$ &
  Drehung um einen Punkt $p$ um Winkel $\ne 0, \pi$ &
  $\{p\}$ &
  keine \\
  \hline
  $\begin{pmatrix} 1 & 0 \\ 0 & -1 \end{pmatrix}$ &
  Spiegelung an einer Geraden $g$, falls $t_1 = 0$  &
  $g$ &
  $g$ und alle Geraden orthogonal zu $g$ \\
   & Schubspiegelung an $g$, falls $t_1 \ne 0$
   & $\emptyset$
   & $g$ \\
   \hline
 \end{tabularx} }
 \[ x \mapsto Ax + t, t = t_1 + t_2 \text{ mit } t_1 \in \ker(A-E), t_2 \in \ker(A-E)^\bot \]
% \caption{Klassifikation der Isometrien des $\real^2$}
%\end{table}

Vorschlag zum zusätzlichen Selbststudium: Die 17 kristallographischen Gruppen der Ebene

\subsection*{Isometrien des $\real^3$}
Zunächst sei $A \in O(3)$ eine orthogonale $3 \times 3$-Matrix. Das charakteristische Polynom $\chi_A$ ist vom Grad 3, hat also mindestens eine reelle Nullstelle. Ferner haben alle reellen oder komplexen Nullstellen den Betrag 1.

Falls das charakteristische Polynom über $\real$ in Linearfaktoren zerfällt, gibt es eine Orthonormalbasis des $\real^3$ aus Eigenvektoren von $A$ und die Matrix $A$ ist \emph{bezüglich dieser Basis} von der Form 
\begin{align*}
 \begin{pmatrix} 1 & 0 & 0 \\ 0 & 1 & 0 \\ 0 & 0 & 1 \end{pmatrix} &\text{ (Identität) oder} \\
 \begin{pmatrix} 1 & 0 & 0 \\ 0 & 1 & 0 \\ 0 & 0 & -1 \end{pmatrix} &\text{ (Spiegelung an einer Ebene) oder} \\
 \begin{pmatrix} 1 & 0 & 0 \\ 0 & -1 & 0 \\ 0 & 0 & -1 \end{pmatrix} &\text{ (Spiegelung an einer Geraden) oder} \\
 \begin{pmatrix} -1 & 0 & 0 \\ 0 & -1 & 0 \\ 0 & 0 & -1 \end{pmatrix} &\text{ (Punktspiegelung)}.
\end{align*}
Andernfalls hat $A$ einen reellen Eigenwert $\pm 1$ und $\chi_A$ hat ferner konjugiert komplexe Nullstellen $\cos \varphi \pm i \sin \varphi$ mit $\varphi \ne 0, \pi$, $\varphi \in [0, 2\pi)$.

Sei $b_1$ der Eigenvektor zum Eigenwert $\pm 1$ mit $\| b_1 \| = 1$. Ergänze zu einer Orthonormalbasis $b_1, b_2, b_3$. Dann ist die Matrix $A$ bezüglich dieser Basis eine orthogonale Matrix und von der Form 
\[ \begin{pmatrix} 1 & 0 & 0 \\ 0 & \cos \varphi & -\sin \varphi \\ 0 & \sin \varphi & \cos \varphi \end{pmatrix}
   \text{ oder }
   \begin{pmatrix} -1 & 0 & 0 \\ 0 & \cos \varphi & -\sin \varphi \\ 0 & \sin \varphi & \cos \varphi \end{pmatrix}. \]
Im ersten Fall ($\det A = 1$) ist die lineare Abbildung eine \emph{Drehung} um die Gerade $\real b_1$ um den Winkel $\varphi$. Im zweiten Fall ($\det A = -1$) ist es eine \emph{Drehspiegelung} mit der \emph{Drehspiegelachse} $\real b_1$ mit \emph{Drehspiegelebene} $\real b_2 + \real b_3$ um den Winkel $\varphi$.

%\begin{table}[ht]
 {\center \small
 \begin{tabularx}{.95\textwidth}{|>{\center}m{3.6cm}|>{\center}m{3cm}|>{\center}m{3cm}|X|}
  \hline
  Zugehörige lineare Abbildung in Normalform 
   & Isometrie des $\real^3$ 
   & Menge der Fixpunkte 
   & Welche Geraden werden auf sich abgebildet? \\
  \hline
  $\begin{pmatrix} 1 & 0 & 0 \\ 0 & 1 & 0 \\ 0 & 0 & 1 \end{pmatrix}$ 
   & Identität, falls $t = t_1 = 0$, 
   & $\real^3$ 
   & Alle Geraden \\
   & Translation, falls $t = t_1 \ne 0$
   & $\emptyset$
   & Alle Geraden parallel zu $\real t$ \\  
  \hline
  $\begin{pmatrix} 1 & 0 & 0 \\ 0 & -1 & 0 \\ 0 & 0 & -1 \end{pmatrix}$
   & Spiegelung an einer Geraden, falls $t_1 = 0$ (Drehung um $\pi$),
   & $g$
   & $g$ und alle Geraden, die $g$ orthogonal schneiden. \\
   & Gleitspiegelung an einer Geraden, falls $t_1 \ne 0$ (Schraubung um $\pi$)
   & $\emptyset$
   & $g$ \\
  \hline
  $\begin{pmatrix} 1 & 0 & 0 \\ 0 & \cos \varphi & - \sin \varphi \\ 0 & \sin \varphi & \cos \varphi \end{pmatrix}$ 
   & Drehung an einer Geraden, falls $t_1 = 0$ 
   & $g$
   & $g$ \\
   & Schraubung, falls $t_1 \ne 0$
   & $\emptyset$
   & $g$ \\
  \hline
  $\begin{pmatrix} -1 & 0 & 0 \\ 0 & -1 & 0 \\ 0 & 0 & -1 \end{pmatrix}$
   & Punktspiegelung am Punkt $p=\frac{t}{2}$
   & $p$
   & Alle Geraden durch $p$ \\
  \hline
  $\begin{pmatrix} 1 & 0 & 0 \\ 0 & 1 & 0 \\ 0 & 0 & -1 \end{pmatrix}$
   & Spiegelung an einer Ebene $E$, falls $t_1 = 0$
   & $E$
   & Alle Geraden in $E$ und alle Geraden $\bot E$ \\
   & Gleitspiegelung an einer Ebene, falls $t_1 \ne 0$.
   & $\emptyset$
   & Alle Geraden in $E$ parallel zu $\real t$ \\
  \hline
  {\footnotesize $\begin{pmatrix} -1 & 0 & 0 \\ 0 & \cos \varphi & - \sin \varphi \\ 0 & \sin \varphi & \cos \varphi \end{pmatrix}$}
   & Drehspiegelung mit Drehspiegelachse $g$ und Fixpunkt $p$
   & $p$
   & $g$ \\
  \hline
 \end{tabularx} }
 \[ x \mapsto Ax + t, A \in O(3), t = t_1 + t_2 \text{ mit } t_1 \in \ker(A-E), t_2 \in \ker(A-E)^\bot \]
 %\caption{Klassifikation der Isometrien des $\real^3$}
%\end{table}

\begin{bem}[Orientierung bei Drehungen]
Äquivalenz über $S \in SO(n)$, das heißt $\det S=1$ statt mit $S \in O(n)$. 
 
 In $\real^2$ ist dann $\varphi$ eindeutig bestimmt:
 \[ S \begin{pmatrix} \cos \varphi & - \sin \varphi \\ \sin \varphi & \cos \varphi \end{pmatrix} S^{-1} = \begin{pmatrix} \cos \varphi & - \sin \varphi \\ \sin \varphi & \cos \varphi \end{pmatrix}, \]
 falls $S \in SO(2)$. Für $S \in O(2) \setminus SO(2)$ zum Beispiel
 \[ \begin{pmatrix} -1 & 0 \\ 0 & 1 \end{pmatrix}
    \begin{pmatrix} \cos \varphi & - \sin \varphi \\ \sin \varphi & \cos \varphi \end{pmatrix}
    \begin{pmatrix} -1 & 0 \\ 0 & 1 \end{pmatrix}
    = 
    \begin{pmatrix} \cos (-\varphi) & - \sin (-\varphi) \\ \sin (-\varphi) & \cos (-\varphi) \end{pmatrix} \]
 
 Bei Drehungen in $\real^3$ kann man (geometrisch) nicht zwischen Drehung um $\varphi$ und $-\varphi$ unterscheiden:
 \[ \begin{pmatrix} -1 & 0 & 0 \\ 0 & -1 & 0 \\ 0 & 0 & 1 \end{pmatrix}
    \begin{pmatrix} -1 & 0 & 0 \\ 0 & \cos \varphi & - \sin \varphi \\ 0 & \sin \varphi & \cos \varphi \end{pmatrix}
    \begin{pmatrix} -1 & 0 & 0 \\ 0 & -1 & 0 \\ 0 & 0 & -1 \end{pmatrix}
    =
    \begin{pmatrix} -1 & 0 & 0 \\ 0 & \cos \varphi & \sin \varphi \\ 0 & -\sin \varphi & \cos \varphi \end{pmatrix}.
 \]
 Bei Drehspiegelungen ist das genauso.
 
 Im Gegensatz dazu: bei Schraubungen wähle den Basisvektor $b_1 \in \ker(A-E)$ (mit $\|b_1\| = 1$), so dass $t_1 = \lambda b_1$ mit $\lambda > 0$. Dann ist $\varphi$ auch eindeutig bestimmt (in der Drehebene, vgl. $\real^2$).
\end{bem}

\subsection*{Normale Endomorphismen des $\real^n$}
\begin{mydef}
Sei $V$ ein unitärer oder euklidischer Raum mit Skalarprodukt $\langle \cdot, \cdot \rangle$ und $f:V \to V$ ein Endomorphismus. Falls $f^*:V \to V$ ein Endomorphismus ist, sodass für alle $x,y \in V$ gilt
\[ \langle f(x), y \rangle = \langle x, f^*(y) \rangle, \]
dann heißt $f^*$ der \emph{zu $f$ adjungierte} Endomorphismus (bezüglich $\langle \cdot, \cdot \rangle$). Falls er exisitiert, dann ist er eindeutig bestimmt.

Falls $\dim V < \infty$, dann existiert $f^*$ immer und es gilt: Falls $A$ die Matrix von $f$ bezüglich einer Orthonormalbasis ist, dann ist $A^* = \obar{A}^T$ die Matrix von $f^*$ bezüglich derselben Basis.

Ein Endomorphismus $f$ heißt \emph{normal}, falls $f^*$ existiert und $f f^* = f^* f$ gilt. Eine Matrix $A \in \compmat{n}{n}$ heißt \emph{normal}, falls $A^* A = A A^*$ gilt.
\end{mydef}

\begin{thm}[Normalform für normale Matrizen]
 Sei $A \in \compmat{n}{n}$, dann sind die folgenden Aussagen äquivalent:
 \begin{enumerate}[(1)]
  \item Es gibt eine Orthonormalbasis aus Eigenvektoren von $A$.
  \item $A$ ist normal.
  \item $A$ ist unitär ähnlich zu einer Diagonalmatrix, das heißt es gibt eine unitäre Matrix\footnote{$S \in U(S)$: Zeilen- und Spaltenvektoren sind orthonormal bzgl. des Standardskalarprodukts, $S^{-1} = S^*$.} $S$ mit $S^{-1} A S = D$, $D$ Diagonalmatrix.
 \end{enumerate}
\end{thm}

\begin{folg}
 Sei $A \in \compmat{n}{n}$, dann sind äquivalent:
 \begin{enumerate}[(1)]
  \item $A = A^*$.
  \item $A$ ist normal und alle Eigenwerte von $A$ sind reell.
 \end{enumerate}
\end{folg}

\begin{folg}
 Sei $A \in \compmat{n}{n}$, dann sind äquivalent:
 \begin{enumerate}[(1)]
  \item $A^* = A^{-1}$, das heißt $A \in U(n)$.
  \item $A$ ist normal und alle Eigenwerte von $A$ haben Betrag 1.
 \end{enumerate}
\end{folg}

\begin{lem}
 Sei $A \in \realmat{n}{n} \subset \compmat{n}{n}$. Dann gilt für $A$ als Matrix in $\compmat{n}{n}$ aufgefasst: Falls $x \in \complex^n$ ein Eigenvektor von $A$ zum Eigenwert $c \in \complex$ ist, dann ist $\obar{x}$ ein Eigenvektor von $A$ zum Eigenwert $\obar{c}$, wobei
 \[ \obar{\begin{pmatrix} x_1 \\ \vdots x_n \end{pmatrix}} := \begin{pmatrix} \obar{x}_1 \\ \vdots \obar{x}_n \end{pmatrix}. \]
\end{lem}

\begin{proof}
 Für $x \ne 0$ gilt $Ax = cx $ $\Rightarrow$  $A\obar{x} = \obar{A}\obar{x} = \obar{Ax} = \obar{cx} = \obar{c} \obar{x}$.
\end{proof}

\begin{thm}[Normalform für über $\complex$ diagonalisierbare Matrizen]
 Sei $A \in \realmat{n}{n}$ und $A$ als Element von $\compmat{n}{n}$ aufgefasst diagonalisierbar\footnote{$A$ ist zu einer Diagonalmatrix $D$ ähnlich, also ex. $S$, so dass $D=S^{-1} A S$.}. Dann gibt es eine Matrix $S \in \GL(n, \real)$ für welche $S^{-1} A S$ die Gestalt
 \[ S^{-1} A S = \begin{pmatrix}
     d_1 &        &     &     &        &      \\
         & \ddots &     &     & 0      &      \\
         &        & d_k &     &        &      \\
         &        &     & \boxed{z_1} &        &      \\
         & 0      &     &     & \ddots &      \\
         &        &     &     &        & \boxed{z_m}
    \end{pmatrix} \]
 hat, wobei $z_\mu = \begin{pmatrix} a_\mu & b_\mu \\ -b_\mu & a_\mu \end{pmatrix}$ für $\mu = 1, \ldots, m$ ist mit $b_\mu \ne 0$, $k+2m = n$. Dabei sind $d_1, \ldots, d_k$ die reellen Eigenwerte von $A$ und $a_\mu \pm i b_\mu$ die komplexen Eigenwerte von $A$ (mit Vielfachheiten).
 
 Ferner gilt: Falls $A$ normal ist, dann kann $S \in O(n)$ gewählt werden, das heißt $S^T = S^{-1}, S \in \realmat{n}{n}$).
\end{thm}

\end{document}