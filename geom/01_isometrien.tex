\section{Isometrien des \texorpdfstring{$\real^n$}{Rn}}
Eine Isometrie eines metrischen Raumes $(X, d)$ ist eine bijektive Abbildung $f: X \to X$ mit
\[ d( f(x), f(y) ) = d(x,y) \text{ für alle } x, y \in X. \]

Wir wollen die Isometriegruppe des $(\real^n, d)$ untersuchen mit $d(x,y) := \| x - y \|$.

\begin{thm}
 Sei $V$ ein euklidischer Vektorraum und $g: V \to V$ eine Isometrie. Dann ist $f: V \to V$ mit $f(x) := g(x) - g(0)$ eine orthogonale Abbildung. Insbesondere ist $g$ eine affine Abbildung.
\end{thm}

\begin{proof}
 Sei $f: V \to V$ definiert durch $f(x) := g(x) - g(0)$. Dann folgt $\| f(x) \| = \| g(x) - g(0) \| = \| x - 0 \| = \| x \|$, also
 \begin{align*}
\| f(x_1) - f(x_2) \|^2 
    &= \| f(x_1) \|^2 + \| f(x_2) \|^2 - 2 \langle f(x_1), f(x_2) \rangle \\
    &= \| x_1 \|^2 + \| x_2 \|^2 - 2 \langle f(x_1), f(x_2) \rangle.
 \end{align*}
 Es gilt auch
 \[ \| g(x_1) - g(x_2) \|^2 = \| x_1 - x_2 \| = \| x_1 \|^2 + \| x_2 \|^2 - 2 \langle g(x_1), g(x_2) \rangle, \]
 also bewahrt $f$ auch das Skalarprodukt.
 
 $f$ ist linear:
 \begin{align*} \| f(x_1 + x_2) - f(x_1) - f(x_2) \|^2 
    =\, &\| f(x_1 + x_2) \|^2 
      - 2 \langle f(x_1 + x_2), f(x_1) \rangle \\
    & - 2 \langle f(x_1 + x_2), f(x_2) \rangle
      - 2 \langle f(x_1 ), f(x_2) \rangle \\
    & + \| f(x_1) \|^2 + \| f(x_2) \|^2 \\
    =\, &\| x_1 + x_2 \|^2 
      - 2 \langle x_1 + x_2, x_1 \rangle
      - 2 \langle x_1 + x_2, x_2 \rangle \\
    & - 2 \langle x_1, x_2 \rangle
      + \| x_1 \|^2 + \| x_2 \|^2 \\
    =\, &\| x_1 + x_2 - x_1 - x_2 \|^2 \\
    =\, &0,
 \end{align*}
 also $f(x_1 + x_2) = f(x_1) + f(x_2)$. Es gilt auch
 \begin{align*} \| f(\lambda x) - \lambda f(x) \|^2 &= \| f( \lambda x ) \|^2 
      - 2 \lambda \langle f( \lambda x ), f( x ) \rangle 
      + \lambda^2 \| f(x) \|^2 \\
    &= \| \lambda x \|^2
      - 2 \lambda \langle \lambda x, x \rangle
      + \lambda^2 \| x \|^2 \\
    &= 0,
 \end{align*}
 also $f(\lambda x) = \lambda f(x)$.
 
 Damit ist $f$ eine (bijektive) Abbildung, die das Skalarprodukt bewahrt, also eine orthogonale Abbildung.
\end{proof}

\begin{mydef}[Ähnlichkeitsbegriff]
 Seien $A, B \in \realmat{n}{n}$. $A$ heißt \emph{ähnlich} zu $B$, falls es $S \in \realmat{n}{n}$ gibt, $S$ invertierbar und $B = S^{-1} A S$. Wir schreiben $A \approx B$.
\end{mydef}

\begin{thm}
 Sei $g: V \to V$ eine affine Abbildung mit $\dim V < \infty$ ($V$ euklidischer Vektorraum) und $f: V \to V$ mit $f(x) = g(x) - t$ mit $t := g(0)$ die zugehörige lineare Abbildung.
 
 Falls 1 \emph{kein} Eigenwert von $f$ ist, dann hat $g$ hat genau einen Fixpunkt $x_0$ und lässt sich schreiben als\footnote{$f$ ist also im obigen Sinn ähnlich zu $g$.}
 \[ g(x) = f(x-x_0) + x_0. \]
\end{thm}

\begin{proof}
 Für einen Fixpunkt von $g$ gilt
 \[ g(x_0) = x_0 \,\Leftrightarrow\, f(x_0) + t = x_0 \,\Leftrightarrow\, (\id - f)(x_0) = t \,\Leftrightarrow\, x_0 = (\id - f)^{-1}(t). \]
 Beachte $(\id -f)$ ist bijektiv, da 1 kein Eigenwert von $f$ ist (und $\dim V < \infty$). Also
 \[ f(x - x_0) + x_0 = f(x) + \underbrace{(\id-f)(x_0)}_{t} = f(x) + t = g(x). \]
\end{proof}

Sei $f:\real^n \to \real^n$ und $f(x) = Ax + t$ eine Isometrie, also $A \in O(n)$. Falls 1 Eigenwert von $A$ ist, dann zerlegen wir $t$ in eine Komponente $t_1 \in \ker(A-E)$ und $t_2 \in ker(A-E)^\bot$, $t = t_1 + t_2$. Das heißt, $t_1$ ist die orthogonale Projektion von $t$ auf $\ker(A-E)$ (Fixraum von $A$), $A_2 = t - t_1$.

\begin{bem}
 Falls $(b_1, \ldots, b_n)$ eine Orthonormalbasis von $ker(A-E)$ ist, dann ist also $t_1 = \sum_{i=1}^k \langle t, b_i \rangle b_i$.
\end{bem}

\begin{thm}
 Falls $f: \real^n \to \real^n$ mit $f(x) = Ax + t$, $A \in O(n)$ ist und $t = t_1 + t_2$ mit $t_1 \in \ker(A-E)$, $t_2 \in \ker(A-E)^\bot$, dann ist $f(x) = A(x-x_0) + x_0 + t_1$ für genau ein $x_0 \in \ker(A-E)^\bot$.
\end{thm}

\begin{proof}
 $A$ bildet $\ker(A-E)$ punktweise auf sich selbst ab. Da ferner $A \in O(n)$, bildet $A$ auch $\ker(A-E)^\bot$ auf sich selbst ab, das heißt die Abbildung $\tilde{f}: \ker(A-E)^\bot \to \ker(A-E)^\bot$ mit $\tilde{f}(x) = Ax + t_2 = f(x) - t_1$ ist eine Isometrie von $\ker(A-E)^\bot$, für die die eingeschränkte Abbildung auf diesen Unterraum nicht den Eigenwert 1 hat.
 
 Nach Satz 1.2 hat also $\tilde{f}$ genau einen Fixpunkt in diesem Unterraum und damit gilt
 \[ \tilde{f}(x) = A(x-x_0) + x_0. \qedhere \]
\end{proof}

\subsection*{Klassifikation der Isometrien des $\real^2$}
Sei $A \in O(2)$, also $\left(\begin{pmatrix} a_{11} \\ a_{21} \end{pmatrix}, \begin{pmatrix} a_{12} \\ a_{22} \end{pmatrix}\right)$ ist eine Orthonormalbasis des $\real^2$ und $a_{11}^2 + a_{21}^2 = 1$. Dann gibt es $\varphi \in [0,2 \pi)$ mit $a_{11} = \cos \varphi$, $a_{21} = \sin \varphi$ und es ist entweder $\begin{pmatrix} a_{12} \\ a_{22} \end{pmatrix} = \begin{pmatrix} - \sin \varphi \\ \cos \varphi \end{pmatrix}$, falls $\det A = 1$ oder $\begin{pmatrix} a_{12} \\ a_{22} \end{pmatrix} = \begin{pmatrix} \sin \varphi \\ - \cos \varphi \end{pmatrix}$, falls $\det A = -1$.

\begin{enumerate}[1)]
 \item Falls $\det A = 1$ beschreibt $A = \begin{pmatrix} \cos \varphi & - \sin \varphi \\ \sin \varphi & \cos \varphi \end{pmatrix}$ eine Drehung um den Winkel $\varphi$. Die Eigenwerte sind komplex:
 \[ \det \begin{pmatrix} \cos \varphi - \lambda & - \sin \varphi \\ \sin \varphi & \cos \varphi - \lambda \end{pmatrix} = \lambda^2 - 2 \lambda \cos \varphi + 1, \]
 also ist $\lambda_{1,2} = \cos \varphi \pm i \sin \varphi = e^{\pm i \varphi}$.
 
 \emph{Spezialfälle:}
 \begin{enumerate}[a)]
  \item $\varphi = 0$, $A = E$. Die Abbildung $f: \real^2 \to \real^2$ mit $f(x) = x + t$ ist also für $t=0$ die \emph{Identität}, für $t \ne 0$ eine \emph{Translation} um $t$ ($t = t_1$).
  \item $\varphi = \pi$, $A = -E$. Hier ist $f$ eine \emph{Punktspiegelung} (Drehung um $180^\circ = \pi$).
 \end{enumerate}
 Mit $\varphi \ne 0$ beschreibt $f$ nach Satz 1.2 also eine \emph{Drehung}.
 \item Falls $\det A = -1$ hat $A = \begin{pmatrix} \cos \varphi & \sin \varphi \\ \sin \varphi & - \cos \varphi \end{pmatrix}$ reelle Eigenwerte:
 \[ \det \begin{pmatrix} \cos \varphi - \lambda & \sin \varphi \\ \sin \varphi & - \cos \varphi - \lambda \end{pmatrix} = \lambda^2 - 1, \]
 also gilt $\lambda_{1,2} = \pm 1$.
 
 Die affine Abbildung $f(x) = Ax + t_1 + t_2$ mit $t_1 \in \ker(A-E)$, $t_2 \in \ker(A-E)^\bot$ beschreibt für $t_1 = 0$ eine \emph{Spiegelung} an einer Geraden und für $t_1 \ne 0$ eine \emph{Schubspiegelung} an einer Geraden.
\end{enumerate}

%\begin{table}[ht]
 {\center \small
 \begin{tabularx}{.95\textwidth}{|>{\center}m{3.5cm}|>{\center}m{3cm}|>{\center}m{3cm}|X|}
  \hline
  Zugehörige lineare Abbildung in Normalform & Isometrie des $\real^2$ & Menge der Fixpunkte & Welche Geraden werden auf sich abgebildet? \\
  \hline
  $\begin{pmatrix} 1 & 0 \\ 0 & 1 \end{pmatrix}$ &
  Identität, falls $t = t_1 = 0$, &
  $\real^2$ &
  Alle Geraden \\
   & Translation, falls $t = t_1 \ne 0$
   & $\emptyset$
   & Alle Geraden parallel zu $\real t$ \\  
  \hline
  $\begin{pmatrix} -1 & 0 \\ 0 & -1 \end{pmatrix}$ &
  Punktspiegelung an $p = \frac{t}{2}$ (Drehung in $p$ um den Winkel $\pi$) &
  $\{p\}$ &
  Alle Geraden durch $p$ \\
  \hline
  $\begin{pmatrix} \cos \varphi & - \sin \varphi \\ \sin \varphi & \cos \varphi \end{pmatrix}$ &
  Drehung um einen Punkt $p$ um Winkel $\ne 0, \pi$ &
  $\{p\}$ &
  keine \\
  \hline
  $\begin{pmatrix} 1 & 0 \\ 0 & -1 \end{pmatrix}$ &
  Spiegelung an einer Geraden $g$, falls $t_1 = 0$  &
  $g$ &
  $g$ und alle Geraden orthogonal zu $g$ \\
   & Schubspiegelung an $g$, falls $t_1 \ne 0$
   & $\emptyset$
   & $g$ \\
   \hline
 \end{tabularx} }
 \[ x \mapsto Ax + t, t = t_1 + t_2 \text{ mit } t_1 \in \ker(A-E), t_2 \in \ker(A-E)^\bot \]
% \caption{Klassifikation der Isometrien des $\real^2$}
%\end{table}

Vorschlag zum zusätzlichen Selbststudium: Die 17 kristallographischen Gruppen der Ebene

\subsection*{Isometrien des $\real^3$}
Zunächst sei $A \in O(3)$ eine orthogonale $3 \times 3$-Matrix. Das charakteristische Polynom $\chi_A$ ist vom Grad 3, hat also mindestens eine reelle Nullstelle. Ferner haben alle reellen oder komplexen Nullstellen den Betrag 1.

Falls das charakteristische Polynom über $\real$ in Linearfaktoren zerfällt, gibt es eine Orthonormalbasis des $\real^3$ aus Eigenvektoren von $A$ und die Matrix $A$ ist \emph{bezüglich dieser Basis} von der Form 
\begin{align*}
 \begin{pmatrix} 1 & 0 & 0 \\ 0 & 1 & 0 \\ 0 & 0 & 1 \end{pmatrix} &\text{ (Identität) oder} \\
 \begin{pmatrix} 1 & 0 & 0 \\ 0 & 1 & 0 \\ 0 & 0 & -1 \end{pmatrix} &\text{ (Spiegelung an einer Ebene) oder} \\
 \begin{pmatrix} 1 & 0 & 0 \\ 0 & -1 & 0 \\ 0 & 0 & -1 \end{pmatrix} &\text{ (Spiegelung an einer Geraden) oder} \\
 \begin{pmatrix} -1 & 0 & 0 \\ 0 & -1 & 0 \\ 0 & 0 & -1 \end{pmatrix} &\text{ (Punktspiegelung)}.
\end{align*}
Andernfalls hat $A$ einen reellen Eigenwert $\pm 1$ und $\chi_A$ hat ferner konjugiert komplexe Nullstellen $\cos \varphi \pm i \sin \varphi$ mit $\varphi \ne 0, \pi$, $\varphi \in [0, 2\pi)$.

Sei $b_1$ der Eigenvektor zum Eigenwert $\pm 1$ mit $\| b_1 \| = 1$. Ergänze zu einer Orthonormalbasis $b_1, b_2, b_3$. Dann ist die Matrix $A$ bezüglich dieser Basis eine orthogonale Matrix und von der Form 
\[ \begin{pmatrix} 1 & 0 & 0 \\ 0 & \cos \varphi & -\sin \varphi \\ 0 & \sin \varphi & \cos \varphi \end{pmatrix}
   \text{ oder }
   \begin{pmatrix} -1 & 0 & 0 \\ 0 & \cos \varphi & -\sin \varphi \\ 0 & \sin \varphi & \cos \varphi \end{pmatrix}. \]
Im ersten Fall ($\det A = 1$) ist die lineare Abbildung eine \emph{Drehung} um die Gerade $\real b_1$ um den Winkel $\varphi$. Im zweiten Fall ($\det A = -1$) ist es eine \emph{Drehspiegelung} mit der \emph{Drehspiegelachse} $\real b_1$ mit \emph{Drehspiegelebene} $\real b_2 + \real b_3$ um den Winkel $\varphi$.

%\begin{table}[ht]
 {\center \small
 \begin{tabularx}{.95\textwidth}{|>{\center}m{3.6cm}|>{\center}m{3cm}|>{\center}m{3cm}|X|}
  \hline
  Zugehörige lineare Abbildung in Normalform 
   & Isometrie des $\real^3$ 
   & Menge der Fixpunkte 
   & Welche Geraden werden auf sich abgebildet? \\
  \hline
  $\begin{pmatrix} 1 & 0 & 0 \\ 0 & 1 & 0 \\ 0 & 0 & 1 \end{pmatrix}$ 
   & Identität, falls $t = t_1 = 0$, 
   & $\real^3$ 
   & Alle Geraden \\
   & Translation, falls $t = t_1 \ne 0$
   & $\emptyset$
   & Alle Geraden parallel zu $\real t$ \\  
  \hline
  $\begin{pmatrix} 1 & 0 & 0 \\ 0 & -1 & 0 \\ 0 & 0 & -1 \end{pmatrix}$
   & Spiegelung an einer Geraden, falls $t_1 = 0$ (Drehung um $\pi$),
   & $g$
   & $g$ und alle Geraden, die $g$ orthogonal schneiden. \\
   & Gleitspiegelung an einer Geraden, falls $t_1 \ne 0$ (Schraubung um $\pi$)
   & $\emptyset$
   & $g$ \\
  \hline
  $\begin{pmatrix} 1 & 0 & 0 \\ 0 & \cos \varphi & - \sin \varphi \\ 0 & \sin \varphi & \cos \varphi \end{pmatrix}$ 
   & Drehung an einer Geraden, falls $t_1 = 0$ 
   & $g$
   & $g$ \\
   & Schraubung, falls $t_1 \ne 0$
   & $\emptyset$
   & $g$ \\
  \hline
  $\begin{pmatrix} -1 & 0 & 0 \\ 0 & -1 & 0 \\ 0 & 0 & -1 \end{pmatrix}$
   & Punktspiegelung am Punkt $p=\frac{t}{2}$
   & $p$
   & Alle Geraden durch $p$ \\
  \hline
  $\begin{pmatrix} 1 & 0 & 0 \\ 0 & 1 & 0 \\ 0 & 0 & -1 \end{pmatrix}$
   & Spiegelung an einer Ebene $E$, falls $t_1 = 0$
   & $E$
   & Alle Geraden in $E$ und alle Geraden $\bot E$ \\
   & Gleitspiegelung an einer Ebene, falls $t_1 \ne 0$.
   & $\emptyset$
   & Alle Geraden in $E$ parallel zu $\real t$ \\
  \hline
  {\footnotesize $\begin{pmatrix} -1 & 0 & 0 \\ 0 & \cos \varphi & - \sin \varphi \\ 0 & \sin \varphi & \cos \varphi \end{pmatrix}$}
   & Drehspiegelung mit Drehspiegelachse $g$ und Fixpunkt $p$
   & $p$
   & $g$ \\
  \hline
 \end{tabularx} }
 \[ x \mapsto Ax + t, A \in O(3), t = t_1 + t_2 \text{ mit } t_1 \in \ker(A-E), t_2 \in \ker(A-E)^\bot \]
 %\caption{Klassifikation der Isometrien des $\real^3$}
%\end{table}

\begin{bem}[Orientierung bei Drehungen]
Äquivalenz über $S \in SO(n)$, das heißt $\det S=1$ statt mit $S \in O(n)$. 
 
 In $\real^2$ ist dann $\varphi$ eindeutig bestimmt:
 \[ S \begin{pmatrix} \cos \varphi & - \sin \varphi \\ \sin \varphi & \cos \varphi \end{pmatrix} S^{-1} = \begin{pmatrix} \cos \varphi & - \sin \varphi \\ \sin \varphi & \cos \varphi \end{pmatrix}, \]
 falls $S \in SO(2)$. Für $S \in O(2) \setminus SO(2)$ zum Beispiel
 \[ \begin{pmatrix} -1 & 0 \\ 0 & 1 \end{pmatrix}
    \begin{pmatrix} \cos \varphi & - \sin \varphi \\ \sin \varphi & \cos \varphi \end{pmatrix}
    \begin{pmatrix} -1 & 0 \\ 0 & 1 \end{pmatrix}
    = 
    \begin{pmatrix} \cos (-\varphi) & - \sin (-\varphi) \\ \sin (-\varphi) & \cos (-\varphi) \end{pmatrix} \]
 
 Bei Drehungen in $\real^3$ kann man (geometrisch) nicht zwischen Drehung um $\varphi$ und $-\varphi$ unterscheiden:
 \[ \begin{pmatrix} -1 & 0 & 0 \\ 0 & -1 & 0 \\ 0 & 0 & 1 \end{pmatrix}
    \begin{pmatrix} -1 & 0 & 0 \\ 0 & \cos \varphi & - \sin \varphi \\ 0 & \sin \varphi & \cos \varphi \end{pmatrix}
    \begin{pmatrix} -1 & 0 & 0 \\ 0 & -1 & 0 \\ 0 & 0 & -1 \end{pmatrix}
    =
    \begin{pmatrix} -1 & 0 & 0 \\ 0 & \cos \varphi & \sin \varphi \\ 0 & -\sin \varphi & \cos \varphi \end{pmatrix}.
 \]
 Bei Drehspiegelungen ist das genauso.
 
 Im Gegensatz dazu: bei Schraubungen wähle den Basisvektor $b_1 \in \ker(A-E)$ (mit $\|b_1\| = 1$), so dass $t_1 = \lambda b_1$ mit $\lambda > 0$. Dann ist $\varphi$ auch eindeutig bestimmt (in der Drehebene, vgl. $\real^2$).
\end{bem}

\subsection*{Normale Endomorphismen des $\real^n$}
\begin{mydef}
Sei $V$ ein unitärer oder euklidischer Raum mit Skalarprodukt $\langle \cdot, \cdot \rangle$ und $f:V \to V$ ein Endomorphismus. Falls $f^*:V \to V$ ein Endomorphismus ist, sodass für alle $x,y \in V$ gilt
\[ \langle f(x), y \rangle = \langle x, f^*(y) \rangle, \]
dann heißt $f^*$ der \emph{zu $f$ adjungierte} Endomorphismus (bezüglich $\langle \cdot, \cdot \rangle$). Falls er existiert, dann ist er eindeutig bestimmt.

Falls $\dim V < \infty$, dann existiert $f^*$ immer und es gilt: Falls $A$ die Matrix von $f$ bezüglich einer Orthonormalbasis ist, dann ist $A^* = \obar{A}^T$ die Matrix von $f^*$ bezüglich derselben Basis.

Ein Endomorphismus $f$ heißt \emph{normal}, falls $f^*$ existiert und $f f^* = f^* f$ gilt. Eine Matrix $A \in \compmat{n}{n}$ heißt \emph{normal}, falls $A^* A = A A^*$ gilt.
\end{mydef}

\begin{thm}[Normalform für normale Matrizen]
 Sei $A \in \compmat{n}{n}$, dann sind die folgenden Aussagen äquivalent:
 \begin{enumerate}[(1)]
  \item Es gibt eine Orthonormalbasis aus Eigenvektoren von $A$.
  \item $A$ ist normal.
  \item $A$ ist unitär ähnlich zu einer Diagonalmatrix, das heißt es gibt eine unitäre Matrix\footnote{$S \in U(S)$: Zeilen- und Spaltenvektoren sind orthonormal bzgl. des Standardskalarprodukts, $S^{-1} = S^*$.} $S$ mit $S^{-1} A S = D$, $D$ Diagonalmatrix.
 \end{enumerate}
\end{thm}

\begin{folg}
 Sei $A \in \compmat{n}{n}$, dann sind äquivalent:
 \begin{enumerate}[(1)]
  \item $A = A^*$.
  \item $A$ ist normal und alle Eigenwerte von $A$ sind reell.
 \end{enumerate}
\end{folg}

\begin{folg}
 Sei $A \in \compmat{n}{n}$, dann sind äquivalent:
 \begin{enumerate}[(1)]
  \item $A^* = A^{-1}$, das heißt $A \in U(n)$.
  \item $A$ ist normal und alle Eigenwerte von $A$ haben Betrag 1.
 \end{enumerate}
\end{folg}

\begin{lem}
 Sei $A \in \realmat{n}{n} \subset \compmat{n}{n}$. Dann gilt für $A$ als Matrix in $\compmat{n}{n}$ aufgefasst: Falls $x \in \complex^n$ ein Eigenvektor von $A$ zum Eigenwert $c \in \complex$ ist, dann ist $\obar{x}$ ein Eigenvektor von $A$ zum Eigenwert $\obar{c}$, wobei
 \[ \obar{\begin{pmatrix} x_1 \\ \vdots \\ x_n \end{pmatrix}} := \begin{pmatrix} \obar{x}_1 \\ \vdots \\ \obar{x}_n \end{pmatrix}. \]
\end{lem}

\begin{proof}
 Für $x \ne 0$ gilt $Ax = cx $ $\Rightarrow$  $A\obar{x} = \obar{A}\obar{x} = \obar{Ax} = \obar{cx} = \obar{c} \obar{x}$.
\end{proof}

\begin{thm}[Normalform für über $\complex$ diagonalisierbare Matrizen]
 Sei $A \in \realmat{n}{n}$ und $A$ als Element von $\compmat{n}{n}$ aufgefasst diagonalisierbar\footnote{$A$ ist zu einer Diagonalmatrix $D$ ähnlich, also ex. $S$, so dass $D=S^{-1} A S$.}. Dann gibt es eine Matrix $S \in \GL(n, \real)$ für welche $S^{-1} A S$ die Gestalt
 \[ S^{-1} A S = \begin{pmatrix}
     d_1 &        &     &     &        &      \\
         & \ddots &     &     & 0      &      \\
         &        & d_k &     &        &      \\
         &        &     & \boxed{z_1} &        &      \\
         & 0      &     &     & \ddots &      \\
         &        &     &     &        & \boxed{z_m}
    \end{pmatrix} \]
 hat, wobei $z_\mu = \begin{pmatrix} a_\mu & b_\mu \\ -b_\mu & a_\mu \end{pmatrix}$ für $\mu = 1, \ldots, m$ ist mit $b_\mu \ne 0$, $k+2m = n$. Dabei sind $d_1, \ldots, d_k$ die reellen Eigenwerte von $A$ und $a_\mu \pm i b_\mu$ die komplexen Eigenwerte von $A$ (mit Vielfachheiten).
 
 Ferner gilt: Falls $A$ normal ist, dann kann $S \in O(n)$ gewählt werden, das heißt $S^T = S^{-1}, S \in \realmat{n}{n}$).
\end{thm}

\begin{proof}
 Das charakteristische Polynom $\chi_A = \det ( XE - A )$ zerfällt über $\complex$ in Linearfaktoren. Da $A \in \realmat{n}{n}$ ist $\det(XE-A)$ ein Polynom mit reellen Koeffizienten und die Nullstellen kommen reell oder in Paaren komplex konjugierter Zahlen vor.
 
 Wir können also die Nullstellen (unter Berücksichtigung der Vielfachheit) sortieren als
 \[ d_1, \ldots, d_n, a_1 + i b_1, a_1 - i b_1, \ldots, a_m + i b_m, a_m - i b_m \]
 mit $d_\mu, a_\mu, b_\mu \in \real$, $b_\mu \ne 0$.
 
 Da $A$ über $\complex$ diagonalisierbar ist, gibt es eine Basis von Eigenvektoren. Falls $u_1, \ldots, u_r$ eine Basis von (komplexem) Eigenvektoren und $A$ zu einem nicht-reellen Eigenwert $a + ib$ von $A$ ist, dann sind $\obar{u}_1, \ldots, \obar{u}_r$ (nach Lemma 1.4) eine Basis von Eigenvektoren zum Eigenwert $a - ib$. 
 
 Falls $A$ normal\footnote{$A^T A = A A^T$} ist, dann gibt es nach Satz 1.1 sogar eine komplexe Orthonormalbasis\footnote{$\scalar{\obar{u}_i, \obar{u}_j} = \scalar{u_i, u_j} = \delta_{ij}$} von Eigenvektoren. Wähle in diesem Fall $u_1, \ldots, u_r$ als Orthonormalbasis von Eigenvektoren von $A$ zum Eigenwert $a+ib$. $\obar{u}_1, \ldots, \obar{u}_r$ sind dann eine Orthonormalbasis von Eigenvektoren zum Eigenwert $a-ib$. 
 
 Man erhält die gewünschte Basis folgendermaßen: Für die reellen Eigenwerte wähle reelle Eigenvektoren (als Orthonormalsystem bei normaler Matrix $A$). Für ein Paar komplex konjugierter Eigenwerte $a \pm ib$ wähle als Basisvektoren
 \[ \rez{\sqrt{2}} ( u_1 + \obar{u}_1 ), \rez{\sqrt{2}} (\obar{u}_1 - u_1), \ldots, 
    \rez{\sqrt{2}} ( u_r + \obar{u}_r ), \rez{\sqrt{2}} (\obar{u}_r - u_r). \]
 Das sind Vektoren aus $\real^n$ ($\sqrt{2} \cdot$ Realteil von $u_\mu$, $\sqrt{2} \cdot$ Imaginärteil von $u_\mu$). Dann gilt
 \begin{align*}
 A \left( \rez{\sqrt{2}} ( u_1 + \obar{u}_1 ) \right) 
      &= \rez{\sqrt{2}} ( (a +ib) u_1 - (a-ib) \obar{u}_1 ) \\
      &= a \left( \rez{\sqrt{2}} ( u_1 + \obar{u}_1 ) \right ) 
       - b \left( \frac{i}{\sqrt{2}}( \obar{u}_1 - u_1 ) \right),
 \end{align*}
 \begin{align*}
  A \left( \rez{\sqrt{2}} ( \obar{u}_1 - u_1 ) \right) 
      &= \frac{i}{\sqrt{2}} ( (a -ib) \obar{u}_1 - (a+ib) u_1 ) \\
      &= b \left( \rez{\sqrt{2}} ( u_1 + \obar{u}_1 ) \right ) 
       + a \left( \frac{i}{\sqrt{2}}( \obar{u}_1 - u_1 ) \right).
 \end{align*}
 Damit erhalten wir also die reelle $2 \times 2$-Matrix $\begin{pmatrix} a & b \\ -b & a \end{pmatrix}$, entsprechend für die anderen Paare von Eigenvektoren. Damit ist die behauptete Normalform gezeigt.
 
 Wir erhalten im Fall von normalen Matrizen eine reelle Orthonormalbasis, denn zu verschiedenen (komplexen) Eigenwerten sind die (komplexen) Eigenräume orthogonal (uni\-täres Standardskalarprodukt auf $\complex^n$. Die Eigenvektoren haben Betrag 1:
 \[ \left\| \rez{\sqrt{2}} ( u_1 + \obar{u}_1 ) \right\| = \scalar{\rez{\sqrt{2}} ( u_1 + \obar{u}_1 ), \rez{\sqrt{2}} ( u_1 + \obar{u}_1 )} = 1, \]
 da $\| u_1 \| = \| \obar{u}_1 \| = 1$ und $\scalar{\obar{u}_1, u_1} = 0$. Genauso
 \[ \left\| \frac{i}{\sqrt{2}} ( u_1 - \obar{u}_1 ) \right\| = 1. \]
 Nun zeigen wir die paarweise Orthogonalität:
 \[ \scalar{\rez{\sqrt{2}} (u_1 + \obar{u}_1), \frac{i}{\sqrt{2}} (\obar{u}_1 - u_1)} = \frac{i}{\sqrt{2}} \scalar{u_1, u_1} - \frac{i}{\sqrt{2}} \scalar{\obar{u}_1, \obar{u}_1} = 0 \]
 Bei mehrfachen komplexen Eigenwerten hatten wir eine Orthonormalbasis gewählt, also gilt zum Beispiel
 \[ \scalar{\rez{\sqrt{2}} ( u_1 + \obar{u}_1 ), \rez{\sqrt{2}} ( u_k + \obar{u}_k )} = 0, \]
 wegen $\scalar{u_1, u_k} = 0$ (auch für $\scalar{u_1, \obar{u}_k}, \ldots$).
 
 Insgesamt haben wir also eine reelle Orthonormalbasis gefunden, bezüglich der die normale Matrix $A$ die gewünschte Normalform hat.
\end{proof}

\subsection*{Geometrische Interpretation}
Sei $z = a \pm ib = |z| e^{\pm i \varphi} = |z| ( \cos \varphi \pm i \sin \varphi)$ ein paar komplex konjugierter Eigenwerte von $A$ mit $b \ne 0$. Die zugehörige $2 \times 2$-Matrix
\[ \begin{pmatrix} |z| \cos \varphi & - |z| \sin \varphi \\ |z| \sin \varphi & |z| \cos \varphi \end{pmatrix}
   = |z| \begin{pmatrix} \cos \varphi & - \sin \varphi \\ \sin \varphi & \cos \varphi \end{pmatrix} \]
beschreibt eine \emph{Drehstreckung} mit Streckungsfaktor $|z| = \sqrt{a^2 + b^2}$ und Drehwinkel $\varphi$ (mit $\varphi \ne 0, pi$).

\textbf{Folgerung aus Satz 1.8:}
\begin{thm}
 \begin{enumerate}[a)]
  \item Sei $A \in O(n)$ und $k = \dim \ker(A-E)$. Dann lässt sich $A$ als Produkt von $n-k$ Spiegelungen an Hyperebenen schreiben.
  \item Jede Isometrie des $\real^n$ lässt sich als Produkt von höchstens $n+1$ Spiegelungen an (affinen) Hyperebenen schreiben.
 \end{enumerate}
\end{thm}

%Aufgabe: Was ist geometrisch das Produkt von zwei Spiegelungen an windschiefen Geraden des $\real^n$?

%Betrachte die zugehörigen linearen Abbildungen: Es ergibt sich eine Drehung um den doppelten Winkel des Winkels zwischen $g_1$ und $g_2$.