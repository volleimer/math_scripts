\documentclass[
 a4paper,
 12pt,
 parskip=half
 ]{scrartcl}

\usepackage{../.tex/settings}

\usepackage{../.tex/mathpkgs}
\usepackage{../.tex/mathcmds}

\usepackage{../.tex/fancy_thm}

\theoremstyle{plain}
%\newtheorem*{thm}{Satz}
%\newtheorem*{lem}{Lemma}
%\newtheorem*{kor}{Korollar}

\theoremstyle{definition}
\newtheorem*{defn}{Definition}
\newtheorem*{exmp}{Beispiel}
\newtheorem*{rmrk}{Bemerkung}
\newtheorem*{folg}{Folgerung}

\numberwithin{equation}{section}

\renewcommand{\thesection}{\Roman{section}} 
\renewcommand{\thesubsection}{\arabic{subsection}}

%opening
\title{Vorlesung\\Analysis}
\subtitle{Wintersemester 2016/2017}
\author{Vorlesung: Prof. Dr. Peter Hornung\\Mitschrift: Jonas Hippold}

\hypersetup{
  pdftitle={Analysis III},
  hidelinks
}

\RedeclareSectionCommand[
  tocnumwidth=1cm
]{section}

\begin{document}

\maketitle

\tableofcontents

\clearpage

\setcounter{section}{10}
\section{Abriss der Maßtheorie}
\subsection*{Motivation und Idee des Lebesgue-Integrals}
\addcontentsline{toc}{subsection}{Motivation und Idee des Lebesgue-Integrals}
\subsubsection*{Erinnerung an das Riemann-Integral auf $[a,b] \subset \real$}
\addcontentsline{toc}{subsubsection}{Erinnerung an das Riemann-Integral auf \texorpdfstring{$[a,b] \subset \real$}{[a,b] in IR}}
Bausteine sind die Treppenfunktionen $\varphi \in \stair$ mit zugehöriger Unterteilung $a = x_0 < x_1 < \ldots < x_n = b$ und $\varphi = c_k$ auf $(x_{k_1},x_k)$:
\[ \int_a^b \varphi := \sum_{k=1}^n c_k \cdot (x_k - x_{k-1}). \]

Das Integral für beliebige (Riemann-integrierbare) $f:[a,b]\to\real$: Approximation von $f$ durch folgendermaßen gewählte Treppenfunktionen $\varphi \in \stair$:
\begin{enumerate}
 \item Wähle (immer feinere) Unterteilung $a = x_0 < x_1 < \ldots < x_n = b$ des \emph{Definitionsbereichs} $[a,b]$.
 \item Wähle $c_1, \ldots, c_n$ in Abhängigkeit von $f$.
\end{enumerate}

\subsubsection*{Lebesgue-Integral auf $[a,b] \subset \real$}
\addcontentsline{toc}{subsubsection}{Lebesgue-Integral auf \texorpdfstring{$[a,b] \subset \real$}{[a,b] in IR}}
Bausteine sind \emph{einfache Funktionen}.

Eine Funktion $\varphi: [a,b] \to \real$ heißt \emph{einfach} $:\Leftrightarrow$ $\varphi$ nimmt nur endlich viele verschiedene Werte an:
\[ \sharp \varphi([a,b]) < \infty. \]
Unterschied zu Treppenfunktionen: $\varphi^{-1}(c_k)$ ist im Allgemeinen kein Intervall.

Lebesgue-Integral für einfache Funktionen $\varphi:[a,b] \to \{ c_1, \ldots, c_n \}$:
\[ \int_a^b \varphi := \sum_{k=1}^n c_k \cdot [ \text{Länge von } \varphi^{-1}(c_k) ]. \]

Lebesgue-Integral für beliebige (Lebesgue-integrierbare) $f:[a,b]\to\real$: Approximation von $f$ durch einfache Funktionen $\varphi$:
\begin{enumerate}
 \item Wähle (immer feinere) Unterteilung des \emph{Wertebereichs} von $f$.
 \item Wähle $A_k := f^{-1}([c_{k-1},c_k])$ und setze $\varphi := c_k$ auf $A_k$.
\end{enumerate}

Diese Definition überträgt sich sofort auf $f: \real^n \to \real$ und sogar auf auf $f: X \to \real^n$ mit beliebigem merischen Raum $X$. Problem dabei ist die Frage: 
\begin{itemize}
 \item Wie ``lang'' ist $f^{-1}((c_{k-1},c_k))$?
 \item Welches ``Volumen'' hat diese Menge? (zum Beispiel $f: \real^3 \to \real^3$)
 \item Allgemein: Welches \emph{Maß} hat sie?
\end{itemize}
Ist ein sinnvoller ``Volumen''-Begriff für jede Menge in $\real^n$ überhaupt möglich? Nein! Siehe Banach-Tarski. Aber für eine geeignete Klasse von (sogenannten messbaren) Teilmengen schon.

\subsection{Äußere Maße}
Im Folgenden bezeichnet $2^X$ die Potenzmenge einer Menge $X$ (Bezeichnung in [MINT]: $\pot(X)$).

Wir bezeichnen außerdem $A \subset B$ $:\Leftrightarrow$ $x \in A \Rightarrow x \in B$, insbesondere ist $A = B$ möglich.

\subsubsection{Erweiterte Zahlengerade}
Wir werden Funktionen betrachten, die Werte in $[-\infty, \infty]$ annehmen. Die Ordnung von $\real$ wird durch folgende Konvention fortgesetzt:
\[ - \infty < a < \infty \text{ für alle } a \in \real. \]
Desweiteren ist 
\[ a + (\pm \infty) = \pm \infty \text{ für alle } a \in \real \]
sowie 
\[ a \cdot (\pm \infty) = \begin{cases}
                           \pm \infty, &\text{falls } a \in (0, \infty] \\
                           \mp \infty, &\text{falls } a \in [-\infty, 0) \\
                           0, &\text{falls } a = 0.
                          \end{cases} \]

\subsubsection{Äußere Maße}
Sei $X$ eine nichtleere Menge. Eine Funktion $\mu: 2^X \to [0,\infty]$ heißt \emph{äußeres Maß} auf $X$ $:\Leftrightarrow$
\begin{enumerate}
 \item $\mu(\emptyset) = 0$.
 \item Monotonie: $A \subset B \subset X\, \Rightarrow \, \mu(A) \le \mu(B)$.
 \item $\sigma$-Subadditivität: Für jede Folge $(A_n)_{n \in \nat} \subset 2^X$ gilt:
  \[ \mu \left( \bigcup_{n=0}^\infty A_n \right) \le \sum_{n=0}^\infty \mu( A_n ) \]
\end{enumerate}

\subsubsection{\texorpdfstring{$\mu$}{Mu}-messbare Mengen}
Sei $\mu$ ein äußeres Maß auf $X$. Eine Menge $E \subset X$ heißt $\mu$-messbar\footnote{Forster Analysis III: ``Caratheodory''-messbar} $:\Leftrightarrow$
 \[ \mu(F) = \mu( F \cap E ) + \mu( F \setminus E ) \text{ für alle } F \subset X. \]

\begin{rmrk}
 Wegen der Subadditivität ist das genau dann der Fall, wenn $\mu(F) \ge \mu(F \cap E) + \mu(F \setminus E)$ ist für alle $F$.
\end{rmrk}

Die Menge aller $\mu$-messbaren Mengen wird mit $\sigma(\mu) \subset 2^X$ bezeichnet.

\subsubsection{\texorpdfstring{$\sigma$}{Sigma}-Algebra}
Sei $X$ eine nichtleere Menge. Eine Familie $\mA \subset 2^X$ heißt \emph{$\sigma$-Algebra} $:\Leftrightarrow$
\begin{enumerate}
 \item $X \in \mA$.
 \item $A \in \mA \Rightarrow X \setminus A \in \mA$.
 \item $(A_n)_{n \in \nat} \subset \mA \Rightarrow \bigcup\limits_{n=0}^\infty A_n \in \mA$.
\end{enumerate}
Das Paar $(X, \mA)$ heißt \emph{Messraum}. 

Analogie: Topologie $\mathcal{T}$ auf einem metrischen Raum.
\[ \mathcal{T} = \{ \text{ offene Mengen auf } X \}. \]

\subsubsection{Borel-Mengen}
Sei $X$ ein metrischer Raum (zum Beispiel $X = \real^n$). 

Erinnerung: $U \subset X$ offen $:\Leftrightarrow$
\[ \forall x \in U \exists \tau > 0: B_\tau(x) \subset U. \]
\begin{itemize}
 \item \emph{Topologie} auf $X$ $:=$ Familie aller offenen Teilmengen von $X$.
 \item $A \subset X$ \emph{abgeschlossen} $:\Leftrightarrow$ $X \setminus A$ offen.
\end{itemize}

Definitionen:
\begin{itemize}
 \item \emph{Borel-$\sigma$-Algebra} $\borel(X)$ $:=$ kleinste $\sigma$-Algebra, die alle offenen Teilmengen von $X$ enthält (Existenz: [MINT]).
 \item Elemente von $\borel(X)$ heißen \emph{Borel-Mengen}.
 \item Borel-Mengen auf $[-\infty,\infty]$ sind per Definition von der Form $B$, $B \cup \{ \pm \infty \}$, $B \cup \{ -\infty, \infty \}$, wobei $B \in \borel(\real)$.
\end{itemize}

\begin{rmrk}
 Alle offenen und alle abgeschlossenen Teilmengen von $X$ sind Borel-mess\-bar. Ebenso $F_\sigma$-Mengen ($=:$ Vereinigungen abzählbar vieler abgeschlossener Mengen) und $G_\delta$-Mengen ($=:$ Durchschnitt abzählbar vieler offener Mengen).
\end{rmrk}

\begin{proof}
 Das folgt direkt aus der Definition der $\sigma$-Algebra.
\end{proof}

\subsubsection{Maß und Maßraum}
Sei $(X,\mA)$ ein Messraum. Ein \emph{Maß} auf $\mA$ ist eine Funktion $\mu: \mA \to [0,\infty]$ mit folgenden Eigenschaften:
\begin{enumerate}
 \item $\mu(\emptyset)=0$.
 \item $\sigma$-Additivität: für jede Folge $(A_n)_{n \in \nat} \subset \mA$ von \emph{paarweise disjunkten} Mengen gilt:
 \[ \mu\left( \bigcup_{n=0}^\infty A_n \right) = \sum_{n=0}^\infty \mu(A_n). \]
\end{enumerate}

Das Tripel $(X, \mA, \mu)$ heißt dann \emph{Maßraum}.

Weitere Definitionen:
\begin{itemize}
 \item Ein Maß $\mu$ auf einem metrischen Raum $X$ heißt \emph{Borel-Maß} $:\Leftrightarrow$ $(X, \borel(X), \mu)$ ist ein Maßraum.
 \item $A \in \mA$ mit $\mu(A) = 0$ heißt \emph{$\mu$-Nullmenge}.
 \item Ein Maßraum $(X, \mA, \mu)$ heißt \emph{vollständig} $:\Leftrightarrow$ Für jede Nullmenge $A \in \mA$ gilt: $F \subset A$, dann ist $F \in \mA$ (und damit auch Nullmenge).
 \item Eine Aussage über Elemente $x \in X$ gilt \emph{fast überall} $:\Leftrightarrow$
 $\{ x \in X :$  Aussage gilt für  $x$ nicht$\}$
 ist eine $\mu$-Nullmenge.
\end{itemize}

\begin{rmrk}
 Abzählbare Vereinigungen von Nullmengen sind wieder Nullmengen.
\end{rmrk}

\subsubsection{Maß und äußeres Maß}
\begin{thm}
 Jedes Maß definiert ein äußeres Maß und jedes äußere Maß definiert ein Maß. Genauer:
 \begin{enumerate}
  \item Sei $(X,\mA,\mu)$ ein Maßraum. Definiere $\mu^\ast:2^X \to [0,\infty]$ durch
  \[ \mu^\ast(A) := \inf \{ \mu(E) : E \subset \mA \text{ und } A \subset E \} \]
  für jedes $A \in 2^X$.
  
  Dann ist $\mu^\ast$ ein äußeres Maß auf $X$ und jede Menge $E \in \mA$ ist $\mu^\ast$-messbar.
  \item Sei $\mu$ ein äußeres Maß auf (nichtleerer) Menge $X$. Dann ist $(X, \sigma(\mu), \mu)$ ein voll\-ständiger Maßraum.
 \end{enumerate}
\end{thm}

Beweis und weitere Eigenschaften von Maßen siehe [MINT].

\subsubsection{Lebesgue-Maß auf \texorpdfstring{$\real^n$}{IRn}}
Setze $Q_r(a) := a + \left[ - \frac{r}{2}, \frac{r}{2} \right)^n$ für jedes $a \in \real^n$ und jedes $r > 0$.\footnote{Das ist ein $n$-dimensionaler Würfel mit Kantenlänge $r$ und Mittelpunkt $a$. Das ``Volumen'' ist $r^n$.}

Für jedes $E \subset \real^n$ definiere
\[ \lebesgue^n (E) := \inf \left\{ \sum_{i=0}^\infty r_i^n : E \subset \bigcup_{i=0}^\infty Q_{r_i}(a_i) \right\}.\footnote{Das lässt sich als ``Volumen'' der ``kleinsten Überdeckung'' von $E$ mit Würfeln interpretieren.} \]

\begin{thm}
 $\lebesgue^n$ ist ein äußeres Maß auf $\real^n$, und jede Borel-Menge ist $\lebesgue^n$-messbar.
\end{thm}

Definitionen:
\begin{itemize}
 \item $\lebesgue^n$ heißt (äußeres) \emph{Lebesgue-Maß} auf $\real^n$.
 \item Die Elemente von $\sigma( \lebesgue^n )$ werden als \emph{Lebesgue-messbare} Mengen bezeichnet\footnote{Satz besagt: Jede Borelmenge in $\real^n$ ist Lebesgue-messbar}.
 \item Das zu $\lebesgue^n$ gehörige Maß auf $\sigma( \lebesgue^n )$ (und auf $\borel( \real^n )$ ) bezeichnen wir ebenfalls mit $\lebesgue^n$.
\end{itemize}

\begin{rmrk}
\begin{itemize}
 \item $\lebesgue^n$ ist das einzige Borel-Maß mit 
 \[ \lebesgue^n( Q_r(a) ) = r^n \text{ für alle } a \in \real^n, r > 0. \]
 \item Es gilt $\lebesgue^n( \partial Q_r(a) ) = 0$\footnote{Allgemein: alle $m$-dimensionalen Mengen mit $m < n$ sind $\lebesgue^n$-Nullmengen.}. Insbesondere also
\[ \lebesgue^n \left(a + \left( - \frac{r}{2}, \frac{r}{2} \right)^n \right) = \lebesgue^n \left(a + \left[ - \frac{r}{2}, \frac{r}{2} \right]^n \right) = \lebesgue^n \left(a + \left( - \frac{r}{2}, \frac{r}{2} \right]^n \right) = r^n. \]
Also für $n=1$ und $b \ge a$ gilt:
\[ \lebesgue^1( (a,b) ) = \lebesgue^1( [a,b) ) = \lebesgue^1( (a,b] ) = \lebesgue^1( [a,b] ) = b-a. \]
\end{itemize}
\end{rmrk}

\subsection{Lebesgue-Integral}
\subsubsection{Charakteristische Funktionen}
Sei $X$ eine nichtleere Menge. Für jedes $A \subset X$ definiert man die \emph{charakteristische Funktion} $\chi_A : X \to \{ 0, 1 \}$ durch 
\[ \chi_A(x) := \begin{cases}
              1, &\text{falls } x \in A \\
              0, &\text{sonst.}
             \end{cases} \]

\subsubsection{Einfache Funktionen}
Sei $X$ eine nichtleere Menge. Eine Funktion $\varphi: X \to \real$ heißt \emph{einfach} $:\Leftrightarrow$ $\varphi$ nimmt nur endlich viele verschiedene Werte an. Mit anderen Worten: Es existiert $m \in \nat$ und $A_1, \ldots, A_m \subset X$ und $c_1, \ldots, c_m \in \real$, so dass
\[ \varphi = \sum_{i=1}^m c_i \chi_{A_i}. \]

\subsubsection{Messbare Funktionen}
Sei $(X, \mA)$ ein Messraum.
\begin{itemize}
 \item Eine Funktion $f: X \to [0, \infty]$ heißt \emph{$\mA$-messbar} $:\Leftrightarrow$ $f^{-1}(E) \in \mA$ für alle Borel-Mengen $E \subset [-\infty,\infty]$.
 \item $f: \real^n \to [-\infty,\infty]$ heißt \emph{Borel-messbar} (bzw. Lebesgue-messbar) $:\Leftrightarrow$ $f$ ist $\borel(\real^n)$-messbar (bzw. $\lebesgue^n$-messbar).
 \item $f: \real^n \to [-\infty,\infty]^n$ heißt \emph{$\mA$-messbar} $:\Leftrightarrow$ jede Komponente von $f$ ist messbar.
\end{itemize}

\begin{rmrk}
 Sei $(X, \mA)$ ein Messraum und $f: X \to [-\infty,\infty]$. Äquivalente Aussagen:
 \begin{enumerate}
  \item $f$ ist messbar.
  \item $\{ x \in X: f(x) > c \} \in \mA$ für alle $c \in \real$.
 \end{enumerate}
 Ebenso für $\le$, $\ge$, $<$ statt $>$.
\end{rmrk}

\begin{thm}
 Sei $(X, \mA)$ ein Messraum und für alle $k \in \nat$ seien $f_k, f, g: X \to [-\infty,\infty]$ messbar. Dann sind ebenfalls messbar:
 \begin{align*}
 &f+g,& &fg,& &\max\{ f, g \},& &\min\{ f, g \}, \\ 
 &\sup_{k \in \nat} f_k, & &\inf_{k \in \nat} f_k,& &\limsup_{k \in \nat} f_k,& &\liminf_{k \in \nat} f_k.
 \end{align*}
 Ebenso $\frac{f}{g}$ sofern $g \ne 0$ auf $X$. Die Grenzwerte und Supremum sind hier \emph{punktweise} zu verstehen.
\end{thm}

\begin{rmrk}
 Jede stetige Funktion $f:\real^n \to \real$ ist Borel-messbar (also auch Lebesgue-messbar).
\end{rmrk}

\begin{proof}
 Wegen der Stetigkeit ist $f^{-1}((c,\infty))$ offen, also Borel-messbar.
\end{proof}

\subsubsection{Integral für nichtnegative einfache Funktionen}
Sei $(X,\mA,\mu)$ ein Maßraum und $\varphi: X \to [0, \infty)$ einfach; genauer: Seien $c_i \ge 0$ und $A_i \in \mA$ für $i=1,\ldots,m$ und $\varphi = \sum_{i=1}^m c_i \chi_{A_i}$. Dann definiert man folgende Zahl\footnote{{$\infty$ ist möglich, da unter Umständen $\mu(A_i) = \infty$ für ein $i$}} in $[0,\infty]$:
\[ \int_X \varphi \diffop \mu := \sum_{i=1}^m c_i \mu( A_i ). \]

\begin{rmrk}
 Das Integral ist wohldefiniert, das heißt wenn
 \[ \sum_{i=1}^m c_i \chi_{A_i} = \sum_{i=1}^m c'_i \chi_{A'_i}, \]
 dann ist 
 \[ \sum_{i=1}^m c_i \mu(A_i) = \sum_{i=1}^m c'_i \mu(A'_i). \]
\end{rmrk}

\subsubsection{Integral für nichtnegative Funktionen}
Sei $(X,\mA,\mu)$ ein Maßraum und $f: X \to [0, \infty]$ messbar. Man definiert
\[ \int_X f \diffop \mu := \sup \left\{ \int_x \varphi \diffop \mu : \varphi: X \to \real \text{ messbar und einfach mit } 0 \le \varphi \le f \right\}. \]

\begin{rmrk}
 Das Integral ist linear und monoton, das heißt
 \[ f \le g \Rightarrow \int f \diffop \mu \le \int g \diffop \mu \]
 auf der Klasse der nichtnegativen messbaren Funktionen.
\end{rmrk}

\subsubsection{Integrierbare Funktionen}
Sei $(X,\mA,\mu)$ ein Maßraum. Eine Funktion $f:X \to [-\infty,\infty]$ heißt \emph{$\mu$-integrierbar} $:\Leftrightarrow$ $f$ ist $\mA$-messbar und für die beiden (dann ebenfalls messbaren\footnote{Das folgt aus Satz 2.3}) Funktionen $f_\pm : X \to [0,\infty]$ gilt:
\[ \int f_\pm \diffop \mu < \infty. \]

Man definiert dann folgende Zahl in $\real$:
\[ \int_X f \diffop \mu := \int_X f_+ \diffop \mu - \int_X f_- \diffop \mu. \]

Hierbei ist $f_\pm := \max \{ \pm f, 0 \}$.\footnote{Beachte: $f_-$ ist nach Definition auch eine nichtnegative Funktion}

\begin{rmrk}
 $f$ ist integrierbar genau dann, wenn $|f|$ integrierbar ist.
 \begin{proof}
  Sei $\int |f| \diffop \mu < \infty$. Da $0 \le f_\pm \le |f|$ folgt aus der Monotonie auf nichtnegativen Funktionen, dass $\int f_\pm \le \int |f| < \infty$.
  
  Umgekehrt sei $f$ integrierbar, das heißt $\int f_\pm < \infty$. Da $|f| = f_+ + f_-$ folgt aus der Linearität $\int |f| = \int f_+ + \int f_i < \infty$.
 \end{proof}
\end{rmrk}

\begin{thm}
 Sei $(X,\mA,\mu)$ ein Maßraum und seien $f,g:X \to [-\infty, \infty]$ messbar. Dann gilt
 \begin{enumerate}[(a)]
  \item $\int_X |f| \diffop \mu = 0$ $\Leftrightarrow$ $f=0$ fast überall.
  \item $f$ ist $\mu$-integrierbar $\Rightarrow$ $f(x) \notin \{ -\infty, \infty \}$ für $\mu$-fast alle\footnotemark $x \in X$.
  \item Wenn $\mu$-fast überall $f=g$ gilt, dann ist $f$ integrierbar genau dann, wenn $g$ integrierbar ist. Dann ist
  \[ \int_X f \diffop \mu = \int_X g \diffop \mu. \]
  \item $f,g$ integrierbar und $c \in \real$ $\Rightarrow$ $cf$, $f+g$ integrierbar.
  \item Das Integral ist linear und monoton auf dem Vektorraum der integrierbaren Funktionen.
 \end{enumerate}
\end{thm}
\footnotetext{$\mu(\{ x \in X: f(x) = \infty$ oder $f(x) = -\infty \} ) = 0$}

\begin{rmrk}
 Keine Äquivalenz in (b), zum Beispiel $f: \real \to \real$ gegeben durch $f(x)$ für alle $x$ mit Maßraum $(\real, \borel(\real), \lebesgue^1 )$.
\end{rmrk}

\subsubsection{Integration über Teilmengen}
 Sei $(X,\mA,\mu)$ ein Maßraum und $Z \in X$ messbar. Eine messbare Funktion $f:X \to [-\infty,\infty]$ heißt \emph{über $Z$ integrierbar} $:\Leftrightarrow$ $\chi_Z f$ ist integrierbar. 
 
 In diesem Fall definiert man
 \[ \int_Z f \diffop \mu := \int_X \chi_Z f \diffop \mu. \]

\begin{exmp}
 $(X,\mA,\mu) = (\real, \borel(\real), \lebesgue^1 )$, $f \equiv 1$ und $Z = [a,b]$ mit $a,b \in \real$, $a \le b$. Dann ist $f$ nicht integrierbar (auf $\real$), aber über $Z$ integrierbar, da $\chi_Z f = \chi_{[a,b]}$ und
 \[ \int_\real \chi_{[a,b]} \diffop \lebesgue^1 = \lebesgue^1( [a,b] ) = b - a. \]
\end{exmp}

\subsection{Konvergenzsätze}
 Riemann-Integral: $f_k \to f$ \emph{gleichmäßig} $\Rightarrow$ $\int f_k \to \int f$. Das ist eine sehr starke Bedingung.
 
\subsubsection{Monotone Konvergenz}
\begin{thm}
 Sei $(X,\mA,\mu)$ ein Maßraum und $f_0 \le f_1 \le \ldots$ eine monoton wachsende Folge integrierbarer Funktionen, $f_k : X \to \real$. Zudem sei
 \[ \sup_{k \in \nat} \int f_k \diffop \mu < \infty. \]
 Dann ist der \emph{punktweise} Grenzwert\footnotemark $f: X \to (-\infty, \infty]$ ebenfalls integrierbar und es gilt
 \[ \lim_{k \to \infty} \int_X f_k \diffop \mu = \int_X f \diffop \mu. \]
\end{thm}
\footnotetext{Das heißt $f(x) := \lim_{k \to \infty} f_k(x)$ für alle $x \in X$.}

Beweis siehe [MINT].

\begin{exmp}
 Siehe nächster Abschnitt (\ref{sect:riemann-lebesgue}) und Tonelli (\ref{sect:tonelli}).
\end{exmp}

\subsubsection{Riemann vs Lebesgue}\label{sect:riemann-lebesgue}
\begin{kor}
 Sei $f:[a,b] \to \real$ Riemann-integrierbar. Dann gilt
 \begin{enumerate}
  \item $f$ ist $\lebesgue^1$ messbar.
  \item Riemann- und Lebesgue-Integral von $f$ stimmen überein:
   \[ \int f(x) \diffop x = \int_{[a,b]} f \diffop \lebesgue^1. \]
 \end{enumerate}
\end{kor}

\begin{proof}
 Da $f$ Riemann-integrierbar ist, existieren Folgen $(\varphi_k), (\psi_k) \subset \stair$, so dass $\varphi_k \le f \le \psi_k$ und
 \[ \int_a^b f = \lim_{k\to\infty} \int_a^b \varphi_k = \lim_{k\to\infty} \int_a^b \psi_k. \tag{$\circ$} \]

 Ohne Einschränkung sei $\varphi_0 \le \varphi_1 \le \ldots$ und $\psi_0 \ge \psi_1 \ge \ldots$. Aber jede Treppenfunktionen ist Borel-messbar und Lebesgue-integrierbar und ihr Riemann- und Lebesgue-Integral stimmen überein, da $\lebesgue^1( (a,b) ) = b - a$.

 Wegen Satz 3.1 und ($\circ$) sind auch die punktweisen Grenzwerte $\varphi := \lim \varphi_k$ und $\psi := \lim \psi_k$ integrierbar und Borel, und
 \[ \int_{[a,b]} \varphi \diffop \lebesgue^1 \stackrel{\mathrm{3.1}}= \lim_{k\to\infty} \int_{[a,b]} \varphi_k \diffop \lebesgue = \lim_{k\to\infty} \sum_{i=1}^{m_k} c_i^{(k)} \mu \left(A_i^{(k)}\right)= \lim_{k\to\infty} \sum_{i=1}^{m_k} c_i^{(k)} \left(a_i^{(k)}-b_i^{(k)}\right) = \]\[ \lim_{k\to\infty} \int_a^b \varphi_k \stackrel{({\circ})}= \int_a^b f. \]
 Analog für $\psi$. Da $\psi \ge \varphi$ also
 \[ \int | \psi - \varphi | \diffop \lebesgue^1 = \int (\psi - \varphi) \diffop \lebesgue^1 = \int \psi \diffop \lebesgue^1 - \int \varphi \diffop \lebesgue^1 = 0. \]
 Also ist $\varphi = \psi$ $\lebesgue^1$-integrierbar fast überall wegen Satz 2.6, also ist $\psi = f = \varphi$ $\lebesgue^1$-integrierbar fast überall, weil ja $\varphi \le f \le \psi$.
\end{proof}

Beispiel einer $\lebesgue^1$-integrierbaren, aber nicht Riemann-integrierbaren Funktion $f:[a,b] \to [0,\infty)$:
\[ f = \chi_{\rat \cap [a,b]} = \begin{cases} 1, &\text{wenn } x \in \rat \text{ und } x \in [a,b], \\
                                 0, &\text{sonst.}
                                \end{cases}
   \qRq \int_{[a,b]} f \diffop \lebesgue^1 = 0. \]
   
\subsubsection*{Uneigentliches Riemann-Integral}
Eine Funktion $f:(0,\infty) \to \real$ heißt uneigentlich Riemann-integrierbar $:\Leftrightarrow$ $\forall 0 < \eps < R < \infty$ gilt $f$ ist Riemann-integrierbar auf $[\eps,R]$. Dann:
\[ \int_0^\infty f := \lim_{\substack{\eps \to 0 \\ R \to \infty}} \int_\eps^R f \]
Es gilt: wenn $f \ge 0$ und uneigentlich Riemann-integrierbar ist auf $(0,\infty)$, dann ist $f$ auch $\lebesgue^1$-integrierbar. auf $(0,\infty)$.

\begin{proof}
 Wende monotone Konvergenz an auf $\chi_{[\eps,R]} f \uparrow f$ für $\eps \to 0$, $R \to \infty$.
\end{proof}

Aber ohne Voraussetzung $f \ge 0$ ist die Aussage im Allgemeinen falsch. Zum Beispiel ist $f(x) := \frac{\sin x}{x}$ uneigentlich Riemann-integrierbar auf $(0,\infty)$, aber nicht $\lebesgue^1$-integrierbar, denn $|f(x)| \ge \rez{2 |x|}$ auf einer Menge $E := \{ x \in \real: |\sin x| \ge \rez{2} \}$. Man kann leicht zeigen, dass
\[ \int_E \rez{|x|} \diffop x = \sum_{i=1}^\infty ( \log b_i - \log a_i ) = \infty. \]

\subsubsection{Majorierte Konvergenz}
\begin{thm}
 Sei $(X, \mA, \mu)$ ein Maßraum und $(f_k)_{k \in \nat}$ eine Folge messbarer Funktionen $X \to \real$, die \emph{punktweise} gegen $f:X \to \real$ konvergiert. Zudem existiere eine Majorante, das heißt eine \emph{integrierbare} Funktion $F: X \to [0, \infty]$, so dass punktweise fast überall $|f_k| \le F$ für alle $k \in \nat$. Dann ist $f$ integrierbar\footnotemark und
 \[ \lim_{k \to \infty} \int_X f_k \diffop \mu = \int_X f \diffop \mu. \]
\end{thm}
\footnotetext{Auch alle $f_k$, das folgt sofort aus der Ungleichung}

\begin{rmrk}
 Wegen $|f_k -f| \le 2F$ und der punktweisen Konvergenz $|f_k - f| \to 0$ folgt durch nochmalige Anwendung des Satzes der stärkere Schluss
 \[ \int_x |f_k - f| \diffop \mu \to 0. \]
\end{rmrk}

\subsubsection*{Notwendigkeit der Majorante}
Zugleich Notwendigkeit der \emph{monotonen} Konvergenz-Annahme in Satz 3.1:

Suche eine Folge $(f_k)$ mit $\int_\real f_k = 1$, aber $f_k \to 0$ punktweise. Wähle zum Beispiel $f_k = k \cdot \chi_{[-1/2k,1/2k]\setminus \{ 0 \}}$. (``Rechtecke'' zentriert um 0). Es gibt keine Majorante, da $f_{k+1} > f_k$ auf $\left[ -\rez{2(k+1)}, \rez{2k+1} \right]$. Es gilt
\[ \int_\real f_k = k \cdot \lebesgue^1 \left( -\rez{2k}, \rez{2k} \right) = k \cdot \rez{k} = 1 \]
und für alle $x \ne 0$ gilt $f_k(x) \to 0$. Damit gilt aber $\int_\real f = 0$!

\subsection{Parameterabhängige Integrale}
Situation: $f(x,t)$ und betrachte
\[ \int_X f(x,t) \diffop \mu(x) := \int_X f( \cdot, t ) \diffop \mu \]
wobei $f( \cdot, t)(x) := f(x,t)$. Wie hängt $\int_X f(x,t) \diffop \mu(x)$ von $t$ ab?

\subsubsection{Stetige Abhängigkeit}
\begin{thm}
 Sei $(X, \mA, \mu)$ ein Maßraum, $U \subset \real^n$ offen und $a \in U$. Weiter sei
 \[ f: X \times U \to \real, \quad (x,t) \mapsto f(x,t) \]
 eine Funktion mit folgenden Eigenschaften:
 \begin{enumerate}
  \item Für jedes feste $t \in U$ ist $f( \cdot, t ): x \mapsto f(x,t)$ integrierbar auf $X$.
  \item Für jedes feste $x \in X$ ist $f( x, \cdot ): t \mapsto f(x,t)$ stetig im Punkt $a$.
  \item Es existiert eine integrierbare Majorante $F: X \to [0,\infty]$ mit $|f(x,t)| \le F(x)$ für alle $(x,t) \in X \times U$.
 \end{enumerate}
 Dann ist die durch
 \[ g(t) := \int_X f(x,t) \diffop \mu(x) := \int_X f( \cdot, t ) \diffop \mu \]
 definierte Funktion $g: U \to \real$ stetig in $a$, das heißt
 \[ \lim_{t \to a} \int_X f(x,t) \diffop \mu(x) = \int_X f(x,a) \diffop \mu(x). \]
\end{thm}

\begin{proof}
 Sei $(t_k)_{k \in \nat} \subset U$ mit $t_k \to a$ und definiere $f_k(x) := f(x, t_k)$ und $f_\infty(x) := f(x, a)$. Nach Voraussetzung ist jedes $f_k$ messbar und majoriert durch $f$:
 \[ |f_k(x)| = |f(x,t_k)| \le F(x) \text{ für alle } x \in X. \]
 Zudem gilt wegen 2. $f_k \to f$ punktweise auf $X$. Daher folgt die Behauptung aus der majorierten Konvergenz.
\end{proof}

\subsubsection{Differenzierbare Abhängigkeit}
\begin{thm}
 Sei $(X, \mA, \mu)$ ein Maßraum und $I \subset \real$ ein nichttriviales Intervall. Weiter sei 
 \[ f: X \times I \to \real, \quad (x,t) \mapsto f(x,t) \]
 eine Funktion mit folgenden Eigenschaften:
 \begin{enumerate}
  \item Für jedes feste $t \in I$ ist $f( \cdot, t )$ integrierbar.
  \item Für jedes feste $x \in X$ ist $f( x, \cdot )$ differenzierbar auf $I$.
  \item Es existiert eine integrierbare Majorante $F: X \to [0,\infty]$, so dass $\pdiff{f}{t}(x,t) \le F(x)$ für alle $(x,t) \in X \times I$.
 \end{enumerate}
 Dann ist die durch
 \[ g(t) := \int_X f(x,t) \diffop \mu(x)  \]
 definierte Funktion $g: I \to \real$ differenzierbar auf $I$ und
 \[ g'(t) = \int_X \pdiff{f}{t}(x,t) \diffop \mu(x) \text{ für alle } t \in I. \]
\end{thm}

\begin{proof}
 Sei $t \in I$ fest und $h \ne 0$, so dass $t+h \in I$. Definiere
 \[ F_h(x) := \rez{h}( f(x, t+h) - f(x,t) ). \]
 Nach dem Mittelwertsatz existiert $\Theta \in [0,1]$, so dass:
 \[ |F_h(x)| = \left| \pdiff{f}{t} (x, t + \Theta h) \right| \stackrel{\mathrm{3.}}\le F(x) \]
 Für alle $x \in X$. Aufgrund der Differenzierbarkeit von $f(x, \cdot)$ konvergiert $F_h(x) \to \pdiff{f}{t}(x,t)$ für alle $x \in X$ (mit $h \to 0$). Also folgt wegen majorierter Konvergenz
 \[ \int_X \pdiff{f}{t}(x,t) \diffop \mu(x) = \lim_{h \to 0} \int_X F_h \diffop \mu =
 \lim_{h\to0} \int_X \frac{1}{h}(f(x,t+h)-f(x,t))\diffop\mu \]\[=
 \lim_{h\to0} \int_X \frac{1}{h}f(x,t+h)\diffop\mu -\frac{1}{h}\int_Xf(x,t)\diffop\mu=
 \left( \int_X f(x,t)\diffop\mu \right)'
 = g'(t). \qedhere \]
\end{proof}

\clearpage

\subsection{Eigenschaften des Lebesgue-Maßes}
\subsubsection{Definition (Erinnerung)}
 \textbf{Äußeres Lebesgue-Maß.} Für jede Menge $E \subset \real^n$ definiert man
 \[ \lebesgue^n( E ) := \inf \left\{ \sum_{k=0}^\infty r_k^n : E \subset \bigcup_{i=0}^\infty Q_{r_k}(a_k) \right\}, \]
 wobei das Infimum über alle Folgen $(r_k)_{k \in \nat} \subset (0, \infty)$ genommen wird, für die eine zugehörige Folge $(a_k)_{k \in \nat} \subset \real^n$ existiert, so dass $E \subset \bigcup_{k=0}^\infty Q_{r_k}(a_k)$. Hierbei ist $Q_{r}(a) := a + \left(-\frac{r}{2}, \frac{r}{2} \right]^n$.

 Lebesgue-Maß: Einschränkung des so definierten äußeren Lebesgue-Maßes auf $\sigma( \lebesgue^n )$.
 
\subsubsection{``Maßraum''-Eigenschaften}
\begin{thm}
 Bezeichne $\lebesgue^n: 2^{\real^n} \to [0, \infty]$ das äußere Lebesgue-Maß auf $\real^n$. Dann gilt:
 \begin{enumerate}
  \item $\lebesgue( \emptyset ) = 0$.
  \item $\sigma$-Subadditivität: Sind $E_0, E_1, \ldots \subset \real^n$, dann
   \[ \lebesgue^n \left( \bigcup_{k=0}^\infty E_k \right) \le \sum_{k=0}^\infty \lebesgue^n (E_k). \]
   Insbesondere ist $\lebesgue^n (E \cup F) \le \lebesgue^n (E) + \lebesgue^n (F)$.
  \item Monotonie: Wenn $E \subset F \subset \real^n$, dann ist $\lebesgue^n (E) \le \lebesgue^n (F)$.
  \item $\sigma$-Additivität: Seien $E_0, E_1, \ldots \in \sigma(\lebesgue^n)$ paarweise disjunkt. Dann ist
   \[ \lebesgue^n \left( \bigcup_{k=0}^\infty E_K \right) = \sum_{k=0}^\infty \lebesgue^n (E_k). \]
   Insbesondere ist $\lebesgue^n (E \cup F) = \lebesgue^n (E) + \lebesgue^n (F)$ für alle $E,F \in \sigma(\lebesgue^n)$ mit $E \cap F = \emptyset$.
  \item Stetigkeit von unten: Seien $E_0, E_1, \ldots \in \sigma(\lebesgue^n)$, sodass $E_k \uparrow E$, das heißt $E_0 \subset E_1 \subset \ldots$ und $E := \bigcup_{k=0}^\infty E_k$, dann ist
   \[ \lim_{k \to \infty} \lebesgue^n(E_k) =  \lebesgue^n (E). \]
  \item Stetigkeit von oben: Seien $E_0, E_1, \ldots \in \sigma(\lebesgue^n)$, sodass $E_k \downarrow E$, das heißt $E_0 \supset E_1 \supset \ldots$ und $E := \bigcap_{k=0}^\infty E_k$ und $\lebesgue^n(E_0) < \infty$, dann ist
   \[ \lim_{k \to \infty} \lebesgue^n(E_k) =  \lebesgue^n (E). \]
 \end{enumerate}
\end{thm}

Beweis siehe [MINT].

\subsubsection{Metrische Eigenschaften}
\begin{thm}
 Bezeichne $\lebesgue^n$ das äußere Lebesgue-Maß.
 \begin{enumerate}
  \item Jede Borelmenge ist $\lebesgue^n$-messbar, das heißt $\borel( \real^n ) \subset \sigma(\lebesgue^n)$. Genauer existiert für jedes $E \in \sigma(\lebesgue^n)$ eine $\lebesgue^n$-Nullmenge $N$ und eine $F_\sigma$-Menge\footnotemark $F$, so dass $E = F \cup N$.
  \item Wenn $E \subset \real^n$ beschränkt ist, dann ist $\lebesgue^n(E) < \infty$. Insbesondere ist $\lebesgue^n (K) < \infty$ für alle $K \subset \real^n$ kompakt.
 \end{enumerate}
\end{thm}
\footnotetext{Eine abzählbare Vereinigung abgeschlossener Mengen. Insbesondere sind Borel-Mengen $F_\sigma$-Mengen.}

\begin{proof}
 \begin{enumerate}
  \item Siehe [MINT] und vergleiche Übungsblatt 1.
  \item Sei $E \subset \real^n$ beschränkt. Dann existiert $r > 0$, so dass $E \subset Q_r(0)$. Also $\lebesgue^n(E) \le \lebesgue^n( Q_r(0) ) = r^n < \infty$. Zudem wissen wir aus Analysis 2: kompakte Teilmengen des $\real^n$ sind beschränkt. \qedhere
 \end{enumerate}
\end{proof}

Es folgt eine weitere maßtheoretische Eigenschaft:
\begin{kor}
 Das äußere Lebesgue-Maß ist $\sigma$-endlich, das heißt für jedes $E \subset \real^n$ existiert $E_0 \subset E_1 \subset \ldots$ mit $\lebesgue(E_k) < \infty$ und $E = \bigcup_{k=0}^\infty E_k$.
\end{kor}

\begin{proof}
 Sei $E \subset \real^n$. Sei $E_k$ der Schnitt von $E$ mit einer Kugel vom Radius $k$, also
 \[ E_k := B_k(0) \cap E. \]
 Dann ist wegen Satz (Teil 2) $\lebesgue^n(E_k) < \infty$ und offensichtlich $E_k \uparrow E$.
\end{proof}

\subsubsection*{Weiterführendes}
\textbf{Literatur:}
\begin{itemize}
 \item Forster: Analyis 2, Analysis 3
 \item Hildebrandt: Analyis 1, Analyis 2
 \item Königsberger: Analysis 1, Analysis 2
 \item Bauer: Maß- und Integrationstheorie
\end{itemize}

\textbf{Ergänzendes Seminar: ``Geometrische Maßtheorie''}
\begin{itemize}
 \item Hausdorff-Maße $\mathscr{H}^k \rightsquigarrow \int f \diffop \mathscr{H}^k$
 \item Flächen- und Ko-Flächenformeln
 \item Rektifizierbarkeit
 \item Teilnahme freiwillig, Vortrag halten möglich, Mail an \href{mailto:peter.hornung@tu-dresden.de}{peter.hornung@tu-dresden.de}
 \item Evans, Gariepy bzw. Ambrosio, Fusco, Pallara
\end{itemize}

\subsubsection{Euklidische Eigenschaften}
\textbf{Formale Anmerkung.} Wenn sich das Volumen einer Menge $E \subset \real^n$ elemtargeometrisch berechnen lässt, dann stimmt es mit dem Lebesgue-Maß $\lebesgue^n(E)$ überein. Mit anderen Worten: Das $n$-dimensionale Lebesgue-Maß ist die natürliche Fortsetzung des Volumenbegriffs auf die Klasse der Borel-messbaren Mengen $\borel(\real^n)$.

\begin{thm}[Verhalten unter affinen Transformationen]
 Sei $T \in \realmat{n}{n}$ invertierbar, sei $a \in \real^n$ und definiere $\Phi:\real^n \to \real^n$ durch $\Phi(x) = a + Tx$. Dann gilt für jede Borel-messbare Menge $E \in \real^n$:
 \begin{enumerate}
  \item $\Phi(E)$ ist Borel-messbar.
  \item $\lebesgue^n( \Phi(E) ) = | \det T | \lebesgue^n (E)$.
 \end{enumerate}
\end{thm}

\emph{Beweisidee.} 
\begin{itemize}
 \item Für $T = \lambda I$ ($I$ ist die Einheitsmatrix) siehe Übung $\lebesgue^n (a + \lambda E) = \lebesgue^n ( \lambda E ) = \lambda^n \lebesgue^n (E)$. 
 \item $\rightsquigarrow$ Naheliegend, dass auch für 
\[ T = \begin{pmatrix} 
        \lambda_0 & 0 & \cdots & \\
        0 & \lambda_1 & \ddots & \\
        \vdots  & \ddots & \ddots & 0 \\
          &  & 0 & \lambda_n
       \end{pmatrix} \]
 die behauptete Formel gilt.
 \item $\lebesgue^n$ ist invariant unter Rotationen und Spiegelungen ($T \in O(n)$).
 \item Jede Matrix $T$ ist von der Form $T = QRDR^{-1}$, wobei $D$ diagonal und $Q,R \in O(n)$.
\end{itemize}

Insbesondere für $T \in SO(n)$ $\Rightarrow$ $\det T = 1$ $\Rightarrow$ $\lebesgue^n$ ist rotationsinvariant.

\clearpage

\subsubsection{Transformationsformel}
Erinnerung an die Substitutionsregel für Riemann-Funktionen in Analysis 2: Sei $I \subset \real$ Intervall, $f:I \to \real$ und $\varphi:[a,b] \to \real$ stetig differenzierbar mit $\varphi([a,b]) \subset I$. Dann
\[ \int_{\varphi(a)}^{\varphi(b)} f = \int_a^b f(\varphi(x)) \varphi'(x) \diffop x. \]

Gegenstück für höhere Dimensionen:
\begin{thm}
 Seien $U,V \subset \real^n$ offen und sei $\Phi: U \to V$ ein $C^1$-Diffeomorphismus\footnotemark. Eine Funktion $f: V \to \real$ ist genau dann integrierbar, wenn $(f \circ \Phi) \cdot | \det \nabla \Phi |$ integrierbar ist und es gilt dann
 \[ \int_V f \diffop \lebesgue^n = \int_U (f \circ \Phi) | \det \nabla \Phi | \diffop \lebesgue^n. \]
\end{thm}
\footnotetext{$\Phi$ ist bijektiv, $\Phi \in C^1$, $\Phi^{-1} \in C^1$}

\begin{proof}
 Siehe [MINT]. Hier Spezialfall $U=V=\real^n$, $f = \chi_A$ und $\Phi(x) = a + Tx$ mit $T \in \realmat{n}{n}$ invertierbar. Aus Satz 5.4 folgt
 \begin{align*}
    \int_{\real^n} f \diffop \lebesgue^n = \int_{\real^n} \chi_A \diffop \lebesgue^n = \lebesgue^n(A)
    &= \lebesgue^n ( \Phi( \Phi^{-1}( A ) )) \\
    &= | \det \nabla \Phi | \lebesgue^n( \Phi^{-1}(A) ) \\
    &= | \det \nabla \Phi | \int_{\real^n} \underbrace{\chi_{\Phi^{-1}(A)}}_{\chi_A \cdot \Phi} \diffop \lebesgue^n \\
    &= \int_{\real^n} | \det\nabla \Phi | f \circ \Phi \diffop \lebesgue^n. \qedhere
 \end{align*}
\end{proof}

\begin{exmp}
 \begin{enumerate}
  \item Translationen: $\Phi(x) = a + x$. Dann $\nabla \Phi = I$ und 
  \[ \int_{a+U} f \diffop \lebesgue^n = \int_U f(a+x) \diffop \lebesgue^n(x). \]
  \item $\Phi(x) = \lambda x$ für $\lambda > 0$. Dann ist $\nabla \Phi = \lambda I$, also $\det \nabla \Phi = \lambda^n$, zum Beispiel mit $U = B_1(0)$ folgt
  \[ \int_{B_\lambda(0)} f \diffop \lebesgue^n = \lambda^n \int_{B_1(0)} f(\lambda x) \diffop \lebesgue^n(x). \]
  \item Polarkoordinaten eines Punktes $x=(x_1,x_2,x_3) \in \real^3$ mit $(x_1,x_2) \ne 0$:
   \begin{itemize}
    \item $r := |x| =$ Länge des Vektors $x$
    \item $\theta :=$ Winkel zwischen $x_3$-Achse und Vektor $x$
    \item $\varphi :=$ Winkel zwischen $x_1$-Achse und Vektor $x$
   \end{itemize}
   Transformation von Polar- in euklidische Koordinaten:$\Phi: [0, \infty) \times [0,\pi] \times [0,2\pi] \to \real^3$, wobei
   \[ (r,\theta, \varphi) \mapsto ( r \sin \theta \cos \varphi, r \sin \theta \sin \varphi, r \cos \theta ). \]
   Rechnung (Übung) ergibt
   \[ \det \nabla \Phi( r, \theta, \varphi) = r^2 \sin \theta. \]
   Setze $U := (0, \infty) \times (0,\pi) \times (0,2\pi)$. Dann ist $\Phi: U \to V$ ein $C^1$-Diffeomorphismus, wobei
   \[ V := \Phi(U) = \real^3 \setminus ( [0,\infty) \times \{ 0 \} \times \real ) \]
   Transformationsformel $\Rightarrow$ eine Funktion $f: \real^3 \to \real$ ist genau dann $\lebesgue^3$-integrierbar, wenn $(f \circ \Phi) | \det \nabla \Phi| : U \to \real$ mit
   \[ (r, \theta, \varphi) \mapsto f( r \sin \theta \cos \varphi, r \sin \theta \sin \varphi, r \cos \theta ) r^2 \sin \theta \]
   integrierbar ist. Dann gilt
   \[ \int_{\real^3} f \diffop \lebesgue^3 = \int_0^\infty \int_0^\pi \int_0^{2 \pi} f( r \sin \theta \cos \varphi, r \sin \theta \sin \varphi, r \cos \theta ) r^2 \sin \theta \diffop \varphi \diffop \theta \diffop r. \]
   Hier wurde Satz \ref{sect:tonelli} verwendet.
   
   Anwendung dieser Formel: 
   \[ \lebesgue^3( B_R(0) ) = \int_{\real^3} \chi_{B_R(0)} \diffop \lebesgue^3 = \int_{\real^3} \chi_{[0,R)}(|x|) \diffop \lebesgue^3 (x). \]
   Hier verwendet: $\chi_{B_R(0)}(0) = \chi_{[0,R)}(|x|)$, denn $x \in B_R(0) \Leftrightarrow |x| \in [0,R)$. Das ist klar, denn
   \[ B_R(0) := \{ x \in \real^n : |x| < R \}. \]
   Also folgt
   \[ \begin{aligned}       
      \lebesgue^3( B_R(0) ) 
      &= \int_0^R \int_0^\pi \int_0^{2 \pi} r^2 \sin \Theta \diffop \varphi \diffop \Theta \diffop r \\
      &= 2 \pi \left( \int_0^R r^2 \diffop r \right) \left( \int_0^\pi \sin \Theta \diffop \Theta \right) \\
      &= \frac{4 \pi}{3} R^3.
      \end{aligned} \]
 \end{enumerate}
\end{exmp}

\subsubsection{Satz von Fubini-Tonelli}\label{sect:tonelli}
``Mehrdimensionale Integrale als iterierte eindimensionale Integrale''

Wir identifizieren in diesem Abschnitt $\real^{k+\ell} = \real^k \times \real^\ell$ und bezeichnen Punkte in $\real^{k + \ell}$ mit $(x,y)$, wobei $x \in \real^k$, $y \in \real^\ell$.

Funktionen $f: \real^{k+\ell} \to \real$ fassen wir auf als Funktionen $\real^k \times \real^\ell \to \real$, $(x,y) \mapsto f(x,y)$. Für $y \in \real^\ell$ definiere 
\[ f( \cdot, y ) : \real^k \to \real, \quad x \mapsto f(x,y). \]
Analog $f( x, \cdot ) := f(x,y)$.

Wir legen den Maßraum $(\real^n, \borel(\real^n), \lebesgue^n)$ zugrunde\footnote{``Integrierbar'' heißt also $\lebesgue^n$-integrierbar und Borel-messbar.}.

\begin{lem}[Cavalieri]
Sei $A \in \real^{k+\ell}$. Dann gilt
\begin{enumerate}
 \item Für alle $y \in \real^\ell$ ist der Schnitt $A_y := \{ x \in \real^k : (x,y) \in A \}$ eine Borelmenge in $\real^k$.
 \item Die Funktion $\real^\ell \to [0,\infty]$, $y \mapsto \lebesgue^n(A_y)$ ist Borel-messbar.
 \item $\lebesgue^{k+\ell} (A) = \int_{\real^\ell} \lebesgue^k (A_y) \diffop \lebesgue^\ell(y)$.
\end{enumerate}
\end{lem}

\begin{proof}
 Siehe [MINT].
\end{proof}

Betrachte zum Beispiel Würfel in $\real^3$
\begin{itemize}
 \item $\real^3 = \real \times \real^2$: $\int_{\real^2} \lebesgue^1 (A_y) \diffop \lebesgue^2(y)$, wir betrachten eine Fläche ($\real^2$) und ermitteln in jedem Punkt die Länge der zugehörigen Linie ($\real^1$).
 \item $\real^3 = \real^2 \times \real$: $\int_{\real^1} \lebesgue^2 (A_y) \diffop \lebesgue^1(y)$, hier gehen wir entlang einer Linie und ermitteln die zugehörige Fläche.
\end{itemize}

\begin{thm}[Tonelli]{}
 Sei $f: \real^{k+\ell} \to [0, \infty]$ Borel-messbar. Dann ist $f(\cdot,y):\real^3 \to [0,\infty]$ Borel-messbar für jedes $y \in \real^\ell$ und die Funktion
 \[ \real^\ell \to [0,\infty], \quad y \mapsto \int_{\real^k} f(x,y) \diffop \lebesgue^k (x) \]
 ist Borel-messbar. Es gilt
 \[ \begin{aligned} \int_{\real^{k+\ell}} f \diffop \lebesgue^{k+\ell} 
    &= \int_{\real^\ell} \left( \int_{\real^k} f(x,y) \diffop \lebesgue^k (x) \right) \diffop \lebesgue^\ell (y) \\
    &= \int_{\real^k} \left( \int_{\real^\ell} f(x,y) \diffop \lebesgue^\ell (y) \right) \diffop \lebesgue^k (x).
 \end{aligned} \]
 Insbesondere ist eine Borel-messbare Funktion $F: \real^k \times \real^\ell \to [-\infty, \infty]$ genau dann $\lebesgue^{k+\ell}$-integrierbar, wenn
 \[ \int_{\real^\ell} \left( \int |F(x,y)| \diffop \lebesgue^k (x) ) \right) \diffop \lebesgue^\ell (y) < \infty. \]
 Analog gilt das auch für $x \leftrightarrow y$ und $k \leftrightarrow \ell$.
\end{thm}

\begin{proof}
 Aufgrund des Lemmas gilt der Satz für die Funktion $f = \chi_A$ mit Borel-Menge $A \subset \real^{k+\ell}$. Wegen Linearität also auch für einfache Funktionen $f$. Nun sei $f$ wie im Satz. Gemäß Übungsblatt 2 existieren einfache Funktionen $\varphi_n : \real^{k+\ell} \to [0, \infty)$, so dass $\varphi_n (x,y) \uparrow f(x,y)$ für alle $(x,y) \in \real^{k+\ell}$ und $\int \varphi_n \diffop \lebesgue^{k+\ell} \rightarrow \int f \diffop \lebesgue^{k+\ell}$.
 
 Sei $y \in \real^\ell$ fest. Dann ist jedes $\varphi_n(\cdot, y)$ Borel-messbar in $\real^k$, weil der Satz für einfach Funktionen $\varphi_n$ gilt. Da $\varphi_n(\cdot, y) \uparrow f( \cdot, y)$ punktweise auf $\real^k$, folgt mit der monotonen Konvergenz:
 \[ \Phi_n(y) := \int \varphi_n(x,y) \diffop \lebesgue^k (x) \uparrow \underbrace{\int f(x,y) \diffop \lebesgue^k (x)}_{:= F(y)}. \]
 
 Nun sei $y$ wieder beliebig. $\Phi_n$ ist messbar wegen Satz für einfache Funktionen $\varphi_n$, daher ist $F$ auch messbar.
 
 Monotone Konvergenz $\Rightarrow$ $\int \Phi_n \diffop \lebesgue^\ell \rightarrow \int F \diffop \lebesgue^\ell$. Daher überträgt sich die Integralzerlegung von $\varphi_n$ auf $f$.
\end{proof}

\begin{kor}[Fubini]
 Sei $f:\real^{k+\ell} \to [-\infty,\infty]$ integrierbar. Dann existiert eine Nullmenge $N \subset \borel(\real^\ell)$, sodass $f(\cdot, y): \real^k \to [-\infty, \infty]$ für jedes $y \in \real^\ell \setminus N$ integrierbar ist, und auch 
 \[ \real^\ell \to [-\infty, \infty],\quad y \mapsto \int_{\real^k} f(x,y) \diffop \lebesgue^k (x) \]
 ($:= 0$ für $y \in N$) integrierbar über $\real^k$. (Dasselbe gilt für $x \leftrightarrow y$).
 
 Zudem gilt
 \[ \begin{aligned}
    \int_{\real^{k+\ell}} f \diffop \lebesgue^{k+\ell} 
    &= \int_{\real^k} \left( \int_{\real^\ell} f(x,y)  \diffop \lebesgue^\ell (y) \right) \diffop \lebesgue^k(x) \\
    &= \int_{\real^\ell} \left( \int_{\real^k} f(x,y)  \diffop \lebesgue^k (x) \right) \diffop \lebesgue^\ell(y).
    \end{aligned} \]
\end{kor}

\begin{proof}
 Wegen Tonelli sind die Funktionen $F_\pm: \real^\ell \to [0,\infty]$, $F_\pm := \int_{\real^k} f_\pm (x,y) \diffop \lebesgue^k(x)$ Borel-messbar mit 
 \[ \int_{\real^\ell} F_\pm \diffop \lebesgue^\ell = \int_{\real^\ell} \left( \int_{\real^k} f_\pm (x,y) \diffop \lebesgue^k(x) \right) \diffop \lebesgue^\ell = \int_{\real^{k+\ell}} f_\pm \diffop \lebesgue^{k+\ell} < \infty. \]
 Also existiert gemäßt Satz 2.6(b) eine Nullmenge $N \subset \real^\ell$, sodass $F_+(y) < \infty$ und $F_-(y) < \infty$ für alle $y \in \real^\ell \setminus N$. Also ist die im Korollar definierte Funktion integrierbar. Die Gleichung am Schluss folgt aus Tonelli, angewandt auf $f_+$ und $f_-$.
\end{proof}

\begin{exmp}
 Aus Cavalieri folgt für jede Borelmenge $A \in \real^{k+\ell}$: $A$ ist Nullmenge genau dann, wenn $\lebesgue^\ell( A_x ) = 0$ für $\lebesgue^k$-fast alle $x \in \real^k$. Hierbei ist $A_x := \{y \in \real^\ell:(x,y) \in A \}$.
\end{exmp}

\begin{folg}
 Graphen messbarer Funktionen sind Nullmengen. Genauer: Ist $U \subset \real^{n-1}$ eine Borelmenge und $f:U \to \real$ Borel-messbar, dann ist der $\mathrm{graph}(f) := \{ (x,f(x)) : x \in U \} \subset \real^n$ Borel-messbar und $\lebesgue^n (\mathrm{graph}(f) ) = 0$.
\end{folg}

\begin{proof}
 Die Borel-Messbarkeit folgt aus $\mathrm{graph}(f) = \phi^{-1}( \{ 0 \} )$, wobei $\phi: U \times \real \to \real$ mit $\phi(x,y) = y -f(x)$ eine Borel-messbare Funktion ist. Er ist eine Nullmenge, weil für alle $x \in U$ $\mathrm{graph}(f)_x = \{ f(x) \}$ eine $\lebesgue^1$-Nullmenge ist.
\end{proof}

\section{Untermannigfaltigkeiten des \texorpdfstring{$\real^n$}{IRn}}

\subsection{Untermannigfaltigkeiten des \texorpdfstring{$\real^n$}{IRn}}
\textbf{Vorspann.} Eine Menge $M$ heißt abzählbar $:\Leftrightarrow$ Es existiert eine injektive Abbildung $M \to \real$. Nicht abzählbare Mengen heißen überabzählbar.

\begin{exmp}
 $\nat, \integer$ und $\rat$ sind abzählbar, aber $\real$ ist überabzählbar.
\end{exmp}

\textbf{Erinnerung.} Ist $M \subset \real$ abzählbar, so ist $\lebesgue^1(M) = 0$. Aber $\lebesgue^1( [0,1] ) = 1$, also ist $[0,1]$ überabzählbar.

Sei $(X,d)$ ein metrischer Raum. 
\begin{itemize}
 \item Die \emph{offene Kugel} mit Radius $r > 0$ um $y \in X$:
  \[ B_r(x) := \{ x \in X : d(x,y) < r \}, \]
 \item $U \subset X$ ist \emph{offen} $:\Leftrightarrow$ $\forall x \in U \exists r > 0 : B_r(x) \subset U$,
 \item $A \subset X$ ist \emph{abgeschlossen} $:\Leftrightarrow$ $X \setminus A$ ist offen,
 \item Für $x \in X$ ist eine \emph{Umgebung von $x$} eine offene Menge, die $x$ enthält.
\end{itemize}

Für beliebige Mengen $M \subset X$ definiert man 
\begin{itemize}
 \item den \emph{Abschluss} $\obar{M} :=$ Durchschnitt aller abgeschlossenen Mengen, die $M$ enthalten ($=$ ``kleinste'' abgeschlossene Menge, die $M$ enthält),
 \item das \emph{Innere} $M^O :=$ Vereinigung aller offenen Teilmengen von $M$, 
 \item der \emph{Rand} $\partial M := \obar{M} \setminus M^O$.
\end{itemize}
\textbf{Beispiel.} $M = [0,1)$ $\Rightarrow$ $\obar{M} = [0,1]$, $M^O = (0,1)$, $\partial M = \{ 0, 1 \}$.

Sei $Y$ ein metrischer Raum. Eine Abbildung $f: X \to Y$ heißt \emph{Homöomorphismus} $:\Leftrightarrow$ $f$ ist bijektiv und $f$ und $f^{-1}$ sind stetig (Anschaulich ``stetige Deformation'').

$K \subset \real^n$ ist \emph{kompakt} $:\Leftrightarrow$ $K$ ist beschränkt und abgeschlossen.

\begin{thm}
 Sei $K \subset \real^n$ kompakt. Sei $I$ eine beliebige\footnotemark Indexmenge und $i \in I$. Sei $U_i \subset \real^n$ offen. Zudem gelte $K \subset \bigcup_{i \in I} U_i$. Dann existiert eine \emph{endliche} Teilmenge $J \subset I$, sodass $K \subset \bigcup_{i \in J} U_i$.
\end{thm}
\footnotetext{Im Allgemeinen eine überabzählbare.}

Diese sogenannte \emph{Heine-Borel-Eigenschaft} ist äquivalent zur Kompaktheit in $\real^n$ und ist in beliebigen metrischen Räumen die \emph{Definition} von Kompaktheit.

Sei $(X,d)$ ein metrischer Raum und $M \subset X$. Eine Teilmenge $V \subset M$ heißt \emph{offen in} (oder relativ zu) $M$ $:\Leftrightarrow$ $M$ ist ein metrischer Raum mit Metrik $d$ $(M,d|_M)$ mit offener Teilmenge $V$.

\begin{rmrk}
$V$ offen in $M \subset X$ $\Leftrightarrow$ Es existiert $U \subset X$ offen, sodass $V = M \cap U$.
\end{rmrk}

\begin{proof}
 Das folgt leicht aus der Tatsache, dass 
 \[ B_r^M(z) = \{ x \in M: d(x,z) < r \} = B_r^X(z) \cap M. \qedhere \]
\end{proof}

\subsubsection{Immersionen}
Sei $U \subset \real^k$ offen. Eine Abbildung $\varphi \in C^1(U, \real^n)$ heißt \emph{Immersion} $:\Leftrightarrow$ Für jedes $x \in U$ sind die Vektoren $\partial_1 \varphi(x), \partial_2 \varphi(x), \ldots, \partial_k \varphi(x)$ linear unabhängig ($\circ$).

\textbf{Idee.} Lokal ist 
\[ \varphi(x) = \varphi(z) + \underbrace{\nabla \varphi(z)}_{\in \real^{n \times k}} x. \]
Daher lineare Algebra für affine $\varphi$ (das heißt $\nabla\varphi \equiv A$)
\[ \begin{aligned}
   \rang(A) &= 2 & &\rightsquigarrow & &\dim \text{Bild}(A) = 2 \text{ (Ebene)} \\
   \rang(A) &= 1 & &\rightsquigarrow & &\dim \text{Bild}(A) = 1  \\
   \rang(A) &= 0 & &\rightsquigarrow & &\dim \text{Bild}(A) = 0 \, (\{0\})
   \end{aligned} \]
   
\begin{rmrk}
 \begin{enumerate}
  \item Lineare Unabhängigkeit kann nur erfüllt sein, wenn $n \ge k$.
  \item Für alle $x \in U$ ist 
   \[ \nabla\varphi(x) = \begin{pmatrix} \partial_1 \varphi(x) & \partial_2 \varphi(x) & \ldots & \partial_k \varphi(x) \end{pmatrix} \in \real^{n \times k}.\]
   Bedingung ($\circ$) ist äquivalent zu
   \[ \rang( \nabla \varphi(x) ) := \text{ Rang von } \nabla \varphi(x) = k. \]
   Mit anderen Worten: Es existieren Indizes $i_1, \ldots, i_k$, sodass $\pdiff{(\varphi_{i_1}, \ldots, \varphi_{i_k})}{(x_1, \ldots, x_k)}$ invertierbar ist.
   
   $\pdiff{(\varphi_{i_1}, \ldots, \varphi_{i_k})}{(x_1, \ldots, x_k)}$ ist an jeder Stelle $x \in U$ eine invertierbare $(k \times k)$-Matrix, das heißt 
   \[ \det \pdiff{(\varphi_{i_1}, \ldots, \varphi_{i_k})}{(x_1, \ldots, x_k)} = \det \nabla \begin{pmatrix} \varphi_{i_1} \\ \ldots \\ \varphi_{i_k} \end{pmatrix}(x) \ne 0. \]
   Da dieser Ausdruck stetig von $x$ abhängt, gilt: Wenn ($\circ$) in $x_0 \in U$ erfüllt ist, dann existiert $r > 0$, sodass ($\circ$) erfüllt ist für alle $x \in B_r(x_0)$.
  \item Wenn $k=n$, dann ist die Immersion $\varphi$ nach dem \emph{Satz über die Umkehrfunktion} lokal invertierbar.
 \end{enumerate}
\end{rmrk}

\begin{defn}
 Sei $U \subset \real^k$ offen und $\varphi: U \to \real^n$ eine Immersion.
 \begin{itemize}
  \item Für alle $x \in U$ definiert man den $k$-dimensionalen Untervektorraum von $\real^n$:
  \[ T_x \varphi := \spn \{ \partial_1 \varphi(x), \ldots, \partial_k \varphi(x) \}. \]
 \item Die von $\varphi$ induzierte Riemannsche Metrik auf $U$ ist die Abbildung $g_\varphi: U \to \real^{k \times k}$, definiert durch $g_\varphi(x) := (\nabla \varphi(x))^T (\nabla \varphi(x))$. Komponentenweise:
  \[ (g_\varphi)_{ij}(x) = \partial_i \varphi(x) \cdot \partial_j \varphi(x), \]
  wobei mit $\cdot$ das Skalarprodukt im $\real^n$ gemeint ist.
 
  \textbf{Beispiel.} $k=2, n=3$, $g = \begin{pmatrix} 1 & 0 \\ 0 & 1 \end{pmatrix}$. Jedes $\varphi$ mit $g_\varphi = g$ ist ``lokal längenerhaltend'' (isometrische Immersion).
 \item Per Definition ist $g_\varphi(x)$ positiv definit und symmetrisch.
 \end{itemize}
\end{defn}

\textbf{Beispiele.}
\begin{enumerate}
 \item Kurven im $\real^n$ ($k=1$). Sei $U = (a,b) \subset \real$. Eine Immersion ist eine Abbildung $\varphi:(a,b) \to \real^n$ mit ``Geschwindigkeitsvektor'' $\varphi'(x) \ne 0$ für alle $x \in (a,b)$.
 \[ T_x \varphi = \spn\{ \varphi'(x)\} \text{ und } g_\varphi(x) = | \varphi'(x) |^2. \]
 \item Graphen im $\real^n$. Sei $U \subset \real^{n-1}$ offen und $f: U \to \real$ stetig differenzierbar. Definiere $\varphi: U \to \real^n$ durch $\varphi(x) := (x, f(x))$, das heißt 
 \[ \varphi(x) = \begin{pmatrix} x_1 \\ \vdots \\ x_{n-1} \\ f(x) \end{pmatrix}. \]
 Dann ist $\varphi$ stetig differenzierbar und $\varphi(U) =  \mathrm{graph}(f)$; $\partial_i \varphi(x) = \vec{e}_i + \partial_i f(x) \vec{e}_n$.
 \[ \begin{aligned}
     (g_\varphi)_{ij} &= \partial_i \varphi \cdot \partial_j \varphi \\
     &= (\vec{e}_i + \partial_i f \vec{e}_n) \cdot (\vec{e}_j + \partial_j f \vec{e}_n) \\
     &= \delta_{ij} + \partial_i f \partial_j f.
    \end{aligned} \]
\end{enumerate}

\subsubsection{Satz}
\begin{thm}
 Sei $U \subset \real^k$ offen, $\varphi \in C^1(U, \real^n)$ eine Immersion und $z \in U$. Dann existiert $r > 0$, sodass die Einschränkung $\varphi|_{B_r(z)} : B_r(z) \to \real^n$ ein Homöomorphismus von $B_r(z)$ auf $\varphi(B_r(z))$ ist.
\end{thm}

\begin{proof}
 Für $k = n$ folgt das aus dem Satz über Umkehrabbildungen. Sei also $k < n$. Ohne Einschränkung sei $\pdiff{(\varphi_{1}, \ldots, \varphi_{k})}{(x_1, \ldots, x_k)}(z)$ invertierbar. Wende den Satz über Umkehrabbildungen an auf $\tilde{\varphi} = ( \varphi_1, \ldots, \varphi_k): U \to \real^k$, da ja $\nabla \varphi = \pdiff{(\varphi_{1}, \ldots, \varphi_{k})}{(x_1, \ldots, x_k)}(z)$ invertierbar ist. Also existiert $r > 0$ und eine offene Menge $V \subset \real^k$, sodass $\tilde{\varphi}: B_r(z) \to V$ bijektiv ist mit stetig differenzierbarer Umkehrabbildung $\tilde{\psi}: V \to B_r(z)$. Da $\tilde{\varphi}|_{B_r(z)}$ injektiv ist, ist auch $\varphi|_{B_r(z)}$ injektiv, also $\varphi:B_r(z) \to \varphi(B_r(z))$ bijektiv.
 
 Die Umkehrabbildung $\psi : \varphi(B_r(z)) \to B_r(z)$, $(y_1, \ldots, y_k) \mapsto \tilde{\psi}(y_1, \ldots, y_k)$ ist stetig, da $\tilde{\psi}$ stetig ist.
\end{proof}

\subsubsection{Definition}
Sei $M \subset \real^n$.
\begin{itemize}
  \item $M$ heißt \emph{Untermannigfaltigkeit} von $\real^n$ $:\Leftrightarrow$ Für jedes $y \in M$ existiert eine Umgebung $V \subset \real^n$ von $y$, eine offene Menge $U \subset \real^n$ und eine Immersion $\varphi: U \to \real^n$, sodass $\varphi: U \to M \cap V$ ein Homöomorphismus ist.
  
  Man nennt dann $\varphi: U \to M \cap V$ eine \emph{lokale Parametrisierung} von $M \cap V$ (oder von $M$ nahe $y$). 
  
  Für jedes $z \in M \cap V$ nennt man $\varphi^{-1}(z)$ die \emph{lokalen Koordinaten} von $z$ unter $\varphi$.
  \item Der \emph{Tangentialraum} an $M$ im Punkt $y = \varphi(x)$ ist der $k$-dimensionale Vektorraum $T_y M := T_y \varphi = \spn \{ \partial_1 \varphi(x), \ldots, \partial_k \varphi(x) \}$. 
  
  Die Elemente von $T_y M$ heißen \emph{Tangentialvektoren} an $M$ im Punkt $y$.
\end{itemize}

\begin{rmrk}
 Für alle $y \in M$ ist $T_y M$ wohldefiniert: Wenn $\varphi_i: U_i \to V \cap M$ ($i = 1,2$) lokale Parametrisierungen nahe $y \in M \cap V$ sind, dann
 \[ T_{\varphi^{-1}_1(y)} \varphi_1 = T_{\varphi^{-1}_2(y)} \varphi_2. \]
\end{rmrk}

\begin{proof}
 Wähle $i \in \{1, \ldots, k\}$ und definiere $\gamma(t) := \varphi_1^{-1}(y) + t e_i$.
 Da $U_1$ offen ist, existiert $\eps > 0$, sodass $\gamma( [-\eps, \eps] ) \subset U_1$. Definiere $\Gamma := \varphi_2^{-1} \circ \varphi_1 \circ \gamma$. In Satz 1.6 werden wir sehen: $\varphi_2^{-1} \circ \varphi_1: U_1 \to U_2$ ist ein $C^1$-Diffeomorphismus. Also ist $\Gamma \in C^1( (-\eps,\eps), \real^k)$ und wegen der Kettenregel
 \[ \partial_1 \varphi_1 ( \varphi^{-1} (y) ) \overset{\text{Def. von } \varphi}{=} (\varphi_1 \circ \gamma) (0) \overset{\text{Def. von } \Gamma}{=} (\varphi_2 \circ \Gamma) (0) = \nabla \varphi_2( \varphi_2^{-1}(y) ) \Gamma'(0), \]
 und die rechte Seite liegt im Bild von $\nabla \varphi_2( \varphi_2^{-1}(y) )$.
 
 Der Beweis für $\partial_2 \varphi_1, \ldots, \partial_k \varphi_1$ verläuft analog.
\end{proof}

\textbf{Beispiele für Untermannigfaltigkeiten.}
\begin{enumerate}
 \item Wenn $U \subset \real^k$ offen und $\varphi: U \to \real^n$ eine \emph{injektive} Immersion ist und $V \subset U$, dann ist $\varphi(V) \subset \real^k$ eine $k$-dimensionale Untermannigfaltigkeit.
 \begin{itemize}
  \item Wenn $\varphi$ nicht injektiv ist, dann ist $V$ im Allgemeinen keine Untermannigfaltigkeit.
  \item $\varphi(U)$ ist im Allgemeinen auch keine Untermannigfaltigkeit, siehe \ref{sect:offen}.
 \end{itemize}
 \item Sphären $\partial B_r(x)$ sind $(n-1)$-dimensionale Untermannigfaltigkeiten, weil $B_r(x) \subset \real^n$ eine $C^1$-berandete Menge ist.
\end{enumerate}

\begin{rmrk}
 Oben definierte Untermannigfaltigkeiten sind genauer sogenannte $C^1$-""Untermannigfaltigkeiten. Eine $C^m$-Untermannigfaltigkeit für $m \in \nat \setminus \{0\}$ erhält man, wenn man die Parametrisierungen in $C^m$ liegen. Hier wird nur $C^1$ betrachtet. Zum Beispiel in der Geometrie wird meist mindestens $C^2$ genutzt, weil man dann Eigenschaften wie die Krümmung übertragen kann.
\end{rmrk}

\newpage

\subsubsection{Äquivalente Definitionen von UM}
\begin{thm}
 Sei $M \subset \real^n$. Dann sind folgende Aussagen äquivalent:
 \begin{enumerate}
  \item $M$ ist $k$-dimensionale Untermannigfaltigkeit.
  \item Für alle $y \in M$ existiert eine Umgebung $V \subset \real^n$, ein offenes $W \subset \real^n$ und ein $f \in C^1(W, \real^{n-k})$, so dass mit $x' = (x_1, \ldots, x_k)$:
  \[ M \cap V = \{ \big(x_1, \ldots, x_k, f_1(x'), \ldots, f_{n-k}(x') \big) : x' \in W \}. \]
  Für $k = n-1$ ist das der Graph von $f$.
  \item Für alle $y \in M$ existiert eine Umgebung $V \in \real^n$ und ein stetig differenzierbares $F:V \to \real^{n-k}$; $\nabla F(x)$ hat Rang $n-k$ für jedes $x \in V \cap M$ und
  \[ V \cap M = \{ x \in V : F(x) = 0 \}. \]
  \item Für alle $y \in M$ existiert eine Umgebung $V \subset \real^n$ und eine offene Menge $W \subset \real^n$ sowie ein $C^1$-Diffeomorphismus $\Phi: V \to W$, so dass
  \[ \Phi(V \cap M) = \{ x \in W : x_{k+1} = \cdots = x_n = 0 \} = W \cap (\real^k \times \{ 0 \}), \]
  wobei $\{ 0 \} \in \real^{n-k}$.
 \end{enumerate}
\end{thm}

\begin{proof}
 \begin{itemize}
  \item \textbf{1. $\Rightarrow$ 2.} Sei also $y \in M$ mit Umgebung $V \subset \real^n$, $U' \subset \real^k$ offen und $\varphi: U' \to \real^n$ eine Immersion, so dass $\varphi: U' \to M \cap V$ ein Homöomorphismus ist. Diese existieren nach der Definition der Untermannigfaltigkeit immer. Ohne Einschränkung\footnote{Evtl. nach Umnummerierung der $\varphi_i$} sei
  \[ \pdiff{ (\varphi_1, \ldots, \varphi_k) }{ (x_1, \ldots, x_k) } \]
  an der Stelle $z := \varphi^{-1}(y)$ invertierbar. Das heißt mit $\tilde{\varphi} := ( \varphi_1, \ldots \varphi_k)$ ist $\nabla \tilde{\varphi}(z)$ invertierbar. Aus dem Satz über die Umkehrfunktion folgt nach Verkleinerung von $U'$, dass eine offene Menge $W \subset \real^k$ existiert, so dass $\tilde{\varphi}: U' \to W$ ein $C^1$-Diffeomorphismus ist. Definiere
  \[ \tilde{f} :=\varphi \circ \tilde{\varphi}^{-1} : W \to \real^n. \]
  Dann ist $\tilde{f}_i(x) = x_i$ für $i = 1, \ldots, k$. Also hat $f := (\tilde{f}_{k+1}, \ldots, \tilde{f}_{n-k})$ die gewünschte Eigenschaft.
  \item \textbf{2. $\Rightarrow$ 3.} Seien $V \subset \real^n$ und $U \subset \real^k$ offen und $f = (f_1, \ldots, f_{n-k}): U \to \real^{n-k}$ in $C^1$, so dass $M \cap V = \{ (x', f_1(x'), \ldots, f_{n-k}(x') ) : x' \in U \}$. Definiere $F: V \to \real^{n-k}$ durch 
  \[ F(x) := ( x_{k+1} - f_1(x'), \ldots, x_n - f_{n-k}(x') ). \]
  Dann ist 
  \[ M \cap V = \{ x \in V : F(x) = 0 \} \]
  und da $(\partial_{k+1} F, \ldots, \partial_n F)$ invertierbar\footnote{Sogar die Einheitsmatrix, da in der $j$-ten Komponente nur $x_j$ vorkommt und $x'$ nur $x_1$ bis $x_k$ enthält.} ist, hat $\nabla F$ Rang $n-k$.
  \item \textbf{3. $\Rightarrow$ 4.} Sei $V \subset \real^n$ Umgebung von $y \in M$ und $F \in C^1(V, \real^{n-k})$, so dass $M \cap V = \{ x \in V : F(x)=0 \} \cap V$. Ohne Einschränkung seien $\partial_{k+1} F(y), \ldots, \partial_n F(y)$ linear unabhängig. Definiere $\Phi: V \to \real^n$ durch
  \[ \Phi(x) := \begin{pmatrix} x_1 \\ \vdots \\ x_k \\ F_1(x) \\ \vdots \\ F_{n-k}(x) \end{pmatrix}. \]
  Dieses $\Phi$ hat die gewünschte Eigenschaft.
  \item \textbf{4. $\Rightarrow$ 1.} Sei $\Phi: V \to W$ wie in 4. gegeben. Dann ist
  \[ \varphi(x_1, \ldots, x_k) := \Phi^{-1}( x_1, \ldots, x_k, \underbrace{0 \ldots, 0}_{n-k} ) \]
  eine lokale Parametrisierung von $M$ nahe $y^{n-k}$. \qedhere
 \end{itemize}
\end{proof}

\subsubsection{Tangentialraum}
\begin{thm}
 Sei $M \subset \real^n$ eine $k$-dimensionale Untermannigfaltigkeit. Dann sind folgende Aussagen äquivalent:
 \begin{enumerate}
  \item $v \in T_y M$.
  \item Es existieren $\eps > 0$ und $\alpha \in C^1((-\eps,\eps),\real^n)$ mit $\alpha((-\eps,\eps)) \subset M$ und $\alpha(0) = y$ und $\alpha'(0) = v$.
  \item Für eine (bzw. jede) Abbildung $F \in C^1(V,\real^{n-k})$ wie in Satz XII.1.4 (3) gilt $v \in \ker \nabla F(y)$.
 \end{enumerate}
\end{thm}

\textbf{Beweisskizze.}
 $v \in T_y M \Rightarrow \exists \mu \in \real^k$, so dass $v = \sum_{i=1}^j \mu_i \partial_i \varphi(z)$, wobei $\varphi$ eine lokale Parametrisierung nahe $y$ und $z := \varphi^{-1}(y)$.
 \[ \rightsquigarrow \alpha(t) := \varphi( \underbrace{z + t\mu}_{\in U \text{ für } |t| < \eps} ), \Rightarrow \alpha'(0) = v. \]
 $\alpha(0) = y$
 \[ \rightsquigarrow 0 = (F \circ \alpha)'(0) = \nabla F( y ) v. \]

\clearpage
 
\subsubsection{Parameter-Transformation}
\begin{lem}
 Sei $M \subset \real^n$ eine $k$-dimensionale Untermannigfaltigkeit, $V \subset \real^n$ eine Umgebung von $y \in M$. Sei $\varphi: U \to V \cap M$ eine lokale Parametrisierung\footnotemark, sei $W \subset \real^n$ offen und sei $\Phi:V \to W$ eine $C^1$-invertierbare Abbildung wie in Satz XII.1.4 (4). Dann ist
 \[ \Phi \circ \varphi : U \to W \cap ( \real^k \times\{ 0 \} ) \]
 $C^1$-invertierbar.
\end{lem}
\footnotetext{Ein Homöomorphismus von $U$ nach $M \cap V$.}

\begin{proof}
 $\varphi: U \to M \cap V$ und $\Phi: V \to W$ sind bijektiv, also $\Phi \circ \varphi: U \to W \cap (\real^k \times \{ 0 \})$ ebenfalls bijektiv. Außerdem sind $\varphi, \Phi$ stetig differenzierbar, also ist auch $\Phi \circ \varphi$ stetig differenzierbar.
 
 Kettenregel $\Rightarrow$ $\forall x \in U$ gilt
 \[ \partial_i ( \Phi \circ \varphi )( x ) = \nabla \Phi(\varphi(x)) \partial_i \varphi(x) \]
 $\nabla \Phi(\varphi(x))$ ist invertierbar, weil $\Phi$ Diffeomorphismus, die $\partial_i \varphi(x)$ sind linear unabhängig für $i=1,\ldots,k$. Also ist $(\partial_1 (\Phi \circ \varphi)(x), \ldots, \partial_k (\Phi \circ \varphi)(x))$ linear unabhängig, das heißt $\nabla( \Phi \circ \varphi )(x)$ invertierbar für alle $x \in U$. Daher ist $\Phi \circ \varphi$ ein $C^1$-Diffeomorphismus wegen des Satzes über die Umkehrabbildung (siehe Analysis 2).
\end{proof}

\begin{thm}
 Sei $M \subset \real^n$ eine $k$-dimensionale Untermannigfaltigkeit, $y \in M$, $V \subset \real^n$ Umgebung von $y$ und seien $\varphi_{1/2}: U_{1/2} \to M \cap V$ lokale Parametrisierungen. Dann ist $\varphi_2^{-1} \circ \varphi_1 : U_1 \to U_2$ ein $C^1$-Diffeomorphismus.
\end{thm}

\begin{proof}
 Als Komposition von Homöomorphismen ist $\varphi_2^{-1} \circ \varphi_1$ auch ein Homöo\-morphis\-mus. Zudem gilt
 \[ \varphi_2^{-1} \circ \varphi_1 = (\Phi \circ \varphi_2)^{-1} \circ (\Phi \circ \varphi_1), \]
 wobei $\Phi$ wie im Lemma definiert ist. Dann sind die $\Phi \circ \varphi_i$ $C^1$-Diffeomorphismen, also auch ihre Umkehrfunktionen und damit ebenfalls $(\Phi \circ \varphi_2)^{-1} \circ (\Phi \circ \varphi_1)$.
\end{proof}

\subsubsection{Offene Mengen mit \texorpdfstring{$C^1$}{C1}-Rand}\label{sect:offen}
Sei $U \subset \real^n$ offen und beschränkt. $U$ hat einen \emph{$C^1$-Rand} $:\Leftrightarrow$ Für alle $z \in \partial U$ existiert $r > 0$ und $F \in C^1( B_r(z) )$, so dass $\nabla F(x) \ne 0$ für alle $x \in B_r(z)$ und
\[ U \cap B_r(z) = \{ x \in B_r(z) : F(x) < 0 \}. \]

\begin{rmrk}
 Es gilt dann
 \[ B_r(z) \cap \partial U = \{ x \in B_r(z) : F(x) = 0 \} \]
 und
 \[ B_r(z) \setminus \obar{U} = \{ x \in B_r(z) : F(x) > 0 \}. \]
\end{rmrk}

\begin{proof}
 Übungsaufgabe.
\end{proof}

\begin{thm}
 Sei $U \subset \real^n$ offen, beschränkt und $C^1$-berandet. Dann ist $\partial U \subset \real^n$ eine $(n-1)$-dimensionale Untermannigfaltigkeit.
\end{thm}

\begin{proof}
  Das folgt aus der Bemerkung und Satz 1.4.
\end{proof}

$\mathbb{S}^{n-1} := \partial B_1(0)$, wobei $B_1(0) \subset \real^n$, also die Oberfläche der $n$-dimensionalen Einheitskugel.

\begin{kor}
 Sei $U \subset \real^n$ offen, beschränkt und $C^1$-berandet. Dann existiert genau eine Abbildung $\nu: \partial U \to \mathbb{S}^{n-1}$, so dass für alle $y \in \partial U$ gilt:
 \begin{enumerate}
  \item $\nu(y) \in (T_y \partial U)^\perp$ und
  \item Es existiert $\eps > 0$, so dass $y + t \nu(y) \notin U$ für alle $t \in (0,\eps)$.
 \end{enumerate}
 Diese Abbildung $\nu(y) : \partial U \to \real^n$ ist stetig.
\end{kor}

\begin{proof}
 Die Existenz folgt einfach durch die Definition
 \[ \nu(y) := \frac{\nabla F(y)}{|\nabla F(y)|}. \]
 
 Wegen  $\dim \partial U = n-1 = \dim T_y \partial U$ für alle $y \in \partial U$ gilt
 \[ \dim( T_y \partial U )^\perp = n - (n-1) = 1. \]
 Weil $|\nu(z)| = 1$ folgt
 \[ \nu(z) = \pm \frac{\nabla F(y)}{|\nabla F(y)|} \]
 und das Vorzeichen wird wie in der Bemerkung festgelegt.
\end{proof}

\begin{exmp}
 \begin{itemize}
  \item $B_r(0)$ (mit $\partial B_r(0) = \{ x \in \real^n : |x| = r \}$) ist $C^1$-berandet. Die Funktion $F \in C^1(\real^n)$ kann sogar global definiert werden, nämlich durch $F(x) = |x|^2 - r$.
  \item Wenn $U$ $C^1$-berandet ist, dann ist $\partial U$ lokal ein Graph; siehe Satz XII.1.4. Das werden wir im Beweis des Satzes von Gauss verwenden.
 \end{itemize}
\end{exmp}

\clearpage

\subsubsection{Kleiner Exkurs: Extrema mit Nebenbedingungen}
\begin{thm}
 Sei $M \subset \real$ eine Untermannigfaltigkeit, sei $V \subset \real^n$ offen, sei $f \in C^1(V)$ und sei $z \in M \cap V$ ein lokales Extremum von $f$ auf $M$, das heißt es existiert $r > 0$, so dass
 \[ f(y) \le f(z) \text{ für alle } y \in M \cap B_r(z) \]
 (oder mit $\ge$). Dann ist
 \[ \nabla f(z) \in (T_z M)^\perp. \]
\end{thm}

\begin{proof}
 Ohne Einschränkung sei $B_r(z) \subset V$ und $U := B_R(0) \subset \real^k$. Sei $\varphi: U \to M \cap B_r(z)$ eine lokale Parametrisierung nahe $z$. Ohne Einschränkung sei $\varphi(0) = z$. Dann hat für jedes $i = 1, \ldots, k$ die Funktion
 \[ \tilde{f}_i:(-R,R) \to \real, \quad \tilde{f}_i(t) := f( \varphi(te_i) ) \]
 ein lokales Extremum in $t=0$. Also ist
 \[ 0 = \tilde{f}'_i(0) = \nabla f( z ) \cdot \partial_i \varphi(0) \]
 für alle $i = 1, \ldots, k$, das heißt $\nabla f( z )$ ist orthogonal zu $\spn \{ \partial_1 \varphi(0), \ldots, \partial_k \varphi(0) \}$.
\end{proof}

\begin{rmrk}
 Daraus folgt die sogenannte Lagrange-Multiplikator-Regel: Sei $M \subset \real^n$ eine $k$-dimensionale Untermannigfaltigkeit und $V \subset \real^n$ eine Umgebung von $z \in M$, so dass $F \in C^1(V, \real^{n-k})$, $F = (F_1, \ldots, F_{n-k})$ existiert und $V \cap M = V \cap \{ F = 0 \}$. Wegen Satz XII.1.5 ist
 \[ T_z M = \ker \nabla F(z) = \{ \underbrace{\nabla F_1(z)}_{\in \real^n}, \ldots, \nabla F_{n-k}(z) \}^\perp. \]
 Also 
 \[ (T_z M)^\perp = \spn \{ \nabla F_1(z), \ldots, \nabla F_{n-k}(z) \}. \]
 
 Für $f$ wie im Satz existieren also sogenannte \emph{Lagrange-Multiplikatoren} $\lambda_1, \ldots, \lambda_{n-k} \in \real$, so dass
 \[ \nabla f(z) = \sum_{i=1}^{n-k} \lambda_i \nabla F_i(z). \]
 
 Formal ist $\nabla f(t) = \sum \lambda_i \nabla F_i(z) = 0$ genau die notwendige Bedingung für ein Extremum der Funktion 
 \[ f - \sum_{i=1}^{n-k} \lambda_i F_i \]
 auf $V$. Das heißt ``$z$ ist Extrempunkt von $f$ auf $M$ $\Rightarrow$ Es existieren $\lambda_i$, so dass $z$ Extrempunkt von $f \sum_{i=1}^{n-k} \lambda_i F_i$ auf $V$''.
 
 $\rightsquigarrow$ Variationsprobleme mit Nebenbedingungen, siehe theoretische Mechanik.
\end{rmrk}

\subsection{Integrale über Untermannigfaltigkeiten}
Sei $M \subset \real^n$ eine $k$-dimensionale Untermannigfaltigkeit und $f: M \to \real$. Wir wollen das Integral $\int_M f$ definieren. Wir beschränken uns auf den Fall, dass $M$ durch endlich viele lokale Parametrisierungen überdeckt werden kann. Wegen der Heine-Borel-Eigenschaft ist das für kompakte Untermannigfaltigkeiten immer der Fall. Zum Beispiel $M = \partial W$, wobei $W \subset \real^n$ offen, beschränkt und $C^1$-berandet ist.

\subsubsection{Lokales Integral}
Wir betrachten zunächst einen Spezialfall. Es existiere $U \subset \real^k$ offen, $V \subset \real^n$ offen und eine Immersion $\varphi : U \to \real^n$ von $M$, so dass $\varphi: U \to V \cap M$ ein Homöomorphismus ist und
\[ \{ y \in M : f(y) \ne 0 \} \subset M \cap V. \]
Ein solches $f$ heißt \emph{integrierbar} über $M$ $:\Leftrightarrow$ die Funktion
\[ (f \circ \varphi) \cdot \sqrt{\det g_\varphi} = (f \circ \varphi) \cdot \sqrt{ \det((\nabla \varphi)^T(\nabla \varphi))} : U \to \real \]
ist über $U$ integrierbar. Dann definiert man
\[ \begin{aligned}
    \int_M f :&= \int_M f \diffop \vol_M := \int_M f \diffop \vol_k := \int_M f(y) \diffop \vol_k(y) \\
    :&= \int_U (f \circ \varphi) \cdot \sqrt{\det g_\varphi} \diffop \lebesgue^k \\
     &= \int_U (f(\varphi(x)) \cdot \sqrt{\det ((\nabla \varphi)^T(\nabla \varphi))} \diffop \lebesgue^k(x).
   \end{aligned} \]

\begin{rmrk}
 Dieses Integral ist wohldefiniert, das heißt es ist unabhängig von der Parametrisierung $\varphi$.
\end{rmrk}

\begin{proof}
 Sei $V \subset \real^n$ offen, $f = 0$ auf $M \setminus V$ und seien $\varphi: U \to M \cap V$ und $\tilde{\varphi}: \tilde{U} \to M \cap V$ Parametrisierungen. Wegen Satz 1.6 existiert ein $C^1$-Diffeomorphismus $\rho: \tilde{U} \to U$, so dass $\tilde{\varphi} = \varphi \circ \rho$. Mit der Transformationsformel folgt
 \[
  \int_U (f \circ \varphi) \cdot \sqrt{\det g_\varphi} \diffop \lebesgue^k = \int_{\tilde{U}} \underbrace{(f \circ \varphi \circ \rho)}_{f \circ \tilde{\varphi}} \cdot \underbrace{\sqrt{ \det g_\varphi \circ \rho } \cdot | \det \nabla \rho |}_{\sqrt{\det g_{\tilde{\varphi}}}} \diffop \lebesgue^k. 
 \]
 Definition von $g_\varphi := \nabla \varphi^T \nabla \varphi$ und $g_{\tilde{\varphi}} := \nabla \tilde{\varphi}^T \nabla \tilde{\varphi}$.
\end{proof}

\subsubsection{Globales Integral und Zerlegung der Eins}
\begin{rmrk}[Messbare Zerlegung der Eins]
 Sei $M \subset \real^n$ und seien $V_1, \ldots, V_m \subset M$ Borel-messbar, $M = V_1 \cup V_2 \cup \ldots \cup V_m$. Dann existieren Borel-messbare Funktionen $\alpha_1, \ldots, \alpha_m : M \to [0,1]$, so dass
 \begin{enumerate}
  \item $\alpha_j = $ auf $M \setminus V_j$.
  \item $\sum_{j=1}^m \alpha_j = 1$ für alle $y \in M$.
 \end{enumerate}
\end{rmrk}

\begin{proof}
 Definiere $W_1 := V_1$ und für $j \ge 2$ definiere 
 \[ W_j := V_j \setminus \bigcup_{i=1}^{j-1} V_i. \]
 Die $W_j$ sind Borel-messbar, paarweise disjunkt und $M = W_1 \cup \ldots W_m$. Definiere $\alpha_j = \chi_{W_j}$.
\end{proof}

Sei nun $M \subset \real^n$ eine $k$-dimensionale Untermannigfaltigkeit, die von endlich vielen Parametrisierungen überdeckt wird, das heißt es existieren $U_1, \ldots, U_m \subset \real^k$ und $V_1, \ldots, V_m \subset \real^n$ offen und Immersionen $\varphi_i : U_i \to V_i \cap M$ (Homöomorphismen) und sei $M = \bigcup_{j=1}^m V_j$.

Eine Funktion $f : M \to \real$ heißt integrierbar über $M$ $:\Leftrightarrow$ Für jedes $j = 1, \ldots, m$ ist $\chi_{V_j} j$ integrierbar über $M$. 

Dann definiert man (mit $\alpha_j$ wie in der Bemerkung):
\[ \int_M f := \int_M f \diffop \vol_k := \sum_{j=1}^m \int_M \alpha_j \cdot f. \]

\begin{rmrk}
 Diese Definition ist unabhängig von der Überdeckung durch Karten und der Zerlegung der Eins.
\end{rmrk}

\begin{proof}
Seien $V_i \subset \real^n$ offen mit Zerlegung der Eins $\alpha_i$, $i = 1, \ldots, m$ und $W_j \subset \real^n$ mit Zerlegung der Eins $\beta_j$, $j = 1, \ldots, l$ jeweils Überdeckungen von $M$. Dann ist 
\[ (V_i \cap W_j)_{\substack{i = 1,\ldots,m \\ j=1,\ldots,l}} \]
eine Überdeckung mit zugehöriger Zerlegung $\alpha_i \beta_j$. Da zum Beispiel
\[ \sum_{i = 1}^m \alpha_i \beta_j = \beta_j, \qquad \sum_{j=1}^l \alpha_i \beta_j = \alpha_i, \]
gilt auch
\[ \sum_{i=1}^m \int_f \alpha_i = \sum_{j=1}^l \sum_{i=1}^m \int f \alpha_i \beta_j = \sum_{j=1}^l f \beta_j. \qedhere \]
\end{proof}

\subsubsection{\texorpdfstring{$k$}{k}-dimensionales Volumen}
Sei $M \subset \real^n$ eine $k$-dimensionale Untermannigfaltigkeit. $A \subset M$ hat \emph{endliches ($k$-)Volumen} (``Fläche'' falls $k=2$, ``Länge'' falls $k=1$) $:\Leftrightarrow$ $\chi_A$ ist integrierbar über $M$. Man definiert
\[ \vol_k (A) := \int_M \chi_A. \]
Eine Funktion $f : M \to \real$ heißt über $A$ integrierbar $:\Leftrightarrow$ $\chi_a f$ ist über $M$ integrierbar. Man setzt $\int_A f := \int_M \chi_A f$.

\begin{exmp}
 Sei $I \subset \real$ ein offenes Intervall und $\varphi \in C^1(I,\real)$ eine Immersion und ein Homöomorphismus von $I$ auf $\varphi(I)$. Für jedes Intervall $J$ mit $\obar{J} \subset I$ ist dann $M := \varphi(J) \subset \real^n$ eine 1-dimensionale Untermannigfaltigkeit; es gilt $g_\varphi(x) = |\varphi'(x)|^2$, also
 \[ \vol_1( \varphi(J) ) = \int_M \chi_{\varphi(J)} = \int_I \chi_{\varphi(J)} \circ \varphi \sqrt{ g_\varphi } = \int_J | \varphi'(x) | \diffop x. \]
\end{exmp}

\clearpage

\subsubsection{Verhalten unter Homothetien}
\begin{thm}
 Sei $M \subset \real^n$ eine $k$-dimensionale Untermannigfaltigkeit und $\lambda > 0$. Dann ist $\lambda M \subset \real^n$ auch eine $k$-dimensionale Untermannigfaltigkeit. Eine Funktion $f: \lambda M \to \real$ ist über $\lambda M$ integrierbar genau dann, wenn $x \mapsto f(\lambda x)$ über $M$ integrierbar ist und es gilt
 \[ \int_{\lambda M} f = \lambda^k \int_M f(\lambda y) \diffop \vol_k (y). \]
 ..
\end{thm}

\begin{proof}
 Sei $V \subset \real^n$ offen. Wenn $\varphi: U \to V \cap M$ eine lokale Parametrisierung ist, dann ist $\lambda \varphi : U \to \lambda (V \cap M) = (\lambda V) \cap (\lambda M)$ auch eine lokale Parametrisierung von $\lambda M$. Offenbar gilt
 \[ (g_{\lambda \varphi})_{ij} = \partial_i (\lambda \varphi) \partial_j (\lambda \varphi) = \lambda^2 \partial_i \varphi \partial_j \varphi = \lambda^2 (g_\varphi)_{ij}. \]
 Also ist $\det g_{\lambda \varphi} = \lambda^{2k}  \det g_\varphi$. Damit folgt
 \begin{align*}
  \int_{\lambda M} f 
  &= \int_U f(\lambda \varphi(x)) \sqrt{ \det g_{\lambda \varphi}(x) } \diffop \lebesgue^k(x) \\
  &= \int_U f(\lambda \varphi(x)) \sqrt{ \lambda^{2k} \det g_{\varphi}(x) } \diffop \lebesgue^k(x) \\
  &= \lambda^k \int_U f(\lambda \varphi(x)) \sqrt{ \det g_{\varphi}(x) } \diffop \lebesgue^k(x) \\
  &= \lambda^k \int_M f(\lambda y) \diffop \vol_k(y). \qedhere
 \end{align*}
\end{proof}

\subsubsection{Koflächen-Formel}
\begin{thm}
 Sei $f: \real^n \to \real$ integrierbar. Dann ist $f$ für $\lebesgue^n$-fast alle $r > 0$ über der Sphäre $\mathbb{S}_r := \partial B_r(0)$ integrierbar und es gilt
 \[ \begin{aligned}
     \int_{\real^n} f \diffop \lebesgue^n
     &= \int_0^\infty \left( \int_{\mathbb{S}_r} f \diffop \vol_{\mathbb{S}_r} \right) \diffop r \\
     &= \int_0^\infty \left( \int_{\mathbb{S}_1} f(r \xi) \diffop \vol_{\mathbb{S}_1}(\xi) \right) r^{n-1} \diffop r.
    \end{aligned} \]
\end{thm}

\begin{exmp}[Rotationssymmetrische Funktionen]
 Sei $\tilde{f}:(0,\infty) \to \real$, so dass die Funktion $f: \real^n \to \real$, $f(x) := \tilde{f}(|x|)$ $\lebesgue^n$-integrierbar ist. Dann folgt aus Satz 2.5:
 \[ \int_{\real^n} f \diffop \lebesgue^n = \int_0^\infty \left( \int_{\mathbb{S}_1} \tilde{f}(r) \diffop \vol_{\mathbb{S}_1} \right) r^{n-1} \diffop r = \vol( \mathbb{S}_1 ) \int_0^\infty \tilde{f}(r) r^{n-1} \diffop r. \]
\end{exmp}

\subsection{Divergenzsatz (Integralsatz von Gauss)}
\subsubsection{Funktionen mit kompaktem Träger}
Sei $U \subset \real^n$ (zum Beispiel $U = \real^n$) und $\varphi: U \to \real$. Der \emph{Träger}\footnote{Englisch support, daher Symbol $\spt \varphi$} von $\varphi$ ist definiert als:
\[ \spt \varphi := \obar{\{ x \in U : \varphi(x) \ne 0 \}}. \]
Der Abschluss wird dabei in $\real^n$ gemacht. Eine Funktion $\varphi: U \to \real$ hat einen \emph{kompakten Träger} in $U$ $:\Leftrightarrow$ $\spt \varphi$ ist kompakt.

\begin{rmrk}
 Äquivalent:
 \begin{enumerate}
  \item $\varphi$ hat kompakten Träger in $U$.
  \item Es existiert $K \subset U$ kompakt, so dass $\varphi = 0$ auf $U \setminus K$.
  \item Die Menge $\{x \in U : \varphi(x) \ne 0 \}$ ist beschränkt.
 \end{enumerate}
\end{rmrk}
 
\begin{proof}
 $1 \Rightarrow 2$ und $2 \Rightarrow 3$ sind klar. $3 \Rightarrow 1$ folgt daraus, dass der Abschluss einer beschränkten Menge auch beschränkt ist. In $\real^n$ sind abgeschlossene, beschränkte Mengen kompakt.
\end{proof}

Für $k \in \nat$ und $U \subset \real^n$ offen schreibt man
\[ C_c^k(U) := \{ \varphi \in C^k(U) : \varphi \text{ hat kompakten Träger in } U \}. \]
Für $k=0$ muss $U$ nicht offen sein (unten $U \cup M$).

\begin{rmrk}
 $C_c^{k+1}(U) \subset C_c^k(U) \subset C_c^k(\real^n)$. Wenn $f \in C_c^0(\real^n)$, dann ist
 \[ \| f \|_{C^0(\real^n)} := \sup \{ |f(x)| : x \in \real^n \} < \infty. \]
\end{rmrk}

\begin{proof}
 Erster Teil klar; letzte Inklusion in dem Sinne, dass wenn $f \in C_c^k(U)$, dann $\tilde{f} \in C^k_c(\real^n)$ ist, wobei
 \[ \tilde{f}(x) := \begin{cases} f(x) &\text{falls } x \in U, \\ 0 &\text{sonst.} \end{cases} \]
 Sei nun $f: \real^n \to \real$ stetig und $f = 0$ auf $\real^n \setminus K$. Da $f$ stetig ist, folgt $\| f \|_{C^0(K)} < \infty$. Aber $f(\real^n) = f(K) \cup \{ 0 \}$.
\end{proof}

$\| f \|_{C^0(\real^n)} := \sup \{ | f (x) | : x \in \real^n \}$ ist endlich.

$\operatorname{spt} \varphi :=$ Abschluss in $\real^n$ von $\{ x \in U : \varphi(x) = 0 \}$.

\subsubsection{Divergenzsatz von Gauß}
\begin{defn}
 Sei $U \subset \real^n$, $F = (F_1, F_2, \ldots, F_n)^T : U \to \real^n$ (man nennt $F$ ein \emph{Vektorfeld}). Für $F \in C^1(U)$ setze für alle $x \in U$
 \[ \div F(x) := \sum_{j=1}^n \partial_j F_j(x) = (\spur \nabla F(x)), \]
 das heißt $\div F: U \to \real$.
\end{defn}

\begin{rmrk}
 Englisch ``diverge'' heißt ``auseinander treiben''.
\end{rmrk}

\begin{exmp} Sei $F: \real^2 \to \real^2$.
 \begin{enumerate}
  \item $F(x) := x = (x_1, x_2)^T$. $\div F(x) = 1 + 1 = 2.$
  \item $F(x) := -x$. $\div F(x) = -2$.
  \item $F(x) := (-x_2, x_1)^T$. $\div F(x) = 0$.
 \end{enumerate}
\end{exmp}

\begin{thm}[Divergenzsatz]
 Sei $U \subset \real^n$ offen, beschränkt und $C^1$-berandet mit äußerem Einheitsnormalenvektorfeld $\nu: \partial U \to \real^n$. Sei $F \in C^0( \obar{U}, \real^n ) \cap C^1( U, \real^n)$. Dann gilt
 \[ \int_U \div F \diffop \lebesgue^n = \int_{\partial U} F \cdot \nu \diffop \vol_{n-1}. \]
\end{thm}

\begin{exmp}
 \begin{enumerate}[a)]
  \item $U = B_r(0) \subset \real^n$, $r > 0$ fest, $F(x) = (1,0)^T$, also $\div F(x) = 0$. Aus dem Satz folgt
  \[ \int_{\partial B_r(0)} F \cdot \nu \diffop \vol_1 = \int_{B_r} \div F \diffop \lebesgue^2 = 0. \]
  \item $F(x) = x$, $U = B_r(0) \subset \real^2$. Also $F \cdot \nu = r > 0$ überall auf $\partial B_r(0)$.
  \[ \begin{aligned}
      \int_{\partial B_r(0)} F \cdot \nu \diffop \vol_1 = r \vol_1( \partial B_r(0) ) = 2 \pi r^2, \\
      \int_{B_r} \div F \diffop \lebesgue^2 = \int_{B_r} (1+1) \diffop \lebesgue^2 = 2 \lebesgue^2(B_r(0)) = 2 \pi r^2. 
     \end{aligned} \]
  \item $F(x) = (-x_2, x_1)^T$, $U = B_r(0)$. Dann ist $F \cdot \nu = 0$ auf ganz $\partial B_r(0)$.
  \[ \int_{B_r} \div F \diffop \lebesgue^2 = \int_{\partial B_r(0)} F \cdot \nu \diffop \vol_1 = 0. \]
 \end{enumerate}
\end{exmp}

\textbf{Interpretation.}
Sei $F:\real^2 \to \real^2$ ein Geschwindigkeitsvektorfeld einer Flüssigkeit, das heißt ein Teilchen im Punkt $x$ bewegt sich mit Geschwindigkeit $F(x)$. In jedem $x \in \partial U$ misst $F(x) \cdot \nu(x)$ den in $x$ aus $U$ ``herausströmenden'' Anteil der Flüssigkeit. Also ist $\int_{\partial U} F \cdot \nu$ das Volumen, welches pro Zeiteinheit aus $U$ heraus fließt.

\begin{rmrk}
 Wenn $n=1$, das heißt $U = (a,b) \subset \real$, so lässt sich $\partial U = \{ a,b \}$ als 0-dimensionale Untermannigfaltigkeit auffassen mit ``Normalenvektor'' $\nu(a) = -1$, $\nu(b) = 1$. Dann gilt für $F: U \to \real$
 \[ \underbrace{\int_U \div F \diffop x}_{F'} = \underbrace{\int_{\partial U} F \diffop \vol}_{F(a)\cdot(-1)+F(b)\cdot(1)}. \]
 Also folgt $\int_U F' = F(b) - F(a)$ und damit der Hauptsatz der Differential- und Integralrechnung. 
\end{rmrk}

\begin{folg}
 Für $U \subset \real^n$ und $F(x) := x$ gilt 
 \[ \div F(x) = \sum_{j=1}^n \underbrace{\partial_i F_i(x)}_1 = n \]
 und damit
 \[ n \cdot \lebesgue^n (U) = \int_U \underbrace{\div F}_n \diffop \lebesgue^n = \int_{\partial U} \underbrace{F \cdot \nu}_{\vec{x} \cdot \nu(x)} \diffop \vol_{n-1}. \]
 Zum Beispiel $U = B_1(0) \subset \real^n$, also $\vol_{n-1}( \partial B_1 ) = n \cdot \lebesgue^n (B_1)$, für $n = 3$ gilt also 
 \[ \vol_2 (\diffop B_r) = 3 \cdot \lebesgue^3 (B_1) = 3 \frac{4}{3} \pi = 4 \pi. \]
 
 Außerdem folgen die ``Green'schen Formeln,'' siehe Übung.
\end{folg}

\subsubsection{Spezialfall 1}
\begin{thm}
 Sei $U \subset \real^n$ offen und beschränkt, sei $f \in C^1_C(U)$. Dann gilt für alle $i \in \{ 1, \ldots, n \}$:
 \[ \int_U \partial_i f \diffop \lebesgue^n = 0. \]
\end{thm}

\begin{folg}
 Für $F \in C^0(\obar{U}, \real^n) \cap C_C^1(U, \real^n)$ gilt
 \[ \int_U \div F \diffop \lebesgue^n = \int_{\partial U} F \cdot \nu \diffop \vol_{n-1} = 0, \]
 wobei $F = 0$ auf $\partial U$.
\end{folg}

\begin{proof}
 Nehme an $i=1$, die anderen $i$ folgen dann analog. Sei $\tilde{f}:\real^n \to \real$ gegeben durch 
 \[ \tilde{f}(x) := \begin{cases} f(x) &\text{für } x \in U \\ 0 &\text{für } x \notin U. \end{cases} \]
 Also gilt $\tilde{f} \in C^1_C(\real^n)$ und damit
 \[ \begin{aligned}
     \int_U \partial_1 f \diffop \lebesgue^n 
     &= \int_{\real^n} \partial_1 \tilde{f} \diffop \lebesgue^1 \\
     &= \int_{\real^{n-1}} \underbrace{ \left( \int_\real \partial_1 \tilde{f}(x_1, \ldots, x_n) \diffop x_1 \right) }_{=0, (*)} \diffop \lebesgue^{n-1}(x_2, \ldots, x_n) = 0 
    \end{aligned} \]
 Zu ($*$): $U$ ist beschränkt, also existiert $R >0$, so dass $U \subset [-R,R]^n$. Damit gilt für alle $(x_2, \ldots, x_n)$
 \begin{align*}
  \int_\real \partial_1 \tilde{f}(x_1, \ldots, x_n) \diffop x_1
  &= \int_{-R}^R \partial_1 \tilde{f} (x_1, \ldots, x_n) \diffop x_1 \\
  &= \Big[ \tilde{f} (x_1, \ldots, x_n) \Big]_{x_1 = -R}^R = 0 - 0 = 0. \qedhere
 \end{align*}
\end{proof}

%%%%%%%

\subsubsection{Spezialfall 2}
\begin{lem}
 Sei $\rho : \real \to \real$ $\lebesgue^1$-integrierbar, sei $a \in \real$ und $f: \real \to \real$ messbar, beschränkt und stetig in $a$. Dann ist für alle $\eps > 0$
 \[ t \mapsto \rho \left( \frac{t-a}{\eps} \right) f(t) \]
 $\lebesgue^1$-integrierbar und es gilt
 \[ \lim_{\eps \to 0} \rez{\eps} \int_\real \rho \left( \frac{t-a}{\eps} \right) f(t) \diffop \lebesgue^1(t) = f(a) \int_\real \rho \diffop \lebesgue^1. \]
\end{lem}

\begin{proof}
 Die Messbarkeit folgt aus der Messbarkeit von $\rho$ und $f$. Definiere $\Phi_\eps(t) := \frac{t-a}{\eps}$. Die Transformationsformel XI.5.5 besagt, da $\rho$ integrierbar ist:
 \[ (\rho \circ \Phi_\eps ) \cdot | \Phi'_\eps | = \rez{\eps} \rho \circ \Phi_\eps \]
 ist integrierbar. Also ist auch $(\rho \circ \Phi_\eps)$ integrierbar. Weil $f$ beschränkt ist, folgt auch die Integrierbarkeit von $(\rho \circ \Phi_\eps) f$ und es gilt
 \[ \rez{\eps} \int_\real \rho( \Phi_\eps(t) ) f(t) \diffop \lebesgue^1(t) =
    \int_\real \rho(t) = f( \underbrace{\Phi^{-1}_\eps (t)}_{a + \eps t} ) \diffop \lebesgue^1 (t). \]
 Da $f$ stetig in $a$ ist, gilt $f(a + \eps t) \rho(t) \to \rho(t) f(a)$ für $\eps \to 0$ für alle $t \in \real$. Wir finden eine Majorante:
 \[ | f(a+\eps t) \rho(t) | \le \left( \sup_\real |f| \right) \cdot | \rho(t) |. \]
 Die rechte Seite ist integrierbar, da $\rho$ integrierbar ist. Also folgt die Behauptung aus der majorierten Konvergenz.
\end{proof}

\clearpage

\begin{thm}
 Sei $V \subset \real^{n-1}$ offen und beschränkt, $g \in C^1(V)$ und für $x' = (x_1, \ldots x_{n-1})^T \in V$, $x_n \in \real$ seien
 \[ \begin{aligned}
     U &:= \{ (x',x_n) \in V \times \real : x_n < g(x') \} \\
     M &:= \operatorname{graph} g := \{ (x', x_n) \in V \times \real : x_n = g(x') \}.
    \end{aligned} \]
 Dann gilt für alle $F \in C^0_C(U \cup M, \real^n) \cap C^1(U,\real^n)$, so dass $\div F$ integrierbar ist über $U$:
 \[ \int_U \div F = \int_M F \cdot \nu. \]
 Hierbei ist $\nu: M \to \real^n$ gegeben durch
 \[ \nu(x', g(x')) := \frac{ (- \nabla' g(x'), 1) }{ | (- \nabla' g(x'), 1) | } \]
 mit $x'= (x_1, \ldots, x_{n-1})$ und $\nabla' = (\partial_1, \ldots, \partial_{n-1})$.
\end{thm}

\begin{proof}
 Übliche Parametrisierung: $\varphi: V \to \real^n$, $\varphi(x') := (x', g(x'))$. Für $i = 1, \ldots, n-1$ gilt $\partial_i \varphi(x') = e_i + \partial_i g(x') e_n$, also
 \[ \det g_\varphi = 1 + | \nabla' g |^2. \]
 Da $\nu \circ \varphi = \frac{(-\nabla' g, 1)}{|(-\nabla' g, 1)|}$, folgt
 \[ \begin{aligned}
     \int_M F \nu \diffop \vol_{n-1} 
     &= \int_V (F \circ \varphi)(\nu \circ \varphi) (\det g_\varphi)^{1/2} \diffop \lebesgue^{n-1} \\
     &= -\int_V G(x') \diffop \lebesgue^{n-1} (x'),
    \end{aligned} \]
 wobei 
 \[ G(x') := -F(x', g(x') ) \cdot \begin{pmatrix} - \nabla' g(x') \\ 1 \end{pmatrix}. \]
 
 Sei $\eta \in C^1(\real)$, so dass $\eta = 1$ auf $(-\infty, 1]$ und $\eta = 0$ auf $[-\rez{2},\infty)$. Setze
 \[ \psi_\eps(x) := \eta \left( \frac{x_n - g(x')}{\eps} \right). \]
 Es ist $F \psi_\eps \in C^1_C(U)$, denn $F$ und $\psi_\eps$ sind $C^1$ und $F \psi_\eps = 0$ außerhalb der kompakten Menge $\{ (x', x_n) : x_n \le g(x') - \frac{\eps}{2} \} \cap \spt F \subset U$.
 
 Aus Satz 3.3 folgt
 \[ 0 = \int_U \div( F \psi_\eps ) = \int_U \psi_\eps \div F + \int_U F \cdot \nabla \psi_\eps. \]
 Da $\psi_\eps \to 1$ auf $U$ punktweise und $| \psi_\eps | \le 1$ gilt, dass $\psi_\eps \div F \to \div F$ punktweise und dass $|\div F|$ eine Majorante ist. Aus der majorierten Konvergenz folgt daher $\int_U \psi_\eps \div F \to \int_U \div F$. 
 
 Es bleibt noch zu zeigen: $\int_U F \nabla \psi_\eps \to \int_V G$ für $\eps \to 0$. Aus dem Satz von Fubini folgt
 \[ \begin{aligned} 
    \int_U F \nabla \psi_\eps 
    &= \int_V \left( \int_{-\infty}^{g(x')} F(x', x_n) \nabla \psi_\eps( x', x_n) \diffop \lebesgue^1(x_n) \right) \diffop \lebesgue^{n-1}(x') \\
    &= \int_V G_\eps(x') \diffop \lebesgue^{n-1}(x'),
    \end{aligned} \]
 wobei
 \[ G_\eps(x') := \rez{\eps} \int_\real F(x', x_n) \cdot \begin{pmatrix} - \nabla' g \\ 1 \end{pmatrix} \eta' \left( \frac{x_n - g(x')}{\eps} \right) \diffop \lebesgue^1(x_n). \]
 Für jedes $x' \in V$ setzt $F(x',x_n) := F(x', g(x'))$ für alle $x_n \ge g(x')$. Damit ist $F(x', \cdot)$ stetig in $g(x')$. Also folgt aus dem Lemma, dass $\lim_{\eps \to 0} G_\eps(x') = G(x')$ für alle $x' \in V$, denn $\int_\real \eta' = -1$ wegen des Hauptsatzes der Integralrechnung. 
 
 Nun wenden wir die majorierte Konvergenz an, um $\int_U F \nabla \psi_\eps \to \int_V G$ für $\eps \to 0$ zu zeigen. Behauptung: Es existiert $M > 0$, so dass
 \[ \left| F(x', x_n) \cdot \begin{pmatrix} \nabla' g \\ 1 \end{pmatrix} \right| \le M \tag{$*$} \]
 für alle $(x',x_n) \in V \times \real$. Dann folgt nämlich
 \[ |G_\eps (x') | \le M \cdot \rez{eps} \left| \int \eta' \left( \frac{x_n - g(x')}{\eps} \right) \diffop \lebesgue^1(x_n) \right| \overset{\text{Lemma}}{=} M. \]

 Um ($*$) zu zeigen, $F$ ist beschränkt, $\nabla' g$ ist stetig, also beschränkt auf einer kompakten Teilmenge $K \subset V$. Wir setzen
 \[ K := \{ x' \in V : \exists x_n \in \real \text{ mit } (x', x_n) \in \spt F \}. \qedhere \]
\end{proof}

\subsubsection{Stetig differenzierbare Zerlegung der Eins}
\begin{thm}
 Sei $K \subset \real^n$ kompakt, $J \in \nat$ und $U_0, \ldots, U_J \subset \real^n$ offen, so dass $K \subset \bigcup_{j=0}^J U_j$. Dann existieren $\varphi_j \in C_C^1(U_j)$ mit $0 \le \varphi_j \le 1$ und $\sum_{j=0}^J \varphi_j(x) = 1$ für alle $x \in K$.
\end{thm}

\subsubsection{Beweis des Divergenzsatzes}
\begin{proof}
Da $U$ $C^1$-berandet ist, existiert für alle $y \in \partial U$ ein $r > 0$, so dass $B_r(y) \cap U$ der Subgraph einer $C^1$-Funktion ist. $\partial U$ ist kompakt, da $\partial U$ beschränkt und abgeschlossen ist. Also existieren wegen Heine-Borel $N \in \nat$, $y_1, \ldots, y_N \in \partial U$ und $r_1, \ldots, r_N > 0$, so dass $\partial U = \bigcup_{i=1}^N B_{r_i}(y_i)$. Mit $U_0 := U$ und $U_i := B_{r_i}(y_i)$ gilt also $\obar{U} \subset \bigcup_{j=0}^N U_j$.

Da $\obar{U}$ kompakt ist, existiert eine Zerlegung der Eins wie in Satz 3.5: $\varphi_1, \ldots, \varphi_N$. Insbesondere gilt $F = \sum_{j=0}^N F_{\varphi_j}$.

Wegen der Linearität der Aussage des Divergenzsatzes genügt es, den Divergenzsatz separat für jedes $F_{\varphi_j}$ zu zeigen. Für $i=0$ haben wir das in Satz 3.3 gemacht und für $i = 1, \ldots, N$ in Satz 3.4. 
\end{proof}

\subsubsection{Satz von Stokes in 2d}
Für ein differenzierbares Vektorfeld $F=(F_1,F_2): U \to \real^2$, wobei $U \subset \real^2$ offen, definiert man die \emph{Rotation} $\rot F : U \to \real$ durch
\[ \rot F(x) := \partial_1 F_2(x) - \partial_2 F_1(x). \]

Wir definieren die \emph{Drehung gegen den Uhrzeigersinn im $\frac{\pi}{2}$} eines Vektors $v = (v_1, v_2) \in \real^2$ durch
\[ v^\perp := (-v_2, v_1). \]
Dann gilt $\rot F = - \div F^\perp$.

Wenn $U \subset \real^2$ offen, beschränkt und $C^1$-berandet ist und $\nu:\partial U \to \real^2$ die äußere Normale bezeichnet, dann ist $\nu^\perp:\partial U \to \real^2$ an jedem Punkt $y \in \partial U$ ein Tangentialvektor an $\partial U$ in $y$.

\begin{thm}
 Sei $U \subset \real^2$ offen, beschränkt und $C^1$-berandet. Sei $F \in C^1( U, \real^2) \cap C^0( \obar{U}, \real^2 )$, so dass $\rot F$ integrierbar\footnotemark über $U$ ist. Dann gilt
 \[ \int_U \rot F = \int_{\partial U} F \cdot \nu^\perp. \]
\end{thm}
\footnotetext{$F \cdot \nu^\perp$ ist dann integrierbar über $\partial U$, weil $\partial U$ eine kompakte Menge ist, also nimmt $F \cdot \nu^\perp$ als stetige Funktion ihr (endliches) Maximum auf $\partial U$ an.}

\begin{proof}
 Divergenzsatz anwenden auf $F^\perp$.
\end{proof}

\subsubsection{Rotation in \texorpdfstring{$\real^3$}{IR3}}
Für $v,w \in \real^3$ definieren wir das \emph{Vektorprodukt} in $\real^3$:
\[ v \times w = 
   \begin{pmatrix} v_1 \\ v_2 \\ v_3 \end{pmatrix} \times
   \begin{pmatrix} w_1 \\ w_2 \\ w_3 \end{pmatrix} :=
   \begin{pmatrix} v_2 w_3 - v_3 w_2 \\ v_3 w_1 - v_1 w_3 \\ v_1 w_2 - v_2 w_1 \end{pmatrix}. \]

Für ein differenzierbares Vektorfeld $F: U \to \real^3$, wobei $U \subset \real^3$ offen, definiert man die Rotation $\rot F: U \to \real^3$ durch
\[ \rot F := \nabla \times F = \begin{pmatrix} \partial_2 F_3 - \partial_3 F_2 \\ \partial_3 F_1 - \partial_1 F_3 \\ \partial_1 F_2 - \partial_2 F_1 \end{pmatrix}. \]

\subsubsection{Orientierbarkeit}
Sei $M \subset \real^3$ eine \emph{Fläche}, das heißt eine 2-dimensionale Untermannigfaltigkeit. Dann heißt $M$ \emph{orientierbar} $:\Leftrightarrow$ Es existiert ein sogenanntes Normalenvektorfeld $\nu \in C^0(M,\real^3)$, so dass für alle $y \in M$ gilt, dass $|\nu(y)| = 1$ und $\nu(y) \in (T_y M)^\perp$.

\begin{rmrk}
 In jedem $V \cap M$, wobei $V \subset \real^3$ offen, so dass eine Parametrisierung (Immersion) $\varphi: U \to \real^2 \xrightarrow{\text{Homöo.}} V \cap M$ existiert, gibt es ein solches stetiges $\nu$, nämlich
 \[ \nu(\varphi(x)) := \frac{\partial_1 \varphi(x) \times \partial_2 \varphi(x)}{|\partial_1 \varphi(x) \times \partial_2 \varphi(x) |}. \]
 Aber im Allgemeinen lässt sich kein global (auf ganz $M$) definiertes stetiges Normalenvektorfeld finden. Ein Beispiel für eine nicht orientierbare Fläche ist das \emph{Möbiusband}.
\end{rmrk}

Sei $\Gamma \in \real^3$ eine \emph{Kurve}, das heißt eine 1-dimensionale Untermannigfaltigkeit. Dann ist $\Gamma$ \emph{orientiert} $:\Leftrightarrow$ Es existiert ein sogenanntes Tangentialvektorfeld $\tau \in C^0(\Gamma, \real^3)$, so dass $|\tau(y)| = 1$ und $\tau(y) \in T_y \Gamma$ für alle $y \in \Gamma$.

Sei nun $U \subset \real^2$ offen, beschränkt und $C^1$-berandet. Sei $V \subset \real^2$ offen und $\obar{U} \subset V$ und sei $\varphi \in C^1(V, \real^3)$ eine Immersion. Zudem sei $\varphi: V \to \real^3$ injektiv. Dann ist wegen der Anmerkung $\varphi(U) \subset \real^3$ eine orientierbare Fläche mit Normale
\[ n := \frac{\partial_1 \varphi \times \partial_2 \varphi}{|\partial_1 \varphi \times \partial_2 \varphi|} \cdot \varphi^{-1}. \tag{1} \]
Zudem ist $\varphi(\partial U)$ eine orientierte Kurve mit Tangentialvektorfeld
\[ \tau := \frac{D_{\nu^\perp} \varphi}{| D_{\nu^\perp} \varphi |} \cdot \varphi^{-1}, \tag{2} \]
wobei $\nu: \partial U \to \real^2$ das äußere Normalenfeld an $U$ ist. Erinnerung:
\[ \begin{aligned}
    D_{\nu^\perp} \varphi(x) :&= \nabla \varphi(x) \nu^\perp (x) \\
    &= \partial_1 \varphi(x) \nu_1^\perp(x) + \partial_2 \varphi(x) \nu_2^\perp(x) \\
    &= \partial_1 \varphi(x) \nu_2^(x) + \partial_2 \varphi(x) \nu_1(x).
   \end{aligned} \]

Sei $M \subset \real^3$ eine orientierbare Fläche mit Normale $n$ und $\Gamma \subset \real^3$ eine orientierte Kurve mit Tangente $\tau$. Dann heißt $(\Gamma, \tau)$ \emph{orientierter Rand} von $(M,n)$ $:\Leftrightarrow$ Für jedes $y \in M \cup \Gamma$ existieren eine Umgebung $V \subset \real^3$, $W \subset \real^2$ offen, eine injektive Immersion $\varphi: W \to \real^3$ und $U \subset \real^2$ offen, beschränkt und $C^1$-berandet, so dass
\[ M \cap V = \varphi(W \cap U) \quad \text{und} \quad \Gamma \cap V = \varphi( W \cap \partial U ) \]
und zudem gelten (1) und (2).

\begin{rmrk}
 Dann gilt $\Gamma = \obar{M} \setminus M$.
\end{rmrk}

\clearpage

\subsubsection{Stokes auf orientierten Flächen im \texorpdfstring{$\real^3$}{IR3}}
\begin{lem}
 Sei $U \subset \real^2$ offen, beschränkt und $C^1$-berandet. Seien $W \subset \real^2$ offen, so dass $\obar{U} \subset W$ und $\varphi:W \to \real^3$ eine injektive Immersion. Wenn es ein offenes $V \subset \real^3$ gibt, so dass $\varphi(W) \subset V$, dann gilt für jedes $F \in C^1(V,\real^3)$:
 \[ \int_{\varphi(U)} \rot F \diffop \vol_2 = \int_{\varphi(\partial U)} F \cdot \tau \diffop \vol_1. \]
 Dabei ist $\tau$ wie zuvor die Tangente an $\varphi(\partial U)$.
\end{lem}

\textbf{Beweisidee.}
Durch ``Zurückziehen'' via $\varphi$ reduziert man das Lemma auf Satz 3.7.

Ähnlich wie der Divergenzsatz aus zwei Spezialfällen durch eine geeignete Zerlegung der Eins folgte, erhalten wir aus dem Lemma:
\begin{thm}
 Sei $M \subset \real^3$ eine orientierbare Fläche mit Normale $n: M \to \real^3$ und $\Gamma \subset \real^3$ eine orientierte Kurve mit Tangente $\tau: \Gamma \to \real^3$ und sei $(\Gamma,\tau)$ der orientierte Rand von $(M,n)$. Zudem sei $V \subset \real^3$ offen und beschränkt, so dass $M \cup \Gamma \subset V$ und sei $F \in C^1(V,\real^3)$. Dann gilt
 \[ \int_M n \cdot \rot F \diffop \vol_2 = \int_\Gamma \tau \cdot F \diffop \vol_1. \]
\end{thm}

%%% Local Variables:
%%% mode: latex
%%% TeX-master: "skript_gdim"
%%% End:


\section{Gewöhnliche Differentialgleichungen}
\subsection{Gewöhnliche Differentialgleichungen erster Ordnung}
Sei $G \subset \real$ und $f:G \to \real$ stetig. Dann nennt man
\[ y' = f(x,y) \]
eine \emph{gewöhnliche Differentialgleichung erster  Ordnung}.

Unter einer \emph{Lösung} versteht man eine auf einem Intervall $I \subset
\real$ differenzierbare Funktion $\varphi:I \to \real$, so dass
\[ \operatorname{graph} \varphi \subset G \quad \text{und} \quad \varphi'(x) =
  f(x,\varphi(x)) \text{ für alle } x \in \real. \]

\textbf{Geometrische Interpretation.} GDG (ordinary differential equation, ODE)
bestimmt ein Richtungsfeld. In jedem Punkt $(x_1, x_2) \in G$ wird eine Steigung
$f(x_1,x_2)$ vorgegeben. Gesucht ist eine differenzierbare Funktion $\varphi$,
deren Graph in jedem seiner Punkte $(x,\varphi(x))$ die vorgegebene Steigung
$f(x,\varphi(x))$ hat.

Nur in ``Spezialfällen'' lässt sich eine Lösung explizit angeben. Oft erhält man
sie durch mehr oder weniger gezieltes ``Raten''. Solche Fälle werden in diesem
Abschnitt behandelt.

``Explizites Hinschreiben'' ist nicht nötig, um relevante Informationen zu
erhalten.

\subsubsection{Trivialer Fall \texorpdfstring{$f(x,y)=f(x)$}{f(x,y)=f(x)}}
Sei $f:(a,b) \to \real$ stetig. Lösungen von $y' = f(x)$ erhält man durch
Integration:
\[ \varphi(x) = c + \int_a^x f(x) \diffop x, \]
wobei $c \in \real$ beliebig.

\subsubsection{Getrennte Variablen \texorpdfstring{$f(x,y) = g(x)h(y)$}{f(x,y) =
    g(x)h(y}}
Sein $I,J$ offene Intervalle und $g:I \to \real$ und $h:J \to \real$ stetig mit
$h \ne 0$ auf $J$. Eine GDG der Form
\[ y' = g(x) h(x) \]
heißt GDG mit \emph{getrennten Variablen}.

Man kann sie auf eine Gleichung reduzieren, in der keine Ableitungen $y'$
vorkommen.

Methode zum Erraten der Lösungsgleichung:
\[ \diff{y}{x} = g(x) h(y), \qquad \frac{\diffop y}{h(y)} = g(x) \diffop x,
  \qquad \int \frac{\diffop y}{h(y)} = \int g(x) \diffop x. \]

\begin{thm}
  Seien $I,J \subset \real$ offene Intervalle und $g:I \to \real$ sowie $h:J \to
  \real$ stetig und $h(y) \ne 0$ für alle $y \in J$. Sei $(x_0, y_0) \in I
  \times J$ und definiere $G(x) := \int_{x_0}^x g$ und $H(y) := \int_{y_0}^y
  \rez{h}$.

  Sei $I' \subset I$ mit $x_0 \in I'$ ung $G(I') \subset H(J)$. Dann existiert
  genau eine Lösung $\varphi: I' \to \real$ der GDG $y' = g(x) h(y)$ mit
  $\varphi(x_0)=y_0$. Diese Lösung erfüllt $H \circ \varphi = G$ auf $I'$.
\end{thm}

\begin{proof}
  Sei $\varphi:I' \to \real$ Lösung mit $\varphi(x_0) = y_0$. Da $\varphi'(x) =
  g(x) h(\varphi(x))$, folgt für alle $x \in I'$: 
  \[ H(\varphi(x)) = \int_{x_0}^x \frac{\varphi'(t)}{h(\varphi(t))} \diffop t =
    \int_{x_0}^x g(t) \diffop t = G(x). \] 
  Da $H' = \rez{h} \ne 0$, ist $H$ streng monoton. Also existiert eine stetig
  differenzierbare Umkehrfunktion (vgl. Analysis II) $H^{-1}:H(J) \to \real$.
  Also gilt
  \[ \varphi = H^{-1} \circ G. \]
  Daraus folgt auch die Eindeutigkeit.

  Dass $\varphi$ eine Lösung mit $\varphi(x_0) = y_0$ ist, folgt durch
  Nachrechnen und Kettenregel.
\end{proof}

\subsubsection{Homogene lineare DGD
  \texorpdfstring{$f(x,y)=a(x)y$}{f(x,y)=a(x)y}}
\begin{thm}
  Sei $I \subset \real$ ein Intervall, $a:I \to \real$ stetig, $x_0 \in I$ und
  $c \in \real$.

  Dann existiert genau eine Lösung $\varphi:I \to \real$ der GDG $y' = a(x) y$
  mit $\varphi(x_0) = c$, nämlich
  \[ \varphi(x) = c \cdot \exp \left( \int_{x_0}^x a(t) \diffop t  \right). \]
\end{thm}

\begin{proof}
  Dass $\varphi$ eine Lösung mit $\varphi(x_0) = c$ ist, folgt durch
  Nachrechnen.

  Eindeutigkeit: Sind $\varphi_1$ und $\varphi_2$ Lösungen von $y'=b(x)+a(x)y$
  mit $\varphi_1(x_0) = \varphi_2(x_0)$, dann ist
  $\tilde{\varphi}=\varphi_1-\varphi_2$ Lösung von $y'=a(x)y$ mit
  $\tilde{\varphi}(x_0) = 0$. Aber gemäß Satz 1.3 gibt es höchstens eine solche
  Lösung, nämlich $\tilde{\varphi} = 0$.
\end{proof}

\begin{exmp}
  $y' = 2xy + x^2$ ist von der Form $y'=a(x)+b(x)$ mit $a(x) = 2x$. Also $A(x) =
  x^2$ (für $x_0=0$) etc.
\end{exmp}

\subsection{Systeme gewöhnlicher Differentialgleichungen}
\subsubsection{System}  
\begin{defn}
  Sei $G \subset \real \times \real^n$ und $f:G \to \real^n$, $(x,y) \mapsto
  f(x,y)$ ein stetiges Vektorfeld. Dann ist $y'=f(x,y)$ ein \emph{System} von
  $n$ gewöhnlichen Differentialgleichungen erster Ordnung.

  Unter einer Lösung versteht man ein auf einem Intervall $I \subset \real$
  differenzierbares Vektorfeld $\varphi:I \to \real^n$, sodass
  $\operatorname{graph} \varphi \subset G$ und $\varphi'(x) = f(x,\varphi(x))$
  für alle $x \in I$.

  System gekoppelter Gleichungen:
  \begin{align*}
    \varphi'_1 &= f_1(x, \varphi_1, \ldots, \varphi_n) \\
               &\vdots \\
    \varphi'_n &= f_1(x, \varphi_1, \ldots, \varphi_n)
  \end{align*}

  Wichtig: weiterhin $x \in \real$, das heißt eindimensional. Für $x \in
  \real^n$ ergeben sich partielle Differentialgleichungen.
\end{defn}

\subsubsection{GDG als Integralgleichungen}
\begin{thm}
  Sei $G \subset \real \times \real^n$ und $f:G \to \real^n$ stetig und sei
  $(a,c) \in G$. Sei $I \subset \real$ ein Intervall mit $a \in I$ und $\varphi:
  I \to \real^n$ stetig und $\operatorname{graph} \varphi \subseteq G$.

  Dann sind äquivalent:
  \begin{enumerate}[1)]
  \item $\varphi$ ist Lösung der GDG $y'=f(x,y)$ mit $\varphi(a) = c$.
  \item $\varphi$ erfüllt die Integralgleichung
    \[ \varphi(x) = c + \int_a^x f(t, \varphi(t)) \diffop t \]
    für alle $x \in I$.
  \end{enumerate}
\end{thm}

\begin{proof}
  $2) \Rightarrow 1):$ $\varphi(a) = c$ ist klar. Da $\varphi$ stetig ist, ist
  auch $t \mapsto f(t,\varphi(t))$ stetig, also folgt aus dem Hauptsatz, dass
  $\varphi$ stetig differenzierbar ist.
  \[ \varphi'(x) = f(x,\varphi(x)) \]
  für alle $x \in I$.

  $1) \Rightarrow 2):$ $\varphi$ ist stetig (also Riemann-integrierbar) mit
  $\varphi'(t) = f(t,\varphi(t))$. Wenn $\varphi(a) = c$, dann folgt
  \[ \varphi(x) - \underbrace{\varphi(a)}_{=c}
    = \int_a^x \varphi'(t) \diffop t
    \overset{\text{GDG}}{=} \int_a^x f(t,\varphi(t)) \diffop t.
    \qedhere \]
\end{proof}

\subsubsection{Lipschitz-stetige Abbildungen}
Seien $X,Y$ metrische Räume und $f:X \to Y$.
\begin{itemize}
\item $f$ heißt \emph{Lipschitz-stetig} (oder einfach Lipschitz)
  $:\Leftrightarrow$ Es existiert $L \in \real^+$, sodass
  \[ d_Y( f(x), f(\tilde{x})) \le L d_X(x,\tilde{x}). \]
  \item Die kleinste Zahl (Infimum) $L$, für die das gilt, heißt
    \emph{Lipschitzkonstante} von $f$.
\end{itemize}

\begin{rmrk}
  Lipschitz-stetige Abbildungen sind stetig.
\end{rmrk}

\textbf{Erinnerung.} Wenn $L < 1$, dann heißt $f$ Kontraktion.

$f$ ist \emph{lokal Lipschitz-stetig} $:\Leftrightarrow$ Für alle $x \in X$
existiert eine offene Umgebung $U \subset X$, sodass $f$ Lipschitz-stetig ist
auf $U$.

In diesem Kapitel: ``Lipschitz-Bedingung''

Sei $G \subset \real \times \real^n$ und $f:G \to \real^n$, $(x,y) \mapsto
f(x,y)$.
\begin{itemize}
\item $f$ erfüllt eine \emph{Lipschitz-Bedingung} $:\Leftrightarrow$ Es
  existiert $L \in \real$, sodass
  \[ | f(x,y) - f(x,\tilde{y}) | \le L |y-\tilde{y}| \]
  für alle $(x,y), (x,\tilde{y}) \in G$.

  Mit anderen Worten: $f$ ist Lipschitz bezüglich $y$ und zwar gleichmäßig
  bezüglich $x$.
\item $f$ erfüllt eine \emph{lokale} Lipschitz-Bedingung $:\Leftrightarrow$ Für
  alle $(a,b) \in G$ existiert eine Umgebung $U \subset \real \times \real^n$,
  sodass $f$ die Lipschitz-Bedingung auf $G \cap U$ erfüllt.
\end{itemize}

\begin{thm}
  Sei $G \subset \real \times \real^n$ offen, $f:G \to \real^n$, $(x,y) \mapsto
  f(x,y)$, sodass für alle $i = 1, \ldots, n$ $\pdiff{f}{y_i}:G \to \real^n$
  existiert und \emph{stetig} sind.

  Dann genügt $f$ lokal einer Lipschitz-Bedingung.
\end{thm}

\begin{proof}
  Sei $(a,b) \in G$. Da $G$ offen ist, existiert $r > 0$, sodass
  \[ V:= \{ (x,y) \in \real \times \real^n:|x-a| \le r \text{ und } |y-b| \le r
    \} \]
  in $G$ enthalten ist.

  Da $V$ kompakt und $\pdiff{f}{y_i}$ stetig ist, ist
  \[ C := \max_{i=1, \ldots, n} \sup \left\{ \left| \pdiff{f}{y_i} \right|(x) :
      x \in V \right\} \]
  endlich. Also folgt aus dem Mittelwertsatz in $\real^n$, dass $f$ die
  Lipschitz-Bedingung auf $V$ erfüllt.
\end{proof}

\subsubsection{Eindeutigkeitssatz}
Das folgende Lemma gilt auf allgemeinen topologischen Räumen.

\begin{lem}
  Sei $I \subset \real$ ein Intervall und sei $J \subset I$ relativ offen und
  relativ abgeschlossen in $I$. Dann ist $J = \emptyset$ oder $J = I$.
\end{lem}

\begin{proof}
  Sei $J$ nichtleer. Wenn $I$ degeniert ist, dann ist nichts mehr zu zeigen.
  Also sei $I$ ein echtes Intervall. Fixiere ein $x_0 \in J$ und setze $x_1 :=
  \sup \{ x \in I : [x_0, x] \subset J \}$.

  Behauptung: $x_1 = \sup I$.

  Angenommen $x_1 < \sup I$, dann existiert $r > 0$, sodass $[x_1, x_1 + r]
  \subset I$. Insbesondere ist $x_1 \in I$, weil $x_0 \in I$ und $x_0 \le x_1 <
  \sup I$. Außerdem ist $[x_0,x_1) \subset J$ wegen Definition von $x_1$.

  Da $J$ abgeschlossen in $I$ ist und $x_1 \in I$, folgt $x_1 \in J$. Da $J$
  offen in $I$ ist, existiert $\delta > 0$, sodass $B_\delta(x_1) \cap I \subset
  J$. Also wenn wir zudem $\delta \in (0,r)$ wählen, dann folgt
  $[x_1,x_1+\delta] \subset J$.

  Damit gilt $[x_0,x_1+\delta] \subset J$. Das widerspricht der Maximalität von
  $x_1$, also $[x_0, \sup I ] \cap I \subset J$. Analog folgt $[\inf I, x_0]
  \cap I \subset J$. Damit gilt $I = J$.
\end{proof}

\begin{thm}
  Sei $G \subset \real \times \real^n$ und $f: G \to \real^n$ lokal
  Lipschitz. Sei $I \subset \real$ ein Intervall und $\varphi, \psi: I \to
  \real^n$ Lösungen der GDG $y' = f(x,y)$. Wenn zudem ein $x_0 \in I$ existiert,
  so dass $\varphi(x_0) = \psi(x_0)$, dann ist $\varphi(x) = \psi(x)$ für alle
  $x \in I$.
\end{thm}

\begin{rmrk}
  Ohne die lokale Lipschitz-Bedingung an $f$ ist die Aussage im Allgemeinen
  falsch.
\end{rmrk}

\begin{proof}
  Setze $J := \{x \in I: \varphi(x) = \psi(x) \}$.

  Behauptung: $J$ relativ abgeschlossen in $I$.

  Da $\varphi, \psi$ stetig ist, ist auch $\varphi-\psi$ stetig. Also ist die
  Urbildmenge $J$ der abgeschlossenen Menge $\{0\}$ abgeschlossen.

  Alternativer Beweis: Sei $x_n \in J$, sodass $x' := \lim x_n$ in $I$
  existiert. Dann gilt wegen der Stetigkeit: $\varphi(x') = \lim \varphi(x_n) =
  \lim \psi(x_n) = \psi(x')$. Also ist $x' \in J$.

  Behauptung: $J$ ist offen in $I$. Mit anderen Worten: für alle $a \in J$
  existiert $\eps > 0$, sodass $B_\eps(a) \cap I \subset J$.

  Sei $a \in J$. Aus Satz 2.2 folgt
  \[ \varphi(x) - \psi(x) = \int_a^x f(t,\varphi(t)) - f(t,\psi(t)) \diffop
    t. \]
  Wegen der lokalen Lipschitz-Bedingung von $f$ existieren $L, \eps > 0$, sodass
  \[ |f(t,\varphi(t)) - f(t,\psi(t)) | \le L |\varphi(t) - \psi(t)| \]
  für alle $t \in B_\eps(a) \cap I$.

  Also gilt für alle $x \in B_\eps(a) \cap I$:
  \[ | \varphi(x) - \psi(x) | \le \int_a^x |f(t,\varphi(t)) - f(t,\psi(t))|
    \diffop t \le L \int_a^x | \varphi(t) - \psi(t) | \diffop t. \]

  Das nun folgende Argument ist ein Spezialfall des sogenannten Gronwall-Lemmas:
  Definiere
  \[ F(r) := \sup \{ | \varphi(t) - \psi(t) | : t \in I \cap B_r(a) \} \]
  für $r > 0$. Offensichtlich ist $F$ monoton wachsend. Aus der letzten
  Ungleichung folgt
  \[ | \varphi(x) - \psi(x) | \le L \cdot |x-a| \cdot F( |x-a| ) \]
  für alle $x \in I \cap B_\delta(a)$.

  Sei nun $\eps \in (0,\delta)$. Dann folgt wegen der Monotonie $|\varphi(x) -
  \psi(x) | \le L \eps F(\eps)$ für alle $x \in B_\eps(a) \cap I$ und damit
  $F(\eps) \le L \eps F(\eps)$ für alle $\eps \in (0,\delta)$.

  Insbesondere existiert $\eps \in (0,\delta)$, sodass $L \eps = \rez{2}$. Also
  gilt $F(\eps) \le \rez{2} F(\eps)$ für dieses $\eps$, also $F(\eps)=0$.

  Das bedeutet gerade: $\varphi = \psi$ auf $I \cap B_\eps(a)$, das heißt $I
  \cap B_\eps(a) \subset J$.

  Insgesamt haben wir gezeigt: $J$ ist relativ offen und abgeschlossen in $I$.
  Also gilt $J=I$, das heißt $\varphi = \psi$ auf $I$.
\end{proof}

\begin{exmp}
  (siehe Übung) $y' = y^{2/3}$ zeigt, dass die Lipschitz-Bedingung an $f$ nicht
  redundant ist.
\end{exmp}

\subsubsection{Der Banachraum \texorpdfstring{$C^0(K)$}{C0(K)}}
Ein Banachraum ist ein normierter Vektorraum, der vollständig ist (als durch die
Norm induzierter metrischer Raum).

\begin{thm}
  Sei $K \subset \real^n$ kompakt und für jedes $f \in C^0(K)$ definiere
  \[ \| f \| := \sup \{ | f(x) : x \in K \}. \]
  Dann ist $(C^0(K), \|\cdot\|)$ ein Banachraum.
\end{thm}

\begin{proof}
  Norm-Eigenschaften von $\|\cdot\|$ folgen aus den Norm-Eigenschaften von
  $|\cdot|$ auf $\real$ und der Tatsache, dass stetige Funktionen auf kompakten
  Mengen beschränkt sind (also $\|f\| < \infty$).

  Die Vollständigkeit folgt aus der Tatsache, dass gleichmäßig konvergente
  Folgen stetiger Funktionen stetig sind.
\end{proof}

\subsubsection{Existenzsatz von Picard-Lindelöf}
\begin{thm}
  Sei $G \subset \real \times \real^n$ offen, erfülle $f:G \to \real^n$ eine
  lokale Lipschitz-Bedingung und sei $(a,c) \in G$. Dann existiert $\eps > 0$
  und eine Lösung $\varphi:[a-\eps,a+\eps] \to \real^n$ des GDG-Systems $y' =
  f(x,y)$ mit $\varphi(a) = c$.
\end{thm}

\begin{proof}
  Da $G$ offen ist, existiert $\delta, r > 0 $, sodass
  \[ Q_{\delta,r} := \{ (x,y) \in \real \times \real^n : |x-a| \le \delta,
    |y-c| \le r \} \]
  in $G$ enthalten ist und $f$ eine Lipschitz-Bedingung auf $Q_{\delta,r}$
  erfüllt. Da $f$ stetig und $Q_{\delta,r}$ kompakt ist, existiert $M > 0$,
  sodass $|f| \le M$ auf $Q_{\delta,r}$. Setze
  \[ \eps := \min \{ \delta, \frac{r}{M}, \rez{2L} \},
    \quad X := C^0([a-\eps,a+\eps], \real^n),
    \quad Y := \{ \psi \in X: \|\psi - c\| \le r \}. \]

  Definiere die Abbildung $T:Y \to X$, $\psi \mapsto T\psi$,
  \[ (T\psi)(x) := c + \int_a^x f(t,\psi(t)) \diffop t. \]
  Beachte, dass $Y$ eine abgeschlossene Teilmenge des vollständigen metrischen
  Raumes $X$ ist, also ist $Y$ vollständig. Damit ist der Fixpunktsatz in $Y$
  anwendbar.

  Zu zeigen: $T(Y) \subset Y$, $T$ ist eine Kontraktion.

  Behauptung: $T$ ist wohldefiniert und $T(Y) \subset Y$.

  Sei $\psi \in Y$. Da $\| \psi - c \| \le r$, gilt $\operatorname{graph} \psi
  \subset Q_{\eps,r} \subset Q_{\delta,r} \subset G$, also im
  Definitionsbereich von $f$. Zudem ist $t \mapsto f(t, \psi(t))$
  Riemann-integrierbar und $T\psi$ ist stetig, also in $X$ und damit ist $T$
  wohldefiniert.

  Da
  \begin{align*}
    | (T\psi)(x) - c )
    &= \left| \int_a^x f(t,\psi(t)) \diffop t  \right| \\
    &\le \int_a^x |f(t,\varphi(t))| \diffop t \\
    &\overset{(\circ)}{\le} M \cdot |x-a| \le M \cdot \eps \le r,
  \end{align*}
  wobei ($\circ$): $|f| \le M$ auf $Q_{\eps,r}$ und damit $\operatorname{graph}
  \psi \subset Q_{\eps,r}$.

  Da das für alle $x \in [a-\eps,a+\eps]$ gilt, ist $T \psi \in Y$.

  Behauptung: $T: Y \to Y$ ist eine Kontraktion.

  Seien $\psi_1, \psi_2 \in Y$. Dann gilt
  \begin{align*}
    |(T\psi_1)(x) - (T\psi_2)(x) |
    &= \left| \int_a^x f(t,\psi_1(t)) - f(t,\psi_2(t)) \diffop t \right| \\
    &\le L \cdot \int_a^x | \psi_1(t) - \psi_2(t) | \diffop t \\
    &\le L |x-a| \| \psi_1 - \psi_2 \| \le \rez{2} \| \psi_1 - \psi_2 \|,
  \end{align*}
  wobei $\psi_1(t) - \psi_2(t) \le \| \psi_1 - \psi_2 \|$ für alle $t \in
  [a-\eps,a+\eps]$ und $|x-a| \le \eps$.

  Also ist $\| T\psi_1 - T\psi_2 \| \le \rez{2} \| \psi_1 - \psi_2$ für alle
  $\psi_1, \psi_2 \in Y$. Wegen des Fixpunktsatzes von Banach (siehe Analysis 2)
  existiert $\varphi \in Y$, sodass $\varphi = T \varphi$, und wegen Satz 2.2
  löst dieses $\varphi$ die GDG.
\end{proof}

\begin{rmrk}
  Gemäß Satz 2.4 ist $\varphi$ die einzige Lösung mit $\varphi(a) = c$.

  Im Beweis des Fixpunktsatzes findet man Fixpunkte iterativ. Hier $\psi_0 := c$
  und $\psi_{k+1}:= c + \int_a^x f(t,\psi_k(t))$ liefert eine gleichmäßig
  konvergente Folge $(\psi_k)$, die gegen die Lösung $\varphi$ konvergiert.

  Die Lösung existiert im Allgemeinen auf einem \emph{kleinen} Intervall.
\end{rmrk}

\subsection{Gewöhnliche Differentialgleichungen höherer Ordnung}
\subsubsection{GDG \texorpdfstring{$n$}{n}-ter Ordnung}

\begin{defn}
  Sei $G \subset \real \times \real^n$ und $f:G \to \real$ stetig. Dann heißt
  \[ y^{(n)} = f(x,y,y',\ldots, y^{(n-1)})\]
  gewöhnliche Differentialgleichung \emph{$n$-ter Ordnung}. Unter einer
  \emph{Lösung} versteht man eine auf dem Intervall $I \subset \real$ $n$-mal
  differenzierbare Funktion $\varphi: I \to \real$, sodass:
  \[ \{ (x, \varphi(x), \varphi'(x), \ldots, \varphi^{(n-1)}(x):x \in I) \}
    \subset G \]
  und für alle $x \in I$ gilt
  \[ \varphi^{(n)}(x) = f(x,\varphi(x), \ldots, \varphi^{(n-1)}(x)). \]
\end{defn}

Beispiel: $y'' = -2y$.

\subsubsection{Reduktion auf System erster Ordnung}
Eine Gleichung $n$-ter Ordnung $y^{(n)} = f(x, \ldots, y^{(n-1)})$ ist äquivalent
zum folgenden System 1. Ordnung:
\begin{align*}
  y'_0 &= y_1 \\
  y'_0 &= y_1 \\
       &\vdots \\
  y'_{n-2} &= y_{n-1} \\
  y'_{n-1} &= f(x,y_0,\ldots, y_{n-1}).
\end{align*}

Diese Äquivalenz ist leicht zu sehen (siehe Forster Analysis 2). Völlig analog
kann man auch jedes \emph{System} $n$-ter Ordnung auf ein System 1. Ordnung
reduzieren.

\subsubsection{Existenz und Eindeutigkeit}
Folgerung aus den Resultaten für Systeme 1. Ordnung:

\clearpage

\begin{kor}
  Sei $G \subset \real \times \real^n$ offen, $f:G \to \real$ stetig und $(x,y)
  \mapsto f(x,y)$ erfülle eine lokale Lipschitz-Bedingung. Dann
  \begin{enumerate}
  \item (Eindeutigkeit) Seien $\varphi, \psi: I \to \real$ Lösungen von
    \[ y^{(n)} = f(x,y,\ldots,y^{(n-1)}).\]
    Für ein $a \in I$ gelte:
    \[ \varphi(a) = \psi(a), \quad \varphi'(a) = \psi'(a), \quad \ldots \quad
      \varphi^{(n-1)}(a) = \psi^{(n-1)}(a). \]
    Dann ist $\varphi(x) =  \psi(x)$ für alle $x \in I$.
  \item (Existenz) Sei $(a,c_0, \ldots, c_{n-1}) \in G$. Dann existiert $\eps >
    0$ und eine Lösung $\varphi:[a-\eps, a+\eps] \to \real$ von $y^{(n)} =
    f(x,y, \ldots, y^{(n-1)})$, die den $n$ Anfangsbedingungen genügt:
    \[ \varphi(a) = c_0, \quad \varphi'(a) = c_1, \quad \ldots \quad
      \varphi^{(n-1)}(a) = c_{n-1}.\]
  \end{enumerate}
\end{kor}

\subsection{Lineare Systeme}
\subsubsection{Homogene und inhomogene lineare Systeme}
Sei $I \subset \real$ ein Intervall und $A:I \to \real^{n \times n}$ stetig.
Dann heißt $y' = A(x) y$ \emph{homogenes} lineares System 1. Ordnung.

Sei zudem $b: I \to \real^n$ stetig. Dann heißt $y' = A(x)y + b(x)$
\emph{inhomogenes} lineares System 1. Ordnung.

Das zum inhomogenen System $y' = A(x)y + b(x)$ gehörige homogene System ist $y'
= A(x) y$.

Komplexe GDG: Sei $A:I \to \complex^{n \times n}$ und $b:I \to \complex^n$. Da
$\complex = \real \times \real$, ist ein System von $n$ komplexen GDG äquivalent
zu einem System von $2n$ reellen GDG. Im Folgenden ist $\koer = \complex$ oder
$\real$.

\subsubsection{Existenz und Eindeutigkeit der Lösung}
 \begin{lem}
  Sei $I \subset \real$ ein  Intervall, $f: I \times \koer^n \to \koer^n$
  erfülle eine (globale) Lipschitz-Bedingung.

  Dann existiert für alle $(x_0,c) \in I \times \koer^n$ genau eine Lösung
  $\varphi:I \to \koer^n$ der GDG $\varphi'=f(x,y)$ mit $\varphi(x_0) = c$.
 \end{lem}

 \begin{proof}
  Die Eindeutigkeit folgt aus Satz 2.4.
  
  Für die Existenz benutzt man den Satz von Picard-Lindelöf. Definiere $\varphi_k: I \to \real^n$ induktiv durch $\varphi_0(x) := c$ und 
  \[ \varphi_{k+1}(x) := c + \int_{x_0}^x f(t, \varphi_k(t)) \diffop t. \]
  Dieser Ausdruck ist wohldefiniert, da $\varphi_k$ induktiv stetig ist, also ist $f( \cdot, \varphi_k(\cdot) )$ Riemann-integrierbar.
  
  Sei $J \subset I$ ein kompaktes Teilintervall (z.B. $I = \real$ und $J = [-R,R]$). Nun wird gezeigt, dass die Folge $(\varphi_k)$ eine Cauchy-Folge im Banach-Raum $C^0(J,\real^n)$ mit Norm $\| f \|_J := \sup \{ |f(x)| : x \in J \}$ ist.
  \[ | \varphi_{k+1}(x) - \varphi_k(x) | \le \frac{L^k | x - x_0 |^k}{k!} \| \varphi_1 - \varphi_0 \|_J \]
  für alle $k \in \nat$, $x \in J$.
  
  Wir zeigen nun die Konvergenz der rechten Seite durch Induktion nach $k$.
  
  Induktionsanfang: $k=0$ klar.
  
  Induktionsschritt:
  \[ \begin{aligned}
     | \varphi_{k+2}(x) - \varphi_{k+1}(x) | 
     &\le \int_{x_0}^x \left| f(t, \varphi_{k+1}(t) ) - f(t, \varphi_k(t)) \right| \diffop t. \\
     &\le L \int_{x_0}^x \frac{L^k}{k!} |t-x_0|^k \diffop t \cdot \|\varphi_0 - \varphi_1\|_J  
     \end{aligned} \]
     
  Aus der Behauptung folgt nun, dass die Reihe $\sum_{k=0}^\infty | \varphi_{k+1}(x) - \varphi_k(x)|$ auf $J$ durch die konvergente Reihe
  \[ \sum_{k=0}^\infty \frac{(L \lebesgue^1(J))^k}{k!} = \exp( L \lebesgue^1(J) ) \]
  majoriert wird. Da $\lim_{k \to \infty} \varphi_k = \sum_{i=0}^\infty (\varphi_{i+1} - \varphi_i)$ folgt, dass $\varphi_k \to: \varphi$ auf $J$ gleichmäßig konvergiert.
  
  Da $J$ beliebig war, haben wir gezeigt, dass $\varphi \in C^0(I, \real^n)$ existiert, so dass $\varphi_k \to \varphi$ gleichmäßig konvergiert auf kompakten Teilmengen (Kompakta).
  
  Grenzübergang $k \to \infty$ in der Definition von $\varphi_{k+1}$ ergibt
  \[ \varphi(x) = c + \int_{x_0}^x f(t, \varphi(t) ) \diffop t \]
  für alle $x \in I$.
  
  Gemäß Satz 2.2 folgt also: $\varphi$ ist Lösung.
 \end{proof}

\begin{rmrk}
 Der Beweis verwendet nur, dass für jedes kompakte Teilintervall $J \subset I$ $f:J \times \koer^n \to \koer^n$ eine (globale) Lipschitz-Bedingung erfüllt (im Beweis durfte $L$ von $J$ abhängen).
\end{rmrk}

\begin{thm}
 Sei $I \subset \real$ ein Intervall, $A: I \to \koer^{n \times n}$ und $b:I \to \koer^n$ seien stetig. Dann existiert für alle $x \in I$ und alle $c \in \koer^n$ genau eine Lösung $\varphi:I \to \koer^n$ mit $\varphi(x_0) = c$ des linearen GDG-Systems
 \[ y' = A(x) y + b(x). \]
\end{thm}

\begin{proof}
 Definiere $f(x,y) := A(x) y + b(x)$ und sei $J \subset I$ ein kompaktes Teilintervall. Dann
 \[ | f(x,y) - f(x,\tilde{y}) | = | A(x) (y-\tilde{y}) | \le L |y-\tilde{y}|, \]
 weil $A$ stetig und $J$ kompakt sind ($L := \sup \{ \|Ax\|: x \in J \} < \infty$).
 
 Die Behauptung folgt nun aus dem Lemma.
\end{proof}

\begin{rmrk}
 Bei \emph{linearen} GDG existiert die Lösung also \emph{global}. Bei \emph{nichtlinearen} GDG hingegen existiert die Lösung im Allgemeinen nur \emph{lokal}.
\end{rmrk}

\subsubsection{Homogene lineare GDG-Systeme}
\begin{thm}
 Sei $I \subset \real$ ein nicht degeneriertes Intervall und $A:I \to \koer^{n \times n}$ stetig. Dann bildet die Menge $L_H$ aller Lösungen $\varphi:I \to \koer^n$ der homogenen linearen Gleichung 
 \[ y' = A(x) y \]
 einen $n$-dimensionalen Vektorraum über $\koer$.
 
 Seien $\varphi_1, \ldots, \varphi_k \in L_H$. Dann sind äquivalent:
 \begin{enumerate}
  \item $\varphi_1, \ldots, \varphi_k$ sind linear unabhängig\footnotemark (als Elemente des Vektorraums aller Abbildungen $I \to \koer^n$),
  \item es existiert $x_0 \in I$, so dass $\varphi_1(x_0), \ldots, \varphi_k(x_0)$ linear unabhängig\footnotemark (in $\koer^n$) sind,
  \item für alle $x \in I$ sind $\varphi_1, \ldots, \varphi_k$ linear unabhängig\footnotemark (in $\koer^n$). 
 \end{enumerate}

\end{thm}
\addtocounter{footnote}{-2}
\footnotetext{%
 Das heißt, wenn $\lambda_1, \ldots, \lambda_k \in \koer$ und $\lambda_1
 \varphi_1(x) + \ldots + \lambda_k \varphi_k(x) = 0$ für alle $x$, dann gilt
 $\lambda_1 = \ldots = \lambda_k = 0$.}
\addtocounter{footnote}{1}
\footnotetext{%
 Das heißt, wenn $\lambda_1, \ldots, \lambda_k \in \koer$ und $\lambda_1
 \varphi_1(x_0) + \ldots + \lambda_k \varphi_k(x_0) = 0$, dann gilt $\lambda_1 =
 \ldots = \lambda_k = 0$.}
\addtocounter{footnote}{1}
\footnotetext{%
 Das heißt, wenn $\lambda_1, \ldots, \lambda_k: I \to \koer$ und $\lambda_1(x)
 \varphi_1(x) + \ldots + \lambda_k(x) \varphi_k(x) = 0$ für alle $x \in I$, dann
 gilt $\lambda_1(x) = \ldots = \lambda_k(x) = 0$ für alle $x \in I$.} 
 
\begin{proof}
 $L_H$ ist ein Vektorraum: Sind $\varphi_1, \varphi_2: I \to \koer$ in $L_H$ (Lösungen), das heißt $\varphi'(x) = A(x) \varphi(x)$ ($i=1,2$), dann ist auch $\varphi_1 + \varphi_2 \in L_H$, denn
 \[ (\varphi_1 + \varphi_2)'(x) = \varphi'_1(x) + \varphi'_2(x) = A(x) \varphi_1(x) + A(x) \varphi_2(x) = A(x) (\varphi_1(x) + \varphi_2(x)). \]
 Die Skalarmultiplikation zeigt man analog.
 
 Behauptung: $\dim L_H \ge n$.
 
 Seien $v_1, \ldots, v_n \in \koer^n$ linear unabhängig und $x_0 \in I$. Gemäß Satz 4.2 existieren $\varphi_1, \ldots, \varphi_n \in L_H$ mit $\varphi_i(x_0) = v_i$ für $i = 1, \ldots, n$. Also ist $\{ \varphi_i \}$ linear unabhängig (als Abb.), weil $\{ \varphi_i(x_0) \}$ linear unabhängig ist.
 
 Behauptung: $\dim L_H \le n$.
 
 Seien $\varphi_1, \ldots, \varphi_{n+1} \in L_H$. Wären sie linear unabhängig, dann existiert gemäß 3. (noch zu zeigen) insbesondere ein $x_0 \in I$, so dass $\varphi_1(x_0), \ldots, \varphi_{n+1} \in \koer^n$ linear unabhängig sind. Widerspruch!
 
 Zu den äquivalenten Aussagen: Es ist klar, dass 3. $\Rightarrow$ 2. $\Rightarrow$ 1. gilt. Also muss noch 1. $\Rightarrow$ 3. gezeigt werden.
 
 Seien also $\varphi_1, \ldots, \varphi_k \in L_H$ linear unabhängig (als Abbildungen $I \to \koer^n$) und sei $x_0 \in I$ beliebig. Wir zeigen nun, dass dann $\varphi_1(x_0), \ldots, \varphi_k(x_0) \in \koer^n$ linear unabhängig sind. 
 
 Seien $\lambda_1, \ldots, \lambda_k \in \koer$ sind, so dass
 \[ \lambda_1 \varphi_1(x_0) + \ldots + \lambda_k \varphi_k(x_0) = 0. \]
 Dann ist $\varphi := \lambda_1 \varphi_1 + \ldots + \lambda_k \varphi_k$ eine Lösung der GDG. Außerdem ist $\varphi(x_0) = 0$. Wegen der Eindeutigkeit von Lösungen der GDG mit gegebenen Anfangsdaten muss dann $\varphi \equiv 0$ für alle $x$ sein, das heißt 
 \[ \lambda_1 \varphi_1 + \ldots + \lambda_k \varphi_k = 0 \]
 als Abbildungen $I \to \koer^n$. Daraus folgt $\lambda_1 = \ldots = \lambda_k = 0$.
\end{proof}

\subsubsection{Lösungsfundamentalsystem}
Eine Basis des Vektorraums aller Lösungen eines homogenen linearen GDG-Systems
heißt \emph{Lösungsfundamentalsystem} (LFS).

Wenn also $A:I \to \koer^n$ und $L_H$ Lösungsraum von $y' = A(x) y$ und
$\varphi_1, \ldots, \varphi_n \in L_H$ linear unabhängig sind, dann ist
$(\varphi_1, \ldots, \varphi_n)$ ein LFS.

Für jede Lösung $\varphi: I \to \koer^n$ existieren eindeutige $c_1, \ldots, c_n
\in \koer$, so dass
\[ \varphi(x) = c_1 \varphi_1(x) + \ldots + c_n \varphi_n(x) \]
für alle $x \in I$. Die $c_i$ werden festgelegt durch den Wert $\varphi(x_0)$ an
einem Punkt $x_0$.

Oft ist diese kompakte Schreibweise zweckmäßig:
\[ \Phi := ( \varphi_1 | \varphi_2 | \cdots | \varphi_n ) =
   \begin{pmatrix}
     (\varphi_1)_1 & \cdots & (\varphi_n)_1 \\
     \vdots & & \vdots \\
     (\varphi_1)_n & \cdots & (\varphi_n)_n
   \end{pmatrix}. \]
 Dann ist $\Phi:I \to \koer^{n \times x}$ eine (matrixwertige) Lösung von
 \[ \Phi' = A(x) \Phi. \]
 Und da $\{ \varphi_i \}_1^n$ linear unabhängig, ist $\Phi(x)$ invertierbar für
 alle $x \in I$.

 \begin{exmp}
   Homogenes lineares System:
   \[ \begin{aligned}
       y'_1 &= - y_2 \\
       y'_2 &= y_1
     \end{aligned} \]
   Das ist von der Form $y' = A(x) y$ mit $A(x) = \begin{pmatrix} 0 & -1 \\ 1 &
     0 \end{pmatrix}$ für alle $x \in I$.

   Lösungen (raten): $\varphi_1 = (\cos x, \sin x)^T$, $\varphi_2 = (- \sin x,
   \cos x)^T$. Sie sind linear unabhängig, weil die Matrix
   \[ \Phi(x) := (\varphi_1(x) | \varphi_2(x)) = \begin{pmatrix} \cos x & - \sin x \\ \sin x & \cos x
     \end{pmatrix} \in SO(2) \]
   ist für alle $x \in I$ und damit invertierbar ($\det \Phi = -1$). Also ist $\varphi_1, \varphi_2$ ein LFS.
 \end{exmp}

 \subsubsection{Inhomogene lineare Systeme}
 \begin{thm}
   Sei $I \subset \real$ ein nicht degeneriertes Intervall und $A: I \to
   \koer^{n \times n}$, $b: I \to \koer^n$ stetige Abbildungen und sei $L_H$ der
   Vektorraum aller Lösungen des homogenen Systems $y' = A(x) y$.

   Dann ist die Menge $L_I$ aller Lösungen $\varphi:I \to \koer^n$ des inhomogenen
   Systems
   \[ y' = A(x) y + b(x) \]
   ein affiner Raum der Form $\psi_0 + L_H$. Dabei ist $\psi_0 \in L_I$ eine
   beliebige Lösung des inhomogenen Systems.
 \end{thm}

 \begin{proof}
   Das folgt aus der Tatsache, dass
   \[ \psi_0, \psi_1 \in L_I \qLRq \psi_0 - \psi_1 \in L_H, \]
   weil
   \[ (\psi_0 - \psi_1)' = \psi'_0 - \psi'_1 = A\psi_0 + b - ()A\psi_1 + b).
     \qedhere \]
 \end{proof}

\subsubsection{Variation der Konstanten - Formel}
Um alle Lösungen $L_I$ zu finden, muss man also $L_H$ kennen und eine spezielle
Lösung $\psi_0 \in L_I$. Letztere definiert man so:

\clearpage

\begin{thm}
  Mit den Bezeichnungen von Satz 4.5 gilt: Sei $\varphi_1, \ldots, \varphi_n \in
  L_H$ ein LFS und setze $\Phi := ( \varphi_1 | \cdots | \varphi_n)$. Sei $x_0
  \in I$ und $c \in \koer^n$. Dann ist
  \[ \psi(x) := \Phi(x) \left(  \Phi^{-1}(x_o) c + \int_{x_0}^x \Phi^{-1}(t)
      b(t) \diffop t \right) \]
  die Lösung der inhomogenen GDG $y' = A(x) y + b(x)$ mit $\psi(x_0) = c$.
\end{thm}

\begin{proof}
  Offenbar erfüllt $\psi$ die Anfangsbedingung.

  Dass $\psi$ eine Lösung ist, erhält man durch Nachrechnen:
  \[ \psi' = \underbrace{\Phi'(\cdots)}_{= A(x) \Phi(x)} + \Phi(x) \Phi^{-1}(x)
    b(x). \qedhere \]
\end{proof}

\begin{rmrk}
  Wenn man nur irgendeine spezielle Lösung der inhomogenen GDG sucht, dann wählt
  man zum Beispiel $c=0$. Es gilt dann
  \[ \psi(x) = \Phi(x) \int_{x_0}^x \Phi^{-1}(t) b(t) \diffop t. \]
\end{rmrk}

\begin{exmp}
  Bestimme die allgemeine Lösung von
  \[ y'_1 = -y_2, \quad y'_2 = y_1 + x. \]
  Das System ist von der Form $y' = Ay + b(x)$, wobei $A := \begin{pmatrix} 0 &
    -1 \\ 1 & 0 \end{pmatrix}$ und $b:\real \to \real^2$, $b(x) := (0,x)^T$.
  
  Lösungsfundamentalsystem: $\Phi(x) = \begin{pmatrix} \cos x & - \sin x \\ \sin
    x & \cos x \end{pmatrix}$.

  Da $\Phi(x) \in SO(2)$, ist $\Phi^{-1}(x) = \Phi^T(x) = \begin{pmatrix} \cos x
    & \sin x \\ - \sin x & \cos x \end{pmatrix}$. Also gilt
  \[ \Phi^{-1}(t) b(t) =  \begin{pmatrix} t \sin t \\ t \cos t \end{pmatrix} \]
  und wir erhalten
  \[ u(x) := \int_{x_0}^x \Phi^{-1}(t) b(t) = \begin{pmatrix} \sin x - x \cos x
      \\ \cos x + x \sin x \end{pmatrix}. \]
  Also ist eine spezielle Lösung 
  \[ \psi(x) := \Phi(x) u(x) = \begin{pmatrix} -x \\ 1 \end{pmatrix}. \]
  Für die allgemeine Lösung der zugehörigen homogenen Gleichung hinzu addieren:
  \[ \varphi(x) = \psi(x) + c_1 \pmat{\sin x \\ \cos x} + c_2 \pmat{-\sin x \\
        \cos x} \]
    mit $c_1, c_2 \in \koer^n$ beliebig.
\end{exmp}

\subsubsection{Lineare GDG \texorpdfstring{$n$}{n}-ter Ordnung}
Sei $I \subset \real$ ein Intervall und $a_0, \ldots, a_{n-1}: I \to \koer$
stetige Funktionen. Dann heißt
\[ y^{(n)} + a_{n-1}(x) y^{(n-1)} + \ldots + a_0(x) = 0 \tag{H} \]
\emph{homogene lineare GDG $n$-ter Ordnung}.

Ist $b:I \to \koer^n$ stetig, dann heißt
\[ y^{(n)} + a_{n-1}(x) y^{(n-1)} + \ldots + a_0(x) = b(x) \tag{I} \]
\emph{inhomogene lineare GDG $n$-ter Ordnung}.

\begin{thm}
  \begin{enumerate}
    \item Die Menge $L_H$ aller Lösungen $\varphi:I \to \koer$ von (H) ist ein
      $n$-dimensionaler Vektorraum über $\koer$.
    \item Die Menge $L_I$ aller Lösungen $\varphi:I \to \koer$ von (I) ist ein
      affiner Raum der Form $\tilde{\varphi}_0 + L_H$.
    \item Seien $\varphi_1, \ldots, \varphi_n \in L_H$. Diese Lösungen sind
      genau dann linear unabhängig, wenn die sogenannte Wronski-Determinante für
      ein (äquivalent: alle) $x \in I$ ungleich Null ist.
      \[ \text{Wronski-Det. } := \det
        \begin{pmatrix}
          \varphi_1 & \cdots & \varphi_n \\
          \varphi'_1 & \cdots & \varphi'_n \\
          \vdots & & \vdots  \\
          \varphi^{(n-1)}_1 & \cdots & \varphi^{(n-1)}_n 
        \end{pmatrix}. \]
    \end{enumerate}
  \end{thm}

  \begin{proof}
    Alles folgt unmittelbar auf den entsprechenden Resultaten für Systeme erster
    Ordnung (via Abschnitt 3.2).

    Betrachte zur Illustration den Fall $n=2$, also die inhomogene lineare GDG
    zweiter Ordnung
    \[ y'' + a_1(x) y' + a_0(x) y = b(x). \]
    Diese Gleichung ist äquivalent zum System ($z_0 := y, z_1 := y'$)
    \[ \begin{aligned}
        z'_0 &= z_1 \\
        z'_1 &= -a_1(x) z_1 - a_0(x) z_0 + b(x) 
      \end{aligned} \]
    Mit $z := \pmat{z_0 \\ z_1}$ wird das zum linearen System $z' = A(x) z +
    B(x)$, wobei
    \[ A(x) := \pmat{0 & 1 \\ -a_0(x) & -a_1(x)}, \quad B(x) := \pmat{0 \\
        b(x)}.\]

    Jeder Lösung $\psi = \pmat{\psi_0 \\ \psi_1}$ dieses Systems entspricht eine
    Lösung $\varphi := \psi_0$ der Gleichung zweiter Ordnung. Umgekehrt: wenn
    $\varphi$ die Gleichung löst, dann löst $\psi := \pmat{\varphi \\ \varphi'}$
    das System.

    Daraus folgt die Behauptung.
  \end{proof}

  \begin{rmrk}
    Für beliebige Funktionen $\varphi_1, \ldots, \varphi_n:I \to \koer$ gilt:
    Wronski-Determinante $\ne 0$ $\Rightarrow$ $\{ \varphi_i \}$ linear
    unabhängig. Aber die Umkehrung gilt nur für Lösungen der GDG.
  \end{rmrk}

\begin{exmp}
  Homogene GDG zweiter Ordnung
  \[ y'' - \rez{2x} y' + \rez{2x^2}y = 0 \]
  auf $I := (0,\infty)$.

  Zwei Lösungen raten: $\varphi_1(x)=x$, $\varphi_2(x) = \sqrt{x}$.
  Wronski-Determinante:
  \[ \det \pmat{\varphi_1 & \varphi_2 \\ \varphi'_1 & \varphi'_2}
    = \det \pmat{ x & \sqrt{x} \\ 1  & \rez{2 \sqrt{x}} }
    = - \frac{\sqrt{x}}{2} \ne 0 \]
  auf $I$. Also ist $\varphi_1, \varphi_2$ ein LFS.

  Um die allgemeine Lösung der inhomogenen Gleichung
  \[ y'' -\rez{2x} y' + \rez{2x^2}y = 1 \]
  zu erhalten, rate eine \emph{spezielle} Lösung der inhomogenen Gleichung:
  $\psi_0(x) := \frac{2}{3} x^2$. Damit ist die \emph{allgemeine} Lösung
  \[ \psi(x) = \frac{2}{3} x^2 + c_1 x + c_2 \sqrt{x}. \]
\end{exmp}

\subsection{Lineare GDG mit konstanten Koeffizienten}

\subsubsection{Lineare Differentialoperatoren mit konstanten Koeffizienten}
Jedem Polynom mit (konstanten) Koeffizienten $a_0, \ldots, a_{n-1} \in
\complex$,
\[ P(z) = a_0 + a_1 z + \ldots + a_{n-1}z^{n-1} + z^n,\]
ordnet man folgenden Differentialoperator der Ordnung $n$ mit konstanten
Koeffizienten zu:
\[ P \left( \diff{}{x} \right) := a_0 + a_1 \diff{}{x} + a_2
  \frac{\diffop^2}{\diffop x^2} + \ldots
  + a_{n-1} \frac{\diffop^{n-1}}{\diffop x^{n-1}}
  + \frac{\diffop^n}{\diffop x^n}. \]
Ein solcher Differentialoperator ordnet einer $n$-mal differenzierbaren Funktion
$\varphi: \real \to \complex$ die Funktion
\[ P \left(  \diff{}{x} \right) \varphi := a_0 \varphi + a_1 \varphi' + \ldots
  + \varphi^{(n)} \]
zu.

Mit $P$ assoziiert man die lineare homogene GDG $P\left( \diff{}{x} \right) y =
0$, das heißt
\[ a_0 y + a_1 y' + \ldots + a_{n-1}y^{(n-1)} + y^{(n)} = 0. \]

Umgekehrt ist jeder lineare gewöhnliche Differentialoperator mit konstanten
Koeffizienten von dieser Form.

\clearpage

\subsubsection{Paarweise verschiedene Nullstellen}
\begin{lem}
  Für jedes $\lambda \in \complex$ und jedes Polynom $P$ gilt
  \[ P\left( \diff{}{x} \right) e^{\lambda x} = P(\lambda) e^{\lambda x}. \]
  Insbesondere gilt: Wenn $P(\lambda) = 0$, dann ist $\varphi:\real \to
  \complex$,
  \[ \varphi(x) := e^{\lambda x} \]
  eine Lösung der GDG $P\left( \diff{}{x} \right) y = 0$.
\end{lem}

\begin{proof}
  Das folgt aus der Tatsache, dass
  \[ \diff{}{x} e^{\lambda x} = \lambda e^{\lambda x}, \]
  also auch $\frac{\diffop^k}{\diffop x^k} e^{\lambda x} = \lambda^k e^{\lambda
    x}$. 
\end{proof}

\begin{prp}
  Seien $\lambda_1, \ldots, \lambda_n \in \complex$ paarweise verschieden. Dann
  sind die Funktionen $\varphi_k : \real \to \complex$, $\varphi_k(x) :=
  e^{\lambda_k x}$ linear unabhängig.
\end{prp}

\begin{proof}
  Induktion nach $n$:

  $n=1$. Aus $\mu e^{\lambda x} = 0$ folgt $\mu = 0$. Die Behauptung gilt also.

  $n-1 \Rightarrow n$. Seien $\mu_1, \ldots, \mu_n \in \complex$ und
  \[ \mu_1 e^{\lambda_1 x} + \ldots + \mu_n e^{\lambda_n x} = 0. \]
  Zu zeigen: $\mu_1 = \ldots = \mu_n = 0$.
  
  Skizze: Zuerst multipliziere mit $e^{-\lambda_1 x}$, dann leite ab und nutze
  die Induktionsvoraussetzung.
\end{proof}

Alternativ kann man die Wronski-Determinante an der Stelle $x \in \real
\setminus \{ 0 \}$ bestimmen.

\begin{thm}
  Sei $P(z) = a_0 + a_1 z + \ldots + a_{n-1} z^{n-1} + z^n$ mit $a_i \in
  \complex$ und habe $P$ $n$ \emph{verschiedene} Nullstellen $\lambda_1, \ldots,
  \lambda_n \in \complex$.

  Dann bilden die Funktionen $\varphi_n: \real  \to \complex$
  \[ \varphi_k := e^{\lambda_k x} \]
  für $k = 1, \ldots, n$ ein LFS der GDG $P\left( \diff{}{x} \right) y = 0$.
\end{thm}

\begin{proof}
  Kombination von Lemma und Proposition.
\end{proof}

\subsubsection{Hauptsatz über Lösung linearer homogener GDG mit konstanten Koeffizienten}
Wegen des Fundamentalsatzes der Algebra kann man jedes Polynom wie folgt
zerlegen:
\[ P(z) = (z-\lambda_1)^{k_1} (z-\lambda_2)^{k_2} \cdots (z-\lambda_r)^{k_r}, \tag{$\ast$} \]
wobei $\lambda_1, \ldots, \lambda_r \in \complex$ paarweise verschieden sind und
$k_1 , \ldots, k_r \in \nat \setminus \{0\}$.

\begin{exmp}
  $P(z) = z^2$. Also $P\left( \diff{}{x} \right) y = y''$. Die GDG $y'' = 0$ hat
  Lösungen der Form $\varphi(x) = ax + b$.

  Aus 5.2 erhalten wir aber nur Lösungen der Form $\varphi(x) = a e^{0x} = a$.
\end{exmp}

\begin{thm}
  $P$ sei ein Polynom mit Koeffizienten in $\complex$ mit $r \in \nat$
  verschiedenen Nullstellen $\lambda_1, \ldots \lambda_r \in \complex$ mit
  Vielfachheiten $k_1, \ldots, k_r \in \nat \setminus \{0\}$.

  Dann besitzt die lineare homogene GDG
  \[ P \left( \diff{}{x} \right) y = 0 \]
  das folgende LFS:
  \[ \varphi_{jm}(x) := x^m e^{\lambda_j x}, \]
  wobei $j=1,\ldots,r$, $m=0,\ldots,k_j-1$.
\end{thm}

Zum Beweis benötigen wir zwei Hilfssätze.

\begin{lem}
  Sei $\lambda \in \complex$ und $k \in \nat \setminus \{0 \}$ und $f: \real \to
  \complex$ sei $k$-mal differenzierbar.

  Dann gilt
  \[ \left( \diff{}{x} - \lambda  \right)^k (f(x)e^{\lambda x}) = f^{(k)}(x)
    e^{\lambda x}.  \]
\end{lem}

\begin{proof}
  Vollständige Induktion nach $k$.

  $k=1$.
  \[ \left( \diff{}{x} - \lambda  \right)^k (f(x)e^{\lambda x})
    = \diff{}{x} (f(x) e^{\lambda x}) - \lambda f(x) e^{\lambda x}
    = f'(x) e^{\lambda x}. \]

  $k \Rightarrow k+1$. Genauso.
\end{proof}

Insbesondere: Wenn $f^{(k)} \equiv 0$, dann ist $\varphi(x) :=) f(x)
e^{\lambda_i x}$ für jedes $i = 1, \ldots, r$ eine Lösung der GDG $P\left(
  \diff{}{x} \right) = 0$, wobei $P$ wie in ($\ast$) ist.

\begin{lem}
  Sei $P$ ein Polynom und $\lambda \in \complex$, sodass $P(\lambda) \ne 0$. Sei
  $g : \real \to \complex$ eine Polynomfunktion. Dann existiert eine
  Polynomfunktion $h:\real \to \complex$ von demselben Grad wie $g$, sodass
  gilt:
  \[ P \left( \diff{}{x}  \right) (g(x) e^{\lambda x}) = h(x) e^{\lambda x}. \]
  Genauer: Es existieren $c_j \in \complex$, sodass
  \[ h = P(\lambda) g + c_1 g' + \ldots c_{k-1} g^{(k-1)}, \]
  wobei $k$ der Grad von $g$ ist.
\end{lem}

\begin{proof}
  Es existieren $n \in \nat$ und $c_j \in \complex$, sodass
  \[ P(z) = \sum_{j=0}^n c_j (z-\lambda)^j. \]
  Gemäß des vorherigen Lemmas gilt also:
  \[ P \left( \diff{}{x} \right) (g(x) e^{\lambda x}) =  \sum_{j=0}^n c_j \left(
    \diff{}{x} - \lambda \right)^j (g(x) e^{\lambda x}) = \sum_{j=0}^n c_j
  g^{(j)}(x) e^{\lambda x} =: h(x).
  \qedhere
  \]
\end{proof}

\begin{prp}
  Sei $r \in \nat \setminus \{ 0 \}$m seien $\lambda_1$, $\lambda_2 \in
  \complex$ paarweise verschieden und seien $k_1, \ldots, k_r \in \nat \setminus
  \{0\}$.

  Dann sind die Funktionen $\varphi_{jm}:\real \to \complex$
  \[ \varphi_{jm}(x) := x^m e^{\lambda_j} \]
  für $j = 1, \ldots, r$, $m = 0, \ldots, k_j - 1$, linear unabhängig.
\end{prp}

\begin{proof}
  Betrachte Linearkombinationen der $\varphi_{jm}$. Seien $q_{jm} \in \complex$
  und betrachte
  \[ \sum_{j=1}^r \sum_{m=0}^{k-1} q_{jm} x^m e^{\lambda_j x} = 0 \]
  für alle $x \in \real$.

  Zu zeigen: $q_{jm} = 0$. Wir beweisen daher folgende allgemeine Tatsache:

  Für $j = 1, \ldots, r$ seien $g_j: \real  \to \complex$ Polynomfunktionen und
  seien $\lambda_1, \ldots, \lambda_r$ paarweise verschieden. Wenn
  \[ \sum_{j=1}^r g_j(x) e^{\lambda_j x} = 0 \]
  für alle $x \in \real$, dann ist $g_j = 0$ für alle $j$.

  Der Beweis erfolgt durch Induktion nach $r$:

  $r=1$: Sei
  \[ g_1(x) e^{\lambda_1 x} = 0 \qRq g_1 \equiv 0 \]
  für alle $j$.

  Induktionsschritt: Wenn $g_j = 0$ für eines der $j$, dann folgt die
  Behauptung aus der Induktionsvoraussetzung. Wir zeigen nun, dass die
  Alternative nicht eintreten kann. Seien alle $g_j$ von der Nullfunktion
  verschieden. Sei $k \in \nat$ echt größer als der Grad der Polynomfunktion
  $g_r$. Setze $Q(z) := (z - \lambda_r)^k$. Da $k > \operatorname{grad} g_r$
  ist, folgt aus Lemma 1:
  \[ Q \left( \diff{}{x} \right) (g_r(x) e^{\lambda_r x}) = g_r^{(k)}(x)
    e^{\lambda_r x} = 0. \]
  Hingegen ist für $j < r$ stets $Q(\lambda_j) \ne 0$, da $\lambda_j =
  \lambda_r$. Also existieren wegen Lemma 2 Polynomfunktionen $h_j \ne 0$ (weil
  $g_j \ne 0$), so dass
  \[ Q\left( \diff{}{x} \right) (g_j(x) e^{\lambda_j x}) = h_j(x) e^{\lambda_j
      x}. \]
  Also ist
  \[ 0 = Q\left( \diff{}{x} \right) \left( \sum_{j=1}^r g_j(x) e^{\lambda_j x}
    \right) = \sum_{j=1}^{r-1} \underbrace{h_j(x)}_{\ne 0} e^{\lambda_j x}. \]
  Aber aus der Induktionsvoraussetzung folgt, dass $h_j \equiv 0$ für alle
  $j=1,\ldots,r-1$. Widerspruch!
\end{proof}

\begin{thm}
  Sei $P(x) = (z- \lambda_1)^{k_1} \cdots (z - \lambda_r)^{k_r}$. Dann bilden
  \[ \varphi_{jm}(x) := x^m e^{\lambda_j x} \]
  für $j=1, \ldots, r$ und $m=0, \ldots,k-1$ ein LFS der GDG $P\left( \diff{}{x}
  \right) y = 0$.
\end{thm}

\begin{proof}
  Dass die $\varphi_{jm}$ linear unabhängig sind, folgt aus der Proposition.

  Es ist noch zu zeigen, dass es Lösungen sind, das heißt
  \[ P \left(  \diff{}{x} \right) \varphi_{jm}(x) = 0 \]
  für alle $j,m$.

  Aber für jedes $j$ existiert ein Polynom $Q_j$, so dass $P(z) = Q_j(z)(z -
  \lambda_j)^{k_j}$. Aber wegen Lemma 1 gilt
  \[ \left( \diff{}{x} - \lambda_j \right)^{k_j} (x^m e^{\lambda_j x}) = 0\]
  für alle $m \le k_j-1$. Also auch
  \[ P \left( \diff{}{x} \right) (x^m e^{\lambda_j x}) =
    Q_j \left( \diff{}{x} \right) \left( \diff{}{x} - \lambda_j \right)^{k_j}
    (x^m e^{\lambda_j x}) = 0. \qedhere\]
\end{proof}

\begin{exmp}
  \begin{itemize}
  \item Sei $y''=0$, also $P(x) = z^2 = (z-0)^2$. Der Satz liefert das LFS
    $\{e^{0x},x \cdot e^{0x}\} = \{ 1, x\}$. Das heißte die Lösungen von $y'' =
    0$ sind genau die affin-linearen Funktionen.
  \item Betrachte
    \[ y^{(4)} + 8 y'' + 16y = 0, \]
    also $P(z) = z^4 + 8 z^2 + 16 = (z-2i)^2(z+2i)^2$. Der Satz liefert das LFS
    \[ \{ e^{-2ix}, e^{2ix}, xe^{-2ix}, xe^{2ix} \}. \]
    Durch geeignete Linearkombination erhält man das reelle LFS
    \[ \{ \cos(2x), \sin(2x), x \cos(x), x \sin(2x) \}. \]
  \end{itemize}
\end{exmp}

\subsubsection{Lineare GDG-Systeme 1. Ordnung mit konstanten Koeffizienten}
% Muss 5.4 sein
Sei $A \in \complex^{n \times n}$, das zugehörige (homogene) GDG-System ist $y' =
Ay$.

\begin{rmrk}
  Sei $A \in \complex^{n \times n}$ und $v \in \complex^n$ ein Eigenvektor zum
  Eigenwert $\lambda \in \complex$. Dann ist $\varphi:\real \to \complex^n$
  \[ \varphi(x) := v e^{\lambda x} \]
  eine Lösung von $y' = Ay$.
\end{rmrk}

\begin{proof}
  Es ist
  \[ \varphi'(x) = \lambda v e^{\lambda x} = A v e^{\lambda x} = A \varphi(x).
    \qedhere \]
\end{proof}

Erinnerung lineare Algebra:
\begin{itemize}
  \item $A \in \complex^{n \times n}$ heißt \emph{diagonalisierbar}
    $:\Leftrightarrow$ Es existiert eine Basis von Eigenvektoren von $A$.

    Das ist äquivalent zur Existenz einer invertierbaren Matrix $S \in
    \complex^{n \times n}$, so dass $S^{-1} A S$ eine Diagonalmatrix ist.
  \item Wenn $A \in \complex^{n \times n}$ $n$ \emph{verschiedene} Eigenwerte
    hat, dann ist $A$ diagonalisierbar.
  \item Nicht jede Matrix $A \in \complex{n \times n}$ ist diagonalisierbar.
    Aber durch Ähnlichkeits\-trans\-for\-mationen kann jede Matrix auf
    \emph{Jordan-Normalform} gebracht werden. Diese besteht dann aus Blöcken der
    Form $\lambda I_k + N_k$, wobei $I_k$ die $(k \times k)$-Einheitsmatrix und
    \[ N_k =
      \begin{pmatrix}
        0 & 1 & \cdots & 0 \\
        & \ddots & \ddots \\
        & & \ddots & 1 \\
        0 & \cdots & 0 & 0
      \end{pmatrix}
      = e_1 \otimes e_2 + \ldots + e_{k-1} \otimes e_k \]
    mit den Einheitsvektoren $e_j \in \real^k$.

    $a,b \in \real^n$. Dann ist $a \otimes b$ die $n \times n$-Matrix mit
    Einträgen $(a \otimes b)_{ij} := a_i b_j$. Zum Beispiel
    \[ e_1 \otimes e_1 = \pmat{1 & 0  & \cdots & 0 \\ 0 & 0 \\ \vdots & & \ddots
        \\ 0 & & & 0}. \]
\end{itemize}

\subsubsection*{Diagonalisierbare Systeme}
\begin{thm}
  Sei $A \in \complex^{n \times n}$ diagonalisierbar mit der Basis von
  Eigenvektoren $\{ v_1, \ldots, v_n \}$ zu den Eigenwerten $\lambda_1, \ldots,
  \lambda_n \in \complex$. Dann bilden die Abbildungen $\varphi_m(x) := v_m
  e^{\lambda_m x}$ ($m=1,\ldots,n$) ein LFS der GDG $y' = Ay$.
\end{thm}

\begin{proof}
Aus der Bemerkung folgt, dass es sich um Lösungen handelt.

Die lineare Unabhängigkeit folgt aus der linearen Unabhängigkeit von $\{ v_1,
\ldots, v_n \}$.
\end{proof}

\begin{thm}[Systeme in Jordan-Block-Gestalt]
  Sei $A = \lambda I + N \in \complex^{n \times n}$. Dann bilden die folgenden
  Abbildungen $\varphi_k: \real \to \complex^n$ ein LFS von $y' = A y$:
  \[ \varphi_k(x) := \left(\sum_{i=1}^k \frac{k^{k-i}}{(k-i)!} e_i\right)
    e^{\lambda x}, \]
  wobei $(e_1, \ldots, e_n)$ die Standard-ONB von $\real^n$ ist.
\end{thm}

\begin{proof}
  Ansatz: $\varphi(x) = \psi(x) e^{\lambda x}$. Einsetzen in $\varphi' = A
  \varphi$ zeigt: $\varphi$ ist Lösung genau dann, wenn $\psi' = N\psi$.
  
  Ausgeschrieben:
  \[ \psi'_1 = \psi_2, \quad \psi'_2 = \psi_3, \quad \ldots \quad \psi'_n =
    0. \]
  Wir können das System zu einer Gleichung der Form $\varphi^{(n)}=0$ umwandeln.
  Die Lösung ist ein Polynom\footnote{%
    $(\varphi^{(n-1)})' = 0$, das heißt $\varphi^{(n-1)}=c_0$, also
    $\varphi^{(n-2)} = c_0 x + c_1$ usw.
  } vom Grad $\le n-1$. Siehe auch Beweis 2.
\end{proof}

\begin{rmrk}
  \begin{itemize}
    \item Wenn $A$ auf mehreren Jordan-Blöcken besteht, dann kann man jeden
      Block separat betrachten, weil die entsprechenden Untersysteme nicht
      miteinander gekoppelt sind.
    \item Wenn $A$ erst noch auf Jordan-Gestalt transformiert werden muss, dann
      verwendet man die folgende Tatsache.
  \end{itemize}
\end{rmrk}

\begin{rmrk}
  Sei $A \in \complex^{n \times n}$ und $S \in \complex^{n \times n}$
  invertierbar. Die Abbildung $\varphi:\real \to \complex^n$ löst $y'=Ay$ genau
  dann, wenn $\psi := S^{-1} \varphi$ das System $(S^{-1} A S)y$ löst.
\end{rmrk}

\textbf{Fazit.} Für lineare GDG mit konstanten Koeffizienten kann man das LFS explizit
angeben.

\subsubsection{Fundamentalsystem \texorpdfstring{$e^{tA}$}{exp(tA)}}
Ab jetzt bezeichnen wir die unabhängige Variable mit $t$ statt $x$, weil man
sich Abbildungen $\real \to \real^n$ oft als zeitabhängige Trajektorien
vorstellt.

Erinnerung Analysis 2, X.2.1: Für jede Matrix $A \in \complex^{n \times n}$
konvergiert die Exponentialreihe
\[ e^A := \exp(A) := \sum_{k=0}^\infty \frac{A^k}{k!} \]
absolut in $\complex^{n \times n}$.

\begin{lem}
  Sei $A \in \complex^{n \times n}$. Dann gilt
  \[ \diff{}{t} (\exp(tA)) = A \exp(tA) = \exp(tA)A. \]
\end{lem}

\begin{proof}
  Definiere $F,F_m:\real \to \complex^{n \times n}$ durch
  \[ F(t) := e^{tA}, \qquad F_m(t) := \sum_{k=0}^m \frac{tA^k}{k!}, \]
  wobei $A^0 := I_{n \times n}$.
  
  Aus Analysis 2, Satz X.2.1 folgt: $F_M$ konvergiert gleichmäßig auf
  Kompakta\footnote{%
    Das heißt auf jedem beschränkten Intervall.
  }
  gegen $F$. Da $AF_m(t) = F_m(t) A$ für alle $m \in \nat$, folgt $A e^{tA} =
  e^{tA} A$. Offenbar gilt
  \[ F'_m(t) = \sum_{k=1}^m \frac{kt^{k-1}A^k}{k!} = A \underbrace{\sum_{k=1}^m
      \frac{(tA)^k}{(k-1)!}}_{F_{m-1}(t)} \to e^{tA}. \]
  Also konvergiert $F'_m$ gleichmäßig auf Kompakta gegen $AF$.

  Wegen $F_m \to F$ gleichmäßig auf Kompakta, folgt also aus Analysis 2 Satz
  VII.3.5, dass $F' = AF$.
\end{proof}

\begin{thm}
  Sei $A \in \complex^{n \times n}$ und setze $\Phi(t) := e^{tA}$. Dann ist
  $\Phi$ (das heißt die Spalten von $\Phi$) ein LFS des GDG-Systems $y' = Ay$.
\end{thm}

\begin{proof}
  Das Lemma besagt, dass $\Phi$ (das heißt jede Spalte $t \mapsto \Phi(t) e_k$)
  eine Lösung ist.

  Noch zu zeigen: Lineare Unabhängigkeit.

  Wegen Satz XIII.4.3 müssen wir die Unabhängigkeit nur an der Stelle $t=0$
  verifizieren. Aber $e^{0A} = I$.
\end{proof}

\subsubsection{Eigenschaften von  \texorpdfstring{$e^A$}{exp(A)}}
\begin{thm}
  Seien $A, B \in C^{n \times n}$. Dann gilt
  \begin{enumerate}
  \item Wenn $AB=BA$, dann ist $e^{A+B} = e^B e^A$.
  \item Wenn $\det B \ne 0$, dann $e^{B^{-1}AB}= B^{-1}e^A B$.
  \item $e^{\diag(\lambda_1, \ldots, \lambda_n)} = \diag(e^{\lambda_1},
    \ldots, e^{\lambda_n})$.
  \end{enumerate}
  Insbesondere:
  \begin{enumerate}[(i)]
  \item $(e^A)^{-1}= e^{-A}$.
  \item $e^{(s+t)A}= e^{sA}e^{tA}$.
  \item $e^{A+\lambda I}= e^\lambda e^A$.
  \end{enumerate}
\end{thm}
Hier bezeichnet
\[ \diag(\lambda_1, \ldots, \lambda_n) := \sum_{i=1}^n \lambda_i e_i \otimes
  e_i = \pmat{ \lambda_1 & & 0 \\ & \ddots \\ 0 & & \lambda_n}. \]

\begin{proof}
  2. folgt aus $(B^{-1}AB)^k = (B^{-1}A^kB)$, was leicht induktiv gezeigt wird.

  3. folgt aus $(\diag(\lambda_1, \ldots, \lambda_n))^k = \diag(\lambda_1^k,
  \ldots, \lambda_n^k)$ für alle $k \in \nat$.

  1. folgt wie in Fall $n=1$ (Cauchy-Produkt, ...).
\end{proof}

\begin{proof}[Zweiter Beweis zu Satz 5.4]
  Sei $A = \lambda I + N$ wie in der Voraussetzung. Das LFS ist
  \[ e^{tA} = e^{t\lambda I + tN} = e^{\lambda t} e^{tN}. \]
  Da aber $N^n = 0$ (also $N$ nilpotent), besteht die Exponentialreihe
  aus nur endlich vielen Summanden:
  \[ \exp(tN) = \sum_{k=1}^{n-1} \frac{(tN^k)}{k!}. \]
  Ausrechnen ergibt:
  \[ \exp(tN) = \begin{pmatrix}
      1 & t & \frac{t^2}{2!} & \cdots & \frac{t^{(n-1)}}{(n-1)!} \\
      0 & 1 & \ddots\\
      \vdots & 0 & \ddots & \ddots \\
      \vdots & \vdots  & & \ddots \\
      0 & 0 & 0 & \cdots & 1
    \end{pmatrix} \]
  Damit folgt die Behauptung des Satzes.
\end{proof}

\subsubsection{Inhomogene Systeme}
Sei $A \in \complex^{n \times n}$ und $b: \real \to \complex^n$ stetig. Lösungen
von $y' = Ay + b(x)$ sind dann gegeben durch die Formel:
\[ \varphi(t) = e^{tA} c + \int_0^F e^{t-s}A b(s) \diffop s. \]

\subsubsection{Beispiel}
Sei $A \in \real^{2 \times 2}$, betrachte $y' = Ay$. Die Eigenwerte von $A$
lösen $\det(A-\lambda I) = 0$, das heißt
\[ \lambda = \frac{\operatorname{Tr} A}{2} \pm \sqrt{ \underbrace{\left(
        \frac{\operatorname{Tr} A}{2} \right)^2 - \det A}_{=: D}}. \]
Fälle:
\begin{enumerate}
\item $D=0$. Dann ist $\lambda = \frac{\operatorname{Tr} A}{2} \in \real$
  Nullstelle mit Vielfachheit 2. Es gibt zwei Fälle:
  \begin{enumerate}[a)]
  \item Es existieren zwei linear unabhängige Eigenvektoren zu $\lambda$. Also
    ist $A$ diagonalisierbar.
  \item $\dim \ker (A-\lambda I) = 1$. Dann existiert $S \in \real^{2 \times 2}$
    invertierbar, sodass $S^{-1}AS= \pmat{\lambda & 1 \\ 0 & \lambda}$.
  \end{enumerate}
\item $D \ne 0$. Dann ist $A$ diagonalisierbar, da zwei verschiedene Eigenwerte
  existieren.
  \begin{enumerate}[a)]
  \item $D > 0$. Dann sind die Eigenwerte und Eigenvektor reell.
  \item $D < 0$. Dann sind die Eigenwerte $\in \complex \setminus \real$ und
    die Eigenvektoren ebenfalls nicht reell.
  \end{enumerate}  
\end{enumerate}

Siehe auch Forster Analysis 2.


\clearpage

\subsection*{Weitere Vorlesungen}
\begin{itemize}
\item Methoden der Analysis (Di. 5. DS, Mi 3. DS), Hornung/Jachan,
  ``Geometrische Analysis'' 
\item Wissenschafltiches Arbeiten, Ausblicke aus der Riemannschen Geometrie (Do
  2. DS, Fr 4. DS), Hardering
\item Vorlesungen aus dem Masterstudium, bauen aufeinander auf.
\end{itemize}

\subsubsection*{Klausurschwerpunkte}
\begin{itemize}
  \item Majorierte Konvergenz
  \item Maße
  \item Untermannigfaltigkeiten (nur Rechnen)
  \item Divergenzsatz
  \item Immersionen
  \item Differentialgleichungen
  \item Explizite Lösungen mit Trennung der Variablen
  \item Existenztheorie (rechte Seite lokal Lipschitz-stetig)
  \item Rechte Seite global Lipschitz
  \item konstante Koeffizienten
  \item Aussagen über Existenz eines LFS, Voraussetzungen für genauere Lösung
\end{itemize}
  
Formelsamlung: 1 A4-Blatt
\end{document}
