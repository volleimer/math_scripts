\section{Aufgaben}
\begin{aufg}
  Zeigen Sie, dass die Familie $\mB^*$ aus Lemma 1.1.8 eine Algebra ist.
\end{aufg}

Wir rechnen die Eigenschaften der Algebra nach:
\begin{enumerate}[(i)]
\item $\Omega \in \mB^*$. Da $\mB$ eine Algebra ist, gilt $\Omega \in \mB$.
  Betrachte die Folge $(A_n)$, $A_i = \Omega$ für alle $i$. Dann gilt
  \[ \lim_{n \to \infty} \pP( A_n \Delta \Omega ) = \pP( \Omega \Delta \Omega)
    = 0.\]
\item $A \in \mB^* \Rightarrow \obar{A} \in \mB^*$. Sei $A \in \mB^*$, dann
  existiert eine Folge $(A_n) \subset \mB_0$, sodass
  \[ \lim_{n \to \infty} \pP( A_n \Delta A ) = 0, \qquad \lim_{n \to \infty}
    \pP( A_n \setminus A ) =  0, \pP( A_n \setminus A ) =  0. \]
  Daraus folgt
  \[ \lim_{n \to \infty} \pP( \obar{A} \setminus \obar{A}_n ) =
    \lim_{n \to \infty} \pP( \obar{A} \cap A_n ) =
    \lim_{n \to \infty} \pP( A_n \setminus A ) = 0 \]
  sowie
  \[ \lim_{n \to \infty} \pP( \obar{A}_n \setminus \obar{A} ) =
    \lim_{n \to \infty} \pP( \obar{A}_n \cap A ) =
    \lim_{n \to \infty} \pP( A \setminus A_n ) = 0, \]
  also gilt
  \[ \lim_{n \to \infty} \pP( \obar{A}_n \Delta \obar{A} ) = 0. \]
  Somit ist $\obar{A} \in \mB^*$.
\item Sei $(A_n)_{n=1}^N \subset \mB^*$. Die zugehörigen Folgen seien
  $(A_{n,i})$. Es gilt
  \begin{align*}
    \lim_{i \to \infty} \pP \left(
      \left( \bigcup_{n=1}^N A_n \right) \Delta
      \left( \bigcup_{n=1}^N A_{n,i} \right)
    \right)
    &=
    \lim_{i \to \infty} \pP \left(
      \left( \bigcap_{n=1}^N \obar{A}_n \right) \Delta
      \left( \bigcap_{n=1}^N \obar{A}_{n,i} \right)
      \right) \\
    \intertext{mit dem Hinweis $(A \cap B) \Delta (C \cap D) \subset (A \Delta B) \cup (C \Delta D)$:}
    &\le
    \lim_{i \to \infty} \pP \left( \bigcup_{n=1}^N
      \left( \obar{A}_n \Delta
      \obar{A}_{n,i} \right)
      \right) \\
    \intertext{und wegen der Subadditivität von Maßen:}
    &\le
    \lim_{i \to \infty} \sum_{n=1}^N \pP \left(
      \obar{A}_n \Delta
      \obar{A}_{n,i}
      \right) \\
    &=
     \sum_{n=1}^N \lim_{i \to \infty} \pP \left(
      \obar{A}_n \Delta
      \obar{A}_{n,i}
      \right) = 0.
  \end{align*}
  Also ist $\bigcup_1^N A_n \in \mB^*$ und damit ist $\mB^*$ eine Algebra.
\end{enumerate}

\begin{aufg}
  Beweisen Sie Satz 1.1.9.

  [Hinweis: Man benutze Lemma 1.1.8.]
\end{aufg}

Zwei Ereignisse $A,B$ sind unabhängig, wenn
\[ \pP(A \cap B) = \pP(A) \cdot \pP(B). \]

Zu zeigen: Die von unabhängigen Algebren $\mA_0, \mB_0$ erzeugten
$\sigma$-Algebren $\mA, \mB$ sind unabhängig.

\begin{proof}
  Für alle $A_0 \in \mA_0$, $B_0 \in \mB_0$ sind $A_0$ und $B_0$ unabhängig.

  Seien $A \in \mA$, $B \in \mB$. Wähle zwei Folgen $(A_n) \subset \mA_0$ und
  $(B_n) \subset \mB_0$ mit
  \[ \lim_{n \to \infty} \pP( A_n \Delta A) = 0, \qquad \lim_{n \to \infty} \pP(
    B_n \Delta B) = 0 \]
  wie in Lemma 1.1.8. Dann gilt auch
  \[ \lim_{n \to \infty} \pP( (A_n \cap B_n) \Delta (A \cap B)) = 0, \]
  weil
  \[ (A_n \cap B_n) \Delta (A \cap B) \subset (A_n \Delta A) \cup (B_n \Delta B). \]

  Nun folgt wegen Lemma 1.1.8(ii) und $\pP(A_n \cap B_n) = \pP(A_n) \cdot \pP(B_n)$:
  \[ \begin{aligned}
      \lim_{n \to \infty} \pP( (A_n \cap B_n) ) &= \pP ( (A \cap B ) ) \\
      \lim_{n \to \infty} \pP(A_n) \cdot \pP(B_n) &= \pP ( (A \cap B ) ) \\
      \pP(A) \cdot \pP(B) &= \pP ( (A \cap B ) )
    \end{aligned} \]
  und damit sind $A$ und $B$ unabhängig.
\end{proof}

\begin{aufg}[Poincaré]
  Man beweise die folgende Verallgemeinerung von Satz 1.1.4 (iii):
  \[ \pP( A_1 \cup \cdots \cup A_n ) = \sum_{k=1}^n (-1)^{k+1} \sum_{1 \le i_1 <
      \cdots < i_k \le n} \pP( A_{i_1} \cap A_{i_2} \cap \cdots \cap
    A_{i_k}). \]
  
  [Hinweis: Induktion nach n.]
\end{aufg}

\begin{proof}
  Induktionsanfang: Sei $n=2$, dann wissen wir aus Satz  1.1.4(iii):
  \[ \pP ( A_1 \cup A_2 ) = \pP( A_1 ) + \pP( A_2 ) - \pP( A_1 \cap A_2 ). \]
  Also gilt die Behauptung für $n=2$.

  Angenommen, die Behauptung gilt für ein $n$. Definiere das Ereignis $B := A_1
  \cup \cdots \cup A_n$. Dann ist
  \[ \pP( B \cup A_{n+1} ) = \pP( B ) + \pP( A_{n+1} ) - \pP( B \cap A_{n+1}
    ). \]
  Wegen
  \[ B \cap A_{n+1} = \left(\bigcup_{j=1}^n A_j\right) \cap A_{n+1} = \bigcap_{j=1}^n (A_i \cap
    A_{n+1}) \]
  folgt
  \[ \pP( B \cap A_{n+1} ) = \sum_{k=1}^n (-1)^{k+1} \sum_{1 \le i_1 < \cdots <
      i_k \le n} \pP( A_{i_1} \cap A_{i_2} \cap \cdots \cap A_{i_k} \cap A_{n+1}
    ) \]
  nach Induktionsvoraussetzung. Damit folgt nun
  \begin{align*}
    \pP( B \cup A_{n+1} ) =
    &\phantom{+} \sum_{k=1}^n (-1)^{k+1} \sum_{1 \le i_1 < \cdots < i_k \le n}
      \pP( A_{i_1} \cap A_{i_2} \cap \cdots \cap A_{i_k}) \\
    &+ \pP( A_{n+1} ) \\
    &- \sum_{k=1}^n (-1)^{k+1} \sum_{1 \le i_1 < \cdots < i_k \le n}
      \pP( A_{i_1} \cap A_{i_2} \cap \cdots \cap A_{i_k} \cap A_{n+1} ) \\
    \pP( B \cup A_{n+1} ) =
    &\phantom{+} \sum_{k=1}^{n+1} (-1)^{k+1}
      \sum_{1 \le i_1 < \cdots < i_k \le n+1} \pP( A_{i_1} \cap A_{i_2} \cap \cdots \cap A_{i_k})
  \end{align*}
  und die Behauptung gilt auch für $n+1$.
\end{proof}

\begin{aufg}
  Wie groß ist die Wahrscheinlichkeit, dass eine Ziehung beim Spiel ``6 aus 49''
  keine benachbarten Zahlen enthält? (Benachbart sind z.B. 2 und 3; 11 und 12).
\end{aufg}

Betrachte eine Lotterie mit den Zahlen 1 bis $n$, $\Omega_n := \{ 1, 2, \ldots,
n \}$. Werden $k$ Zahlen gezogen und der Größe nach geordnet, so ist die Menge
der gültigen Ereignisse
\[ \mA_k = \{ (x_1, x_2, \ldots, x_k) : x_1 < x_2 < \cdots < x_k, x_i \in \Omega
  \}. \]

Es gilt $|\mA_k| = \binom{n}{k}$. Für eine Lotterie 6 aus 49 gibt es also
$\binom{49}{6}$ mögliche Ereignisse.

Die Funktion $f: \Omega_{44} \to \Omega_{49}$ sei definiert durch
\[ f({x_1, \ldots, x_6}) = \{x_1, x_2 + 1, x_3 + 2, x_4 + 3, x_5 + 4, x_6 +5
  \}. \]
Diese Funktion ist offenbar bijektiv. Außerdem sind die Urbilder aller
Ziehungen, die keine aufeinander folgenden Zahlen enthalten, gültige Ereignisse
in der ``Urlotterie''. Also ist die Menge der gewünschten Ziehungen isomorph zur
Menge der gültigen Ereignisse $\mA_{44}$. Damit können wir die
Wahrscheinlichkeit bestimmen, dass eine Ziehung keine benachbarten Zahlen
enthält:
\[ \pP( \text{Keine benachbarten Zahlen} ) = \frac{\binom{44}{6}}{\binom{49}{6}}
  = \frac{7059052}{13983816} \approx 0.5048. \] 

%%% Local Variables:
%%% TeX-master: "master"
%%% End:
