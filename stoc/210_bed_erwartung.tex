\section{Bedingte Erwartung}
In diesem Abschnitt: $(\Omega, \mA, \pP)$ ein $\pW$-Raum.

Ist $\mB \subset \mA$ eine $\sigma$-Algebra, so bezeichnet $\pP|_\mB$ die
Einschränkung ovn $\pP$ auf $\mB$.

\begin{prgp}[Aufgabe]
  Sei $\mB \subset \mA$ eine $\sigma$-Algebra. Eine $\mB$-messbare Zufallsgröße
  $X$ (reell oder komplex) ist genau dann $\pP$-integrierbar, wenn sie
  $\pP|_\mB$-integrierbar ist. Im Fall der Integrierbarkeit gilt
  \[ \int_B X \diffop \pP = \int_B X \diffop \pP|_\mB, \quad B \in \mB. \]
\end{prgp}

\begin{thm}
  Sei $\mB \subset \mA$ eine $\sigma$-Algebra und $X$ eine $\pP$-integrierbare
  Zufallsgröße. Dann existiert eine $\mB$-messbare $\pP$-integrierbare
  Zufallsgröße $Z$ mit
  \[ \int_B X \diffop \pP = \int_B Z \diffop \pP|_\mB,  \quad B \in \mB.
    \tag{1} \]
  Diese Zufallsgröße ist bis auf $\pP$-fast sichere Gleichheit eindeutig
  bestimmt.
\end{thm}

\begin{proof}
  Wir definieren das Maß $\mu$ auf $\mB$ durch
  \[ \mu( B) := \int_B X \diffop \pP, \quad B \in \mB. \]
  Ist $B$ eine $\pP|_\mB$-Nullmenge, so ist sie auch eine $\pP$-Nullmenge und
  folglich ist $\mu(B) = 0$. Also ist $\mu$ absolut stetig bezüglich $\pP|_\mB$.

  Die Existenz und Eindeutigkeit von $Z$ folgt aus dem Satz von Radon-Nikodim
  (und aus (2.10.1)).
\end{proof}

\begin{defn}
  Die Zufallsgröße in 2.10.2 heißt \emph{bedingte Erwartung} von $X$ \emph{unter
    der Bedingung} oder \emph{bezüglich} $\mB$ und wird mit $\pE(X|\mB)$ oder
  $\pE_\mB(X)$ oder $\pE^\mB(X)$ bezeichnet.

  Falls $\mB = \sigma( Y_1, \ldots, Y_n)$, wobei die $Y_j$ Zufallsvariablen
  sind, schreiben wir auch $\pE(X|_{Y_1, \ldots, Y_n})$.

  Ist $X = \ind_A$, $A \in \mA$, so heißt $\pE(X|\mB)$ die \emph{bedingte
    Wahrscheinlichkeit} von $A$ unter der Bedingung oder bezüglich $\mB$ und
  wird mit $\pP(A|\mB)$ oder $\pP_\mB(A)$ oder $\pP^\mB(A)$ bezeichnet.
\end{defn}

\begin{rmrk}
  \begin{enumerate}[(a)]
  \item Zwei Extremfälle für $\mB$: $\mB = \mA$ und $\mB = \{ \emptyset, \Omega
    \}$.

    Wenn $\mB = \mA$: $\pE_\mB X = X$.

    Wenn $\mB = \{ \emptyset, \Omega \}$: $\pE_\mB X = \pE X$.
  \item Extremfall für $X$: $X$ ist $\mB$-messbar, dann ist $\pE_\mB X = X$.
  \item Extremfall in (1) mit $B = \Omega$: Die Zufallsgröße $X$ und $pE_\mB(X)$
    haben denselben Erwartungswert, $\pE X = \pE \pE_\mB X$.
  \item $X = Y$ fast sicher $\Rightarrow$ $\pE(X|\mB) = \pE(Y|\mB)$ fast sicher.
  \end{enumerate}
\end{rmrk}

\begin{prgp}[Aufgabe]
  \begin{enumerate}[(a)]
  \item Wird $\mB$ von disjunkten Mengen $B_1, B_2, \ldots \in \mA$ mit
    $\bigcup_n B_n = \Omega$ erzeugt, so ist jede $\mB$-messbare Zufallsgröße
    konstant auf den Mengen $B_n$. Beschreiben Sie die $\sigma$-Algebra $\mB$.
  \item Sei $\mB$ wie in (a), dann gilt (fast sicher):
    \[ \pP(A|\mB) = \pE( \ind_A | \mB ) = \sum_{n : \pP(B_n) > 0} \pP(A | B_n)
      \cdot \ind_{B_n} \]
    und
    \[ \pE(X|\mB) = \sum_{} \pE_{B_n} X \cdot \ind_{B_n},\]
    wobei $X$ eine integrierbare Zufallsgröße ist und
    \[ \pE_{B_n}(X) = \rez{\pP(B_n)} \int_{B_n} X \diffop \pP. \]

    Spezialfall: $B_1 = B, B_2 = B^C$.
    
    Hinweis: Verwenden Sie (a).
  \item  Für jedes Ereignis $A \in \mA$ gilt
    \[ \int_B \pP( A | \mB ) \diffop \pP = \pP(A \cap B), \quad B \in \mB. \]
    Hinweis:
    \[ \int_B \ind_A \diffop \pP = \pP(A \cap B), \]
    siehe 2.10.2.1.
  \end{enumerate}
\end{prgp}

\begin{thm}
  Sei $\mB \subset \mA$ eine $\sigma$-Algebra, sowie $X$ und $Y$ integrierbare
  Zufallsgrößen auf $(\Omega, \mA, \pP)$. Die folgenden Eigenschaften bestehen
  fast sicher:
  \begin{enumerate}[(i)]
  \item $\pE_\mB( aX + bY ) = a \cdot \pE_\mB(X) + b \cdot \pE_\mB(Y)$, $a,b \in
    \real$.
  \item Wenn $X \ge Y$, dann auch $\pE_\mB(X) \ge \pE_\mB(Y)$; speziell gilt
    $\pE_\mB(x) \ge 0$, wenn $X \ge 0$.
  \item Ist $X$ $\mB$-messbar, so gilt $\pE_\mB(X) = X$; speziell gilt
    \[ \pE_\mB( a \cdot \ind_\Omega) = a \cdot \ind_\Omega \quad \text{und}
      \quad \pE_\mB[ \pE_\mB(X) ] = \pE_\mB(X). \]
  \item $|\pE_\mB(X)| \le \pE_\mB(|X|)$ und $\|\pE_\mB(X)\|_1 \le \|X\|_1.\footnotemark$
  \end{enumerate}
\end{thm}
\footnotetext{Das heißt, die Abbildung $X \to \pE_\mB(X)$ ist ein linearer
  Operator von $L^1(\pP)$ in $L^1(\pP)$. Dieser Operator ist nichtnegativ,
  kontraktiv und idempotent, die $\mB$-messbaren Zufallsgrößen sind Fixpunkte.}

\begin{proof}
  (i) bis (iii) folgen sofort aus der Definition des bedingten Erwartungswertes.

  Kontraktivität: Linear und nichtnegativ.
  \[ - |X| \le X \le |X| \qRq |\pE_\mB(X)| \le \pE(|X|) \]
  Aus der Definition von $\pE_\mB$ (D) folgt
  \begin{align*}
  \|X\|_1 &= \int_\Omega |X| \diffop \pP \overset{\text{(D)}}{=} \int_\Omega
            \pE_\mB(|X|) \diffop \pP \\
          &\ge \int_\Omega |\pE_\mB(X)| \diffop \pP|_\mB = \| \pE_\mB(X) \|_1. \qedhere
  \end{align*}
\end{proof}

\begin{exmp*}
  Sei $(\Omega, \mA, \pP) := ([0,1], \borel([0,1]), \lambda)$, $X(\omega) :=
  \omega$, $\omega \in [0,1]$.
  \begin{itemize}
  \item $\mB_1 := \{ \emptyset, \Omega \}$
  \item $\mB_2 := \{ \emptyset, [0,1/2], (1/2,1], [0,1] \}$
  \end{itemize}
\end{exmp*}

\begin{thm}
  Seien $\mB, \mB_1, \mB_2$ Teil-$\sigma$-Algebren von $\mA$, sowie $X$, $X_n$
  und $Y$ integrierbare Zufallsgrößen sowie $Z$ eine beschränkte Zufallsgröße
  auf $(\Omega, \mA, \pP)$. Die folgenden Eigenschaften bestehen fast sicher:
  \begin{enumerate}[(i)]
  \item Ist $Z$ messbar bezüglich $\mB$, so gilt
    \[ \pE_\mB( X \cdot Z ) = \pE_\mB(X) \cdot Z, \]
    speziell
    \[ \pE_\mB( X \cdot \pE_\mB(Y)) = \pE_\mB(X) \cdot \pE_\mB(Y), \]
    falls $\pE_\mB(Y)$ beschränkt ist.
  \item Ist $\mB_1 \subset \mB_2$, so gilt
    \[ \pE_{\mB_1}( \pE_{\mB_2} (X) ) = \pE_{\mB_2}( \pE_{\mB_1}(X) ) =
      \pE_{\mB_1}(X). \]
  \item Sind $B$ und $\sigma(X)$ unabhängig, so gilt
    \[ \pE_\mB(X) = \pE(X). \]
  \item Gilt $X_n \to X$ und $|X_n| \le Y$ für alle $n$, dann folgt
    \[ \pE_\mB(X_n) \to \pE_\mB(X). \]
  \end{enumerate}
\end{thm}

\begin{proof}
  (i): Es gilt für $B \in \mB$:
  \[ \int_B \pE_\mB(X \cdot Z) \diffop \pP|_\mB = \int_B X \cdot Z \diffop \pP
    = \int_B Z \cdot \pE_\mB(X) \diffop \pP|_\mB. \]
  Spezialfall: $Z = \ind_A$, $A \in \mB$. Die zweite Gleichung hat die Form
  \[ \int_{A \cap B} X \diffop \pP = \int_{A \cap B} \pE_\mB(X) \diffop
    \pP|_\mB \]
  und sie ist wegen $A \cap B \in \mB$ richtig.

  Der allgemeine Fall folgt durch Approximation von $Z$ durch Treppenfunktionen.

  (ii): Die Gleichung $\pE_{\mB_2}( \pE_{\mB_1}(X) ) = \pE_{\mB_1}(X)$ folgt aus
  (2.10.6.iii), da $\pE_{\mB_1}(X)$ messbar ist bezüglich $\mB_2$. Für $B \in
  \mB_1 \subset \mB_2$ gilt:
  \[ \int_B \pE_{\mB_1}(\pE_{\mB_2}(X)) \diffop \pP =
    \int_B \pE_{\mB_2}(X) \diffop \pP = \int_B X \diffop \pP = \int_B
    \pE_{\mB_1}(X) \diffop \pP. \]
  Der erste und der letzte Integrand sind $\mB_1$-messbar, ihre Integrale sind
  für alle $B \in \mB_1$ gleich, also sind sie fast sicher gleich.

  (iii): Annahme, $X$ und $\ind_B$ sind unabhängige Zufallsgrößen, dann gilt
  \begin{align*}
    \int_B \pE(X) \diffop \pP
    &= \pE(X) \cdot \pP(B) = \pE(X) \cdot \pE(\ind_B) \\
    &= \pE(X \cdot \ind_B) = \int_B X \diffop \pP = \int_B \pE_\mB(X) \diffop \pP,
  \end{align*}
  also sind die Integranden gleich.

  (iv): Nach (2.10.6.iv) ist
  \[ |\pE_\mB(X_n) - \pE_\mB(X)| \le \pE_\mB(|X_n - X|) \le \pE_\mB(Y_n), \]
  wobei $Y_n := \sup_{k \ge n} |X_n - X|$.

  Wir zeigen, dass die rechte Seite
  $\xrightarrow{n \to \infty} 0$. Es gilt $0 \le Y_n \le 2Y$ und $Y_n\to 0$. Die
  Folge $\{ Y_n \}$ und damit auch $\{\pE_\mB(Y_n)\}$ sind monoton fallend. Also
  \[ 0 \le \int \lim_n \pE_\mB(Y_n) \diffop \pP = \lim_n \int \pE_\mB(Y_n)
    \diffop \pP = \lim_n \int Y_n \diffop \pP = \int \lim_n Y_n \diffop \pP =
    0. \qedhere \]
\end{proof}

\begin{thm}[Faktorisierungslemma]
  Sei $T: \Omega \to \Omega'$ eine Abbildung in einem Messraum $\Omega', \mA')$
  und sei $X : \Omega \to \realext$ eine Zufallsgröße. Dann ist $X$ genau
  $\sigma(T)$-messbar, wenn es eine Zufallsgröße $Y : \Omega' \to \realext$ gibt
  mit $X = Y(T)$.
\end{thm}

\begin{proof}
  Siehe Bauer, 1990, S. 71.
\end{proof}

\begin{folg}
  Für beliebige Zufallsgrößen $X, Y_1, \ldots, Y_n$ existiert eine
  Borel-messbare Funktion $g : \real^n \to \real$, sodass
  \[ \pE( X | Y_1, \ldots, Y_n) = g(Y_1, \ldots, Y_n) \]
  fast sicher.
\end{folg}

\clearpage

\begin{thm}
  Es seien $X$ eine integrierbare Zufallsgröße, $Y$ eine beliebige Zufallsgröße
  und $g$ die nach (2.10.9) existierende, Borel-messbare Funktion mit
  \[ \pE( X | Y )(\omega) = g(Y(\omega)) \quad \text{fast sicher.} \tag{1} \]
  Dann ist $g$ bezüglich der Verteilung $\mu_Y$ von $Y$ integrierbar und genügt
  der Gleichung
  \[ \int_B g \diffop \mu_Y = \int_{[Y \in B]} X \diffop \pP \tag{2} \]
  für jede Borel-Menge $B \subset \real$. Sie ist hierdurch $\mu_Y$-fast sicher
  bestimmt.

  Ist umgekehrt $g$ eine reelle, Borel-messbare, $\mu_Y$-integrierbare Funktion
  mit (2), so gilt (1).
\end{thm}

\begin{proof}
  Wegen (1) ist die Funktion $g(Y)$ integrierbar. Nach dem Transformationssatz
  gilt
  \[ \int_B g \diffop \mu_Y = \int_{[Y \in B]} g(Y) \diffop \pP, \quad B \in
    \borel(\real). \tag{$\circ$} \]
  Aus (1) und $[Y \in B] \in \sigma(Y)$ folgt, dass
  \[ \int_{[Y \in B]} g(Y) \diffop \pP = \int_{[Y \in B]} X \diffop \pP \]
  und damit (2).

  Die Eindeutigkeit folgt daraus, dass die Dichtefunktion bezüglich $\mu_Y$ des
  Maßes $\nu := g \diffop \mu_Y$ $\mu_Y$-fast sicher bestimmt.

  Gilt nun (2) mit eine Funktion $g$, so folgt aus ($\circ$), dass
  \[ \int_{[Y \in B]} X \diffop \pP = \int_{[Y \in B]} g(Y) \diffop \pP \]
  und damit (1).
\end{proof}

\begin{prgp}[Aufgabe]
  Für die Umkehrung im vorhergehenden Satz genügt es zu fordern, dass
  \[ \int_{-\infty}^x g \diffop \mu_Y = \int_[Y \le X] X \diffop \pP, \quad x
    \in \real. \]
\end{prgp}

\begin{defn}
  Seien $X$, $Y$ und $g$ wie in (2.10.10). Für jedes $y \in \real$ heißt $g(y)$
  die \emph{bedingte Erwartung von $X$} unter der Hypothese, dass $Y$ gleich $y$
  ist, in Zeichen
  \[ \pE(X | Y = y) := g(y), \quad y \in \real. \]
\end{defn}

\begin{rmrk*}
  $\pE(X|Y)$ ist eine Zufallsgröße, $\pE(X|Y=y)$ ist eine Zahl.
\end{rmrk*}

Aus der Definition folgt:
\[ \pE(X|Y) = \pE(X | Y = Y(\omega)) \quad (\text{fast sicher}). \]

In einem für viele Anwendungen wichtigen Spezialfall kann man die Funktion $g$
und damit $\pE(X|Y = y)$ und $\pE(X|Y)$ relativ einfach berechnen.

\begin{thm}
  Es seien $X$ und $Y$ Zufallsgrößen, deren gemeinsame Verteilung $\mu_{X,Y}$
  eine Dichte $p$ besitzt. Ferner sei $X$ integrierbar und für alle $y \in
  \real$ sei die Dichte von $Y$
  \[ p_Y (y) := \int_{-\infty}^\infty p(x,y) \diffop x > 0. \]

  Dann gilt für die Funktion $g(y) = \pE(X|Y=y)$
  \[ \pE(X|Y=y) = \rez{p_Y(y)} \int_{-\infty}^\infty x \cdot p(x,y) \diffop x
    \tag{1} \]
  für $\mu_Y$-fast alle $y \in \real$.

  Ferner gilt
  \[ \pE(X|Y)(\omega) = \rez{p_Y(Y(\omega))} \cdot \int_{-\infty}^\infty x
    \cdot p(x,Y(\omega)) \diffop x \tag{2}\]
  $p$-fast sicher.
\end{thm}

\begin{rmrk*}
  $p(x|y) = \frac{p(x,y)}{p_Y(y)}$ heißt die \emph{bedingte Dichtefunktion unter
    der Bedingung $Y=y$}. Aus (1) folgt:
  \[ \pE( X | Y = y ) = \int_{-\infty}^\infty x \cdot p( x | y ) \diffop x \]
  $\mu_Y$-fast sicher.
\end{rmrk*}
