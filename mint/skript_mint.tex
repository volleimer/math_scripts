\documentclass[
 a4paper,
 12pt,
 parskip=half
 ]{scrreprt}

\usepackage{../.tex/settings}

\usepackage{../.tex/mathpkgs}
\usepackage{../.tex/mathcmds}

\theoremstyle{plain}
\newtheorem{thm}{Satz}[section] % reset theorem numbering for each chapter

\theoremstyle{definition}
\newtheorem{defn}[thm]{Definition} % definition numbers are dependent on theorem numbers
\newtheorem{exmp}[thm]{} % same for example numbers

\numberwithin{equation}{section}

%opening
\title{Vorlesung\\Maß und Integral}
\subtitle{Wintersemester 2016/2017}
\author{Vorlesung: Prof. Dr. Zoltán Sasvári\\Mitschrift: Jonas Hippold}

\begin{document}

\maketitle

\tableofcontents

\section*{Organisatorisches}
 \begin{itemize}
  \item \emph{Aufgaben:} Auf der Vorlesungsseite
  \item \emph{Prüfungsvorleistung:} 50 \% der Punkte in den Pflichtaufgaben und eine vorgetragene Übungslösung in der Übung
  \item \emph{Abgabe der Pflichtaufgaben:} Dienstag, 9:20 Uhr, Briefkasten 
  \item \emph{Schriftliche Prüfung:} Definitionen, Sätze, Beweise \\
   Aufgaben: ähnlich wie in den Übungen
  \item \emph{Literatur:}
   \begin{itemize}
    \item G. B. Folland: Real Analysis
    \item H. Bauer: Wahrscheinlichkeitstheorie und Grundzüge der Maßtheorie
    \item R. L. Schilling: Maß und Integral (2015)
   \end{itemize}
 \end{itemize}

\chapter{Maße}
\section{Einführung}
\begin{exmp}
 Eine sehr alte Aufgabe der Geometrie: Man bestimme Länge, Flächeninhalt, Volumen von gewissen Gebieten, Körpern in $\real$, in der Ebene, im Raum.
 
 Mit Hilfe des Riemann-Integrals kann man dieses Problem für \emph{gewisse} Mengen in $\real^2$ und $\real^3$ lösen: Mengen, die durch glatte Kurven oder Flächen berandet sind.
 
 Mit diesem Zugang kann man aber kompliziertere Mengen nicht behandeln.
\end{exmp}
 
\subsection*{Axiomatischer Zugang}
Sei $d \in \nat$ beliebig. 

Anstelle von Länge, Flächeninhalt oder Volumen sprechen wir vom \emph{Maß} einer Menge.

Welche Eigenschaften sollte ein Maß haben?

Ein Maß $\mu$ soll \emph{jeder Menge} $E \subset \real^d$ eine Zahl $\mu(E) \in [0, \infty]$ zuordnen, so dass
\begin{enumerate}[(i)]
 \item Ist $E_1, E_2, \ldots$ eine endliche oder unendliche Folge von \emph{disjunkten} Mengen $E_j \subset \real^d$, so gilt:
 \[ \mu(E_1 \cup E_2 \cup \ldots ) = \mu(E_1) + \mu(E_2) + \ldots \]
 \item Sind $E$ und $F$ \emph{kongruent}\footnote{$F$ entsteht aus $E$ mit Hilfe von Verschiebung, Spiegelung oder Rotation}, so ist $\mu(E) = \mu(F)$.
 \item Der $d$-dimensionale Einheitswürfel soll das Maß 1 haben:
  \[ \mu([0,1]^d) = 1. \]
\end{enumerate}

Wir zeigen, dass ein solches Maß nicht existiert! Wir betrachten nur den Fall $d=1$, der allgemeine Fall lässt sich analog behandeln.

Wir definieren eine Äquivalenzrelation $\sim$ auf $[0,1)$ durch
\[ x \sim y \qRq x-y \in \rat. \]
Sei $N$ eine Teilmenge von $[0,1)$, die genau ein Element von jeder Äquivalenzklasse enthält. Sei $R := \rat \cap [0,1)$ und für jedes $r \in R$ sei 
\[ N_r := \{ x + r: x \in N \cap [0,1-r) \} \cup \{ x + r - 1: x \in N \cap [1-r,1) \}. \]
Um $N_r$ zu erhalten, wird $N$ um $r$ nach rechts verschoben; der Teil, der dabei $[0,1)$ verlässt, wird um 1 nach links verschoben.

Wir zeigen:
\[ [0,1) = \bigcup_{r \in R} N_r \text{ und } N_r \cap N_s = \emptyset,\, r \ne s. \]

\begin{proof}
 Sei $x \in [0,1)$ beliebig und sei $y$ das Element von $N$, das zur Äquivalenzklasse von $x$ gehört. Dann ist $x \in N_r$, wobei 
 \[ r := \begin{cases}
          x-y, & \text{wenn } x \ge y, \\
          x-y+1 & \text{wenn } x < y.
         \end{cases} \]
 Also ist
 \[ [0,1) = \bigcup_{r \in R} N_r. \]
 
 Ist $x \in N_r \cap N_s$, so sind $x-r$ (oder $x-r+1$) und $x-s$ (oder $x-s+1$) verschiedene Elemente von $N$, die zur selben Äquivalenzklasse gehören. Das ist nicht möglich. Also ist
 \[ N_r \cap N_s = \emptyset,\, r \ne s. \]
\end{proof}

Wir zeigen nun noch:
\[ \mu( N ) = \mu(N_r),\, r \in R. \]

\begin{proof}
 Aus den geforderten Eigenschaften von $\mu$ folgt
 \[ \mu(N) \overset{(i)}{=} \mu( N \cap [0,1-r) ) + \mu( N \cap [1-r, 1) ) \overset{(ii)}{=} \mu(N_r). \] 
 
 Nun ist
 \[ 1 \overset{(iii)}{=} \mu( [0,1) ) \overset{(i)}{=} \sum_{r \in R} \mu( N_r ). \]
 Das ist nicht möglich, da
 \[ \sum_{r \in R} \mu( N_r ) = 
   \begin{cases}
    \infty, & \text{falls } \mu(N) > 0, \\
    0, & \text{falls } \mu(N) = 0.
   \end{cases} \]
\end{proof}

Sei $I \ne \emptyset$ eine beliebige Menge und $a_i \in \real$. Was versteht man unter $\sum_{i \in I} a_i$?

Ein Ausweg wäre (i) nur für endliche Folgen zu fordern (was die Nützlichkeit von $\mu$ sehr einschränken würde). Das geht auch nicht!

\begin{thm}[Banach-Tarski, 1924]
 Seien $U$ und $V$ beliebige, nichtleere, beschränkte, offene Mengen in $\real^d$, $d \ge 3$.
 
 Dann gibt es ein $k \in \nat$ und Teilmengen $E_1, \ldots, E_k, F_1, \ldots, F_k$, so dass
 \begin{enumerate}[(i)]
  \item Die $E_j$ sind disjunkt und ihre Vereinigung ist $U$.
  \item Die $F_j$ sind disjunkt und ihre Vereinigung ist $V$.
  \item $E_j$ ist kongruent zu $F_j$ für alle $j$.
 \end{enumerate}
\end{thm}

Das bedeutet: Die Mengen $U$ und $V$ haben mit den obigen Forderungen das selbe Maß, egal wie man $U$ und $V$ wählt.
\end{document}