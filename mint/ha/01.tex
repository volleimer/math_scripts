\documentclass[
 a4paper,
 12pt,
 parskip=half
 ]{scrreprt}

\usepackage{../../.tex/settings}

\usepackage{../../.tex/mathpkgs}
\usepackage{../../.tex/mathcmds}

%%%%%%%%%%%%%%%%%%%%%%%%%%%%%%%%%%%%%%%%%%%%%%%%%%%%%%
%%%%%%%%%%%%%% EDIT THIS PART %%%%%%%%%%%%%%%%%%%%%%%%
%%%%%%%%%%%%%%%%%%%%%%%%%%%%%%%%%%%%%%%%%%%%%%%%%%%%%%
\newcommand{\Fach}{\textbf{Maß und Integral}}
\newcommand{\Name}{Hippold, Jonas}
\newcommand{\Seminargruppe}{...}
\newcommand{\Matrikelnummer}{3381170}
\newcommand{\Semester}{WS 2016}
\newcommand{\Uebungsblatt}{} %  <-- UPDATE ME
%%%%%%%%%%%%%%%%%%%%%%%%%%%%%%%%%%%%%%%%%%%%%%%%%%%%%%
%%%%%%%%%%%%%%%%%%%%%%%%%%%%%%%%%%%%%%%%%%%%%%%%%%%%%%

\usepackage{scrpage2}
\renewcommand{\headfont}{}

\usepackage{hyperref}

%%%%%%%%%%%%%%%
%% Aufgaben-COMMAND
\newcommand{\Aufgabe}[1]{
  {
  \vspace*{0.3cm}
  \textsf{\textbf{#1.}}
  }
}

%\setlength{\parindent}{0em}
%\topmargin -1.0cm
%\oddsidemargin 0cm
%\evensidemargin 0cm
%\setlength{\textheight}{9.2in}
%\setlength{\textwidth}{6.0in}

%%%%%%%%%%%%%%
\hypersetup{
    pdftitle={\Fach: Hausaufgabe \Uebungsblatt{}},
    pdfauthor={\Name},
    pdfborder={0 0 0}
}

\title{Hausaufgabe}
\author{\Name{}}

\pagestyle{scrplain}
\ihead{\large \Fach \\ \Semester }
\ohead[\small \Name, Übungsgruppe \Seminargruppe{}]{\small \Name, Matrikel \Matrikelnummer{} \\ Imma 2016, Bachelor Mathematik \\ Übungsgruppe \Seminargruppe{}}

\begin{document}
\thispagestyle{scrheadings}

\vspace*{0cm}
\begin{center}
\LARGE \textbf{Hausaufgabe \Uebungsblatt{}}
\end{center}
\vspace*{0.1cm}

\Aufgabe{(1.2.1)} Für eine beliebige Teilmenge $E \subset \real^d$ sei:
\begin{enumerate}[a)]
 \item $\mu(E) = 1$ wenn $0 \in E$ und $\mu(E) = 0$ sonst.
 \item $\nu(E) =$ Anzahl der Elemente von $E$.
\end{enumerate}
Welche Eigenschaften von (1.1.1.i-iii) besitzen $\mu$ und $\nu$?
\begin{enumerate}[a)]
 \item 
  \begin{enumerate}[(i)]
   \item Nein. Gegenbeispiel: Betrachte $E_1, E_2 \subset \real$, $E_1 := [-1, 1]$, $E_2 := [-2, 2]$.
    \begin{align*}
     \mu( E_1 \cup E_2 ) &\overset{?}{=} \mu(E_1) + \mu(E_2) \\
     \mu( [-2, 2] ) &= \mu( [-1,1] ) + \mu( [-2, 2] ) \\
     1 &= 1 + 1 = 2.
    \end{align*}
    Widerspruch!
   \item Nein. Gegenbeispiel: Betrachte $E_1$ wie oben, $E_2 := \{ x + 2 : x \in E_1 \}$. Dann ist $0 \notin E_2$ und somit $\mu(E_2) = 0$, obwohl $E_1$ und $E_2$ kongruent sind.
   \item Ja. $0 \in [0,1]^d$, also ist $\mu( [0,1]^d ) = 1$.
  \end{enumerate}
 \item 
  \begin{enumerate}[(i)]
   \item Nein. Gegenbeispiel: Betrachte $E_1, E_2 \subset \real$, $F_1 := \{1\}$, $F_2 := [0,1]$. Es ist $\nu(F_1) = 1$, $\nu(F_2) = \infty$.
    \begin{align*}
     \nu( F_1 \cup F_2 ) &\overset{?}{=} \nu(F_1) + \nu(F_2) \\
     \mu( [0, 1] ) &= \nu( \{ 1 \} ) + \nu( [0,1] ) \\
     \infty &= 1 + \infty.
    \end{align*}
    Widerspruch, der Ausdruck $1+\infty$ ist nicht definiert.
   \item Ja. Für kongruente $F_1$, $F_2$ gilt:
    \[ \exists \tau: \real^d \to \real^d \forall f_1 \in F_1 \exists! f_2(f_1) \in F_2 : \tau(f_1) = f_2. \]
    Es existiert eine eineindeutige Zuordnung zwischen beiden Mengen. Also ist die Menge der Elemente in beiden Mengen gleich, also ist $\nu(F_1) = \nu(F_2)$.
   \item Nein. Der Einheitswürfel enthält unendlich viele Elemente, also ist $\nu( [0,1]^d ) = \infty \ne 1$.
  \end{enumerate}
\end{enumerate}


\end{document}
