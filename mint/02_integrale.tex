\chapter{Integrale}
\section{Messbare Abbildungen}
Zu jeder Abbildung $f:X \to Y$ zwischen zwei Mengen gehört eine \emph{inverse Abbildung} $f^{-1}:\pot(Y) \to \pot(X)$ definiert durch $f^{-1}(E) = \{ x \in X : f(x) \in E \}.$ Die inverse Abbildung kommutiert mit Vereinigung, Durchschnitt und Komplement:
\begin{equation}
  \left.
 \begin{aligned}
 f^{-1} \left( \bigcap_\alpha E_\alpha \right) &= \bigcap_\alpha f^{-1}(E_\alpha) \\
 f^{-1} \left( \bigcup_\alpha E_\alpha \right) &= \bigcup_\alpha f^{-1}(E_\alpha) \\
 f^{-1} ( E^C ) &= (f^{-1}(E) )^C
\end{aligned}
\qquad
\right\}
 \tag{$\ast$}
\end{equation}

\begin{lem}
 Ist $\mN \subset \pot(Y)$ eine $\sigma$-Algebra, so ist auch
 \[ f^{-1}(\mN) := \{ f^{-1}(E) : E \in \mN \} \]
 eine $\sigma$-Algebra.
\end{lem}

\begin{proof}
 Das folgt aus den drei Gleichungen ($\ast$).
\end{proof}

\begin{defn}
 Sind $(X,\mM)$ und $(Y,\mN)$ Messräume, so heißt eine Abbildung $f: X \to Y$ \emph{$(\mM,\mN)$-messbar}\footnote{Das heißt $f^{-1}(\mN) \subset \mM$.} (oder einfach messbar), wenn $f^{-1}(E) \in \mM$, $E \in \mN$.
 
 \emph{Vergleich zur Stetigkeit:} Urbilder von offenen Mengen sind offen.
\end{defn}

\begin{lem}
 Sind $(X,\mM)$, $(Y,\mN)$ und $(Z,\mO)$ Messräume und $f:X \to Y$, $g: Y \to Z$ $(\mM,\mN)$-messbare bzw. $(\mN,\mO)$-messbare Abbildungen, so ist $f \circ g = g(f)$ $(\mM,\mO)$-messbar.
\end{lem}

\begin{proof}
 Einfach.
\end{proof}

\clearpage

\begin{lem}
 Ist $\mE$ ein Erzeuger für die $\sigma$-Algebra $\mN$, so ist $f$ genau dann $(\mM,\mN)$-messbar, wenn $f^{-1}(\mE) \subset \mM$.
\end{lem}

\begin{proof}
 Die eine Richtung folgt aus der Definition der Messbarkeit von $f$. Nehmen wir an, dass $f^{-1}(\mE) \subset \mM$. Die Familie $\{ E \subset Y : f^{-1}(E) \in \mM \}$ ist wegen ($\ast$) eine $\sigma$-Algebra, die $\mE$ enthält, folglich enthält sie auch $\mN$.
\end{proof}

\begin{folg}
 Sind $X$ und $Y$ metrische Räume, so ist jede stetige Abbildung $f: X \to Y$ $(\borel(X),\borel(Y))$-messbar.
\end{folg}

\begin{proof}
 $f$ stetig $\Leftrightarrow$ $f^{-1}(U)$ offen für beliebige offene Menge $U \subset Y$. Die Behauptung folgt deshalb aus Lemma 2.1.4.
\end{proof}

Sei $(X,\mM)$ ein Messraum. Eine \emph{reellwertige} oder \emph{komplexwertige} Funktion $f$ auf $X$ heißt $\mM$-messbar (oder einfach messbar), wenn sie $(\mM,\borel(\real))$- oder $(\mM,\borel(\complex))$-messbar ist. Standardmäßig betrachten wir also die Borelmengen im Wertebereich.

\textbf{Spezialfälle.}
\begin{itemize}
 \item $f: \real \to \real$ heißt \emph{Lebesgue}-messbar, wenn sie $(\mM_\lambda, \borel(\real))$-messbar ist.
 \item $f: \real \to \real$ heißt \emph{Borel}-messbar, wenn sie $(\borel(\real), \borel(\real))$-messbar ist.
 \item Analog für komplexwertige Funktionen.
\end{itemize}

\textbf{Warnung.}
Sind $f,g: \real \to \real$ Lebesgue-messbar, so braucht $g(f)$ nicht Lebesgue-messbar zu sein\footnote{Lemma 2.1.3 ist hier aufgrund der Definition der Lebesgue-Messbarkeit nicht anwendbar.}!

\begin{lem}
 Sei $f$ eine rechtsstetige Funktion auf dem Messraum $(X,\mM)$. Die folgenden Aussagen sind äquivalent:
 \begin{enumerate}[(i)]
  \item $f$ ist $\mM$-messbar (genauer $(\mM,\borel(\real))$-messbar).
  \item $f^{-1}((a,\infty)) \in \mM$, $a \in \rat$.
  \item $f^{-1}((a,\infty)) \in \mM$, $a \in \rat$.
  \item $f^{-1}([a,\infty)) \in \mM$, $a \in \rat$.
  \item $f^{-1}((-\infty,a)) \in \mM$, $a \in \rat$.
  \item $f^{-1}((-\infty,a]) \in \mM$, $a \in \rat$.
 \end{enumerate}
\end{lem}

\begin{proof}
 Aufgabe (Folgt aus Lemma 2.1.4 und Aufgabe 1.4.4).
\end{proof}

\clearpage

\begin{lem}
 Seien $(X,\mM)$ und $(Y_\alpha, \mN_\alpha)$, $\alpha \in A$, Messräume, $Y := \prod_{\alpha \in A} Y_\alpha$ und $\mN := \bigotimes_{\alpha \in A} \mN_\alpha$.
 \begin{enumerate}[(i)]
  \item Die Koordinatenprojektionen $\Pi_\alpha: Y \to Y_\alpha$ sind $(\mN,\mN_\alpha)$-messbar.
  \item Eine Abbildung $f: X \to Y$ ist genau dann $(\mM,\mN)$-messbar, wenn die Abbildung $f_\alpha := \Pi_\alpha(f)$ $(\mM,\mN_\alpha)$-messbar ist für alle $\alpha$.
 \end{enumerate}
\end{lem}

Zum Beispiel $Y = Y_1 \times Y_2$, $f(x) = ( f_1(x), f_2(x) )$, $\Pi_1(f(x)) = f_1(x)$.

\begin{proof}
 (i) folgt aus der Definition von $\mN$ und Lemma 2.1.4.
 \[ \Pi^{-1}_\alpha ( N_\alpha ) = \prod_{\beta \in A} Z_\beta \in \mN, \quad \text{für alle } N_\alpha \in \mN_\alpha \]
 $Z_\alpha = N_\alpha$, $Z_\beta = Y_\beta$ für $\beta \ne \alpha$.
 (ii) Ist $f$ messbar, so ist auch $f_\alpha$ messbar nach Lemma 2.1.3. Ist umgekehrt $f_\alpha$ messbar für alle $\alpha$, so gilt
 \[ f^{-1}( \Pi_\alpha^{-1}( E_\alpha ) ) = f_\alpha^{-1} (E_\alpha) \in \mM, \quad \text{für alle } E_\alpha \in \mN_\alpha. \]
 Nach Lemma 2.1.4 ist $f$ messbar.
\end{proof}

\begin{folg}
 $f: X \to \complex$ ist genau dann $\mM$-messbar, wenn $\Re f$ und $\Im f$ messbar sind.
\end{folg}

\begin{proof}
 $\complex = \real \times \real$, also ist $\borel(\complex) = \borel( \real \times \real ) = \borel( \real ) \otimes \borel(\real)$ nach 1.9.5.
\end{proof}

\begin{folg}
 Seien $f_1, \ldots, f_d$ reell- oder komplexwertige Funktionen auf $X$. Die Abbildung $f := (f_1, \ldots, f_d) \in \real^d$ bzw. $\complex^d$ ist genau dann Borel-messbar, wenn $f_1, \ldots, f_d$ Borel-messbar sind.
\end{folg}

\begin{proof}
 Das folgt auch aus 1.9.5.
\end{proof}

\begin{folg}
 Sind $f,g: X \to \complex$ $\mM$-messbar, so sind auch $f+g$, $fg$ und $f/g$ (falls definiert) $\mM$-messbar.
\end{folg}

\begin{proof}
 $F := (f,g)$ ist $(\mM,\borel(\complex \times \complex))$-messbar nach 2.1.9. $\psi(x,y) := x + y$, $x,y \in \complex$ ist stetig und damit Borel-messbar. Also ist $f+g = \psi(F)$ $\mM$-messbar.
 
 $fg$ und $f/g$ folgen analog.
\end{proof}

Es ist manchmal zweckmäßig, Funktionen zu betrachten, deren Werte aus
\[ \realext := [-\infty,\infty] := \real \cup \{-\infty, \infty\} \]
sind. Wir definieren die Borel-$\sigma$-Algebra in $\realext$ durch 
\[ \borel(\realext) := \{ E \subset \realext : E \cap \real \in \borel( \real ) \} \]
Es ist leicht zu sehen, dass die Familie der Mengen $(a, \infty]$ oder $[-\infty,a)$, $a \in \real$ (oder $\rat$), Erzeuger für $\borel(\realext)$ sind.

\begin{lem}
 Ist $\{f_n\}$ eine Folge messbarer, $\realext$-wertiger Funktionen auf $(X, \mM)$, so sind die Funktionen 
 \begin{align*}
  g_1(x) &= \sup_j f_j(x), & g_2(x) &= \inf_j f_j(x), \\
  g_3(x) &= \limsup_{j} f_j(x), & g_4(x) &= \liminf_j f_j(x)
 \end{align*}
 messbar.
 
 Existiert $f(x) = \lim_j f_j(x)$ für alle $x$, so ist auch $f$ messbar.
\end{lem}

\begin{proof}
 Es gilt
 \begin{align*}
   g_1^{-1}( (a, \infty] ) &= \bigcup_{j} f_j^{-1} ((a,\infty]), & g_2^{-1}([-\infty,a)) = \bigcup_j ([-\infty,a)).
 \end{align*}
 Nach 2.1.4 sind $g_1$ und $g_2$ messbar. Hieraus folgt, dass $h_k(x) := \sup_{j > k} f_j(x)$ messbar ist, also auch $g_3 = \inf_{k \ge 0} h_k(x)$. Analog für $g_4$.
 
 Wenn der Grenzwert $\lim_j f_j(x)$ existiert, dann ist $f = g_3 = g_4$ und damit $f$ messbar.
\end{proof}

\begin{folg}
 Sind $f,g : X \to \realext$ messbar, so auch $\max(f,g)$ und $\min(f,g)$.
\end{folg}

\begin{folg}
 Ist $\{f_i\}$ eine Folge komplexwertiger, messbarer Funktionen und existiert $\lim_i f_i(x) = f(x)$ für alle $x$, so ist $f$ messbar.
\end{folg}

\begin{proof}
 Das folgt aus 2.1.8.
\end{proof}

\begin{deno}
 \begin{itemize}
  \item Für eine Funktion $f: X \to \realext$ definieren wir den \emph{positiven} und den \emph{negativen} Teil von $f$ durch 
  \[ f^+ = \max( f(x), 0 ), \qquad f^- = \max( -f(x), 0 ). \]
  \item Ist $f$ messbar, so auch $f^+$ und $f^-$ (siehe Folgerung 2.1.12). 
  \item Ist $f:X \to \complex$, so gilt die \emph{Polarzerlegung}
  \[ f = \sgn f \cdot |f|, \]
  wobei $\sgn z = \frac{z}{|z|}$, $z \ne 0$ und $\sgn 0 = 0$.
  \item Ist $f$ messbar, so auch $\sgn(f)$ und $|f|$.
  \item Sei $X, \meas$ ein Messraum und $E \subset X$. Die Funktion $\ind_E$, die auf $E$ den Wert 1 und sonst den Wert 0 annimmt, heißt \emph{Indikatorfunktion} von $E$. Sie ist genau dann messbar, wenn $E \in \meas$.\footnote{Es gilt $\ind^{-1}( B ) \in \{ \emptyset, E, E^C, X \}$.}
  \item Eine \emph{einfache Funktion} (oder Treppenfunktion) $f$ auf $X$ ist eine endliche Linearkombination von Indikatorfunktionen von Mengen aus $\meas$: 
  \[ f = \sum_{j=1}^n c_j \ind_{E_j}, \quad c_j \in \complex, \quad E_j \in \meas. \]
  \item Summe und Produkt von einfachen Funktionen sind einfach.
 \end{itemize}
\end{deno}

\begin{thm}
 Sei $(X, \meas)$ ein Messraum.
 \begin{enumerate}[(i)]
  \item Ist $f: X \to [0, \infty]$ messbar, so existiert eine Folge $\{ \varphi_n \}$ von einfachen Funktionen mit $0 \le \varphi_1 \le \varphi_2 \le \ldots \le f$, $\varphi_n \to f$, punktweise und $\varphi_n \to f$ gleichmäßig auf jeder Menge, wo $f$ beschränkt ist.
  \item Ist $f: X \to \complex$ messbar, so existiert eine Folge $\{ \varphi_n \}$ von einfachen Funktionen mit $0 \le |\varphi_1| \le |\varphi_2| \le \ldots \le |f|$, $\varphi_n \to f$, punktweise und $\varphi_n \to f$ gleichmäßig auf jeder Menge, wo $f$ beschränkt ist.
 \end{enumerate}
\end{thm}

\begin{proof}
 \begin{enumerate}[(i)]
  \item Für $n \in \nat$, $0 \le k \le 2^{2n}-1$ seien $E_n^k := f^{-1} (( k \cdot 2^{-n} , (k+1) \cdot 2^{n-} ] )$, $F_n := f^{-1}((2^n, \infty])$, 
  \[ \varphi := \sum_{k=0}^{2^{2n} - 1} k \cdot 2^{-n} \ind_{E_n^k} + 2^n \ind_{F_n}. \]
  Es ist leicht zu sehen: Für alle $n \in \nat$ gilt: $0 \le f - \varphi_n \le 2^{-n}$ auf der Menge, wo $f \le 2^n$, $\varphi_n \le \varphi_{n+1}$.
  \item Sei $f = g + i \cdot h = (g^+ - g^-) + i( h^+ - h^- )$. Nach (i) wählen wir die entsprechenden Folgen $\psi_n^+$, $\psi_n^-$, $\chi_n^+$ und $\chi_n^-$. Dann setzen wir $\varphi_n := (\psi_n^+  - \psi_n^-) + i(\chi_n^+ - \chi_n^-)$. \qedhere
 \end{enumerate}
\end{proof}

\begin{lem}
 Sei $\mu$ ein vollständiges Maß, $f,f_n,g: X \to \real^d$ oder $\complex^d$.
 \begin{enumerate}[(i)]
  \item Ist $f$ messbar und $f=g$ fast überall, so ist $g$ messbar.
  \item Ist $f_n$, $n \in \nat$ messbar und $f_n \to f$ fast überall, so ist $f$ messbar.
 \end{enumerate}
\end{lem}

\textbf{Anmerkung.}
Ist $\mu$ nicht vollständig, so sind (i) und (ii) falsch!

\begin{proof}
 \begin{enumerate}[(i)]
  \item Es gilt
  \[ g^{-1}(B) = \Big( [g=b] \cap f^{-1}(B) \Big) \cup \Big( [g \ne b] \cap g^{-1}(B) \Big), \]
  die Menge $[g=b] \cap f^{-1}(B)$ ist messbar aufgrund der Voraussetzung und $[g \ne b] \cap g^{-1}(B)$ ist eine Nullmenge und damit messbar, weil $\mu$ vollständig ist.
  \item Verläuft ähnlich. Sei $N$ eine $\mu$-Nullmenge und $N_0 \subset N$ mit $N_0 \notin \meas_\mu$. Dann ist $g := \ind_{N_0} = 0 =: f$ fast überall, aber $\ind_{N_0}$ ist nicht messbar. \qedhere
 \end{enumerate}
\end{proof}

\section{Aufgaben}
Siehe \verb+Aufgaben-2-2-Teil-1.pdf+ und \verb+Aufgaben-2-2-Teil-2.pdf+.

\section{Integration nichtnegativer Funktionen}
In diesem Abschnitt ist $(X,\meas,\mu)$ ein Maßraum.

Die Menge $L^+$ ist die Menge aller messbaren Funktionen $f:X \to [0, \infty]$.

\begin{defn}
 Sei $\varphi \in L^+$ eine einfache Funktion mit Standard-Darstellung $\varphi = \sum_1^n a_j \ind_{E_j}$. Wir definieren das \emph{Integral} von $\varphi$ bezüglich $\mu$ durch
 \[ \int \varphi \diffop \mu := \sum_{j=1}^n a_j \mu( E_j ). \]
\end{defn}

\begin{rmrk*}
 \begin{itemize}
  \item $\int \varphi \diffop \mu = \infty$ ist möglich, zum Beispiel $\int \ind_\real \diffop \lambda = \infty$.
  \item Kurze Schreibweise: $\int \varphi$, wenn keine Verwechselungsgefahr.
  \item Angabe der Funktionsvariable: $\int \varphi(x) \diffop \mu(x)$, zum Beispiel wenn mehrere Variablen vorkommen:
  \[ \int \varphi( x_1, x_2, x_3 ) \diffop \mu(x_2). \]
  Auch üblich: $\int \varphi(x) \mu(\diffop x)$.
 \end{itemize}
\end{rmrk*}

Ist $A \in \meas$ und $\varphi$ wie oben, so ist 
\[ \varphi \cdot \ind_A = \sum_{j=1}^n a_j \ind_{E_j \cap A} \in L^+. \]
Wir schreiben
\[ \int_A \varphi \diffop \mu := \int \varphi \cdot \ind_A \diffop \mu \]
\emph{Integral über $A$}.

\begin{lem}
 Für einfache Funktionen $\varphi, \si \in L^+$ gilt:
 \begin{enumerate}[(i)]
  \item $\int c \varphi = c \int \varphi$, $c \ge 0$,
  \item $\int(\varphi + \psi) = \int \varphi + \int \psi$,
  \item wenn $\varphi \le \psi$, dann $\int \varphi \le \int \psi$,
  \item $A \to \int_A \varphi$ ist ein Maß auf $\meas$.
 \end{enumerate}
\end{lem}

\begin{proof}
 \begin{enumerate}[(i)]
  \item Ist einfach.
  \item Seien $\varphi = \sum_{j=1}^n a_j \ind_{E_j}$, $\psi = \sum_{j=1}^m b_j \ind_{F_j}$ die Standard-Darstellungen. Aus 
  \[ E_j = \bigcup_{k=1}^m ( E_j \cap F_k ), \quad F_k = \bigcup_{j=1}^n ( E_j \cap F_k ), \]
  wobei die Vereinigungen disjunkt sind, folgt wegen der Additivität von $\mu$ 
  \[ \begin{aligned}
      \int \varphi + \int \psi &= \sum_{j=1}^n a_j \mu( E_j ) + \sum_{k=1}^m b_k \mu( F_k ) \\
      &= \sum_{j,k} (a_j + b_k) \mu (E_j \cap F_k ).
     \end{aligned} \]
  Dieselbe Überlegung zeigt, dass die Summe rechts gleich $\int(\varphi + \psi)$ ist.
  \item Wenn $\varphi \le \psi$ und $E_j \cap F_k \ne \emptyset$, dann $a_j \le b_k$ $\Rightarrow$
  \[ \int \varphi = \sum_{j,k} a_j \mu( E_j \cap F_k ) \le \sum_{j,k} b_k \mu (E_j \cap F_k) = \int \psi. \]
  \item Seien $A_j \in \meas$, $j \in \nat$ disjunkt und $A := \bigcup_1^\infty A_j$.
  \[ \begin{aligned}
      \int_A \varphi &= \sum_j a_j \mu( A \cap E_j ) \\
      &= \sum_{j,k} a_j \mu ( A_j \cap E_j ) \\
      &= \sum_k \int_{A_k} \varphi
     \end{aligned} \]
  und damit folgt die Behauptung. \qedhere
 \end{enumerate}
\end{proof}

\begin{defn}
 Für eine beliebige Funktion $f \in L^+$ sei
 \[ \int f \diffop \mu := \sup \left\{ \int \varphi \diffop \mu : 0 \le \varphi \le f, \varphi \text{ einfach } \right\}. \]
\end{defn}

\textbf{Anmerkungen.} 
Aus 2.3.2(iii) folgt, dass die zwei Definitionen für $\int f$ übereinstimmen, wenn $f$ einfach ist. Aus der Definition ist außerdem klar:
\[ \int f \le \int g, \text{ wenn } f \le g, \qquad \int f = c \cdot \int f, c \ge 0. \]

\begin{thm}[Monotone Konvergenz, Beppo Levi]
 Ist $\{f_n\}$ eine Folge in $L^+$ mit $f_j \le f_{j+1}$ für alle $j$ und $f = \lim_{n \to \infty} f_n$, so gilt
 \[ \int f = \lim_{n \to \infty} \int f_n.\footnotemark \]
\end{thm}
\footnotetext{Beide Grenzwerte existieren wegen der Monotonie, $\infty$ ist möglich.}

\begin{proof}
 Für alle $n$ gilt $\int f_n \le \int f$ $\Rightarrow$ $\lim \int f_n \le \int f$. Es bleibt noch zu zeigen, dass $\lim \int f_n \ge \int f$. 
 
 Sei $\alpha \in (0,1)$ beliebig, sei $\varphi$ eine einfache Funktion mit $0 \le \varphi \le f$ und $E_n := \{ x : f_n(x) \ge \alpha \cdot f(x) \}$. Dann gilt $E_n \subset E_{n+1}$, weil $f_n \le f_{n+1}$, $\bigcup E_n = X$ und
 \[ \int f_n \ge \int_{E_n} f_n \ge \alpha \int_{E_n} \varphi. \]
 Nach 2.3.2(iv) und der Stetigkeit von unten ist
 \[ \lim \int_{E_n} \varphi = \int \varphi \qRq \lim \int f_n \ge \alpha \int \varphi \]
 für alle $\alpha \in (0,1)$, also auch für $\alpha = 1$. 
 
 Wir bilden das Supremum über alle einfachen Funktionen $0 \le \varphi \le f$ und erhalten:
 \[ \lim \int f_n \ge \int f. \qedhere \]
\end{proof}

\begin{thm}
 Ist $\{ f_n \}$ eine endliche oder unendliche Folge in $L^+$ und $f = \sum_n f_n$, so gilt
 \[ \int f =  \sum_n \int f_n. \]
\end{thm}

\begin{proof}
 Wir betrachten zunächst zwei Funktionen $f_1$ und $f_2$. Nach Satz 2.1.15 existieren monoton wachsende  Folgen $\{ \varphi_j \}$ und $\{ \psi_j \}$ von einfachen Funktionen, die gegen $f_1$ bzw. $f_2$ konvergieren. Dann konvergiert die \emph{monoton wachsende Folge} $\{ \varphi_j + \psi_j \}$ gegen $f_1 + f_2$. Nach dem Satz über monotone Konvergenz (m) und der Additivität für einfache Funktionen (a) gilt:
 \[ \int (f_1 + f_2 ) \overset{(\text{m})}{=} \lim \int (\varphi_j + \psi_j) \overset{(\text{a})}{=} \lim \int \varphi_j + \lim \int \psi_j \overset{(\text{m})}{=} \int f_1 + \int f_2. \]
 
 Induktion über $N$ liefert
 \[ \int \sum_{n=1}^N f_n = \sum_{n=1}^N \int f_N. \]
 
 Ist die Folge unendlich, so bilden wir den Grenzwert $N \to \infty$:
 \[ \int \sum_{n=1}^\infty f_n \overset{(\text{m})}{=} \sum_{n=1}^\infty \int f_n. \qedhere \]
\end{proof}

\begin{thm}
 Für eine Funktion $f \in L^+$ gilt $\int f = 0$ genau dann, wenn $f=0$ fast überall.
\end{thm}

\begin{proof}
 Sei zuerst $f = \sum_j a_j \ind_{E_j}$, $a_j \ge 0$ eine einfache Funktion. Dann gilt $\int f = \sum_j a_j \mu( E_j ) = 0$ genau dann, wenn für alle $j$ entweder $a_j = 0$ oder $\mu( E_j ) = 0$, also ist $f = 0$ fast überall.
 
 Allgemein: Wenn $f=0$ fast überall und $\varphi$ eine einfache Funktion ist mit $0 \le \varphi \le f$, dann ist $\varphi = 0$ fast überall. Also
 \[ \int f = \sup \left\{ \int \varphi \right\} = 0. \]
 
 Andere Richtung: Es gilt
 \[ \{ x : f(x) > 0 \} = \bigcup_{n=1}^\infty E_n, \]
 wobei $E_n := \{ x : f(x) > \rez{n} \}$. Wäre die Aussage $f=0$ fast überall falsch, so hätten wir $\mu(E_n) > 0$ für ein $n$. Wegen $f \ge n \cdot \ind_{E_n}$ gilt dann 
 \[ \int f \ge n \mu( E_n ) > 0, \]
 Widerspruch.
\end{proof}

\begin{folg}
 Sei $f \in L^+$ und $\{ f_n\}$ eine monoton wachsende Folge in $L^+$, die fast überall gegen $f$ konvergiert. Dann gilt $\int f = \lim \int f_n.$
\end{folg}

\begin{proof}
 Übungsaufgabe.
\end{proof}

\begin{thm}[Lemma von Fatou]
 Sei $(f_n)_{n \in \nat}$ eine Folge in $L^+$. Dann gilt
 \[ \int \liminf_{n \to \infty} f_n \diffop \mu \le \liminf_{n \to \infty} \int f_n \diffop \mu. \]
\end{thm}

\begin{proof}
 Für alle $k \in \nat$ gilt
 \[ \inf_{n \ge k} f_n \le f_j \]
 für alle $j \ge k$. Damit folgt 
 \begin{align*}
     \int \inf_{n \ge k} f_n \diffop \mu &\le \int f_j \diffop \mu, \\
     \Rightarrow \int \inf_{n \ge k} f_n \diffop \mu &\le \inf_{j \ge k} \int f_j \diffop \mu. \tag{$*$}
 \end{align*}
 Wir bilden den Grenzwert $k \to \infty$ und wenden den Satz von der monotonen Konvergenz (m) an:
 \begin{align*}
     \int \liminf_{n \to \infty} f_n \diffop \mu &= \int \lim_{n \to \infty} \underbrace{\inf{n \ge k} f_n}_{\text{aufsteigend, messbar, } > 0} \diffop \mu \\
     &\overset{(\text{m})}{=} \lim_{k \to \infty} \int \inf_{n \ge k} f_n \diffop \mu \\
     &\overset{(*)}{=} \lim_{k \to \infty} \inf_{n \ge k} \int f_n \diffop \mu \\
     &= \liminf_{k \to \infty} f_k \diffop \mu. \qedhere
 \end{align*}
\end{proof}

\begin{folg}
 Sei $(f_n)_n$ eine Folge in $L^+$, die fast überall gegen eine Funktion $f \in L^+$ konvergiert. Dann ist
 \[ \int f \diffop \mu \le \liminf_{n \to \infty} \int f_n \diffop \mu. \]
\end{folg}

\begin{proof}
 Gilt die Konvergenz überall, dann ist
 \[ f(x) = \lim_{n \to \infty} f_n(x) = \liminf_{n \to \infty} f_n(x) \]
 für alle $x \in X$ und somit folgt die Behauptung aus dem Lemma von Fatou. 
 
 Konvergenz überall können wir erreichen, indem wir $f_n$ und $f$ auf einer Nullmenge ändern\footnote{Passe $f_n$ und $f$ so an, dass $f'_n$ auch gegen $f'$ konvergiert auf der Nullmenge, wo $f_n$ \emph{nicht} gegen $f$ konvergiert.}, das ändert nicht die Integrale.
\end{proof}

\begin{lem}
 Sei $f \in L^+$ und $\int f \diffop \mu < \infty$. Dann
 \begin{enumerate}[(i)]
  \item $\{ x \in X,  f(x) = \infty \}$ ist eine Nullmenge.
  \item $\{ x \in X, f(x) > 0 \}$ ist $\sigma$-endlich.
 \end{enumerate}
\end{lem}

\begin{proof}
 Übungsaufgabe.
\end{proof}

\section{Aufgaben}
Siehe \verb+Aufgaben-2-4.pdf+.

\section{Integration komplexer Funktionen}
In diesem Abschnitt ist $(X, \meas, \mu)$ ein Maßraum.

\begin{defn}
 Sei $f:X \to \realext$ eine messbare Funktion. Ist $\int f^+ \diffop \mu < \infty$ oder $\int f^- \diffop \mu < \infty$, so definieren wir das \emph{Integral von $f$} bezüglich $\mu$ durch
 \[ \int f \diffop \mu := \int f^+ \diffop \mu - \int f^- \diffop \mu. \]
 Sind $\int f^+ \diffop \mu < \infty$ und $\int f^- \diffop \mu < \infty$, so heißt $f$ \emph{integrierbar}. 
 
 Wegen $0 \le f^2 \le |f| = f^+ + f^-$ ist $f$ genau dann integrierbar, wenn $\int |f| \diffop \mu < \infty.$
\end{defn}

\begin{lem}
 Die Menge der reellwertigen integrierbaren Funktionen bildet einen reellen linearen Raum\footnotemark und das Integral ist ein lineares Funktional\footnotemark auf diesem Raum.
\end{lem}
 \addtocounter{footnote}{-2} %3=n
 \stepcounter{footnote}\footnotetext{Ein linearer Raum $X$ ist abgeschlossen bezüglich Skalarmultiplikation und Addition}
 \stepcounter{footnote}\footnotetext{Sei $V$ ein $\mathbb{K}$-Vektorraum. Ein lineares Funktional ist eine lineare Abb. $T: V \to \mathbb{K}$.}

\begin{proof}
 \textbf{Linearer Raum:} Das folgt aus
 \[ | af + bg | \le |a| \cdot |f| + |n| \cdot |g| \]
 für alle $a,b \in \real$, $f,g$ Funktionen.
 
 Sind $f,g$ integrierbar, so folgt daraus, dass $af + bg$ integrierbar ist.
 
 \textbf{Lineares Funktional:} $\int (af) \diffop \mu = a \int f \diffop \mu$ für alle $a \in \real$ ist trivial.
 
 Seien $f,g$ integrierbar und $h = f+g$. Dann ist $h = h^+ - h^- = (f^+ - f^-) + (g^+ - g^-)$, also
 \[ h^+ + f^- + g^- = h^- + f^+ + g^+ \ge 0. \]
 Nach Satz 2.3.5 ist
 \[ \int h^+ \diffop \mu + \int f^- \diffop \mu + \int g^- \diffop \mu = \int h^- \diffop \mu + \int f^+ \diffop \mu + \int g^+ \diffop \mu \]
 und damit
 \[ \int h \diffop \mu = \int f \diffop \mu + \int g \diffop \mu. \qedhere \]
\end{proof}

\begin{defn}
 Eine messbare Funktion $f: X \to \complex$ heißt \emph{integrierbar}, falls $\int |f| \diffop \mu < \infty$.
 
 \textbf{Allgemeiner:} $f$ heißt auf $E \in \meas$ integrierbar, falls $\int_E |f| \diffop \mu$ endlich ist. Wegen $|f| \le |\Re f| + |\Im f| \le 2 |f|$ folgt, dass $f$ genau dann integrierbar ist, wenn $\Re f$ und $\Im f$ integrierbar sind. In diesem Fall sei
 \[ \int f \diffop \mu = \int \Re f \diffop \mu + i \int \Im f \diffop \mu. \]
 Insbesondere gilt
 \[ \Re \left( \int f \diffop \mu \right) = \int \Re f \diffop \mu, \qquad
    \Im \left( \int f \diffop \mu \right) = \int \Im f \diffop \mu. \]
\end{defn}

\begin{lem}
 Die Menge aller komplexwertigen integrierbaren Funktionen ist ein komplexer linearer Raum und das Integral ist ein lineares Funktional auf diesem Raum.
\end{lem}

\begin{proof}
 Einfach, ähnlich wie der Beweis von 2.5.2.
\end{proof}

Bezeichnung für den obigen Raum: $\intf^1(X)$, $\intf^1(X,\mu)$, $\intf^1(\mu)$ oder einfach $\intf^1$.

\begin{lem}
 Für $f \in \intf^1$ gilt
 \[ \left| \int f \diffop \mu \right| \le \int |f| \diffop \mu. \]
\end{lem}

\begin{proof}
 Ist $\int f \diffop \mu = 0$, dann ist die Behauptung trivial.
 
 \textbf{Fall 1:} $f$ ist reellwertig. Dann
 \[ \begin{aligned}
     \left| \int f \diffop \mu \right| &\overset{\text{Def.}}{=} \left| \int f^+ \diffop \mu - \int f^- \diffop \mu \right| \\
     &\le \left| \int f^+ \diffop \mu \right| + \left| \int f^- \diffop \mu \right|
     = \int f^+ \diffop \mu + \int f^- \diffop \mu \\
     &= \int ( f^+ + f^- ) \diffop \mu = \int |f| \diffop \mu.
    \end{aligned} \]
 \textbf{Fall 2:} $f$ ist komplexwertig. Sei 
 \[ \alpha = \frac{ \obar{\int f  \diffop \mu} }{ | \int f \diffop \mu |}. \]
 Dann gilt $|\alpha| \le 1$ und
 \[ \left| \int f \diffop \mu \right| \overset{(\text{z})}{=} \alpha \int f \diffop \mu = \int \alpha f \diffop \mu, \]
 wobei (z): $z\obar{z} = |z|^2, z := \int f \diffop \mu$. Also
 \begin{align*}
  \left| \int f \diffop \mu \right| &= \Re \left| \int f \diffop \mu \right| = \Re \left( \int \alpha f \diffop \mu \right) \\
  &= \int \Re (\alpha f) \diffop \mu \le \int \underbrace{|\alpha f|}_{|\alpha|\cdot|f| \le |f|} \diffop \mu \\
  &\le \int |f| \diffop \mu. \qedhere
 \end{align*}
\end{proof}
 
\begin{lem}
 Ist $f \in \intf^1$, dann gilt $\{ x, f(x) \ne 0 \}$ ist $\sigma$-endlich.  
\end{lem}

\begin{proof}
 Übungsaufgabe.
\end{proof}

\begin{lem}
 Für $f,g \in \intf^1$ sind äquivalent:
 \begin{enumerate}[(i)]
  \item $\int_E f \diffop \mu =  \int_E g \diffop \mu$ für alle $E \in \meas$.
  \item $\int |f-g| \diffop \mu = 0$.
  \item $f=g$ fast überall.
 \end{enumerate}
\end{lem}

\begin{proof}
 Übungsaufgabe.
\end{proof}

\begin{rmrk}
 Das Integral einer Funktion $f$ ändert sich also nicht, wenn wir $f$ auf einer Nullmenge modifizieren. Demzufolge können wir Funktionen integrieren, die nur fast überall definiert sind. In den nicht definierten Punkten können wir die Funktion zum Beispiel gleich 0 setzen.
 
 In diesem Sinne können wir $\realext$-wertige Funktionen, die fast überall endlich sind, als $\real$-wertig betrachten.
\end{rmrk}

\textbf{Bezeichnung.} 
$L^1(\mu)$ ist die Menge der Äquivalenzklassen von fast überall definierten $\mu$-integrierbaren Funktionen, wobei $f$ und $g$ äquivalent heißen, falls $f = g$ fast überall.

Dann ist $L^1(\mu)$ ein komplexer linearer Raum und sogar ein metrischer Raum mit der Metrik
\[ \rho( f,g ) := \int | f -g | \diffop \mu. \]
Man verwendet die (eigentlich nicht korrekte) Schreibweise $f \in L^1(\mu)$ für die Äqui\-valenz\-klasse von $f$.

\begin{thm}[Dominierte Konvergenz, Satz von Lebesgue]
 Seien $f_n, g \in \intf^1$, $n \in \nat$, sodass $f_n \to f$ fast überall und $|f_n| \le g$ für alle $n \in \nat$. Dann gilt
 \[ \int f \diffop \mu = \int \lim_{n \to \infty} f_n \diffop \mu = \lim_{n \to \infty} \int f \diffop \mu. \]
\end{thm}

\begin{proof}
 Die Funktion $f$ ist messbar und $|f| \le g$ fast überall, also ist $f \in \intf^1$. 
 
 Es genügt, reellwertige Funktionen zu betrachten (man bildet sonst Real- und Imaginärteil). Es gilt
 \[ g + f_n \ge 0 \quad \text{und} \quad g - f_n \ge 0 \]
 für alle $n \in \nat$. Damit folgt mit dem Lemma von Fatou (f):
 \[ \begin{aligned}
  \int(g+f) \diffop \mu &= \int \liminf_{n \to \infty} (g + f_n) \diffop \mu \\
  &\overset{(\text{f})}{\le} \liminf_{n \to \infty} \int (g+f_n) \diffop \mu = \int g \diffop \mu + \liminf_{n \to \infty} \int f_n \diffop \mu 
    \end{aligned} \]
  sowie
 \[ \begin{aligned}
 \int(g-f) \diffop \mu &= \int \liminf_{n \to \infty} (g - f_n) \diffop \mu \\
  &\overset{(\text{f})}{\le} \liminf_{n \to \infty} \int(g - f_n) \diffop \mu = \int g \diffop \mu - \limsup_{n \to \infty} \int f_n \diffop \mu.
    \end{aligned} \]
 Folglich gilt
 \[ \liminf_{n \to \infty} \int f_n \diffop \mu \ge \int f \diffop \mu \ge \limsup_{n \to \infty} \int f_n \diffop \mu. \]
 Daraus folgt
 \[ \lim_{n \to \infty} \int f_n \diffop \mu = \int f \diffop \mu. \]
 Hier wurde verwendet, dass $\lim_{n \to \infty} a_n$ existiert $\Leftrightarrow$ $\liminf_{n \to \infty} a_n = \limsup_{n \to \infty} a_n$.
\end{proof}

%%%

\begin{thm}
 Es sei $\{ f_j \}$ eine Folge in $\intf^1$ mit $\sum_1^\infty |f_j| < \infty$. Dann konvergiert $\sum_1^\infty f_j$ fast überall gegen eine Funktion in $\intf^1$ und es gilt
 \[ \int \sum_{j=1}^\infty f_j = \sum_{j=1}^\infty \int f_j. \]
\end{thm}

\begin{proof}
 Nach Satz 2.3.5 gilt
 \[ \int \sum_{j=1}^\infty | f_j | = \sum_{j=1}^\infty \int |f_j| < \infty \]
 und folglich $\sum_1^\infty | f_j | \in \intf^1$.
 
 Wegen Lemma 2.3.10 ist $\sum_1^\infty |f_j|$ fast überall endlich, also konvergiert $\sum_1^\infty f_j$ fast überall.  Für alle $n$ gilt
 \[ \left| \sum_{j=1}^n f_j \right| \le \sum_{j=1}^n | f_j | \le \sum_{j=1}^\infty |f_j| =: g. \]
 Die Behauptung folgt nun aus dem Satz über dominierte Konvergenz.
\end{proof}

\begin{thm}\label{thm:2-5-11}
 \begin{enumerate}[(i)]
  \item Für jede Funktion $f \in \intf^1$ und jedes $\eps > 0$ existiert eine integrierbare, einfache Funktion $\varphi = \sum a_j \ind_{E_j}$ mit $\int | f -\varphi | < \eps$.\footnotemark (Abstand $d(f,\varphi)$ in $\intf^1$)
  \item Ist $\mu$ ein Lebesgue-Stieltjes-Maß auf $\real$, so können wir für die $E_j$ offene Intervalle nehmen.
  \item Ist $\mu$ wie in (ii), so existiert eine stetige Funktion $g$, die außerhalb eines beschränkten Intervalls verschwindet\footnotemark, mit
  \[ \int |f-g| \diffop \mu < \eps.\footnotemark \]
 \end{enumerate}

\end{thm}
\addtocounter{footnote}{-3} %3=n
 \stepcounter{footnote}\footnotetext{Das heißt die integrierbaren einfachen Funktionen sind \emph{dicht} in $\intf^1$.}
 \stepcounter{footnote}\footnotetext{$g$ hat einen \emph{kompakten Träger}, $\operatorname{supp} g := \obar{\{ x \in \real : g(x) \ne 0 \}}$.}
 \stepcounter{footnote}\footnotetext{Also sind auch die stetigen Funktionen mit kompaktem Träger dicht in $\intf^1$.}

\begin{proof}
 \begin{enumerate}[(i)]
  \item Sei$\{ \varphi_j \}$ wie in Satz 2.1.15(ii). Nach dem Satz über dominierte Konvergenz gilt $\int |\varphi_n - f| < \eps$, wenn $n$ hinreichend groß\footnote{Es ist $|\varphi_n| \le |f|$, $\varphi_j \to f$ und damit $| \varphi_n - f| \le |\varphi_n| + |f| \le 2 |f| =: g$.} ist.
  \item Sei $\varphi = \sum a_j \ind_{E_j}$, wobei die $E_j$ disjunkt und die $a_j \ne 0$ sind. Dann ist 
  \[ \mu(E_j)  = \rez{|a_j|} \int_{E_j} |\varphi| \le \rez{|a_j|} \int |f| < \infty. \]
  Weiterhin gilt $\mu( E \Delta F ) = \int( \ind_E - \ind_F )$. 
  
  Nach Aufgabe 1.12.1 können wir $\ind_{E_j}$ in der $\intf^1$-Metrik beliebig approximieren durch endliche Linearkombinationen von Funktionen $\ind_{I_k}$, wobei die $I_k$ offene Intervalle sind.
  \item Ist $I_k = (a_k, b_k)$ wie in (ii), so können wir $\ind_{I_k}$ in der $\intf^1$-Metrik beliebig durch stetige Funktionen approximieren, die außerhalb von $(a,b)$ verschwinden (auch Differenzierbarkeit ist möglich). \qedhere
 \end{enumerate}
\end{proof}

\begin{exmp*}
 Sei $g(x) = 0$ für $x \notin (a_k,b_k)$, $g(x) = 1$ für $x \in [a_k + \rez{n}, b_k - \rez{n}]$ und sei $g$ linear auf $[a, a + \rez{n}]$ und $[b-\rez{n},b]$. Dann gilt
 \[ \int | g - \ind_{(a_k,b_k)} | = \mu(( a, a+\rez{n})) + \mu( (b-\rez{n},b) )) \xrightarrow{n \to \infty} 0 \]
 nach dem Stetigkeitssatz (von oben).
\end{exmp*}

\begin{thm}[Stetigkeit und Differenzierbarkeit von Parameterintegralen]
 Sei $f:X \times [a,b] \to \complex$, $-\infty < a < b < \infty$, so das $f( \cdot, t ) : X \to \complex$ integrierbar ist. Wir schreiben
 \[ F(t) := \int_X f( x, t ) \diffop \mu(x), \quad t \in [a,b]. \]
 \begin{enumerate}[(i)]
  \item Nehmen wir an, dass ein $g \in \intf^1(\mu)$ existiert mit $|f(x,t)| \le g(x)$ für alle $x$ und $t$, und dass $\lim_{t \to t_0} f(x,t) = f(x,t_0)$ für alle $x$. Dann gilt
  \[ \lim_{t \to t_0} F(t) = F(t_0). \]
  Ist speziell $f(x, \cdot)$ stetig für alle $x$, so ist $F$ stetig.
  \item Nehmen wir an, dass $\pdiff{f}{t}$ und ein $g \in \intf^1$ existieren, so dass $\left| \pdiff{f}{t} (x,t) \right| \le g(x)$ für alle $x$ und $t$. Dann ist $F$ differenzierbar und 
  \[ F'(t) = \int \pdiff{f}{t} \diffop \mu(x), \quad t \in [a,b]. \]
 \end{enumerate}
\end{thm}

\begin{proof}
 \begin{enumerate}[(i)]
  \item Wir wenden den Satz über dominierte Konvergenz auf $f_n(x) := f(x, t_n)$ an, wobei $\{ t_n \}$ eine beliebige Folge in $[a,b]$ mit $t_n \to t_0$ ist. 
  \item Sei $\{ t_n \}$ wie oben. Es gilt
  \[ \pdiff{f}{t}(x,t_0) = \lim_{n \to \infty} h_n(x), \qquad \text{wobei } h_n(x) := \frac{f(x,t_n) - f(x,t_0)}{t_n -t_0} \]
  und $t_n \to t_0$. Hieraus folgt, dass $\pdiff{f}{t}$ messbar ist.
  
  Mit dem Mittelwertsatz (m) folgt
  \[ |h_n(x)| \overset{(\text{m})}{\le} \sup_{t \in [a,b]} \left| \pdiff{f}{t} (x,t) \right| \le g(x) \]
  und damit folgt
  \[ F'(t_0) = \lim_{n \to \infty} \frac{F(t_n) - F(t_0)}{t_n - t_0} = \lim_{n \to \infty} \int h_n(x) \diffop \mu(x) \overset{(\text{d})}{=} \int \pdiff{f}{t} (x,t) \diffop \mu(x) \]
  mit dem Satz über dominierte Konvergenz (d). \qedhere
 \end{enumerate}
\end{proof}

\begin{prgp}[Vergleich Lebesgue- und Riemann-Integral auf $\real$]
 Sei $[a,b]$ ein beschränktes Intervall. Eine \emph{Zerlegung} von $[a,b]$ ist eine endliche Folge $P = \{ t_j \}_0^n$ mit $a = t_0 < \ldots < t_n = b$. Sei $f$ eine beliebige, beschränkte, reellwertige Funktion auf $[a,b]$. Für jede Zerlegung $P$ sei
\[ S_P f := \sum_{j=1}^n M_j(t_j-t_{j-1}), \qquad s_P f := \sum_{j=1}^n m_j(t_j-t_{j-1}), \]
wobei $M_j$ bzw. $m_j$ das Supremum bzw. Infimum von $f$ über $(t_{j-1},t_j]$ bezeichnen. Gilt
\[ i(f) := \inf_{p} S_p f = \sup_p s_p f =: I(f), \]
so heißt $f$ \emph{Riemann-integrierbar} und das \emph{Riemann-Integral} von $f$ über $[a,b]$ wird definiert durch
\[ \int_a^b f(x) \diffop x := I(f) = i(f). \]
\end{prgp}

\begin{thm}
 Sei $f$ eine beschränkte, reellwertige Funktion auf $[a,b]$.
 \begin{enumerate}[(i)]
  \item Ist $f$ Riemann-integrierbar, dann ist $f$ Lebesgue-integrierbar und
  \[ \int_a^b f(x) \diffop x = \int_{[a,b]} f(x) \diffop \lambda(x). \]
  \item $f$ ist genau dann Riemann-integrierbar, wenn das Lebesgue-Maß der Menge
  \[ \{ x \in [a,b] : f \text{ ist nicht stetig in } x \} \]
  gleich Null ist.
 \end{enumerate}
\end{thm}

\begin{proof}
 \begin{enumerate}[(i)]
  \item Für jede Partition $P$ sei
  \[ G_P := \sum_{j=1}^n M_j \ind_{(t_{j-1},t_j]}, \qquad g_P := \sum_{j=1}^\infty m_j \ind_{(t_{j-1},t_j]}. \]
  Dann gilt
  \[ \int_{[a,b]} G_P \diffop \lambda = S_P f, \qquad \int_{[a,b]} g_P \diffop \lambda = s_P f. \]
  Wir wählen eine Folge $\{ P_k \}$ von Partitionen, so dass
  \begin{itemize}
   \item ihre Feinheit $\max_j (t_j - t_{j-1}$ gegen Null geht,
   \item $P_{k+1}$ eine Verfeinerung von $P_k$ ist, dann ist $g_{P_k}$ \emph{monoton wachsend} und $G_{P_k}$ \emph{monoton fallend},
   \item $S_{P_k} f$ und $s_{P_k} f$ gegen $\int_a^b f(x) \diffop x$ konvergiert.
  \end{itemize}
  Sei $G := \lim G_{P_k}$ und $g := \lim g_{P_k}$. Dann ist $g \le f \le G$ und nach dem Satz über dominierte Konvergenz gilt
  \[ \int G \diffop \lambda = \int g \diffop \lambda = \int_a^b f(x) \diffop x. \]
  Folglich ist
  \[ \int \underbrace{(G-g)}_{\ge 0} \diffop \lambda = 0 \qRq G \overset{\text{f.ü.}}{=} g \qRq G=g=f \text{ fast überall.} \]
  Da $G$ messbar ist (Grenzwert von einfachen Funktionen) und $\lambda$ vollständig ist, ist auch $f$ messbar und
  \[ \int_{[a,b]} f \diffop \lambda = \int_{[a,b]} G \diffop \lambda = \int_a^b f(x) \diffop x. \]
  \item Wir definieren für $x \in [a,b]$
  \[ h(x) := \liminf_{y \to x} f(y), \qquad H(x) := \limsup_{y \to x} f(y). \]
  Dann gilt $h \le f \le H$ und $h(x) = H(x)$, wenn $f$ in $x$ stetig ist.
  
  Zu zeigen ist also: $f$ ist Riemann-integrierbar $\Leftrightarrow$ $h = H$ fast überall.
  
  Sei $f$ Riemann-integrierbar. Mit den Bezeichnungen von (i) gilt dann $H = G$ fast überall und $h=g$ fast überall (Warum? Übungsaufgabe!). Folglich sind $h$ und $H$ Lebesgue-messbar und
  \[ \int_{[a,b]} h \diffop \lambda = i(f) = I(f) = \int_{[a,b]} H \diffop \lambda. \]
  Da $h \le H$ folgt hieraus, dass $h = H$ fast überall.
  
  Ist $f$ fast überall stetig, so ist $f$ messbar (Warum? Übungsaufgabe!). Folglich sind auch $h$ und $H$ messbar und $h = H$ fast überall. Die Gleichung oben gilt auch in diesem Fall, das heißt $i(f) = I(f)$, also ist $f$ Riemann-integrierbar. \qedhere
 \end{enumerate}
\end{proof}

\begin{prgp}[Uneigentliches Riemann-Integral]
 Ist $f$ Riemann-integrierbar über $[a,b]$ für jedes $b > 0$ und Lebesgue-integrierbar über $[0,\infty)$, so gilt nach dem Satz über dominierte Konvergenz\footnote{Wähle eine Folge $b_n \to \infty$, definiere $f_n := \ind_{[a,b_n]} \cdot f$, also $\int_0^{b_n} f = \int_0^\infty f_n$, $|f_n| \le |f| =: g$.}
 \[ \int_{[0,\infty)} f \diffop \lambda = \lim_{b \to \infty} \int_a^b f(x) \diffop x \overset{\text{Def.}}{=} \int_0^\infty f(x) \diffop x. \]
\end{prgp}

Der Grenzwert oben kann also auch dann existieren, wenn $f$ nicht Lebesgue-integrierbar ist auf $[0,\infty)$, zum Beispiel
\[ \sum_{n=1}^\infty (-1)^{n+1} \cdot \rez{n} \cdot \ind_{(n,n+1)}(x) := f(x). \]
Es gilt: $f$ ist integrierbar $\Leftrightarrow$ $|f|$ ist integrierbar, aber
\[ \int |f| = \int_{[0,\infty)} \sum_{n=1}^\infty \rez{n} \cdot \ind_{(n,n+1)} = \sum_{n=1}^\infty \rez{n} = \infty. \]
Also ist $f$ nicht Lebesgue-integrierbar. 

Wir können aber das uneigentliche Riemann-Integral berechnen:
\[ \int_0^{N+1} f(x) \diffop x = \int_0^{N+1} \sum_{n=1}^\infty (-1)^{n+1} \cdot \rez{n} \cdot \ind_{(n,n+1)}(x) \diffop x = \sum_{n=1}^N (-1)^{n+1} \cdot \rez{n} \xrightarrow{N \to \infty} \ln 2. \]
Die Reihe konvergiert nach dem Leibniz-Kriterium.

Die Indikatorfunktion $f$ der Menge der rationalen Zahlen in $[0,1]$ ist Lebesgue-messbar, $f=0$ fast überall und damit ist sie Lebesgue-integrierbar mit $\int f = 0$. $f$ ist keinem Punkt stetig, also auch nicht Riemann-integrierbar.
\[ 0 = i(f) \ne I(f) = 1. \]

\section{Aufgaben}
Siehe \verb+Aufgaben-2-6.pdf+.

\section{Konvergenzarten}
In diesem Abschnitt ist $X$ eine beliebige Menge, $\{ f_n \}$ eine Folge von komplexen Funktionen auf $X$.

$f_n \to t$, $n \to \infty$ kann bedeuten:
\begin{itemize}
 \item Punktweise Konvergenz: 
  \[ \lim_{n \to \infty} f_n(x) = f(x) \text{ für alle } x \in X, \]
 \item Gleichmäßige Konvergenz:
  \[ \lim_{n \to \infty} \| f_n - f \| = 0, \text{ wobei } \| g \| := \sup_{x \in X} | g(x) |. \]
\end{itemize}

Ist $(X, \mA, \mu)$ ein Maßraum, so können wir auch über \emph{Konvergenz fast überall} oder $L^1$-Konvergenz sprechen:
\[ \lim_{n \to \infty} \| f_n - f \|_1 = 0, \text{ wobei } \| g \|_1 := \int_X |g| \diffop \mu. \]

Die folgenden Beispiele sind nützlich ($X = \real$):
\begin{enumerate}[(i)]
 \item $f_n = \rez{n} \cdot \ind_{(0, n]}$, $\int f_n \diffop \lambda = 1$,
 \item $f_n = \ind_{(n,n+1)}$, $\int f_n \diffop \lambda = 1$,
 \item $f_n = n \cdot \ind_{(0,\rez{n}]}$, $\int f_n \diffop \lambda = 1$,
 \item $f_n = \ind_{[j/2^k,(j+1)/2^k]}$, wobei $k \in \nat$, $0 \le j \le 2^k$, $n = j + 2^k$, $\int f_n \diffop \lambda = \rez{2^k}$ (``Wandernde Hüte'')
\end{enumerate}

In (i), (ii) und (iii) gilt $f_n \to 0$ punktweise, in (i) sogar gleichmäßig, aber nicht in (ii) und (iii).

In (i), (ii) und (iii) konvergiert $f_n$ fast überall, aber keine $L^1$-Konvergenz, da $\| f_n - f \|_1 = \| f_n \|_1 = 1 \nrightarrow 0$ für $n \to \infty$.

In (iv) gilt dagegen $f_n \to 0$ in der $L^1$-Metrik (Norm), aber die Folge konvergiert \emph{in keinem Punkt}.

\textbf{Andererseits:} Wenn $f_n \to f$ fast überall und $| f_n | \le g \in L^1$ für alle $n$, dann $f_n \to f$ in $L^1$.
\[ |f_n - f| \le | f_n | + | f | \le g + g \text{ f. ü. } \qRq \lim_{n \to \infty} \int | f_n - f | = \int \underbrace{\lim_{n \to \infty} | f_n - f |}_{= 0 \text{ f. ü.}} = 0.\]

\begin{defn}
 Eine Folge $\{f_n\}$ messbarer, komplexer Funktionen auf einem Maßraum $(X,\mA,\mu)$ ist eine \emph{Cauchy-Folge nach Maß}, wenn für jedes $\eps > 0$ gilt:
 \[ \mu( \{ x : | f_n(x) - f_m(x) | \ge \eps \} ) \to 0; \quad n, m \to \infty. \]
 Die Folge $\{ f_n \}$ \emph{konvergiert nach Maß} gegen eine messbare Funktion $f$, wenn
 \[ \mu( \{ x : | f_n(x) - f(x) | \ge \eps \} ) \to 0; \quad n \to \infty. \]
\end{defn}

\textbf{Beispiel.} Die Folgen (i), (iii) und (iv) (\emph{nicht} (ii)) konvergieren gegen 0 nach Maß; (ii) ist keine Cauchy-Folge nach Maß.

\begin{lem}
 Gilt $f_n \to f$ in $L^1$, so gilt auch $f_n \to f$ nach Maß.
\end{lem}

\begin{proof}
 Definiere $E_{n,\eps} := \{ | f_n(x) - f(x) | \ge \eps \}$. Es gilt:
 \[ \underbrace{\int | f_n - f |}_{\to 0 \text{ für } n \to \infty} \ge \int_{E_{n,\eps}} | f_n - f | \ge \eps \cdot \mu( E_{n,\eps} ). \qedhere \]
\end{proof}

Die Umkehrung ist falsch, siehe die Beispiele (i) und (iii).

\begin{thm}
 Sei $\{ f_n \}$ eine Cauchy-Folge nach Maß. Dann existiert eine messbare Funktion $f$, sodass $f_n \to f$ nach Maß, und eine Teilfolge $\{ f_{nj} \}$ von $\{ f_n \}$, die fast überall gegen $f$ konvergiert. Gilt $f_n \to g$ nach Maß, so ist $g = f$ fast überall.
\end{thm}

\begin{proof}
 Wir wählen eine Teilfolge $\{ g_j \} := \{ f_{nj} \}$, sodass $\mu( E_j ) \le \rez{2^j}$, wobei 
 \[ E_j := \{ x : | g_j(x) - g_{j+1}(x) | \ge \rez{2^j} \}, \quad j \in \nat. \]
 Mit $F_k := \bigcup_k^\infty E_j$ ist
 \[ \mu( F_k ) \le \sum_{j=k}^\infty \rez{2^j} = 2^{1-k}. \]
 Für $k \le j \le i$ und $x \notin F_k$ gilt
 \[ | g_j(x) - g_i(x) | \le \sum_{l=j}^{i-1} | g_{l+1}(x) - g_l(x) | \le \sum_{l=j}^{i-1} \rez{2^l} = 2^{1-j}. \]
 Also ist $\{g_j\}$ punktweise Cauchy auf $F_k^c$. Mit $F := \bigcap_1^\infty F_k$ ist $\mu(F) = 0$ und damit ist die Funktion
 \[ f(x) := \begin{cases} \lim_{i \to \infty} g_j(x), &x \notin F, \\ 0 &x \in F \end{cases} \]
 messbar und $g_j \to f$ fast überall.
 
 Die obige Ungleichung zeigt, dass
 \[ | g_j(x) - f(x) | \le 2^{1-j}, \quad j \ge k, \quad x \notin F_k. \]
 Wegen $\mu(F_k) \to 0$, $k \to \infty$ folgt hieraus, dass $g_j \to f$ nach Maß. Dann gilt aber auch $f_n \to f$ nach Maß, da
 \[ \{ x : | f_n(x) - f(x) | \ge \eps \} \subset \left\{ x : | f_n(x) - g_j(x) | \ge \frac{\eps}{2} \right\} \cup \left\{ x : | g_j(x) - f(x) | \ge \frac{\eps}{2} \right\} \]
 für alle $n$.
 
 Nehmen wir an, dass $f_n \to g$ nach Maß. Dann gilt
 \[ \{ x : | f(x) - g(x) | \ge \eps \} \subset \left\{ x : | f(x) - f_n(x) | \ge \frac{\eps}{2} \right\} \cup \left\{ x : | f_n(x) - g(x) | \ge \frac{\eps}{2} \right\} \]
 für alle $n$. Also ist
 \[ \mu( \{ x : | f(x) - g(x) | \ge \eps \} ) = 0 \]
 für alle $\eps > 0$. Damit gilt $f = g$ fast überall.
\end{proof}

\begin{thm}[Jegorow]
 Sei $(X, \mA, \mu)$ ein endlicher Maßraum und $f_n, f$ messbar mit $f_n \to f$ fast überall.
 
 Dann gilt: Für alle $\eps > 0$ existiert $E \in \mA$ mit $\mu(E) < \eps$, so dass $f_n \to f$ gleichmäßig auf $E^c$, das heißt
 \[ \sup_{x \in E^c} | f_n(x) - f(x) | \xrightarrow{n \to \infty} 0. \]
\end{thm}

\begin{proof}
 Indem wir $f_n$, $f$ gegebenenfalls auf Nullmengen modifizieren, können wir o.B.d.A. annehmen, dass $f_n \to f$ (überall).
 
 Definiere 
 \[ E_n(h) := \bigcup_{m \ge n} \left\{x_i | f_m(x) -f(x) | \ge \rez{h} \right\}. \]
 Es gilt $E_{n+1}(h) \subseteq E_n(h)$ und $\bigcup_{n \ge 1} E_n(h) = \emptyset$ (für $x \in X$ existiert wegen $f_n(x) \to f(x)$ also ein $N \in \nat$, sodass $| f_n(x) - f(x) | < \rez{h}$ für $n \ge N$, folglich ist $x \notin E_{n+1}(h)$).
 
 Da $\mu$ ein endliches Maß ist, folgt aus der Stetigkeit von oben:
 \[ \mu( E_n(h) ) \xrightarrow{n \to \infty} \mu(\emptyset) = 0. \]
 Für $\eps > 0$, $k \in \nat$, wähle $n_k = n_k(\eps)$ mit
 \[ \mu( E_{n_k}(k) < \eps \cdot 2^{-k}. \]
 Für $E := \bigcup_{k \ge 1} E_{n_k}(k)$ gilt $\mu(E) < \eps$, denn
 \[ \mu(E) \le \sum_{k\ge 1} \underbrace{\mu( E_{n_k}(k) )}_{< \eps \cdot 2^{-k}} < \eps \]
 und $| f_n(x) - f(x) | < \rez{k}$ für alle $x \in E^c$ und $n \ge n_k$.
\end{proof}

\section{Aufgaben}
Siehe \verb+Aufgaben-2-8.pdf+.

\section{Produktmaße}
In diesem Abschnitt sind $(X, \mM, \mu)$ und $(Y, \mN, \eta)$ Maßräume.

Wir haben bereits die Produkt-$\sigma$-Algebra
\[ \mM \otimes \mN := \sigma( \{ M \times N; M \in \mM, N \in \mN \} ) \]
auf $X \times Y$ konstruiert. Wir werden jetzt ein Maß auf $\mM \otimes \mN$ konstruieren, das in natürlicher Weise als Produkt von $\mu$ und $\eta$ angesehen werden kann.

Ein Rechteck $R$ ist eine Menge der Form $A \times E$ für $A \in \mM$, $E \in \mN$.

Natürliche Idee: Das Produktmaß eines solchen Rechtecks durch $\mu(A) \cdot \eta(E)$ zu definieren.

\begin{lem}
 Die Familie $\mA$ aller endlichen disjunkten Vereinigungen von Rechtecken ist eine Algebra und ein Erzeuger von $\mM \otimes \mN$.
\end{lem}

\begin{proof}
 Die erste Aussage folgt aus
 \[ (A \times E) \cap (B \times F) = (A \cap B) \times (E \cap F) \]
 wegen Lemma 1.11.1, dass
 \[ (A \times E)^C = (X \times E^C) \cup (A^C \times Y). \]
 
 Die zweite Aussage folgt aus der Definition von $\mM \otimes \mN$.
\end{proof}

\begin{lem}
 Sei $B \in \mA$ die disjunkte Vereinigung der Rechtecke $A_1 \times E_1, \ldots, A_n \times E_n$. Dann hängt die Zahl
 \[ \pi( B ) := \sum_{j=1}^n \mu(A_j) \eta(E_j) \]
 nur von $B$ ab, nicht aber von der speziellen Wahl der $A_j$ und $E_j$. Weiterhin ist $\pi$ ein Prämaß auf $\mA$.
\end{lem}

\begin{proof}
 Nehmen wir an, dass $B = A \times E \ne \emptyset$. Für $x \in X, y \in Y$ gilt
 \[ \ind_A(x) \cdot \ind_E(y) = \ind_{A \times E}(x,y) = \sum_{j=1}^n \ind_{A_j \times E_j} (x,y) = \sum_{j=1}^n \ind_{A_j}(x) \cdot \ind_{E_j}(y). \]
 Wir integrieren bezüglich $x$ und verwenden Satz 2.3.5:
 \[ \begin{aligned}
     \mu(A) \cdot \ind_E (y) 
     &= \int \int_A(x) \diffop \mu(x) \cdot \ind_E(y) \\
     &= \int \sum_{j=1}^n \ind_{A_j} (x) \cdot \ind_{E_j}(y) \diffop \mu(x) \\
     &= \sum_{j=1}^n \mu(A_j) \cdot \ind_{E_j} (y).
    \end{aligned} \]
 Wir integrieren bezüglich $y$:
 \[ \mu(A) \cdot \eta(E) = \sum_{j=1}^n \mu(A_j) \eta(E_j). \]
 Damit ist der Spezialfall bewiesen.
 
 Aus diesem Spezialfall folgt: Betrachten wir zwei Darstellungen einer Menge $B \in \mA$, wobei die eine Darstellung eine Verfeinerung der anderen ist, so sind die Summen gleich.
 
 Der allgemeine Fall folgt nun aus der Tatsache, dass zwei beliebige Darstellungen einer Menge $B \in \mA$ immer eine gemeinsame Verfeinerung besitzen.
 
 Die letzte Aussage folgt daraus, dass die obigen Gleichungen auch für abzählbar viele $A_j$, $E_j$ gelten.
\end{proof}

\begin{defn}
 Nach dem vorliegenden Lemma und Satz 1.7.7 lässt sich $\pi$ zu einem Maß auf $\mM \otimes \mN$ fortsetzen. Dieses Maß heißt das \emph{Produkt} von $\mu$ und $\eta$. Es wird mit $\mu \times \eta$ bezeichnet.
\end{defn}

\begin{rmrk}
 Sind $\mu$ und $\eta$ $\sigma$-endlich, das heißt $X = \bigcup_j A_j$, $Y = \bigcup_j E_j$ mit $\mu(A_j) < \infty$, $\eta(E_j) < \infty$, dann gilt:
 \[ X \times Y = \bigcup_j \bigcup_k (A_j \times E_k), \]
 also $\mu \times \eta$ $\sigma$-endlich.
 
 Nach Satz 1.7.7 ist dann $\mu \times \eta$ das \emph{einzige} Maß auf $\mM \otimes \mN$ mit 
 \[ (\mu \times \eta)(A \times E) = \mu(A) \cdot \eta(E) \]
 für alle Rechtecke $A \times E$.
\end{rmrk}

\begin{rmrk}
 Dieselbe Konstruktion kann man mit endlich vielen Faktoren durch\-führen.
 
 Seien $(X_j, \mM_j, \mu_j)$, $j = 1, \ldots, n$ Maßräume. Unter einem Rechteck verstehen wir Mengen der Form
 \[ A_1 \times \ldots \times A_n, \quad A_j \in \mM_j. \]
 
 Das Analogon von Lemma 2.9.1 bleibt gültig und dieselbe Konstruktion liefert dann ein Maß $\mu_1 \times \ldots \times \mu_n$ auf $\mM_1 \otimes \ldots \otimes \mM_n$ mit
 \[ (\mu_1 \times \ldots \times \mu_n)( A_1 \times \ldots \times A_n ) = \prod_{j=1}^n \mu_j(A_j). \]
 Sind die $\mu_j$ $\sigma$-endlich, so ist auch $\mu_1 \times \ldots \times \mu_n$ $\sigma$-endlich.
 
 Im Falle der Eindeutigkeit gelten die Assoziativgesetze. 
 
 Beispiel: Wenn wir $X_1 \times X_2 \times X_3$ mit $(X_1 \times X_2) \times X_3$ identifizieren, dann gilt
 \[ \mM_1 \otimes \mM_2 \otimes \mM_3 = (\mM_1 \otimes \mM_2) \otimes \mM_3. \]
 Die Gleichheit $\mu_1 \times \mu_2 \times \mu_3 = ( \mu_1 \times \mu_2 ) \times \mu_3$ folgt nun daraus, dass diese Maße auf Mengen der Form $A_1 \times A_2 \times A_3$ übereinstimmen, also wegen der Eindeutigkeit gleich sind.
\end{rmrk}

Im Weiteren betrachten wir nur zwei Faktoren.

Für $E \subseteq X \times Y$, $x \in X$, $y \in Y$ schreiben wir
\[ E_x := \{ y \in Y; (x,y) \in E \}, \quad E_y := \{ x \in X; (x,y) \in E \}. \]

Für eine Funktion $f: X \times Y \to \real$ sei
\[ f_x(y) := f^y(x) := f(x,y). \]

\begin{lem}
 \begin{enumerate}[(i)]
  \item Wenn $E \in \mM \otimes \mN$, dann gilt $E_x \in \mN$ und $E^y \in \mM$ für alle $x \in X$, $y \in Y$.
  \item Ist $f$ $\mM \otimes \mN$-messbar, so ist $f_x$ $\mN$-messbar für alle $x \in X$ und $f^y$ $\mM$-messbar für alle $y \in Y$.
 \end{enumerate}
\end{lem}

\begin{proof}
 \begin{enumerate}[(i)]
  \item Sei 
   \[ \mR := \{ E \subseteq X \times Y, E_x \in \mN, E^y \in \mN \text{ für alle } x,y \}. \]
   Dann enthält $\mR$ alle Rechtecke (so ist $(A \times B)_x = B$ oder $(A \times B)_x = \emptyset$ je nachdem, ob $x \in A$ oder $x \notin A$).
   
   Wegen $(\bigcup_j E_j)_x = \bigcup_j (E_j)_x$ und $(E_x)^C = (E^C)_x$ ist $\mR$ eine $\sigma$-Algebra, also $\mM \otimes \mN \subseteq \mR$.
  \item Folgt aus (i) wegen
   \[ (f_x^{-1})(B) = \{ y \in Y; f(x,y) \in B \} = \{ y: (x,y) \in f^{-1}(B) \} = (\underbrace{f^{-1}(B)}_{\in \mM \otimes \mN}) \in \mN. \]
   Analog für $f^y$. \qedhere
 \end{enumerate}
\end{proof}

\begin{rmrk}
 Auch wenn $\mu$ und $\eta$ vollständig sind, ist das Produktmaß $\mu \times \eta$ im Allgemeinen nicht vollständig.
 
 \textbf{Beispiel:} Nehmen wir an, dass $E \in \pot(Y) \setminus \mN$ und eine nichtleere Menge $A \in \mM$ mit $\mu(A) = 0$ existieren (z.B. für $X=Y=\real$, $\mu = \eta$ Lebesgue-Maß). Nach dem Lemma 2.9.6 ist $A \times E \notin \mM \otimes \mN$, obwohl $A \times E \subseteq A \times Y$ und $(\mu \times \eta)(A \times Y) = 0$.
\end{rmrk}

\begin{defn}
 Eine \emph{monotone Klasse} auf $X$ ist eine Familie $\mC \subseteq \pot(X)$, die abgeschlossen ist bezüglich abzählbarer, monoton wachsender Vereinigungen und abzählbarer, monoton fallender Durchschnitte. Jede $\sigma$-Algebra ist eine monotone Klasse und der Durchschnitt von monotonen Klassen ist eine monotone Klasse. Folglich existiert für eine beliebige Familie $\mE \subseteq \pot(X)$ eine eindeutige kleinste monotone Klasse, die $\mE$ enthält, die durch $\mE$ erzeugte monotone Klasse.
\end{defn}

\begin{lem}
 Ist $\mA$ eine Algebra, so ist die durch $\mA$ erzeugte monotone Klasse $\mC$ gleich der $\sigma$-Algebra $\mM$, die durch $\mA$ erzeugt wird.
\end{lem}

\begin{proof}
 Da $\mM$ eine monotone Klasse ist, die $\mA$ enthält, gilt $\mC \subseteq \mM$. 
 
 Für $E \in \mC$ sei:
 \[ \mC(E) := \{ F \in \mC; E \setminus F, F \setminus E, F \cap E \in \mC \}. \]
 Offenbar sind $\emptyset, E \in \mC(E)$.
 
 Wenn $E \in \mA$, dann gilt $F \in \mC(E)$ für alle $F \in \mA$ (da $\mA$ eine Algebra ist). Dann $E \in \mC(F)$ genau dann, wenn $F \in \mC(E)$. Weiterhin ist $\mC(E)$ eine monotone Klasse. 
 
 Ist $E \in \mA$, dann gilt 
 \[ \mA \subseteq \mC(E) \qRq \mC \subseteq \mC(E), \]
 das heißt wenn $F \in \mC$, dann gilt $F \in \mC(E)$ für alle $E \in \mA$. Also $\mA \subseteq \mC(F)$ für alle $F \in \mC$ und damit $\mC \subseteq \mC(F)$ für alle $F \in \mC$.
 
 Hieraus folgt, dass für alle $E,F \in \mC$ gilt $E \setminus F, F \setminus E, E \cap F \in \mC$. Wegen $X \in \mA \subseteq \mC$ folgt, dass $\mC$ eine Algebra ist. 
 
 Wenn $(E_j) \in \mC$, dann ist auch $F_n := \bigcup_1^n E_j \in \mC$. Damit folgt aus der Abgeschlossenheit von $\mC$ unter monoton wachsenden Vereinigungen, dass $\bigcup_n F_n = \bigcup_j E_j \in \mC$. Folglich ist $\mC$ eine $\sigma$-Algebra, also $\mM \subseteq \mC$.
 
 Aus $\mC \subseteq \mM$ und $\mM \subseteq \mC$ folgt $\mM = \mC$.
\end{proof}

\begin{thm}
 Seien $(X,\mM,\mu)$ und $(Y,\mN,\nu)$ $\sigma$-endliche Maßräume und $E \in \mM \otimes \mN$.
 
 Die Funktionen $x \mapsto \nu(E_x)$ und $y \mapsto \mu(E^y)$ sind messbar auf $X$ bzw. $Y$ und
 \begin{align*}
  (\mu \times\nu)(E) &= \int \nu(E_x) \diffop \mu(x), \tag{1} \\
  (\mu \times\nu)(E) &= \int \mu(E^y) \diffop \nu(y). \tag{2}
 \end{align*}
\end{thm}

\begin{proof}
 \textbf{1. Fall:} $\mu$ und $\nu$ endlich. 
 
 Sei $\mC$ die Menge aller $E \in \mM \otimes \mN$, für welche die Aussage des Satzes gilt.
 
 Rechtecke der From $E= A \times B$, $A \in \mM$, $B \in \mN$, gehören zu $\mC$, da
 \[ \nu(E_x) = \ind_A (x) \nu(B), \qquad \mu(E^y) = \ind_B (y) \mu(A). \]
 Aus der Additivität folgt, dass endliche, disjunkte Vereinigungen von Rechtecken zu $\mC$ gehören.
 
 Nach Lemma 2.9.1 und Lemma 2.9.9 genügt es zu zeigen, dass $\mC$ eine monotone Klasse ist. Ist $(E_n)_{n \in \nat}$ eine monoton wachsende Folge in $\mC$ und $E = \bigcup_{n \in \nat} E_n$, so sind $f_n: y \mapsto \mu(E_n^y)$ messbar und konvergieren punktweise und monoton wachsend gegen $f: y \mapsto \mu(E^y)$.
 
 Folglich ist $f$ messbar und nach dem Satz von der monotonen Konvergenz (m) und wegen $E_n \in \mC$ (c) gilt
 \[ \begin{aligned}
     \int \mu(E^y) \diffop \nu(y) 
     &\overset{(\text{m})}{=} \lim_{n \to \infty} \int \mu((E_n)^y) \diffop \nu(y) \\
     &\overset{(\text{c})}{=} \lim_{n \to \infty} \int (\mu \times \nu)(E_n) \\
     &= (\mu \times \nu)(E).
    \end{aligned} \]
 Analog sieht man, dass
 \[ (\mu \times \nu)(E) = \int \nu(E_x) \diffop \mu(x) \]
 gilt. Also ist $E \in \mC$.
 
 Ist $(E_n)_{n \in \nat}$ eine monoton fallende Folge von $\mC$ und $E = \bigcap_{n \in \nat} E_n$, so gilt $\mu((E_n)^y) \le \mu(X) < \infty$ und $\nu(Y) < \infty$. Daher ist die Funktion $y \mapsto \mu((E_n)^y)$ in $\lebesgue'(\nu)$. Damit liefert analog zum vorhergehenden Schritt der Satz über dominierte Konvergenz, dass $E \in  \mC$ gilt.
 
 \textbf{2. Fall:} $\mu, \nu$ $\sigma$-endlich.
 
 Dann lässt sich $X \times Y$ als Vereinigung einer monoton wachsenden Folge $(X_j,Y_j)_{j \in \nat}$ von Rechtecken mit endlichem Maß darstellen.
 
 Ist $E \in \mM \otimes \mN$, so können wir die Überlegungen des ersten Falls auf $E \cap (X_j \times Y_j)$, $j \in \nat$ anwenden:
 \[ \begin{aligned}
     (\mu \times \nu)(E \cap (X_j \times Y_j)) 
     &= \int \ind_{X_j}(x) \nu( E_x \cap Y_j ) \diffop \mu(x) \\
     &= \int \ind_{Y_j}(y) \mu( E^y \cap X_j ) \diffop \nu(x).
    \end{aligned} \]
 Mit dem Stetigkeitssatz (Anwenden auf die linke Seite) und dem Satz über monotone Konvergenz (rechte Seite) liefert eine Grenzwertbildung $j \to \infty$ die Behauptung.
\end{proof}

\begin{thm}[Fubini-Tonelli]
 Seien $(X,\mM,\mu)$ und $(Y,\mN,\nu)$ $\sigma$-endliche Maßräume.
 
 \begin{enumerate}[(i)]
  \item Tonelli: Ist $f \in \lebesgue^+(X \times Y)$, so sind die Funktionen $g:x \mapsto \int f_x \diffop \nu$ und $h:y \mapsto \int f^y \diffop \mu$ in $\lebesgue^+(X)$ bzw. $\lebesgue^+(Y)$ und
  \begin{align*}
   \int f \diffop (\mu \times \nu)
     &= \int \left[ \int f(x,y) \diffop \nu(y) \right] \diffop \mu(x), \tag{1} \\
     &= \int \left[ \int f(x,y) \diffop \mu(x) \right] \diffop \nu(y). \tag{2}
  \end{align*}
  \item Fubini: Ist $f \in \lebesgue'(\mu \times \nu)$, so gilt $f_x \in \lebesgue'(\nu)$ für fast alle $x \in X$, $f^y \in \lebesgue'(\mu)$ für fast alle $y \in Y$. Die fast überall definierten Funktionen $g:x \mapsto \int f_x \diffop \nu$ und $h:y \mapsto \int f^y \diffop \mu$ sind in $\lebesgue'(X)$ bzw. $\lebesgue'(Y)$ und es gelten die Gleichungen (1) und (2).
 \end{enumerate}
\end{thm}

\begin{proof}
 \begin{enumerate}[(i)]
  \item Ist $f$ Indikatorfunktion, so folgt die Aussage aus dem gerade bewiesenen Satz 2.9.10. Wegen Linearität gilt (i) damit für alle nicht-negativen, einfachen Funktionen.
 
  Sei $f \in \lebesgue^+(X \times Y)$ beliebig. Wir wählen eine Folge $(f_n)_{n \in \nat}$ von nicht-negativen, einfachen Funktionen, die punktweise und monoton wachsend gegen $f$ konvergieren (Satz 2.1.15).
  
  Nach dem Satz über monotone Konvergenz (m) konvergieren die zugehörigen $g_n$ und $h_n$ gegen $g$ und $h$. Folglich sind $g$ und $h$ messbar und es gilt
  \[ \int g \diffop \mu \overset{(\text{m})}{=} \lim_{n \to \infty} g_n \diffop \mu \overset{(1)}{=} \lim_{n \to \infty} \int f_n \diffop (\mu \times \nu) \overset{(\text{m})}{=} \int f \diffop (\mu \times \nu), \]
  \[ \int h \diffop \nu \overset{(\text{m})}{=} \lim_{n \to \infty} h_n \diffop \nu \overset{(2)}{=} \lim_{n \to \infty} \int f_n \diffop (\mu \times \nu) \overset{(\text{m})}{=} \int f \diffop (\mu \times \nu). \]
  \item Aus (i) erhalten wir: Ist $f \in \lebesgue^+ (X \times Y)$ und $\int f \diffop (\mu \times \nu) < \infty$, so gilt fast überall $g < \infty$, $h < \infty$.
  
  Also ist $f_x \in L'(\nu)$ für fast alle $x$ bzw. $f^y \in L'(\mu)$ für fast alle $y$.
  
  Ist nun $f \in L'(\mu \times \nu)$ beliebig, so können wir die gerade
  gezeigten Aussagen für $(\Re f)^\pm$ bzw. $(\Im f)^\pm$ anwenden. 
  \qedhere
 \end{enumerate}
\end{proof}

\begin{rmrk}
 \begin{enumerate}[(a)]
  \item Die Klammern ``$[\,]$'' in (1) und (2) werden gelegentlich auch
    weggelassen, man schreibt auch 
   \[ \int \left[ \int f(x,y) \diffop \mu(x) \right] \diffop \nu(y) 
      = \iint f(x,y) \diffop \mu(x) \diffop \nu(y)
      = \iint f \diffop \mu \diffop \nu. \]
  \item Sei $X = Y = [0,1]$, $\mM = \mN = \borel([0,1])$, $\mu = \lambda
    |_{\borel([0,1])}$, $\nu$ Zählmaß und $D = \{ (x,y) \in [0,1] \times [0,1] :
    x = y  \}$. 
  
  Dann gilt 
  \[ \iint \ind_D \diffop \lambda \diffop \nu = 0, \qquad \iint \ind_D \diffop \nu \diffop \lambda = 1. \]
  Man kann sogar zeigen, dass
  \[ \int \ind_D \diffop (\lambda \times \nu) = (\lambda \times \nu)(D) = \infty. \]
  
  Also darf die $\sigma$-Additivität für die Anwendbarkeit des Satzes nicht
  weggelassen werden! 
  \item Die Sätze von Fubini und Tonelli werden oft zusammen verwendet. Man
    möchte häufig die Reihenfolge der Integration in einem Doppelintegral $\iint
    f  \diffop \mu \diffop \nu$ vertauschen. Zuerst zeigt man, dass $\iint |f|
    \diffop \mu \diffop \nu < \infty$, indem man das Integral nach Tonelli als
    iteriertes Integral auswertet. Danach wendet man den Satz von Fubini an. 
 \end{enumerate}
\end{rmrk}

Eine ``vollständige'' Version des Satzes von Fubini-Tonelli folgt ohne Beweis.
\begin{thm}
 Seien $(X, \mM, \mu)$ und $(Y, \mN, \nu)$ \emph{vollständige} $\sigma$-endliche Maßräume und sei $(X \times Y, \mA, \sigma)$ die Vervollständigung von $(X \times Y, \mM \otimes \mN, \mu \times \nu)$. Ist $f$ $\mA$-messbar und 
 \[ \text{entweder} \quad \text{(a)} \quad f \ge 0 \quad \text{oder} \quad \text{(b)} \quad f \in L'(\sigma), \] 
 so gilt: $f_x$ ist $\mN$-messbar für fast alle $x$ bzw. $f^y$ ist $\mM$-messbar für fast alle $y$. Im Falle (b) sind diese Funktionen fast überall integrierbar.
 
 Weiterhin sind $x \mapsto \int f_x \diffop \nu$ und $y \mapsto \int f^y \diffop \mu$ messbar, im Falle (b) auch integrierbar und
 \begin{align*}
  \int f \diffop \sigma
  &= \int \left[ \int f(x,y) \diffop \nu(y) \right] \diffop \mu(x) \tag{1}\\
  &= \int \left[ \int f(x,y) \diffop \mu(x) \right] \diffop \nu(y). \tag{2}
 \end{align*}
\end{thm}

\section{Aufgaben}
Siehe \verb+Aufgaben-2.10.pdf+.

\section{Das Lebesgue-Integral auf \texorpdfstring{$\real^n$}{IRn}}
\begin{defn}
 Das \emph{Lebesgue-Maß auf $\real^n$} ist die Vervollständigung des
 Produktmaßes $\lambda \times \ldots \times \lambda$ auf $\borel(\real) \otimes
 \ldots \otimes \borel(\real)$. 
\end{defn}
 
Die $\sigma$-Algebra der $\lambda_n$-messbaren Mengen wird mit $\lebesgue_n$
bezeichnet, ihre Elemente heißen \emph{Lebesgue-messbare Mengen}. 
 
Wenn keine Verwechselungsgefahr besteht, werden wir einfach $\lambda$ statt
$\lambda_n$ und $\int f(x) \diffop x$ statt $\int f \diffop \lambda_n$
schreiben.
 
Ist $E  = \prod_1^n E_j$ ein Rechteck in $\real^n$, so wird $E_j \subseteq
\real$ als \emph{Seite} von $E$ bezeichnet.

\clearpage

\begin{thm}\label{thm:2-11-2}
  Sei $E \in \lebesgue_n$. Dann gelten:
  \begin{enumerate}[(i)]
  \item $\lambda(E) = \inf \{ \lambda(U) : U \supset E, U$ offen $\} = \sup \{
    \lambda(K) : K \subset E, K $ kompakt $\}$.
  \item $E = A_1 \cup N_1 = A_2 \setminus N_2$, wobei $A_1$ eine
    $F_\sigma$-Menge, $A_2$ eine $G_\delta$-Menge und $N_1, N_2 \in \lebesgue_n$
    mit $\lambda(N_1) = \lambda(N_2)=0$.
  \item Ist $\lambda(E) < \infty$, so existiert für jedes $\eps > 0$ eine
    endliche Familie $\{ R_1, \ldots, R_N\}$ von disjunkten Rechtecken $R_j$,
    deren Seiten Intervalle sind, so dass
    \[ \lambda \left(E \Delta \bigcup_{j=1}^\infty R_j \right) < \eps. \]
  \end{enumerate}
\end{thm}

\begin{proof}
  Zu (i): Sei $\eps > 0$ beliebig. Nach Definition des Produktmaßes existiert
  eine Folge $(R_j)_j$ von Rechtecken, so dass $E \subset \bigcup_1^\infty
  R_j$ und $\sum_1^\infty \lambda R_j \le \lambda(E) + \eps$.

  Wir wenden Satz 1.11.6 für die Seiten der $R_j$ an. Für jedes $j \in \nat$
  existiert ein Rechteck $S_j \supset R_j$, dessen Seiten offene Mengen
  sind, mit $\lambda(S_j) \le \lambda(R_j) + \frac{\eps}{2^j}$.

  Die Menge $U := \bigcup_1^\infty S_j$ ist offen und
  \[ \lambda(U) \le  \sum_{j=1}^\infty \lambda (S_j) \le \sum_{j=1}^\infty
    \lambda (R_j) + \eps = \lambda(E) + 2 \eps. \]
  Hieraus folgt die erste Gleichung in (i).

  Die zweite Gleichung in (i) und (ii) werden genauso wie in den Sätzen
  1.11.6 und 1.12.3 bewiesen.

  Zu (iii): Sei $\lambda(E) < \infty$ und $S_j$ wie oben. Dann ist $\lambda(S_j)
  < \infty$ und $E \subset \bigcup_1^\infty S_j$.

  Die Seiten von $S_j$ sind abzählbare Vereinigungen offener Intervalle. Indem
  wir geeignete Untervereinigungen auswählen, erhalten wir Rechtecke $T_j
  \subset S_j$, deren Seiten endliche Vereinigungen von Intervallen sind und
  $\lambda(T_j) > \lambda(S_j) - \frac{\eps}{2^j}$.

  Ist $N$ hinreichend groß, so gilt
  \[ \begin{aligned}
      \lambda \left( E \Delta \bigcup_{j=1}^N \right)
      &= \lambda \left( E \setminus \bigcup_{j=1}^N T_j \right)
      + \lambda \left( \bigcup_{j=1}^N T_j \setminus E \right) \\
      &\le \underbrace{\lambda \left( \bigcup_{j=1}^N S_j \setminus T_j
        \right)}_{\le \sum_1^\infty e/2^j}
      + \underbrace{\lambda \left( \bigcup_{j=N+1}^\infty S_j \right)}_{< \eps}
      + \underbrace{\lambda \left( \bigcup_{j=1}^\infty S_j \setminus E \right)}_{< 2
        \eps} \\
      &\le 4 \eps.
    \end{aligned} \]
  Die Aussage folgt nun daraus, dass $\bigcup_1^N T_j$ als endliche disjunkte
  Vereinigung von Rechtecken $R_j$ dargestellt werden kann, deren Seiten
  Intervalle sind.
\end{proof}

\begin{thm}
  Für $f \in \intf^1(\lambda)$ und $\eps > 0$ existiert eine Funktion $\varphi
  := \sum_1^N a_j \ind_{R_j}$, $a_j \in \complex$, wobei $R_j$ ein Produkt von
  Intervallen ist, mit
  \[ \int |f - \varphi| \diffop \lambda < \eps. \]
  Weiterhin existiert eine stetige Funktion $g$ mit kompaktem Träger, so dass
  \[ \int |f-g| \diffop \lambda < \eps. \]
\end{thm}
\begin{proof}
  Wie im Beweis zu Satz \ref{thm:2-5-11} approximieren wir zuerst $f$ durch
  einfache Funktionen. Dann wenden wir (iii) aus Satz \ref{thm:2-11-2} an, um
  die einfache Funktion durch Funktionen $\varphi$ der gewünschten Form zu
  approximieren.

  Für die letzte Aussage approximieren wir die $\varphi$ durch stetige
  Funktionen ähnlich wie im Beweis zu Satz \ref{thm:2-5-11}.
\end{proof}

\begin{thm}[Verschiebungsinvarianz des Lebesgue-Maßes]\label{thm:2-11-4}
  Ist $E \in \lebesgue_n$, $x \in \real^n$, so gilt $E+x\in \lebesgue_n$ und
  $\lambda(E+x) = \lambda(E)$.

  Ist $f$ eine Lebesgue-messbare Funktion und entweder $f \ge 0$ oder $f$
  Lebesgue-integrierbar, so ist $y \mapsto f(x+y)$ Lebesgue-messbar und
  \[ \int f(x+y) \diffop y = \int f(y) \diffop y. \]
\end{thm}

\begin{proof}
  Ist $E$ ein Rechteck, dann folgt die Aussage sofort aus dem eindimensionalen
  Resultat (siehe Satz 1.11.7). 

  Wegen der Konstruktion des Produktmaßes folgt die Aussage für Borel-Mengen.
  Speziell ist die Familie der Borel-Nullmengen verschiebungsinvariant, woraus
  die Aussage für beliebige Lebesgue-messbare $E$ folgt.

  Für den zweiten Teil des Satzes: Ist $f$ eine Indikatorfunktion, so folgt die
  Aussage aus dem ersten Teil des Satzes. Wegen der Linearität ist sie für alle
  einfachen Funktionen gültig. Der allgemeine Fall folgt aus der Definition des
  Integrals.
\end{proof}

\begin{prgp}\label{prgp:2-11-5}
  Als nächstes untersuchen wir das Verhalten des Lebesgue-Maßes unter linearen
  Transformationen. Dazu identifizieren wir eine lineare Abbildung $T: \real^n
  \to \real^n$ mit der Matrix
  \[ (T_{ij}) = (\angles{T(e_j) \cdot e_j})_{i,j=1}^n, \]
  wobei $(e_j)_{j=1}^n$ die kanonische Basis des $\real^n$ bezeichnet.

  $GL(n,\real)$ ... Gruppe der invertierbaren linearen Abbildungen auf
  $\real^n$.

  Aus der linearen Algebra ist bekannt, dass sich jedes $T \in GM(n,\real)$ als
  Produkt von endlich vielen Transformationen der folgenden drei Typen
  darstellen lässt:
  \[ \begin{aligned}
    T_1(x) &= (x_1, \ldots, cx_j, \ldots, x_n), & \qquad &1 \le j \le n, &\quad&
    c \ne 0 \\
    T_2(x) &= (x_1, \ldots, x_j + cx_k, \ldots, x_n), & \qquad &1 \le j \le n,
    &\quad& k \ne j, \quad c \ne 0 \\
    T_1(x) &= (x_1, \ldots, x_k, \ldots, x_j, \ldots, x_n), & \qquad &1 \le j <
    k \le n,
  \end{aligned} \]
  wobei $x = (x_1, \ldots, x_n)$.
\end{prgp}

\begin{thm}
  Sei $T \in GL(n,\real)$.
  \begin{enumerate}[(i)]
    \item Ist $f$ eine Lebesgue-messbare Funktion auf $\real^n$, so ist auch
      $f(T)$ Lebesgue-messbar. Ist $f \ge 0$ oder $f$ Lebesgue-integrierbar, so
      gilt
      \[ \int f(x) \diffop x = | \det T | \int f(T(x)) \diffop x. \tag{1} \]
    \item Ist $E \subset \real^n$ Lebesgue-messbar, so auch $T(E)$ und
      \[ \lambda(T(E)) = \| \det T | \lambda(E). \]
  \end{enumerate}
\end{thm}

\begin{proof}
  Zu (i): Ist $f$ Borel-messbar, so ist auch $f(T)$ Borel-messbar, da $T$ stetig
  ist. Gilt (1) für $T, S \in GL(n,\real)$, so auch für $TS$, da
  \[ \begin{aligned}
      \int f(x) \diffop x
    &= |\det T| \int f(T(x))\diffop x = |\det T||\det S| \int f(T(S(x)))
    \diffop x \\
    &= | \det TS |\int f(TS(x)) \diffop x.
  \end{aligned} \]
  Deshalb genügt es, (1) für Transformationen des Typs $T_1$, $T_2$ und $T_3$
  aus \ref{prgp:2-11-5} zu zeigen. Dies folgt leicht aus dem Satz von Fubini-Tonelli.

  Im Fall von $T_3$ vertauschen wir die Reihenfolge der Integrale in den
  Variablen $x_j$ und $x_k$; im Fall von $T_1$ und $T_2$ integrieren wir zuerst
  bezüglich $x_j$ und benutzen die eindimensionalen Resultate:
  \[ \int f(t) \diffop t = | c | \int f(ct) \diffop t, \qquad \int
    f(t+a)\diffop t = \int f(t) \diffop t. \]
  (vgl. Satz 1.11.7). Dann folgt (1) aus $\det T_1= c$, $\det T_2 = 1$, $\det
  T_3 = -1$.

  Zu (ii): Ist $E$ eine Borel-Menge, so auch $T(E)$ (da $T^{-1}$ stetig). Mit
  $f=\ind_{T(E)}$ erhalten wir $\lambda (T(E)) = | \det T | \cdot \lambda(E)$.
  Also sind Borel-Nullmengen invariant unter $T$ und $T^{-1}$ und dasselbe gilt
  für $\lebesgue_n$.

  Nun ist es leicht zu sehen, dass die Aussagen auch für Lebesgue-messbare
  Funktionen und Mengen gelten.
\end{proof}

Ist $T$ eine lineare Abbildung, so gilt
\[ \begin{aligned}
    \int f(x) \diffop x &= | \det T | \cdot \int f(T(x)) \diffop x, \\
    \lambda(T(E)) = |\det T| \cdot \lambda(E)
  \end{aligned} \]
für eine Lebesgue-messbare Menge $E$.

Für eine Rotation $R$ (das heißt $RR^T = I$) gilt $|\det R| = 1$. Wir erhalten:
\begin{folg}
  Das Lebesgue-Maß ist \emph{rotationsinvariant}.
  \[ \lambda (R(E)) = \lambda(E) \]
  für alle $E$ Lebesgue-messbar.
\end{folg}

Als nächstes verallgemeinern wir Satz 2.11.6 auf differenzierbare Funktionen.
\begin{defn}
  Sei $G = (g_1, \ldots, g_n):\Omega \to \real^n$ eine Abbildung, wobei $\Omega
  \subseteq \real^n$ offen ist und die $g_i: \Omega \to \real$ zu $C^1$ gehören.
  Für $x \in \Omega$ sei $D_x G$ die lineare Abbildung, die durch die Matrix
  \[ \left(  \pdiff{g_i}{x_j} \right)^n_{i,j=1} \]
  definiert wird.
\end{defn}

\begin{rmrk*}
  Ist $G$ linear, so ist $D_x G = G$. Die Abbildung $G$ wir ein
  $C^1$-Diffeomorphismus genannt, wenn $G$ injektiv ist und $D_x G$ für alle $x
  \in \Omega$ invertierbar ist. Nach dem Satz über die inverse Funktion ist die
  inverse Abbildung $G^{-1} : G(\Omega) \to \Omega$ ein $C^1$-Diffeomorphismus,
  \[ D_x G^{-1} = (D_{G^{-1}(x)} G)^{-1} \]
  für $x \in G(\Omega)$.
\end{rmrk*}

\begin{thm}
  Sei $\emptyset \ne \Omega \subseteq \real^n$ offen und $G: \Omega \to \real^n$
  ein $C^1$-Diffeomorphismus.
  \begin{enumerate}[(i)]
    \item Ist $f$ eine Lebesgue-messbare Funktion auf $G(\Omega)$, so ist $f(G)$
      Lebesgue-messbar. Ist $f \ge 0$ oder $f \in L^1(G(\Omega),\lambda)$, dann
      ist
      \[ \int_G f(x) \diffop x = \int_\Omega f(G(x)) \cdot | \det D_xG |
        \diffop x. \]
    \item ist $E \subseteq \Omega$ Lebesgue-messbar, dann ist $G(E)$
      Lebesgue-messbar und es gilt
      \[ \lambda( G(E) ) = \int_E |\det D_x G| \diffop x. \]
  \end{enumerate}
\end{thm}

\begin{exmp}
  \begin{enumerate}[a)]
  \item Polarkoordinaten ($n=2$):
    \[ x = r \cdot \cos \varphi, \quad y = r \cdot \sin \varphi. \]
    Mit Satz 2.11.9 folgt $\diffop x \diffop y = r \diffop r \diffop
    \varphi$.
  \item Kugelkoordinaten ($n=3$):
    \begin{align*}
      x &= r \cdot \sin \varphi \cdot \cos \theta, \\
      y &= r \cdot \sin \varphi \cdot \sin \theta, \\
      z &= r \cdot \cos \varphi.
    \end{align*}
    Mit Satz 2.11.9 folgt $\diffop x \diffop y \diffop z = r \sin \varphi
    \diffop r \diffop \varphi$.
  \end{enumerate}
\end{exmp}

\begin{deno}
  \[ \sphere^{n-1} = \{ x \in \real^n : \| x \| = 1 \}. \]
  Ist $x \in \real^n \setminus \{0\}$, so schreiben wir $r := \|x\|$ und $x' :=
  \frac{x}{\|x\|}$.

  Die Abbildung
  \[ \Phi(x) := (r,x') \]
  ist eine stetige eindeutige Abbildung von $\real^n\setminus\{0\}$ auf
  $(0,\infty) \times \sphere^{n-1}$. Die inverse Abbildung
  \[ \Phi^{-1}(r, x') := r \cdot x' \]
  ist auch stetig.

  Wir bezeichnen mit $\lambda_*$ das Borel-Maß auf $(a,\infty)
  \times \sphere^{n-1}$ definiert durch
  \[ \lambda_* := \lambda(\Phi^{-1}(E)), \quad E \in \borel((a,\infty) \times
    \sphere^{n-1},\]
  das sogenannte Bildmaß.

  Wir definieren $\rho = \rho_n$ auf $(0,\infty)$ durch
  \[ \rho(A) := \int_A r^{n-1} \diffop r, \quad A \in \borel((0,\infty)). \]
\end{deno}

\begin{thm}
  Auf $\sphere^{n-1}$ existiert ein eindeutiges Borel-Maß $\sigma = \sigma_{n-1}$
  mit $\lambda_* = \rho \times \sigma$. Ist $f$ Borel-messbar mit $f \ge 0$ oder
  $f \in L^1(\lambda)$, dann ist
  \[ \int_{\real^n} f(x) \diffop x =  \int_0^\infty \int_{\sphere^{n-1}}
    f(r,x') r^{n-1} \diffop \sigma(x') \diffop r. \tag{1} \]
\end{thm}

\begin{proof}
  Ist $f$ eine Indikatorfunktion, dann ist (1) gleichwertig mit
  \[ \lambda_* = \rho \times \sigma. \]
  Gilt (1) für Indikatorfunktionen, so auch für beliebige Funktionen
  (Linearität des Integrals und Approximation durch einfach Funktionen).

  Deshalb genügt es, die erste Aussage zu zeigen. Sei $E \subset \sphere^{n-1}$
  eine Borel-Menge und
  \[ E_a := \Phi^{-1} ((0,a]) \times E) = \{ rx' : 0 < r \le a, x' \in E \}. \]
  Wenn (1) für $f = \ind_{E_1}$ gilt
  \begin{align*}
    \lambda(E_1) &= \int_0^1 \int_E r^{n-1} \diffop \sigma(x') \diffop r \\
                 &= \sigma(E) \cdot \int_0^1 r^{n-1} \diffop r = \frac{\sigma(E)}{n}.
  \end{align*}
  
  Deshalb definieren wir $\sigma(E) = n \cdot \lambda(E_1)$. Die Abbildung $E
  \to E_1$ bildet Borel-Mengen auf Borel-Mengen ab. Sie vertauscht mit
  Vereinigungen, Durchschnitten und Kompositionen.

  Also ist $\sigma$ ein Borel-Maß auf $\sphere^{n-1}$.

  Die Menge $E_a$ ist das Bild von $E_1$ unter der Abbildung $x \mapsto ax$.
  Nach Satz 2.11.6 gilt
  \[ \lambda(E_a) = a^n \cdot \lambda(E_1). \]
  Für $0 < a \le b$ gilt:
  \begin{align*}
    \lambda_*( (a,b] \times E )
    &= \lambda (E_b \setminus E_a) = \lambda(E_b) - \lambda(E_a) \\
    &= (b^n - a^n) \lambda(E_1) = (b^n - a^n) \frac{\sigma(E)}{n} \tag{2} \\
    &= \sigma(E) \int_a^b r^{n-1} \diffop r = \sigma(E) \rho((a,b]) \\
    &= (\rho \times \sigma) ((a,b] \times E). \tag{3}
  \end{align*}

  Sei $E \in \borel(\sphere^{n-1}$ fest und bezeichne $\mA_E$ die Familie aller
  endlichen disjunkten Vereinigungen von Mengen der Form $(a,b] \times E$. Nach
  Lemma 1.11.1 ist $\mA_E$ eine Algebra auf $(0,\infty) \times E$, die die
  $\sigma$-Algebra
  \[ \mM_E := \{ A \times E : A \in \borel((0,\infty)) \} \]
  erzeugt. Nach der obigen Rechnung gilt $\lambda_* = \rho \times \sigma$ auf
  $\mA_E$. Wegen der Eindeutigkeitsaussage in Satz 1.7.7 folgt, dass $\lambda_*
  = \rho \times \sigma$ auf $\mM_E$. Andererseits ist $\bigcup\{\mM_E:E \in
  \borel(\sphere^{n-1})\}$ die Menge aller Rechtecke auf $(0,\infty) \times
  \sphere^{n-1}$. Nochmaliges Anwenden des Eindeutigkeitssatzes zeigt, dass
  $\lambda_* = \rho \times \sigma$ auf Borel-Mengen gilt.
\end{proof}

\begin{rmrk}
  Nicht schwer zu zeigen: Die Gleichung (1) gilt auch für Lebesgue-messbare
  Funktionen, wenn man die Vervollständigung von $\sigma$ betrachtet.
\end{rmrk}

\begin{folg}
  Sei $f$ eine messbare Funktion auf $\real^n$, die nicht-negativ oder
  integrierbar ist, sodass $f(x) = g(\|x\|)$ mit einer messbaren Funktion $g$.
  Dann gilt
  \[ \int f(x) \diffop x = \sigma(\sphere^{n-1}) \cdot \int_0^\infty g(r)
    r^{n-1} \diffop r. \]
\end{folg}

\begin{folg}
  Seien $a > 0$, $f$ eine messbare Funktion und $B := \{ x : \|x\|<a\}$.
  \begin{enumerate}[(i)]
    \item Wenn $|f(x)| \le C \|x\|^{-\alpha}$ für ein $C > 0$, $\alpha < n$, dann
      ist $f \in L^1(B)$.
    \item Wenn $|f(x)| \ge C \|x\|^{-n}$ für ein $C > 0$, dann ist $f \notin
      L^1(B)$.
    \item Wenn $|f(x)| \le C \|x\|^{-\alpha}$ auf $B^C$ für ein $C > 0$, $\alpha
      > n$, dann ist $f \in L^1(B^C)$.
    \item Wenn $|f(x)| \ge C \|x\|^{-n}$ auf $B^C$ für ein $C > 0$, dann ist $f
      \notin L^1(B^C)$.
  \end{enumerate}
\end{folg}  

\begin{proof}
  Wende 2.11.14 an auf $\|x\|^{-\alpha} \ind_B$ bzw. $\|x\|^{-\alpha}
  \ind_{B^C}$.
\end{proof}

\begin{prgp}
  \textbf{Aufgabe.}
  Beschreibe $\sigma$ für $n=1$ und $n=2$.

  Aus der Beschreibung folgt $\sigma(\sphere^1) = 2 \pi$.
\end{prgp}

Als nächstes berechnen wir $\sigma(\sphere^{n-1})$ sowie das Lebesgue-Maß von
Kugeln.

\begin{lem}
  Für $a > 0$ gilt
  \[ \int_{\real^n} \exp(-a\|x\|^2) \diffop x = \left(  \frac{\pi}{a}
    \right)^{n/2}. \]
\end{lem}

\begin{proof}
  Sei $I_n := \int_{\real^n} \exp(-a\|x\|^2) \diffop x$. Nach Folgerung 2.11.14
  gilt für $n=2$
  \[ I_2 = \sigma(\sphere^1) \int_0^\infty r \exp(-ar^2) \diffop r = -
    \frac{\pi}{a} \exp(-ar^2) \big|_0^\infty  = \frac{\pi}{a}. \]
  Da $\exp(-a \|x\|^2) = \prod_1^n \exp(-a \|x_j\|^2)$ zeigt der Satz von
  Tonelli:
  \[ I_n = I_1^n. \]
  Insbesondere ist $I_1 = \sqrt{I_2} = \sqrt{\pi/a}$. Folglich ist
  \[ I_n = I_1^n = \left(  \frac{\pi}{a} \right)^{n/2}. \qedhere \]
\end{proof}

\begin{lem}
  Es gilt
  \[ \sigma(\sphere^{n-1}) = \frac{2\pi^{n/2}}{\Gamma\left(  \frac{n}{2} \right)}. \]
\end{lem}

\begin{proof}
  Mit Lemma 2.11.17 (l), Folgerung 2.11.14 (f) und der Substitution  $s := r^2$
  (s) zeigen wir:
  \begin{align*}
    \pi^{n/2}
    &\overset{(\text{l})}{=} \int_{\real^n} \exp(-\|x\|^2) \diffop x
      \overset{(\text{f})}{=} \sigma(\sphere^{n-1}) \int_0^\infty r^{n-1} \diffop r \\
    &\overset{(\text{s})}{=} \rez{2} \sigma( \sphere^{n-1}) \int_0^\infty s^{\frac{n-2}{2}}\exp(-s) \diffop s \\
    &= \rez{2} \sigma( \sphere^{n-1}) \Gamma\left( \frac{n}{2} \right). \qedhere
  \end{align*}
\end{proof}

\begin{folg}
  Das Lebesgue-Maß der Einheitskugel in $\real^n$ ist gleich
  \[ \frac{2 \pi^{n/2}}{\Gamma\left(\frac{n}{2}+1\right)}.\]
\end{folg}

\begin{proof}
  Es gilt
  \begin{align*}
    \lambda(\{x \in \real^n : \| x \| \le 1 \} )
    &= \sigma(\sphere^{n-1}) \underbrace{\int_0^1 r^{n-1} \diffop r}_{\rez{n}}
      = \frac{\sigma(\sphere^{n-1})}{n} \\
    &= \frac{2 \pi^{n/2}}{n \cdot \Gamma\left( \frac{n}{2} \right)}
      = \frac{2 \pi^{n/2}}{\Gamma\left( \frac{n}{2} + 1\right)}.
      \qedhere
  \end{align*}
\end{proof}

%%% Local Variables:
%%% TeX-master: "skript_mint"
%%% End:
