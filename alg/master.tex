\documentclass[
 a4paper,
 12pt,
 parskip=half
 ]{scrreprt}

\usepackage{../.tex/settings}

\usepackage{../.tex/mathpkgs}
\usepackage{../.tex/mathcmds}

\renewcommand{\thesection}{\arabic{section}} 
\renewcommand{\thesubsection}{\arabic{section}.\arabic{subsection}}

\usepackage[numbers,with_chapter]{../.tex/fancy_thm}

\swapnumbers
\theoremstyle{plain}
%\newtheorem{thm}{Satz}[section] % reset theorem numbering for each chapter
\newtheorem*{prp*}{Proposition}    

\theoremstyle{definition}
\newtheorem{defn}[thm]{Definition} 
\newtheorem{folg}[thm]{Folgerung} 
\newtheorem{rmrk}[thm]{Bemerkung} 
\newtheorem{deno}[thm]{Bezeichnungen}
\newtheorem{exmp}[thm]{Beispiel} 
\newtheorem{prgp}[thm]{} % Numbered paragraph

\newtheorem*{rmrk*}{Bemerkung}
\newtheorem*{exmp*}{Beispiel}
\newtheorem*{defn*}{Definition}

\DeclareMathOperator{\ch}{char}
\DeclareMathOperator{\Grad}{Grad}
\newcommand{\gzz}{\mathbb{Z}}

\numberwithin{equation}{chapter}
\numberwithin{thm}{chapter}

\hypersetup{
  pdftitle={ALGZTH},
  pdfauthor={Jonas Hippold},
  hidelinks
}

\title{Vorlesung\\Algebra und Zahlentheorie}
\subtitle{Sommersemester 2018}
\author{Vorlesung: Prof. Thom\\Mitschrift: Jonas Hippold}

\begin{document}

\maketitle

\tableofcontents

\clearpage

VL-Termine: 12.04.(Do), 13.04.(Fr), 20.04.(Fr), 26.04.(Do), 27.04.(Fr),
03.05.(Do), 11.05.(Fr), 17.05.(Do)

\chapter*{Motivation}
\[ p(t) = t^n + a_1 t^{n-1} + \ldots + a_n, \]
wobei $a_1, \ldots, a_n \in \integer$. Zum Beispiel:
\[ p(t) = t^3 + t + 1. \]

\begin{itemize}
\item Kann man $p$ lösen? Wo?
\item Wie kann man die Lösungen beschreiben?
\item Wie hängen verschiedene Lösungen miteinander zusammen?
\end{itemize}

\begin{exmp*}
  \[ p(t) = t^2 + at + b, \qquad a,b \in \integer. \]
  Lösung:
  \[ t_\pm = - \frac{a}{2} \pm \sqrt{\frac{a^2}{4} - b}. \]
  Es gilt $\frac{a^2}{4} - b \in \rat$.
\end{exmp*}

Fakt: Alle Polynome bis Grad 4 können durch Ausdrücke ``mit Wurzeln'' gelöst
werden. Dies gilt \emph{nicht mehr} ab Grad 5!

Die Galois-Theory beantwortet diese Fragen.

\subsubsection*{Fragen}
1. Würfelverdoppelung: Ist es möglich, aus einem Würfel mit Kantenlänge 1
einen Würfel mit dem doppelten Volumen zu konstruieren?

Genauer: Kann man eine Strecke mit Länge $\sqrt[3]{2}$ mit Zirkel und Lineal
konstruieren?

Galois-Theorie: Nein!

2. Quadratur des Kreises: Kann man ein Quadrat konstruieren (mit Zirkel und
Lineal), welches den gleichen Flächeninhalt wie ein gegebener Kreis hat?

Galois-Theorie: Nein!

3. Winkeldreiteilung: Kann man einen Winkel mit Zirkel und Lineal in drei gleich
große Winkel teilen?

Galois-Theorie: Nein!

Literatur:
\begin{itemize}
\item J.S. Milne: Fields and Galois Theory.
\end{itemize}

\chapter{Grundlegende Definitionen und Resultate}
\begin{defn}
  Ein \emph{Ring} ist eine Menge $R$ mit Verknüpfungen $+: R \times R \to
  R$, $\cdot: R \times R \to R$, so dass
  \begin{enumerate}[a)]
  \item $(R,+)$ ist eine abelsche Gruppe.
  \item $\cdot$ ist assoziativ und es existiert $1_R \in R$ mit
    \[ 1_R \cdot a = a \cdot 1_R = a. \]
  \item Es gilt ein Distributivgesetz:
    \[ (a+b) \cdot c = a \cdot c + b \cdot c, \qquad
      a \cdot (b+c) = a \cdot b + a \cdot c. \]
  \end{enumerate}
  Notation: $ab = a \cdot b$, $1 = 1_R$.

  $S \subseteq R$ heißt \emph{Teilring} (Unterring) von $R$, falls $1_R \in S$
  und $S$ unter $+$ und $\cdot$ abgeschlossen ist. Ebenso soll $a \in S$
  implizieren $-a \in S$. Dann ist $S$ selbst ein Ring.

  Seien $R$, $R'$ Ringe. Eine Abbildung $\alpha: R \to R'$ heißt
  \emph{Ringhomomorphismus}, wenn gilt:
  \[ \alpha( a+b ) = \alpha(a) + \alpha(b), \qquad
    \alpha( ab ) = \alpha(a) \alpha(b). \]
  und $\alpha( 1_R ) = 1_{R'}$.

  Wenn $\alpha: R \to R'$ ein Ringhomomorphismus ist, dann setzen wir
  \[ \ker(\alpha) := \{ a \in R : \alpha(a) = 0_{R'} \}, \]
  wobei $0_{R'}$ das neutrale Element von $(R',+)$ ist.

  $R$ heißt \emph{kommutativ}, falls $ab = ba$ für alle $a, b \in R$.

  $R$ heißt \emph{Integritätsbereich}, wenn $ab = 0$ nur gilt, wenn $a = 0$ oder
  $b = 0$. Ist $R$ Integritätsbereich, so folgt
  \[ ab = ac, \quad a \ne 0 \qRq b = c \]
  und
  \[ a(b-c) = 0 \qRq b -c = 0 \qRq b = c. \]

  Beispiel:
  \begin{itemize}
  \item $(\rkr{p}, +, \cdot)$ ist Integritätsbereich, wenn $p$ eine Primzahl
    ist.
  \item $(\rkr{4}, +, \cdot)$, $[2] \cdot [2] = 0$, kein Integritätsbereich.
  \end{itemize}
  
  Eine Teilmenge $I \subseteq R$ heißt \emph{Ideal} in $R$, falls
  \begin{itemize}
  \item $a,b \in I$ $\Rightarrow$ $a + b \in I$,
  \item $0 \in I$,
  \item $a \in I$, $b \in R$ $\Rightarrow$ $ab \in I$ und $ba \in I$.
  \end{itemize}

  Wenn $a_1, \ldots, a_n \in R$ dann bezeichnen wir das von $a_1, \ldots, a_n$
  erzeugte Ideal mit $(a_1, \ldots, a_n)$.

  Beispiel: Wenn $R$ kommutativ $a \in R$, dann gilt:
  \[ (a) = a \cdot R := \{ ab | b \in R \}. \]

  Ein Ideal $I \subseteq R$ heißt \emph{maximal}, falls $I \ne R$ und für jedes
  Ideal $J \subseteq R$ gilt:
  \[ I \subset J \qRq J = R. \]

  Ein Ideal $I \subseteq R$ ist \emph{prim}, falls für alle $a,b \in R$ gilt:
  \[ ab \in I \qRq a \in I \quad \text{oder} \quad b \in I. \]

  Beispiel: $R = \integer$, $I = (m) = m \integer$ ist prim genau dann, wenn $m$
  oder $-m$ eine Primzahl ist oder $m = 0$.

  Beobachtung: $(0) \subset R$ ist Primideal $\Leftrightarrow$ $R$ ist
  Integritätsbereich. 

  Ein \emph{Körper} ist ein kommutativer Ring, so dass gilt
  \[ a \ne 0 \qRq \exists b \in R : ab = 1. \]
  Wir schreiben $b := a^{-1}$

  Frage: Kann ein Körper Nullteiler haben? Nein!
  \[ ab = 0, a \ne 0 \qRq \exists a^{-1} \in R : b = a^{-1} a b = a^{-1} 0 =
    0. \]
  Also ist ein Körper nullteilerfrei.

  Ein \emph{Teilkörper} ist ein Teilring, der unter Inversen abgeschlossen ist.
\end{defn}

\begin{lem}
  Sei $R$ ein kommutativer Ring mit $0 \ne 1$. Dann ist $R$ ein Körper genau
  dann, wenn die einzigen Ideale in $R$ $(0)$ und $R$ sind.
\end{lem}

\begin{proof}
  ``$\Rightarrow$'': Sei $R$ ein Körper $I \subset R$ ein Ideal, $I \ne (0)$.
  Dann existiert $a \in I$, $a \ne 0$. Weil $R$ ein Körper ist, existiert auch
  $a^{-1} \in R$. Dann gilt:
  \[ b = \underbrace{b a^{-1}}_{\in R} \underbrace{a}_{\in I} \in I \]
  für alle $b \in R$. Also ist $I = R$.

  ``$\Leftarrow$'': Angenommen, $(0)$ und $R$ sind die einzigen Ideale. Aus $(a)
  = R$ folgt, dass $b \in R$ existiert, so dass $ab = 1_R$.
\end{proof}

\begin{exmp}
  $\rat$, $\real$, $\complex$, $\mathbb{F}_p := \rkr(p)$ ($p$ prim) sind Körper.
\end{exmp}

\section*{Die Charakteristik eines Körpers}

Man prüft leicht, dass $\alpha: \integer \to F$, wobei $F$ ein Körper ist,
\[ \alpha(m) = 1_F + 1_F + \ldots + 1_F =: m \cdot 1_F, \]
einen Ringhomomorphismus definiert.

Dann ist $\ker(\alpha)$ ein Ideal in $\integer$. Warum?
\[ m', m \in \ker(\alpha) \qRq -m, m + m' \in \ker(\alpha). \]
Für $m \in \ker(\alpha)$ und $n \in \integer$ gilt $mn \in \ker(\alpha)$,
\[ \alpha( mn ) = \alpha(m) \alpha(n) = 0 \alpha(n) = 0. \]

Es gilt: $\ker(\alpha) = (p)$, wobei $p$ eine Primzahl ist oder $p = 0$.

Weil $\integer$ Hauptidealring ist, gilt $\ker(\alpha) = (m)$ mit $m \in
\integer$. Falls $m = ab$, dann gilt
\[ 0 = \alpha(m) = \alpha(ab) = \alpha(a) \alpha(b)
  \qRq \alpha(a) = 0 \quad \text{oder} \quad \alpha(b) = 0. \]
Also muss $\alpha(a) \in \ker(\alpha)$ oder $\alpha(b) \in \ker(\alpha)$ sein.
Damit ist $\ker(\alpha)$ ein Primideal und $m$ ist prim.

Falls $\ker(\alpha) = (0)$, dann heißt $F$ von \emph{Charakteristik} 0. Falls
$\ker(\alpha) = p$, dann heißt $F$ von \emph{Charakteristik} $p$.

Beobachtung: Falls $F$ Charakteristik $p$ hat, so enthält $F$ den Körper
$\mathbb{F}_p$ als Teilkörper. Falls die Charakteristik 0 ist, so enthält $F$
den Körper $\rat$ als Teilkörper.
\[ \alpha' : \rat \to F, \quad
  \alpha' \left( \frac{p}{q} \right)
  = \frac{\alpha(p)}{\alpha(q)}. \]

\begin{exmp*}
  $\ch(\rat) = \ch(\complex) = 0$, $\ch(\mathbb{F}_p) = p$.
\end{exmp*}

Wir nennen $\mathbb{F}_2, \mathbb{F}_3, \mathbb{F}_5, \ldots, \rat$ die
\emph{Primkörper}.

\begin{rmrk}
  Für jeden Körper $K$ der Charakteristik $p \ne 0$ definiert die Abbildung
  $F(a) = a^p$ einen Homomorphismus $F: K \to K$. $F$ heißt
  \emph{Frobenius-Homomorphismus}.
\end{rmrk}

Warum ist $F$ Homomorphismus?
\begin{align*}
  F(ab) &= (ab)^p = a^p b^p = F(a) F(b), \\
  F(a+b) &= (a+b)^p = a^p + \binom{p1} a^{p-1} b + \binom{p}{2} a^{p-2} b^2
           + \ldots + \binom{p}{p-1} ab^{p-1} + b^p \\
        &= a^p + b^p = F(a) + F(b)
\end{align*}
$p | \binom{p}{k}$ für $0 < k < p$.

Es gilt $F(1) = 1^p = 1$. $\ker(F)$ ist Ideal in $K$, also gilt
\[ \ker(F) = (0) \qRq F \text{ injektiv}. \]

Seien $K$, $K'$ Körper und $\alpha : K \to K'$ ein Homomorphismus. Dann heißt
$\alpha$ \emph{Endomorphismus}, falls $K = K'$.

$\alpha$ heißt \emph{Isomorphismus}, falls $\alpha$ bijektiv ist ($\alpha^{-1}$
ist dann auch ein Isomorphismus).

Die Abbildung $\alpha$ heißt \emph{Automorphismus}, falls $\alpha$ Isomorphismus
und Endomorphismus ist.

Beobachtung: Falls $K$ ein endlicher Körper ist und $\ch(K) = p$, dann ist $F: K
\to K$ ein Automorphismus.

\section*{Rückblick auf Polynome}
Sei $F$ ein Körper. $F[X]$ ist der Polynomring in $X$ mit Koeffizienten in $F$,
ein Vektorraum mit Basis $1, X, X^2, \ldots$
\[ \left( \sum_i a_i X^i \right) \cdot \left( \sum_j b_j X^j \right)
  = \sum_k \left( \sum_{i+j=k} a_i b_j \right) X^k. \]

Beobachtung: Sei $R$ ein Ring, der F als Teilring enthält und sei $r \in R$.
Dann gibt es einen eindeutigen Homomorphismus $\alpha : F[X] \to R$ mit
$\alpha(a) = a$ für alle $a \in F$ und $\alpha(X) = r$.
\[ \alpha \left( \sum_i a_i X^i \right) = \sum_i a_i r^i. \]

\subsubsection*{Divisionsalgorithmus nach Euklid}
Seien $f,g \in F[X]$, $g \ne 0$. Dann existiert $q \in F[X]$, $r \in F[X]$ mit 
$\Grad(r) < \Grad(g)$ und $f = g \cdot q + r$

Mit diesen Eigenschaften sind $q$ und $r$ eindeutig definiert.
 
Beispiel: $f \in F[X]$, $a \in F$. $g(X) = X - a$. Division mit Rest liefert $q
\in F[X]$, $r = c \in F$,
\[ f(X) = (X-a) q(X) + c. \]
Falls $f(a)= 0$ für $a \in F$, dann gilt $c = 0$.

Falls $a$ Nullstelle von $f$ (also $f(a)=0$) ist, dann teilt $(X-a)$ das Polynom
$f$. Daraus kann man ableiten, dass $f$ höchstens $\Grad(f)$ Nullstellen haben
kann.

\subsubsection{Euklidischer Algorithmus zur Bestimmung des größten gemeinsamen
  Teiler zweier Polynome}
Seien $f, g \in F[X]$. Der Euklidische Algorithmus konstruiert Polynome $a, b
\in F[X]$ mit $\Grad(a) < \Grad(g)$, $\Grad(b) < \Grad(f)$ und $af + bg =
\ggT(f,g)$.

Vorgehen: Ohne Einschränkung $\Grad(f) \ge \Grad(g)$.
\begin{align*}
  f &= q_0 g + r_0 & \Grad(r_0) &< \Grad(g) \\
  g &= q_1 r_0 + r_1 & \Grad(r_1) &< \Grad(r_0) \\
  r_0 &= q_1 r_1 + r_2 & \Grad(r_2) &< \Grad(r_1) \\
  &\vdots
\end{align*}
Nach höchstens $\Grad(g) =: n$ Schritten ist $r_{n+1} = 0$ und dann gilt $r_{n-1}
= q_{n+1} r_n$.
Dann gilt: $r_n$ teilt $r_{n-1}$ und wegen $r_{n-2} = q_n r_{n-1} + r_n$ folgt
dann $r_n | r_{n-2}$, $r_n | r_{n-3}$, $\ldots$, $r_n | g$, $r_n | f$.
\[ r_n = r_{n-2} - q_n r_{n-1} = r_{n-2} - q_n( r_{n-3} - q_{n-1} r_{n_2} =
  \ldots = af + bg. \]
Jedes Ideal in $F[X]$  ist von der Form $(f)$ für ein eindeutig bestimmtes,
\emph{normiertes} Polynom $f \in F[X]$,
\[ f(X) = X^n  + a_1 X^{n-1} + \ldots + a_n \]
oder $(0)$.

Wenn $I \subseteq F[X]$ ein Ideal ist, dann gitl $I = (f)$, wenn $f$
kleinstmöglichen Grad hat und $f$ normiert ist.

\begin{proof} $(f) \subseteq I$ klar. Sei $h \in I$, $h = qf + r$, $\Grad(r) <
  \Grad(f)$, dann folgt $r = h - qf \in I$ $\Rightarrow$ $r = 0$.
\end{proof}

\subsubsection*{Faktorisierung von Polynomen}
\begin{defn*}
  Sei $F$ ein Körper, $f \in F[X]$. Dann heißt $f$ \emph{irreduzibel}, falls $f$
  kein Produkt von Polynomen kleineren Grades (in $F[X]$) ist.
\end{defn*}

\begin{exmp*}
  \[ f(X) = X^2 - 2 \in \rat[X] \]
  ist irreduzibel, aber
  \[ f(X) = X^2 - 2 = (X - \sqrt{2})(X + \sqrt{2}) \in \real[X].\]
\end{exmp*}

\begin{prp*}
  Sei $q(X) = a_m X^m + a_{m-1} X^{m-1} + \ldots + a_0 \in \rat[X]$ mit $a_0,
  \ldots, a_m \in \integer$. Sei $r = \frac{c}{d} \in \rat$ eine Nullstelle von
  $q$ und gelte $\ggT(c,d) = 1$. Dann gilt $c|a_0$ und $d | a_m$.
\end{prp*}

\begin{proof}
  \[ q(r) = 0 \qRq
    a_m c^m + \underbrace{a_{m-1} c^{m-1} d + \ldots + a_1 c d^{m-1} a_0
      d^m}_{\text{durch $d$ teilbar}} = 0, \]
  also $d | a_m$. Analog folgt $c | a_0$.
\end{proof}

\begin{exmp*}
  \[ f(X) = X^3 - 3X - 1 \in \rat[X] \]
  ist irreduzibel. Die Proposition impliziert, dass die einzigen möglichen
  Nullstellen in $\rat$ $\pm 1$ sind, aber $f( \pm 1 ) \ne 0$. $f$ ist also
  irreduzibel, weil für jede Faktorisierung $f = gh$ mit $\Grad(g) < 3$ und
  $\Grad(h) < 3$ ein Faktor Grad 1 haben müsste. Dann gäbe es aber eine
  Nullstelle von $f$ in $\rat$.
\end{exmp*}

\subsubsection*{Eisensteins Kriterium}
  Sei $f(X) = a_m X^m + a_{m-1} X^{m-1} + \ldots + a_0$ mit $a_0, \ldots, a_m
  \in \integer$. Angenommen, es gibt eine Primzahl $p$ mit
  \begin{enumerate}
  \item $p \nmid a_m$,
  \item $p | a_0, a_1, \ldots, a_{m-1}$,
  \item $p^2 \nmid a_0$.
  \end{enumerate}
  Dann ist $f \in \rat[X]$ irreduzibel.

\begin{rmrk*}
  Sei $f \in \rat[X]$ normiert. Dann gibt es immer $\alpha_1, \ldots, \alpha_n \in
  \complex$, sodass
  \[ f(X) = (X-\alpha_1) \cdot \ldots \cdot (X-\alpha_n). \]
\end{rmrk*}

\subsubsection*{Faktorisierung von Polynomen}
Sei $F$ ein Körper, $F[X]$ der Polynomring mit Koeffizienten in $F$. $f \in
F[X]$ ist \emph{irreduzibel}, wenn \emph{kein} $g,h \in F[X]$ existiert, so dass
\[ f = gh, \qquad \Grad(g) < \Grad(f), \qquad \Grad(h) < \Grad(f). \]
Mit anderen Worten: $f$ hat keine nicht triviale Faktorisierung in $F[X]$.

\begin{lem}[Gauss]
  Sei $f \in \integer[X]$ und $f$ besitze eine nicht triviale Faktorisierung in
  $\rat[X]$, dann besitzt es auch eine nicht triviale Faktorisierung in
  $\integer[X]$. 
\end{lem}

\begin{proof}
  Sei $f \in \integer[X]$ und $g,h \in \rat[X]$. Es existieren $m,n \in \integer
  \setminus \{0\}$, so dass  $m \cdot g, n \cdot h \in \integer[X]$. Dann gilt
  \[ m \cdot n \cdot f = (m \cdot g)(n \cdot h). \]
  Das ist eine nicht triviale Faktorisierung.

  Sei $p$ ein Primfaktor von $mn$. Dann betrachten wir $\bar{g}, \bar{h} \in
  (\rkr{p})[X]$, welche entstehen, indem man die Koeffizienten von $mg$ und $nh$
  modulo $f$ reduziert. Es gilt dann
  \[ \bar{g} \cdot \bar{h} = 0 \qquad \text{in } (\rkr{p})[X]. \]

  Ein Polynomring in einem Körper ist ein Integritätsbereich, das heißt $\bar{g}
  = 0$ oder $\bar{h} = 0$. O.B.d.A. sei $\bar{g} = 0$ in $(\rkr{p})[X]$, das
  heißt jeder Koeffizient von $mg$ war durch $p$ teilbar. Also ist $\frac{m}{p}
  g \in \integer[X]$. Es gilt
  \[ \frac{mn}{p} f = \left( \frac{m}{p} \right) \cdot g \cdot (nh). \]

  Iterieren dieses Prozesses liefert $f = g' h'$ mit $g', h' \in \integer[X]$.
\end{proof}

\begin{kor}
  Sei $f \in \integer[X]$ normiert und sei $g \in \rat[X]$ ein normierter Faktor
  von $f$. Dann ist $g \in \gzz[X]$ und $\frac{f}{g} \in \gzz[X]$.
\end{kor}

\begin{proof}
  Es gilt $f = gh$ in $\rat[[X]]$ und $g$ ist normiert. Dann ist auch $h$
  normiert. Seien $m, n$ kleinstmöglich mit $mg, nh \in \gzz[X]$. Wenn $p | mn$
  gilt, dann miuss $p$ schon Teiler von \emph{allen} Koeffizienten von $mg$ oder
  $nh$ sein (siehe Beweis zum Lemma) von Gauß). O.B.d.A. teile $p$ die
  Koeffizienten von $mg$. Dann ist
  \[ \left( \frac{m}{p} \right) g \in \gzz[X]. \]
  Weil $g$ normiert, der Führungskoeffizient also 1 war, folgt $p | m$. Das
  heißt
  \[ \frac{m}{p} \in \gzz, \quad \left( \frac{m}{p} \right) g \in \gzz[X]. \]
  Widerspruch! Also existiert kein solches $p$ und damit gilt $m = n = 1$ (oder
  $-1$).
\end{proof}

\subsubsection*{Kriterien für Irreduzibilität}
\begin{dis}[Eisenstein-Kriterium]
  Sei
  \[ f(X) = a_m X^m + a_{m-1} X^{m-1} + \cdots + a_0, \quad a_i \in \integer. \]
  Angenommen, es existiert eine Primzahl $p$, so dass
  \begin{enumerate}[a)]
  \item $p \nmid a_m$,
  \item $p | a_{m-1}, \ldots, a_0$,
  \item $p^2 \nmid a_0$,
  \end{enumerate}
\end{dis}
\end{document}
