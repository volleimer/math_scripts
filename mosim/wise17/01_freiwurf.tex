\chapter{Modellierungsbeispiel: Basketball-Freiwurf}
Trefferquote:
\begin{itemize}
\item Dirk Nowitzki: 89 \%
\item Shaquille O'Neal: 54,4 \%
\end{itemize}
Fragestellung (noch unpräzise): ``Beste Art einen Freiwurf zu werfen?''

Anwendungsproblem: ``Wie sollte ein Basketballspieler einer bestimmten Größe
beim Freiwurf den Ball werfen, damit eine größtmögliche Abweichung beim Abwurf
noch zu einem Treffer führt?''

\section{Erstes Modell: Der beste Abwurfwinkel}
\subsection{Analyse des Anwendungsproblems}
Beobachtung: Nach Verlassen der Wurfhand wird die Flugbahn lediglich durch
physikalische Gesetzmäßigkeiten (Bewegungsgleichungen) bestimmt.
\begin{itemize}
\item Abwurfposition, Anfangsgeschwindigkeit und -richtung, Drehimpuls
\item Luftwiderstand
\item Verschiedene Arten, ins Netz zu gehen
\end{itemize}
$\rightsquigarrow$ ``exakte'' Beschreibung sehr aufwendig bzw. kompliziert.

Vereinfachte Annahmen:
\begin{enumerate}
\item Nur ``nearly nothing but net''-Würfe, entweder direkt ins Netz oder Ball
  trifft Hinterkante des Rings und geht dann ins Netz.
\item Vernachlässigung des Luftwiderstands.
\item Vernachlässigung des Spins.
\item Kein seitlicher Fehler beim Abwurf (2D-Betrachtung statt 3D).
\item Kein Fehler in der Abwurfgeschwindigkeit (zunächst nur Winkel
  untersuchen).
\item Idealer Wurf geht durch das Zentrum des Rings (legt ``ideale''
  Geschwindigkeit fest).
\end{enumerate}

Präzisierte Problemstellung: Mit welchem Abwurfwinkel sollte ein Spieler den
Ball werfen, sodass eine größtmögliche Abweichung vom Winkel immer noch zu einem
Treffer führt?

\subsection{Herleitung eines (ersten) mathematischen Modells}
\subsubsection*{Systemparameter}
(hier Geometrie)

Relevante Maße: (Regelwerk NBA) Foot [\si{ft}], bzw. Inch [\si{in}]
(\SI{1}{in} $=$ \SI{2.54}{\cm}, \SI{1}{ft} $=$ \SI{12}{in} $=$ \SI{0.3048}{\m})
\[ l = \SI{13.5}{ft} = \SI{4.1148}{\m}, \qquad h = \SI{10}{ft} - \frac{5}{4}
  \cdot h_P, \qquad g = \SI{9.807}{\m \per \second \squared}. \]
$h_p$ ... Spielergröße

\subsubsection*{Variablen zur Zustandsbeschreibung}
(Mittelpunkt des Balls)

Ort:
\[ s(t) = \pmat{x(t) \\ y(t)} \in \real^2  \]
wegen Annahme 4.

Geschwindigkeit: $v(t)$ \\
Newtonsche Bewegungsgleichungen (Flugbahn)
\begin{align*}
  v_x(t) &= v_x^0 = \mathrm{const}. \\
  \dot{v}_y(t) &= \diff{}{t} v_y(t) = -g \\
  v_y(t) &= v_y^0 - g \cdot t
\end{align*}

Abwurfwinkel: $\vartheta^0$
\[ \pmat{ v_x^0 \\ v_y^0 } = \pmat{ v^0 \cdot \cos \vartheta^0 \\ v^0 \cdot \sin
    \vartheta^0 } \]
$v^0$: Betrag der Abwurfgeschwindigkeit.

Idealer Wurf durch Zentrum des Rings zum Zeitpunkt $T$:
\[ s(t) = \int_0^t v(\obar{t}) \diffop \obar{t} = \pmat{ v_x^0 t \\ v_y^0 t -
    \rez{2} g t^2 }, \qquad t \ge 0. \]
Forderung:
\begin{align*}
  l &\overset{!}{=} x(T) = v^0 \cdot \cos (\vartheta^0) \cdot T \\
  h &\overset{!}{=} y(T) = v^0 \cdot \sin (\vartheta^0) \cdot T - \rez{2} g T^2
      \tag{$\circ$}
\end{align*}
Die erste Gleichung nach $T$ umstellen, in die zweite einsetzen:
\[  l \cdot \tan(\vartheta^0) - h = \rez{2} g \left( \frac{l}{ v^0
      \cos(\vartheta^0)} \right)^2 \]
und damit folgt
\[  v^0 = \frac{l}{\cos(\vartheta^0)} \sqrt{\frac{g}{2(l \tan(\vartheta^0) -
      h)}}. \]
Eine physikalische Lösung existiert nur für $(l \tan(\vartheta^0) - h) > 0$,
also muss
\[ \arctan \left( \frac{h}{l} \right) < \vartheta^0 < \frac{\pi}{2} \]
gelten.

\subsubsection*{Erlaubte Abweichung vom Abwurfwinkel?}
(bei festgehaltenem $v^0$)

$\tilde{T}$ Zeitpunkt des Erreichens der Höhe $h$, Abwurfwinkel $\vartheta$.
\begin{align*}
  x(\tilde{T}) \tan \vartheta - h
  &= \rez{2} g \left( \frac{x(\tilde{T})}{v^0 \cos \vartheta} \right)^2 \\
  &= \frac{(v^0 \cos \vartheta)^2}{g} \left(
    \tan \vartheta + \sqrt{
    (\tan \vartheta)^2 - \frac{2gh}{(v^0 \cos \vartheta)^2}
    }\right).
\end{align*}
Anmerkung: Auflösen von ($\circ$) liefert
\[ \tilde{T} = \rez{g} \left(
    v_y^0 + \sqrt{(v_y^0)^2 - 2gh}
  \right) \]

\subsubsection*{Rekapitulation}
Bewegungsgleichung:
\[ \pmat{ \ddot{x} \\ \ddot{y} } = \pmat{0 \\ -g } \]
Randbedingung:
\[ \pmat{ x(0) \\ y(0) } = \pmat{ 0 \\ 0 }, \quad \pmat{x(t) \\ y(T)}
  = \pmat{l \\ h}. \]
Flugzeit $T$ ist freier Parameter.

Bedingung für ``Ball geht ins Netz'':
\begin{enumerate}
\item Wurf darf nicht zu lang sein:
  \[ x(\tilde{T}) + R_b \le l + R_r, \]
  $R_b $ ... Radius des Balls, $R_r$ ... Radius des Korbrings.
\item Wurf darf nicht zu kurz sein
  \[ x(\tilde{T}) - R_b \le l - R_r. \]
\item Ball darf Vorderseite des Rings nicht berühren:
  \[ a(t) := \left\| \pmat{x(t) \\ y(t)} - \pmat{l - R_r \\ h} \right\|_2 > R_b. \]
  Anmerkung: Es reicht, das Zeitintervall $[t_0, \tilde{T}]$ mit $t_0 =
  \rez{v_x}(l - R_r - R_b)$ zu überprüfen.
\end{enumerate}

$\vartheta_{\min}$, $\vartheta_{\max}$: Kleinster bzw. größter Winkel, bei dem
der Ball gerade noch in den Korb geht (gegen Referenzwinkel $\vartheta^0$).
\[ e( \vartheta^0 ) := \min \{ v_{\max} - \vartheta^0, \vartheta^0 -
  \vartheta_{\min} \}. \]
Zielfunktional (soll maximiert werden):
\[ \vartheta_{\mathrm{opt}}^0 = \argmax_{\vartheta^0} e( \vartheta^0). \]

Zur Implementierung und Simulation siehe die Python-Beispiele im Netz:\\
\url{https://tu-dresden.de/mn/math/wir/mendl/studium/courses/mosim_2017_wise}

Bis jetzt: Sehr viele Annahmen, um ein relativ einfach zu lösendes Modell zu
erhalten (insbesondere Annahme 5: kein Fehler in der Abwurfgeschwindigkeit).

\section{Zweites Modell: beste Wurfbahn}
\subsubsection*{Beste Wurfbahn}
Annahmen 1, 2, 3 und 4 wie bisher, aber jetzt ohne Annahmen 5 und 6. Variiere
das Tupel $(v^0, \vartheta^0)$ (anstatt $v^0$ fest zu halten).

Problemformulierung: ``Bestimme diejenige Wurfbahn (festgelegt durch $(v^0,
\vartheta^0)$), die die größtmögliche Abweichung erlaubt, sodass der Ball immer
noch in den Korb geht.''

\subsection{Analyse des Anwendungsproblems}
Zu gegebenem $\vartheta^0$ bestimme:
\[ v^0_{\text{front}}, v^0_{\text{back}}, \]
$v^0_{\text{front}}$ ... Ball berührt Korbring vorne (und geht ins Netz), \\
$v^0_{\text{back}}$ ... Ball berührt Korbring hinten (und geht ins Netz).

\subsubsection*{Berechnung von $v^0_{\text{back}}$}
Wir definieren $\bar{T}$ als den Zeitpunkt, zu dem gilt:
\[ x(\bar{T}) = l + R_r - R_b, \qquad y(\bar{T}) = h. \]
Nach analoger Rechnung wie zuvor (mit $x(\bar{T}$ statt $l$) erhalten wir
\[ v^0_{\text{back}} = \frac{l+R_r-R_b}{\cos \vartheta^0}
  \sqrt{\frac{g}{2 ((l + R_r - R_b) \tan \vartheta^0 - h)}}. \]

\subsubsection*{Berechnung von $v^0_{\text{front}}$}
Betrachte $a(t)$: Abstand des Ballmittelpunkts von der Vorderseite des Rings
\[ a_{\min} = \min_{t \in [0, \bar{T}]} \{ a(t) \}, \qquad v^0_{\text{front}}
  := \argmin_{v^0} | a_{\min}(v^0, \vartheta^0) - R_b |. \]
Ball berührt Vorderseite, falls Zielfunktion den Minimalwert 0 erreicht.

Nebenbedingung: $x(\bar{T}) \ge l - R_r$ (Ball fliegt über Vorderseite hinweg).

Berücksichtung der Nebenbedingung im Zielfunktional? Idee: Addiere sie als
``Strafterm'' hinzu.
\[ v^0_{\text{front}}
  := \argmin_{v^0} | a_{\min}(v^0, \vartheta^0) - R_b | + \max \{ 0, l - R_r -
  x(\bar{T})\}. \]
Der Strafterm verschwindet, falls die Nebenbedingung erfüllt wird.

\subsection{Mathematisches Modell: Mehrzieloptimierung}
Relative Abweichungen (sonst nicht vergleichbar)
\[ \begin{aligned}
    e_\vartheta(v^0, \vartheta^0) &:= \min \left(
      \frac{\vartheta_{\max}(\vartheta^0) - \vartheta^0}{\vartheta^0},
      \frac{\vartheta^0 - \vartheta_{\min}(\vartheta^0)}{\vartheta^0}
    \right) \\
    e_v(v^0, \vartheta^0) &:= \min \left(
      \frac{v_{\text{back}}(v^0) - v^0}{v^0},
      \frac{v^0 - v_{\text{front}}(v^0)}{v^0}
    \right)
  \end{aligned} \]

Visualisierung ergibt: Erlaubte Abweichung im Winkel ist größer ($\sim 10
\%$) als erlaubte Abweichung in Geschwindigkeit ($\sim 1 \%$).

Wie sollen $e_\vartheta$ und $e_v$ gleichzeitig optimiert werden?
``Mehrzieloptimierung''
\[ e( v^0, \vartheta^0 ) := \min \{ e_\vartheta(v^0, \vartheta^0), w \cdot
  e_v(v^0, \vartheta^0) \} \]
mit Gewichtungsfaktor $w$.

Mathematisches Optimierungsproblem:
\[ (v^0_{\text{opt}}, \vartheta^0_{\text{opt}}) = \argmax_{(v^0, \vartheta^0)}
  e(v^0, \vartheta^0). \]

\subsubsection*{Auswertung und Interpretation}
Für \SI{2.13}{\m} Spielergröße, $w = 5$:
\[ v^0_{\text{opt}} = \SI{6.7}{\m \per \s}, \qquad \vartheta^0_{\text{opt}} =
  \ang{52.5}. \]