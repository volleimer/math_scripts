\chapter{Diskretisierung partieller Differentialgleichungen}
Partielle Differentialgleichungen (PDE) treten in fast allen quantitativen
Naturwissenschaften auf.

\section{Exemplarische Beispiele partieller Differentialgleichungen}
\subsubsection*{Beispiel 1: Stationäres elliptisches Problem}
$u(x) \in \real^d$, $s \in \Omega \subseteq \real^m$ Ortsvariable (oft $m = 2,
3$).

An $\partial \Omega$ werden Daten vorgegeben, etwa
\[ u |_{\partial \Omega} = 0, \qquad \pdiff{u}{n} = 0, \qquad \ldots \]

Anwendung: Form einer Seifenblase

Gegeben sei die Höhe des Drahtes als Funktion $\varphi: \partial \Omega \to
\real$.

Gesucht ist die Höhe $u: \Omega \to \real$ der Seifenblase.

Physikalisches Prinzip der Seifenblasenhaut: Minimierung der Oberfläche
\[ J := \int_\Omega \sqrt{1 + (\partial_{x_1} u)^2 + (\partial_{x_2} u)^2}
  \diffop^2 x \to \min. \]

%% Hopp

Plateau'sches Problem:
\[ \left\{ \quad \begin{aligned}
      &J \to \min \\
      &\text{Randbedingung } u|_{\partial \Omega} = \varphi
    \end{aligned}
  \right.
\]

Bei kleinen Gradienten:
\[ \sqrt{ 1 + (\partial_{x_1} u)^2 + (\partial_{x_2} u)^2}
  \approx
  1 + \rez{2} ((\partial_{x_1} u)^2 + (\partial_{x_2} u)^2). \]
Damit lässt sich das Problem vereinfachen:
\[ \left\{ \quad \begin{aligned}
      &J := \rez{2} \int_\Omega (\partial_{x_1} u)^2 + (\partial_{x_2} u)^2
      \diffop x \to \min \\
      &u|_{\partial \Omega} = \varphi
    \end{aligned}
  \right.
\]
Mit der Euler-Lagrange-Gleichung der Variationsrechnung erhält man
\[ \left\{ \quad \begin{aligned}
      &\partial_{x_1} u)^2 + (\partial_{x_2} u)^2 = 0 \\
      &u|_{\partial \Omega} = \varphi
    \end{aligned}
  \right.
\]
Schreibweise: $\Delta u = 0$.

\subsubsection*{Beispiel 2: Hyperbolische Kontinuitätsgleichung}
\begin{itemize}
\item $\rho(x,t) \in \real$: Dichte
\item $u(x,t) \in \real^2$: Geschwindigkeitsfeld
\end{itemize}

Massenerhaltung:
\begin{align*}
  \pdiff{}{t} \int_V \rho \diffop^3 x
  &= - \int_{\partial V} \rho( u \cdot n ) \diffop^2 S, \\
  \begin{aligned}
    &\text{Zeitliche Änderung} \\
    &\text{der Masse in $V$}
  \end{aligned}
  &= \begin{aligned}
    &\text{Flüssigkeit, die in das Volumen} \\
    &\text{hinein bzw. herausströmt}
  \end{aligned}
\end{align*}

Gauß:
\[ \int_{\partial V} \rho(u \cdot n) \diffop^2 S
  = \int_V \nabla \cdot (\rho u) \diffop^3 x. \]
Weil $V$ beliebig ist:
\[ \pdiff{}{t} \rho + \nabla \cdot (\rho u) = 0. \]
Das ist die Kontinuitätsgleichung für die Dichte.

``hyperbolisch'': Orts- und Zeitableitung haben dieselbe Ordnung, hier 1.

\subsubsection*{Beispiel 3: Parabolische Wärmeleitungsgleichung}
$u(x,t) \in \real^d$, $x \in \real^d$, $d = 1,2,3$.
\[ \partial_t u = k \Delta u \]
mit dem Laplace-Operator $\Delta u = \sum_{i=1}^d \partial_{x_i}^2 u$.

Cauchy-Problem (Anfangs- und/oder Randbedingungen vorgegeben):
\[ u(x, t = 0) = u_0(x) \]

Fundamentallösung (in 1D):
\[ u(x,t) = \rez{\sqrt{4\pi k t}} \int_{-\infty}^\infty
  e^{-\frac{(x-y)^2}{4 k t}} u_0(y) \diffop y. \]

\subsubsection*{Bemerkungen}
\begin{enumerate}[(a)]
\item Beispiele 1, 2 und 3 sind \emph{lineare} PDEs, das heißt sie sind linear
  in der gesuchten Lösung, zum Beispiel:
  \[ \Delta( \alpha u(x,t) + v(x,t) )
    = \alpha \cdot \Delta u(x,t) + \Delta v(x,t). \]
  Die Navier-Stokes-Impulsgleichung ist eine nichtlineare PDE
  \[ \rho \left( \partial_t u_i + \sum_{j=1}^3 u_j \partial_{x_j} u_i \right)
    + \partial_{x_i} P = \ldots \]
  Der Term $u_j \partial_{x_j} u_i$ ist quadratisch in $u(x,t)$ das heißt
  \[ u(x,t) \to \alpha u(x,t), \qquad
    \sum_{j=1}^3 u_j \partial_{x_j} u_i \to \alpha^2
    \sum_{j=1}^3 u_j \partial_{x_j} u_i. \]
\item Die Typeneinteilnug (elliptisch, hyperbolisch, parabolisch usw.) bestimmt
  die Art der Zusatzdaten (Anfangs- oder Randwerte), die das Problem
  \emph{wohlgestellt}\footnote{%
    Das heißt die Lösung ist stetig bezüglich der Anfangs- bzw. Randwerte.
  } machen.
\end{enumerate}

\section{Typische Schritte der numerischen Lösung}
\begin{enumerate}
\item Geometrieaquisition
\item Diskretisierung (Gitter und numerische Methode)
\item Gleichungssystem aufstellen
\item Gleichungssystem lösen
\item Lösung bewerten (Fehlerabschätzung)
\item Visualisierung
\end{enumerate}
Adaptive Systeme kehren automatisiert von Schritt 5 wieder zu Schritt 2 zurück.

\section{Disketrisierungsprinzipien} %% 6.3
\begin{itemize}
\item Finite Differenzen: Verallgemeinerungen des Differentialquotienten
  \[ \frac{f(x+h) - f(x)}{h} \]
  auf höhere Dimensionen bzw. Ordnungen
\item Finite Volumen: Anwendung vor allem auf Erhaltungsgleichungen
  \[ \int_V \text{Ausdruck } 1 = \int_V \text{Ausdruck } 2 \]
  für alle Volumen $V$ in $\Omega$

  Idee: Ersetze ``alle $V$'' durch ausgewählte, endlich viele Testvolumen $E_j$,
  wobei
  \[ \Omega = E_1 \hat{\cup} E_2 \hat{\cup} \cdots \hat{\cup} E_N, \]
  wobei $\hat{\cup}$ die disjunkte Vereinigung ist.
\item Finite Elemente
  Variationsformulierung $\rightsquigarrow$ PDE
  \[ u \in X : J(u) \overset{!}{=} \min_{v \in X} J(v) \]
  $X$: Funktionenraum (zum Beispiel alle stetig differenzierbaren Funktionen auf
  $\Omega$).

  Wähle endlich-dimensionalen Unterraum $X_h \subset X$ und betrachte
  \[ u_h \in X_h : J(u_h) \overset{!}{=} \min_{v_h \in X_h} J(v_h). \]
  Für $J$ quadratisch ist das äquivalent zum linearen Gleichungssystem
  \[ L_h u_h = f_h. \]

  Speziell für Finite Elemente: $X_h$ ist der Raum der stückweisen Polynome in
  $\Omega$, $\Omega$ wird zerlegt (Triangulierung), in 2D Zerlegung in Dreiecke.
\end{itemize}

\section{Finite Differenzen}
Idee:
\begin{enumerate}[a)]
\item Diskretisiere $\Omega$ durch (uniformes) Gitter mit Gitterkonstante $h$.
  Notation: $\Omega_h$.
\item Approximiere Differentialoperatoren durch Differenzenquotienten.
\end{enumerate}

Differentialoperatoren: Zum Beispiel symmetrische Differenzen für $u(x,y)$:
\[ \partial_x u =
  \underbrace{\frac{u(x+h,y) - u(x-h,y)}{2h}}_{=: \partial_{h,x} u} + O(h^2)
  \quad \text{für } u \in C^3, \]
\[ \partial_x^2 u = \frac{u(x+h,y) - 2 u(x,y) + u(x - h,y)}{h^2} + O(h^2)
  \quad \text{für } u \in C^4. \]
Ableitung nach $y$ analog.
\begin{align*}
  - \Delta u &= -(\partial_x^2 u + \partial_y^2 u) \\
             &= \frac{4 u(x,y) - u(x+h,y) - u(x-h,y) - x(x,y+h) - u(x,y-h)}
               {h^2} + O(h^2) \\
             &\text{für } u \in C^4.
\end{align*}
Der diskrete Laplace-Operate $- \Delta_h u$ wird als \emph{Fünf-Punkte-Stern}
bezeichnet.

Abgekürzte Notation in gittereingebetteter Matrix-Schreibweise:
\begin{align*}
  - \Delta_h
  &= \rez{h^2}
    \begin{bmatrix}
      0 & -1 & 0 \\
      -1 & 4 & -1 \\
      0 & -1 & 0
    \end{bmatrix}, \\
  \partial_{h,x}
  &= \rez{2 h}
    \begin{bmatrix}
      0 & 0 & 0 \\
      -1 & 0 & 1 \\
      0 & 0 & 0
    \end{bmatrix} =
              \begin{bmatrix}
                -1 & 0 & 1
              \end{bmatrix}, \\
  \partial_{h,x}
  &= \rez{2 h}
    \begin{bmatrix}
      1 \\ 0 \\ -1
    \end{bmatrix}.
\end{align*}

\subsection{Modellproblem Poisson-Gleichung}
Modell für elliptisches Randwertproblem
\[ \text{Poisson-Gleichung } \left\{
    \begin{aligned}
      - \Delta u &= f & \text{auf } \Omega \subset \real^d \text{ Gebiet} \\
      u|_{\partial \Omega} &= 0 & \text{Homogene Dirichlet'sche Randbedingung}
    \end{aligned}
  \right. \]
$\Omega$ sei zum Beispiel das Einheitsquadrat $[0,1]^2$ für $d = 2$.

Wir diskretisieren $\Omega$ mit der Gitterkonstante $h = \rez{n}$, $n \in \nat$.
Für $n = 5$ erhalten wir ein $6 \times 6$-Gitter
\[ \Omega_h =
  \big\{ (x,y) \in \Omega : x = ih, y = jh; i,j \in \{0,1,2,3,4,5\} \big\}. \]
Die Randpunkte von $\Omega_h$ werden in $\Gamma_h$ zusammen gefasst.

Die diskrete Lösung
\[ u_h : \Omega_h \setminus \Gamma_h \to \real \]
ist nur auf den inneren Punkten definiert wegen der Randbedingung $u|_{\partial
  \Omega} = 0$. Das heißt $u_h$ ist ein Vektor der Dimension $(n-1)^2$.

Die diskretisierte rechte Seite ist
\[ f_h := f|_{\Omega_h \setminus \Gamma_h}. \]

Die Diskretisierung mit Fünf-Punkte-Stern-Operator führt zu einem linearen
Gleichungssystem
\[ L_h u_h = f_h. \]
Wie sieht die Matrix $L_h$ aus? Das hängt von der Nummerierung der Gitterpunkte
ab. Typischerweise wählt man eine lexikographische Nummerierung der inneren
Punkte:
\[ \begin{bmatrix}
    & & & \cdots & (n-1)^2 \\
    \vdots & \vdots & \vdots &  & \vdots \\
    n & n+1 & n+2 & \cdots & 2(n-1) \\
    1 & 2 & 3 & \cdots & n-1
  \end{bmatrix}, \qquad
  u_h = \pmat{
    u(h,h) \\ u(2h,h) \\ \vdots \\
    u((n-1)h,h) \\ u(h,2h) \\ \vdots \\
    u( (n-1)h, (n-1)h) }
\]
Damit ergibt sich
\[ L_h = \rez{h^2}
  \begin{pmatrix}
    4 & -1 & 0 & \cdots & -1 & 0 & \cdots \\
    -1 & 4 & -1 & & \cdots & -1 & 0 & \cdots \\
    0 & -1 & 4 & -1 & & \cdots & -1 & 0 & \cdots \\
    & & & \ddots & \ddots & & \ddots \\
    & & & & 4 & 0 & & -1 \\
    -1 & & & & 0 & 4 & -1 & & -1 \\
    & -1 & & & & -1 & 4 & -1 & & -1 \\
    & & \ddots & & & & \ddots & \ddots & \ddots \\
  \end{pmatrix}
\]
Die Hauptdiagonale ist überall 4. Auf den $-1$-Nebendiagonalen gibt es wegen der
Randpunkte einige Null-Einträge. Insgesamt hat $L_h$ eine pentadiagonale
Struktur und ist dünn besetzt.

Das Aufstellen von $L_h$ mittels Python erfolgt mit dieser Idee: Nutze die
Faktorisierung bezüglich $x$- und $y$-Richtung aus.
\[ -\Delta_h = \rez{h^2}
  \begin{bmatrix}
    0 & -1 & 0 \\
    -1 & 4 & -1 \\
    0 & -1 & 0
  \end{bmatrix}
  = \underbrace{
    \rez{h^2}
    \begin{bmatrix}
      -1 & 2 & -1
    \end{bmatrix}}_{(-\partial_{h,x}^2) \otimes I_y} +
  \underbrace{
    \rez{h^2}
    \begin{bmatrix}
      -1 \\ 2 \\ -1
    \end{bmatrix}}_{I_x \otimes (-\partial_{h,y}^2)}
\]
mit dem Kronecker-produkt $\otimes$ (numpy.kron) und $I_x$, $I_y$ $(n-1) \times
(n-1)$-Einheitsmatrix.

1. Zeile von $L_h$:
\[ \rez{h^2} ( 4u(h,h) - u(2h,h) - u(h,2h)), \]
2. Zeile von $L_h$:
\[ \rez{h^2} ( 4u(2h,h) - 3(3h,h)- u(h,h)  u(2h,2h). \]

\paragraph{Aufstellen von $L_h$ mittels Python}
Idee: Nutze Faktorisierung bezüglich $x$- und $y$-Richtung aus.
\[ -\Delta_h = \rez{h^2}
  \begin{bmatrix}
    & -1 \\
    -1 & 4 & -1 \\
    & -1
  \end{bmatrix}
  = \rez{h^2} \begin{bmatrix} -1 & 2 & -1 \end{bmatrix}
  + \rez{h^2} \begin{bmatrix} -1 \\ 2 \\ -1 \end{bmatrix}
\]
Man schreibt
\begin{align*}
  \rez{h^2} \begin{bmatrix} -1 & 2 & -1 \end{bmatrix}
  &=: -\partial_{h,x}^2 \otimes I_y), \\
  \rez{h^2} \begin{bmatrix} -1 \\ 2 \\ -1 \end{bmatrix}
  &=: -\partial_{h,y}^2 \otimes I_x.
\end{align*}

  
Eindimensionaler Fall:
\[ - \partial_{h,x}^2 = \rez{h^2}
  \begin{bmatrix} -1 & 2 & -1 \end{bmatrix} =
  \begin{pmatrix}
    2 & -1 & 0 & 0 & \cdots & 0 \\
    -1 & 2 & -1 & 0 & \cdots & 0\\
    & -1 & 2 & -1 & \cdots & 0 \\
    & & \ddots & \ddots & \ddots \\
    & & & -1 & 2 & -1 \\
    & & & & -1 & 2
  \end{pmatrix} \]
\[ - \partial_{h,x}^2 = \rez{h^2}
  \begin{bmatrix} -1 \\ 2 \\ -1 \end{bmatrix} =
  \text{Gleiche Matrix wie oben.}\]

\subsection{Lösbarkeit des linearen Gleichungssystems
  $L_h u_h = f_h$}
%% 6.4.2

Elliptische Probleme allgemein: symmetrisch, positiv definit (alle Eigenwerte
$>0$).

Die Konditionszahl von $L_h$ ist
\[ \kappa(L_h) = \| L_h^{-1} \| \cdot \| L_h = O(h^{-2}). \]
Die iterative Lösung des Gleichungssystems, zum Beispiel mittels CG-Verfahren
(conjugate gradients), benötigt $\sim \sqrt{\kappa(L_h)} = O(h^{-1})$
Iterationen. Das ist sehr aufwendig für kleine $h$. Als Ausweg verwendet man
\emph{Mehrgitterverfahren} (multigrid methods), eine Hierarchie von
gröberen/feineren Gittern.

\section{Finite Volumen} %% 6.5
Erhaltungsprinzip
\[ \int_V \text{Ausdruck 1} = \int_V \text{Ausdruck 2} \quad
  \text{für alle Volumen $V$ in $\Omega$} \]
Beispiel Masseerhaltung:
\[ \partial_t \int_V \rho \diffop^3 x = \int_V \nabla \cdot (\rho \vec{u})
  \diffop^3 x. \]

Idee:
\begin{enumerate}
\item Ersetze ``alle $V$'' durch ausgewählte, endlich viele Testvolumen $E_i$,
  wobei
  \[ \Omega = \dot\bigcup_{i=1}^n E_n \]
  mit der disjunkten Vereinigung $\dot\cup$.

  Für jedes Teilvolumen speichert man Freiheitsgrade ab:
  \begin{enumerate}[(a)]
  \item all-center (ein Freiheitsgrad in der Mitte jedes Volumens)
  \item all-edge (ein Freiheitsgrad in der Mitte jeder Kante)
  \item all-vertex (ein Freiheitsgrad an jedem Eckpunkt)
  \end{enumerate}
\item Integriere die Differentialgleichung über $E_j$, wende den Gauß'schen Satz
  an  und approximiere, zum Beispiel
  \[ - \Delta u = f \qRq
    - \int_{E_j} \Delta u \diffop x = \int_{E_j} f \diffop x \]
  Gauß angewendet auf $\nabla u$:
  \[ \Delta u = \nabla \cdot( \nabla u ) = ( \partial_x^2 + \partial_y^2 +
    \partial_z^2) u. \]
  Also erhalten wir
  \[ - \int_{E_j} \Delta u \cdot \vec{n} \diffop S =
    \int_{E_j} f \diffop x. \]
  Benutze Freiheitsgrade, um $\nabla u \cdot \vec{n}$ durch Differenzen zu
  approximieren. 
\end{enumerate}

\subsection{Modellproblem: Poisson-Gleichung} %% 6.5.1
Teile die Ebene in Quadrate auf und wähle eine all-center
%% Grid 5 x 5, Seitenlänge h = 1/n
%% BIld

Ersetze Normalenableitung  $\nabla u \cdot \vec{n}$ durch 1. Differenzen
zwischen Zellmittelpunkten:
\[ \int_{E_j} f \diffop x = - \int_{\partial E_j} \partial_{h,\vec{u}} u_h
  \diffop S = - \int_{\Gamma_1 \cup \Gamma_2 \cup \Gamma_3 \cup \Gamma_4}
  \partial_{h,\vec{u}} u_h \diffop S \tag{$*$}. \]
Die Integrale über die einzelnen Seiten sind
\begin{align*}
  \int_{\Gamma_1} \partial_{h,\vec{u}} \diffop x
  &= h \cdot \frac{u(x_j + h, y_j) - u(x_j, y_j)}{h}, \\
  \int_{\Gamma_2} \partial_{h,\vec{u}} \diffop x
  &= h \cdot \frac{u(x_j, y_j + h) - u(x_j, y_j)}{h}, \\
  \int_{\Gamma_3} \partial_{h,\vec{u}} \diffop x
  &= h \cdot \frac{u(x_j - h, y_j) - u(x_j, y_j)}{h}, \\
  \int_{\Gamma_4} \partial_{h,\vec{u}} \diffop x
  &= h \cdot \frac{u(x_j, y_j - h) - u(x_j, y_j)}{h}.
\end{align*}
Der Faktor $h$ kommt daher, dass die Länge des Integrationspfads $\Gamma_1$
genau die Seitenlänge ist.

Eingesetzt in ($*$), teile durch $\rez{h^2}$:
\[ \rez{h^2} \int_{E_j} f \diffop x =
  \underbrace{
    \begin{aligned}
      \rez{h^2} \Big( 4 u(x_j,y_j)
      &- u(x_j+h,y_j) - u(x_j-h,y_j) \\
      &- u(x_j,y_j+h) - u(x_j,y_j-h) \Big) 
    \end{aligned}
  }_{
    -\Delta_h u } \]

Beobachtung: Wir erhalten die gleiche Matrix $L_h$ wei bei finiten Differenzen,
aber die rechte Seite ist
\[ f(x_j,y_j) = \oint_{E_j} f \diffop x := \frac{\int_{E_j} f \diffop
    x}{\int_{E_j} \diffop x} \]
%% Average int!
Integraler Mittelwert.

\subsection{Hyperbolische Erhaltungsgleichungen in 1D} %% 6.5.2
Allgemeine Formulierung:
\[ \left\{ \quad \begin{aligned}
      &\partial_t u (x,t) + \partial_x f(u(x,t)) = 0 \\
      &u(x,0) = u_0(x)
    \end{aligned} \right. \tag{$\circ$} \]
Gesucht ist $u(x,t)$, also haben wir dieselbe Situation wie Massenerhaltung in
D.
\[
  \begin{array}{cccccc}
    \partial_t \rho & + & \nabla & \cdot & (\rho u) & = 0 \\
    \updownarrow & & \updownarrow & & \updownarrow \\
    u & & \partial_x & & f(u(x,t))
  \end{array} \]

Einfachster Fall: Lineare Konvektionsgleichung
\[ f(u) = a \cdot u, \quad a \text{ konstant}. \]
Entsprechende Differentialgleichung:
\[ \left\{ \quad \begin{aligned}
      &\partial_t u (x,t) + a \partial_x u(x,t) = 0 \\
      &u(x,0) = u_0(x)
    \end{aligned} \right. \]
Für differenzierbares $u$ ist die Lösung:
\[ u(x,t) = u_0(x-at), \]
die Anfangsverteilung verschiebt sich mit Geschwindigkeit $a$ in Richtung des
Vorzeichens von $a$.

\subsubsection*{Beispiel: Reibungsfreie Burgensgleichung}
\[ f(u) = \rez{2} u^2 \qRq \partial_x f(u(x,t)) = u(x,t) \partial_x u(x,t). \]
Trick: Betrachte Lösungen der gewöhnlichen Differentialgleichung für $z(t)$:
\[ z'(t) = u(z(t),t), \tag{$*$} \]
sogenannte \emph{Charakteristiken}.
\begin{align*}
  \diff{}{t} u(z(t),t) 
  &= \partial_t u(z(t),t) + \partial_x u(z(t),t) z'(t) \\
  &= \partial_t u(z(t),t) + \underbrace{u(z(t),t) \partial_x u(z(t))}_{
    \partial_x f(u(z(t),t))
    } = 0
\end{align*}
gemäß der Differentialgleichung ($\circ$). Also ist $u(z(t),t)$ zeitlich
konstant, insbesondere ist nach ($*$) auch $z'(t)$ konstant. Damit folgt
\[ z(t) = z_0 + t \cdot u(z(t),t) = z_0 + t \cdot u( z_0, 0 ). \]

Also ist $z(t)$ abhängig von der Anfangsverteilung von $u$. Das führt dazu, dass
die Verteilung von $u$ umso schneller transportiert wird, je größer $u(z_0)$
ist. Es entstehen \emph{Schockwellen} (Stoßwellen, shock waves) mit senkrechtem
Abfall auf einer Seite.

Problem aus numerischer Sicht: Die Lösung $u(x,t)$ entwickelt eine Unstetigkeit
durch diese Schockwelle. Dann ist die Beschreibung mittels Differenzenquotienten
nicht mehr sinnvoll. Deswegen wurde für solche Anwendungen eine eigene
numerische Lösungstheorie entwickelt, siehe zum Beispiel Randall LeVoque:
Numerical Methods for Conservation Laws.

\section{Finite Elemente} %% 6.6
\subsection{Formulierung als Variationsgleichung}
Idee: Schreibe PDE als Variationsgleichung um.

\subsubsection*{Beispiel 1}
Poisson-Gleichung auf dem Gebiet $\Omega \in \real^d$, $d=2,3$:
\[
  \left\{
    \begin{aligned}
      - \Delta u &= g, &\text{in } \Omega, \\
      u |_{\partial \Omega} &= 0
    \end{aligned}
  \right. \]
Nun betrachtet man $V := $ Menge aller stetig differenzierbaren
Funktionen\footnote{%
  Das ist der sogenannte Sobolev-Raum $H_0^1(\Omega)$.}
$v$ auf $\Omega$ ($v: \Omega \to \real$ stetig differenzierbar) mit
$v|_{\partial \Omega} = 0$. Multipliziere die PDE mit beliebigem $v \in V$
(``Testfunktion'') und wende den Gauß'schen Integralsatz an:
\[ - \int_\Omega (\Delta u) \cdot v \diffop x
  = \int_\Omega g \cdot v \diffop x
  \quad \text{für alle } v \in V. \]
Gauß angewendet auf $(\nabla u) v$ liefert
\[ - \int_\Omega (\Delta u) \cdot v \diffop x
  = \int_\Omega (\nabla u) \cdot (\nabla v) \diffop x
  - \underbrace{\int_{\partial \Omega} v \cdot \nabla u \cdot \vec{n} \diffop
    S}_{=0}.
\]
Der zweite Term ist $=0$, weil wir $v|_{\partial \Omega} = 0$ voraussetzen. Also
gilt
\[ \underbrace{\int_\Omega \nabla u \cdot \nabla v \diffop x}_{=: a(u,v)}
  = \underbrace{\int_\Omega g \cdot v \diffop x}_{=: f(x)}
  \quad \text{für alle } v \in V. \]
Also: Bestimme $u \in V$, so dass
\[ a(u,v) = f(v) \quad \text{für alle } v \in V, \]
das ist die ``Variationsgleichung'' mit der ``Testfunktion'' $v$.

\begin{rmrk*}
  Die Randbedingung $u |_{\partial \Omega} = 0$ wurde direkt in $V$
  ``eingebaut''.
\end{rmrk*}

\subsubsection*{Beispiel 2}
\[ \left\{
    \begin{aligned}
      - \Delta u + \vec{w} \cdot \nabla u &= g &\text{in } \Omega, \\
      \pdiff{u}{n} &= 0 &\text{auf } \textcolor{red}{\partial \Omega}
    \end{aligned}
  \right. \]
$\pdiff{u}{n}$ bezeichnet die Ableitung in Normalenrichtung:
\[ \pdiff{u}{n} = \nabla u \cdot \vec{n}. \]
Diese Randbedingung wird als \emph{Neumann'sche Randbedingung} bezeichnet.

Definiere $V :=$ Menge aller stetig differenzierbaren Funktionen auf $\Omega$.
Im Unterschied zu Beispiel 1 wird nun \emph{nicht} $v|_{\partial \Omega} = 0$
gefordert.

Multipliziere wieder die PDE mit $v \in V$ und wende den Gauß'schen Integralsatz
an:
\[ - \int_\Omega (\Delta u) \cdot v \diffop x
  + \int_\Omega (\vec{w} \cdot \nabla u) v \diffop x
  = \int_\Omega g v \diffop x. \]
Gauß angewendet auf $(\nabla u) v$ liefert wieder
\[ - \int_\Omega (\Delta u) \cdot v \diffop x
  = \int_\Omega \nabla u \cdot \nabla v \diffop x
  - \underbrace{\int_{\partial \Omega} v \nabla u \cdot \vec{n} \diffop S}_{=0},
\]
wobei diesmal der zweite Term aufgrund der Neumann'schen Randbedingung wegfällt.
Also folgt
\[ \underbrace{\int_\Omega \nabla u \cdot \nabla v
    + (\vec{w} \cdot \nabla u) v \diffop x}_{=: a(u,v)} =
  \underbrace{\int_\Omega g v \diffop x}_{=: f(v)}. \]
Bestimme nun $u \in V$, so dass
\[ a(u,v) = f(v) \quad \text{für alle } v \in V. \]

\subsection{Abstrakte Variationsgleichung: Mathematische Grundlagen}
``Zutaten''
\begin{itemize}
\item $V$: Funktionenraum, zum Beispiel differenzierbare Funktionen auf $\Omega$
  oder (typischerweise für finite Elemente) Sobolev-Raum $H^1(\Omega)$.
\item $a(u,v)$: Bilinearform auf $V$, das heißt $a : V \times V \to \real$ ist
  linear in $u$ und $v$.
  \[ a(u, \gamma v_1 + v_2) = \gamma a(u,v_1) + a(u,v_2), \]
  \[ a(\gamma u_1 + u_2, v) = \gamma a(u_1,v) + a(u_2,v). \]

  Beispiel:
  \[ a(u,v) = \int_\Omega \nabla u \cdot \nabla v \diffop x. \]
\item $f$: Linearform auf $V$, das heißt $f : V \to \real$ ist linear.

  Beispiel:
  \[ f(v) = \int_\Omega g \cdot v \diffop x \]
  mit fest gewählter Funktion $g$.
\item Problemstellung: Gesucht ist $u \in V$, so dass
  \[ a(u,v) = f(v) \quad \text{für alle } v \in V. \]  
\end{itemize}

\subsubsection*{Mathematische Grundlage}
\begin{thm}[Satz von Lax-Milgram]
  Sei $V$ ein Hilbertraum, $a$ eine Bilinearform auf $V$, die
  \begin{enumerate}[a)]
  \item koerziv ist, das heißt es existiert $c_1 > 0$, so dass
    \[ a(v,v) \ge c_1 \| v \|^2
      \quad \text{für alle } v \in V, \]
  \item stetig ist, das heißt es existiert $c_2 > 0$, so dass
    \[ |a(u,v)| \le c_2 \|u\| \cdot \| v \|
      \quad \text{für alle } u, v \in V \]
  \end{enumerate}
  sowie $f$ eine stetig Linearform, das heißt es existiert $c > 0$, so dass
  \[ |f(v)| \le c \|v\|
    \quad \text{für alle } v \in V. \]
  Dann existiert eine eindeutige Lösung $u \in V$ von
  \[ a(u,v) = f(v) \quad \text{für alle } v \in V. \]  
\end{thm}

\begin{rmrk*}
  Die Sobolev-Räume  $H^1(\Omega)$ und $H_0^1(\Omega)$ ($u|_{\partial \Omega} =
  0$) sind Hilberträume, also ist der Satz direkt anwendbar.
\end{rmrk*}

\subsection{Diskretisierung} %% 6.6.3
Idee: Wähle einen endlich-dimensionalen Teilraum $V_h \in V$, zum Beispiel die
stückweise linearen Funktionen auf der Triangulierung von $\Omega$ und bestimme
$u_h \in V_h$, so dass
\[ a(u_h, v_h) = f(v_h) \quad \text{für alle } v_h \in V_h. \]
Das heißt, die Testfunktionen $v_h$ werden aus dem Teilraum $V_h$ anstelle von
$V$ gewählt.

Vorteil: Viele Eigenschaften der kontinuierlichen Formulierung (typisch $V =
H_0^1(\Omega)$) lassen sich direkt auf die Diskretisierung übertragen,
insbesondere der Satz von Lax-Milgram (angewendet auf $V_h$).

Anforderung an die Triangulierung von $\Omega$: Es dürfen keine Ecken auf den
Kanten anderer Dreiecke liegen.

Berechnung von $u_h$: Gegeben Basis $\varphi_1, \varphi_2, \ldots, \varphi_n$
von $V_h$ kann man $u_h$ eindeutig darstellen als
\[ u_h = \sum_{i=1}^n u_i \varphi_i. \]
Weil $(\varphi_i)_{i=1}^n$ eine Basis ist, gilt
\begin{align*}
  a( u_h, v_h )
  &= f(v_h) \quad \text{für alle } v_h \in V_h 
    \quad \Leftrightarrow \\
  a( u_h, \varphi_j )
  &= f( \varphi_j ) \quad \text{für } j = 1,\ldots,n.
\end{align*}
Einsetzen von $u_h = \sum_{i=1}^n u_i \varphi_i$:
\[ \sum_{i=1}^n u_i
  \underbrace{a(\varphi_i, \varphi_j)}_{=: a_{ji}}
  = f(\varphi_j) \forall j = 1,
  \ldots, n. \]
$\Leftrightarrow$ Lineares Gleichungssystem
\[ A_h \pmat{u_1 \\ u_2 \\ \vdots \\ u_n} =
  \pmat{f(\varphi_1) \\ f(\varphi_2) \\ \vdots \\ f(\varphi_n)} \]
mit $A_h = (a_{ij})_{i,j=1}^n$.

Überlegung: $A_h$ sollte dünn besetzt sein, um das Gleichungssystem für große
$n$ (effizient) lösen zu können. Also wähle die Basisfunktionen $\varphi_j$ so,
dass sie jeweils nur in einem kleinen Teilgebiet von $\Omega$ von 0 verschieden
sind.
\[ \Rightarrow \quad a( \varphi_i, \varphi_j ) = 0, \quad \text{falls die
    Teilgebiete von $\varphi_i$ und $\varphi_j$ disjunkt sind.}\]

%% Bild

Basis von $V_h$: stückweise lineare Funktionen $\varphi_j$, die im zentralen
Knoten 1 sind und auf allen anderen Knoten 0.
\[ \varphi_j \in V_h, \quad \varphi_j (x_j) = 1, \quad \varphi_j(x_k) = 0 \quad
  \text{für alle} k \ne j. \]
Bezeichnung: Lagrange/Courant-Basis

\begin{rmrk*}
  \begin{itemize}
  \item Vorteil der Formulierung mittels Sobolev-Räumen: Es gilt $\varphi_j \in
    H^1(\Omega)$, da die $\varphi_j$ stückweise stetig differenzierbar, aber
    $\varphi_j \notin C^1(\Omega)$.
  \item In Anwendungen werden auch häufig Basisfunktionen mit höherem
    Polynomgrad verwendet. Die Lagrange-Basis mit stückweise linearen Funktionen
    hat den Polynomgrad 1.
  \end{itemize}
\end{rmrk*}

\section{Diskussion der drei Disketrisierungsprinzipien} %% 6.7
\begin{footnotesize}
\begin{tabular}{p{4.5em}|p{9em}|p{4em}|p{4.5em}|p{5em}|p{12em}}
  & PDE-Typ
  & Konv.-ordnung
  & Geometrie von $\Omega$
  & Struktur\-erhaltung
  & Sonstiges
  \\
  \hline
  Finite Differenzen
  & universell
  & hoch
  & starr achsenparallel
  & nur äquidistant
  & sehr einfach realisierbar für simple Probleme
  \\
  \hline
  Finite Volumen
  & Erhaltungsprinzip oder Divergenzform
  & niedrig
  & äußerst flexibel
  & eingebaut
  & 
  \\
  \hline
  Finite Elemente
  & Variationsprinzip (Variationsgleichung)
  & hoch
  & sehr flexibel
  & eingebaut
  & Implementierung sehr aufwändig selbst für einfache Probleme,
    Software z.B. AMDiS, DUNE
  \\
\end{tabular}
\end{footnotesize}