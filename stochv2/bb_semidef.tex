\setcounter{chapter}{1}
\chapter{Semidefinite Matrizen}
\renewcommand{\thethm}{B.\arabic{thm}}

\begin{defn} %% B.1
  $X \ne \emptyset$ sei eine beliebige Menge und $\Phi: X \times X \to \complex$
  ein Kern. Wir definieren $\Phi^*$ durch
  \[ \Phi(x,y) = \obar{\Phi(y,x)} \]
  für alle $x, y \in X$.

  $\Phi$ heißt \emph{hermitesch}, wenn $\Phi = \Phi^*$.

  $\Phi$ heißt \emph{positiv definit}, wenn
  \[ \sum_{j,k = 1}^n \Phi(x_j, x_k) c_j \obar{c_k} > 0 \]
  für alle $n \in \nat$, $x_1, \ldots, x_n \in X$ mit $x_j \ne x_k$ und $c_1,
  \ldots, c_n \in \complex$ nicht alle $=0$.

  $\Phi$ heißt \emph{positiv semidefinit}, wenn
  \[ \sum_{j,k = 1}^n \Phi(x_j, x_k) c_j \obar{c_k} \ge 0. \]
  In diesem Fall gilt $ge$ auch ohne die Bedingungen $x_j \ne x_k$ und $c_j \ne
  0$.

  Eine \emph{quadratische Matrix} der Ordnung $n$ ist ein Kern auf $\{1, \ldots,
  n\} \times \{1, \ldots, n\}$.

  $\diag(\lambda_1, \ldots, \lambda_n)$ ist eine Diagonalmatrix $(d_{ij})$ mit
  \[ d_{ii} = \lambda_i, \qquad d_{ij} = 0, \quad i \ne j. \]
\end{defn}

\begin{thm} %% B.2
  Jeder positiv semidefinite Kern $\phi$ auf $X$ ist hermitesch und
  \[ \phi(x, x) \ge 0, \qquad x \in X. \]
\end{thm}

\begin{exmp*}
  $A = (a_{jk}) = \pmat{ 1 & 1 \\ -1 & 1}$ ist nicht hermitesch, aber
  \[ \sum_{j,k=1}^2 a_{jk} r_j r_k = r_1^2 + r_2^2 > 0 \]
  wenn nicht beide $= 0$ sind.
\end{exmp*}

\clearpage

\begin{thm} %% B.3
  Ein reeller Kern $\phi$ auf $X$ ist genau dann positiv semidefinit, wenn
  $\phi$ symmterisch ist, das heißt $\phi(x,y) = \phi(y,x)$ und
  \[ \sum_{j,k = 1}^n \phi(x_j, x_k) r_j r_k \ge 0 \]
  für alle $n \in \nat$, $x_j \in X$, $r_j \in \real$.
\end{thm}

\begin{defn}
  Sei $A = (a_{jk})$ eine hermitesche Matrix der Ordnung $n$. Die \emph{Anzahl
    der negativen Quadrate} ist die Anzahl der negativen Eigenwerte, gezählt mit
  Vielfachheit.

  Sei $E \subseteq \complex^n$ ein linearer Raum mit
  \[ \sum_{j,k=1}^n a_{jk} x_j \obar{x_k} = (Ax, x) < 0 \]
  für alle $x \in E \setminus \{ 0 \}$. Dann heißt $E$ \emph{negativer
    Unterraum}.

  Analog für $\le 0$, $\ge 0$, $> 0$.
\end{defn}

\begin{rmrk}
  Bekannt ist bereits, dass eine hermitesche $n \times n$-Matrix $A$ $n$ reelle
  Eigenwerte $r_1, \ldots, r_n$ besitzt. Weiterhin existiert eine orthonormale
  Basis von $complex$, die aus Eigenvektoren von $A$ besteht. Eigenvektoren mit
  negativem Eigenwert erzeugen negative Unterräume.
\end{rmrk}

\begin{rmrk*}
  $\det(A) = r_1 \cdot \ldots \cdot r_n$.
\end{rmrk*}

\begin{thm}
  Ist $A$ eine hermitesche Matrix, so ist die Anzahl der negativen Quadrate
  gleich der maximalen Dimension von negativen Unterräumen von $A$.
\end{thm}

\begin{thm}
  Ist $A$ eine nicht singuläre $n \times n$ hermitesche Matrix, so ist die
  Anzahl der negativen Quadrate gleich der Anzahl der Vorzeichenwechsel der
  Folge
  \[ 1, \det A_1, \ldots, \det A_n, \]
  wobei $A_k = (a_{ij})_{i,j = 1}^k$.
\end{thm}

\begin{folg}
  Eine hermitesche Matrix $A$ der Ordnung $n$ ist genau dann positiv definit,
  wenn $\det(A_k) = 0$ für $k = 1, \ldots, n$.
\end{folg}

\begin{exmp*}
  $A = \pmat{0 & 0 \\ 0 & -1}$ ist hermitesch, $\det( A_k ) \equiv 0$, aber nicht
  positiv semidefinit.
\end{exmp*}

\begin{thm} %% B.9
  Ein hermitescher Kern $\Phi : X \times X \to \complex$ ist genau dann positiv
  semidefinit, wenn
  \[ \det (( \Phi(x_j, x_k))_{j,k \le n}) \ge 0 \]
  für alle $n$ und für alle $x_1, \ldots, x$
\end{thm}

\begin{thm} %% B.10
  Eine $n \times n$-Matrix $A$ ist genau dann positiv semidefinit, wenn eine $n
  \times n$ hermitesche Matrix $B$ existiert, so dass $A = B^2$. Ist $A$ reell,
  so kann $B$ auch reell gewählt werden.
\end{thm}

\begin{proof}
  Wenn $A = B^2$, dann ist
  \[ (Ax,x) = (BBx,x) = (Bx,B^* x) = (Bx, Bx) \ge 0. \]

  Sei $A$ positiv semidefinit und sei $\{e_1, \ldots, e_n\}$ eine orthonormale
  Basis von $\complex^n$ (bzw. $\real^n$), wobei $e_j$ ein Eigenvektor von $A$
  zum Eigenwert $r_j \ge 0$\footnote{%
    Wegen $(A e_j, e_j) = (r_j e_j, e_j) = r_j$ muss $r_j \ge 0$ gelten, wenn
    $A$ positiv semidefinit ist.
  } ist. Definiere
  \[ Q := [ e_1, \ldots, e_n] \]
  mit den $e_j$ als Spaltenvektoren. Das ist eine $n \times n$-Matrix und es
  gilt
  \[ Q^* Q = I_n, \]
  wobei $I_n$ die $n \times n$-Einheitsmatrix ist. Es folgt
  \[ Q \diag (r_1, \ldots, r_n) Q^* e_j = r_j e_j = A e_j \]
  für alle $j$. Also ist
  \[ A = Q \diag( r_1, \ldots, r_n) Q^*, \]
  damit ist
  \[ B := Q \diag(\sqrt{r_1}, \ldots, \sqrt{r_n}) Q^* \]
  hermitesch und es gilt $BB = A$.
\end{proof}

\begin{thm} %% B.11
  Sind $A = (a_{ij})$ und $B = (b_{ij})$ positiv semidefinite $n \times
  n$-Matrizen, so ist es auch das \emph{Hadamard-Produkt}
  \[ C = (a_{ij} \cdot b_{ij})_{i,j = 1}^n. \]
\end{thm}