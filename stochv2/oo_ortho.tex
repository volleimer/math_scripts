\setcounter{chapter}{14}
\chapter{Orthogonalität}
Sei $X$ ein linearer Raum, $(\cdot, \cdot)$ das Skalarprodukt. $x,y \in X$
heißen \emph{orthogonal}, wenn
\[ (x,y) = 0. \]
Schreibweise $x \perp y$.

Satz von Pythagoras: Für $x \perp y$ gilt
\[ \| x + y \|^2 = \| x \|^2 + \| y \|^2. \]

\begin{proof}
  \[ \|x+y\|^2 = (x+y,x+y) = \ldots \qedhere \]
\end{proof}

Parallelogrammidentität:
\[ \|x+y\|^2 + \|x-y\|^2 = 2(\|x\|^2 + \|y\|^2). \]

Sei $x \in X$, $A,B \subset X$. $x$ ist \emph{orthogonal} zu $A$, $x \perp A$,
wenn $x \perp a$ für alle $a \in A$. $A$ ist \emph{orthogonal} zu $B$, $A \perp
B$, wenn $a \perp b$ für alle $a \in A$, $b \in B$.

Das \emph{orthogonale Komplement} von $A$ ist
\[ A^\perp := \{ x \in X : x \perp A \}. \]

\renewcommand{\thethm}{O.\arabic{thm}}
\setcounter{thm}{0}
\begin{aufg} %% O.1
  \begin{enumerate}
  \item $A^\perp$ ist ein abgeschlossener Teilraum von $X$.
  \item $\{ 0 \}^\perp = X$ und $X^\perp = \{ 0 \}$.
  \item Aus $A \subset B$ folgt $B^\perp \subset A^\perp$.
  \item $A^\perp = (\obar{A})^\perp$ (Abschluss) und $A^\perp = L(A)^\perp =
    (\obar{L(A)})^\perp$ (lineare Hülle).
  \end{enumerate}
\end{aufg}

$M \subset X$ heißt \emph{total}, wenn die lineare Hülle $L(M)$ dicht in $X$
ist:
\[ \obar{L(M)} = X. \]

Beispiel: $X = \real^n$, $M := \{e_1, \ldots, e_n\}$ eine Basis, dann ist
$L(M) = \real^n$.

\clearpage

\begin{thm}[Approximationssatz] %% O.2
  Sei $X$ ein Hilbertraum, $A$ eine abgeschlossene konvexe Teilmenge von $X$.
  Dann gibt es zu jedem $x \in X$ genau eine beste Approximation in $A$, das
  heißt es gibt genau ein $y \in A$ mit
  \[ \| x - y \| = d(x,A) := \inf \{ \| x - z \| : z \in A \}. \]
\end{thm}

\begin{proof}
  Nach der Definition von $d(A,x)$ gibt es eine Folge $\{y_n\}$ aus $A$ mit
  \[ \| x - y_n \| \xrightarrow{n \to \infty} d(x,A). \]
  Wir zeigen: $\{y_n\}$ ist eine Cauchy-Folge.

  Parallelogrammidentität $x \to x - y_n$ und $y \to x - y_n$.
  \begin{align*}
    \| y_n - y_m \|^2
    &= 2 \big( \|x-y_n\|^2 + \|x-y_m\|^2 \big) - \| 2x - (y_n + y_m) \|^2 \\
    &= 2 \big( \|x-y_n\|^2 + \|x-y_m\|^2 \big)
      - 4 \left\| x - \frac{y_n+y_m}{2} \right\|^2 \\
    &= 2 \big( \|x-y_n\|^2 + \|x-y_m\|^2 \big)
      - 4 d(x,A)^2 \xrightarrow{n,m \to \infty} 0. \\
  \end{align*}
  Es gilt $\frac{y_n+y_m}{2} \in A$ wegen der Konvexität von $A$.

  Also ist $\{y_n\}$ eine Cauchy-Folge und damit existiert ein $y \in A$ mit
  $y_n \to y$ und
  \[ \| x - y \| = \lim_{n \to \infty} \| x - y_n \| = d(x,A). \]
  Damit ist die Existenz bewiesen.

  Es fehlt noch die Eindeutigkeit: Angenommen, für $y_1$ und $y_2$ gilt
  \[ \| x - y_1 \| = \| x - y_2 \| = d(x,A), \qquad y_1 \ne y_2. \]

  Oder: Der erste Teil des Beweises zeigt: $y_1, y_2, y_1, y_2, \ldots$ ist eine
  Cauchy-Folge $\Rightarrow$ $y_1 = y_2$.
\end{proof}

\begin{thm}[Projektionssatz] % O.3
  Sei $X$ ein Hilbertraum, $M$ ein abgeschlossener Teilraum. Dann gilt:
  \begin{enumerate}[a)]
  \item Jedes $x \in X$ lässt sich eindeutig in der Form $x = y + z$ schreiben
    mit $y \in M$ und $z \in M^\perp$. $y$ heißt die \emph{orthogonale
      Projektion} von $x$ auf $M$.
  \item $(M^\perp)^\perp = M$.
  \end{enumerate}
\end{thm}

\clearpage

\begin{aufg} % O.4
  Sei $X$ ein Hilbertraum und $A \subset X$.
  \begin{enumerate}[a)]
  \item $(A^\perp)^\perp = \obar{L(A)}$.
  \item $A^\perp = \{0\}$ $\Leftrightarrow$ $\obar{L(A)} = X$.
  \end{enumerate}
\end{aufg}

\begin{proof}[Beweis zu Satz O.3]
  Existenz: Wir wählen für $y$ die beste Approximation von $x$ in $M$. Definiere
  $z := x - y$.

  Noch zu zeigen: $z \in M^\perp$, das heißt $(z,w) = 0$ für alle $w \in M$.
  O.B.d.A. $w \ne 0$. Es gilt $y + aw \in M$ für alle $a \in K$ ($=\complex$
  oder $\real$).
  \begin{align*}
    d(x,M)^2
    &\le \| x - (y+aw) \|^2 = \| z - aw \|^2 \\
    &= \underbrace{\| z \|^2}_{=d(x,M)^2} - 2 \Re( a(z,w) + |a|^2 \|w\|^2 )
  \end{align*}
  für alle $a \in K$. Mit $a = \| w \|^{-2} (w,z)$ folgt
  \[ 0 \le -2 \frac{|(z,w)|^2}{\|w\|^2} + \frac{|(z,w)|^2}{\|w\|^2}
    = - \frac{|(z,w)|^2}{\|w\|^2}. \]
  Also muss $(z,w) = 0$ sein.

  Eindeutigkeit: Gilt auch $x = y' + z'$ mit $y' \in M$, $z' \in M^\perp$, so
  gilt $y - y' \in M$ und $z - z' \in M^\perp$. Wegen
  \[ y + z = y' + z' \]
  gilt
  \[ y - y' = z' - z \in M \cap M^\perp = \{ 0 \}. \]
  Also gilt $y = y'$ und $z = z'$.
\end{proof}

\begin{denos*}
  Seien $M_1$, $M_2$ Teilräume von $X$ mit $M_1 \cap M_2 = \{ 0 \}$.
  \[ M_1 + M_2 = \{ y_1 + y_2 : y_1 \in M_1, y_2 \in M_2 \} \]
  ist die sogenannte \emph{direkte Summe} $M_1 + M_2$ (das heißt, jedes Element
  aus $M_1 + M_2$ hat \emph{genau eine} Darstellung der Form $y_1 + y_2$, $y_j
  \in M_j$).
  Sind $M_1$ und $M_2$ orthogonal, so gilt $M_1 \cap M_2 = \{0\}$. In diesem
  Fall schreibt man $M_1 \oplus M_2$, die \emph{orthogonale Summe}.

  Man schreibt auch $M_1 = M \ominus M_2$, das heißt $M_1$ ist das
  \emph{orthogonale Komplement} von $M_2$ bezüglich $M$.

  Spezialfall: $Y^\perp = X \ominus Y$.
\end{denos*}

\clearpage

\begin{thm}
  Sei $X$ ein Hilbertraum.
  \begin{enumerate}[a)]
  \item Sind $M_1$ und $M_2$ orthogonale Teilräume, so ist $M_1 \oplus M_2$
    genau dann abgeschlossen, wenn $M_1$ und $M_2$ abgeschlossen sind.
  \item Sind $M_1$ und $M$ abgeschlossene Teilräume mit $M_1 \subset M$, so
    existiert ein abgeschlossener Teilraum $M_2 \subset M$ mit $M = M_1 \oplus
    M_2$. 
  \end{enumerate}
\end{thm}

\begin{proof}
  a) folgt aus der Tatsache, dass $\{ y_{1,n} + y_{2,n} \}_{n=1}^\infty$ mit
  $y_{j,n} \in M_J$ genau dann eine Cauchy-Folge ist, wenn
  $\{y_{j,n}\}_{n=1}^\infty$ Cauchy-Folgen sind.

  b) O.B.d.A. $M = X$. $M_2 = M_1^\perp$ hat die gewünschte Eigenschaft.
\end{proof}

\begin{rmrk*}
  Die Orthogonalität in a) ist notwendig. Sei $H = \ell_2$, das heißt
  \[ x \in \ell_2 : \{ x = (x_0, x_1, \ldots ) \} \]
  mit $\sum_j |x_j|^2 < \infty$.

  Seien $X_1 := \{$ alle Folgen in $\ell_2$ mit $ x_{2n} = 0\}$ und $X_2 := \{$
  alle Folgen in $\ell_2$ mit $x_{2n+1} = n x_{2n} \}$. Dann gilt
  \begin{enumerate}
  \item $X_1$ und $X_2$ sind abgeschlossene, lineare Teilräume.
  \item $X_1 \cap X_2 = \{ 0 \}$.
  \item $X_1 \dot{+} X_2$ ist dicht in $\ell_2$ (Hinweis: Enthält alle Folgen
    mit endlichem Träger)
  \item $X_1 \dot{+} X_2 \ne \ell_2$ (Hinweis: $x_n := \rez{n+1} \in \ell_2$,
    aber $\notin X_1 \dot{+} X_2$)
  \end{enumerate}
  Aufgabe.
\end{rmrk*}

\begin{exmp}
  Ist $X = L^2(a,b)$, $a < c < b$, so gilt
  \[ L^2(a,b) = L^2(a,c) \oplus L^2(c,b), \]
  da
  \[ f = f \cdot \ind_{[a,c]} + f \cdot \ind_{[c,b]} \]
  und
  \[ \int_a^b f \cdot \ind_{[a,c]} \cdot f \cdot \ind_{[c,b]} \diffop \lambda =
    0. \]
\end{exmp}

\begin{exmp}
  In $X = L^2(-a,a)$ seien
  \[ L_g^2(-a,a) := \{ f \in L^2(-a,a) :
    f(x) = f(-x) \text{ fast überall} \} \]
  und
  \[ L_g^2(-a,a) = \{ f \in L^2(-a,a) :
    f(x) = -f(-x) \text{ fast überall} \}. \]
  Dann gilt
  \[ L^2(-a,a) = L^2_g(-a,a) \oplus L^2_u(-a,a), \]
  denn
  \[ f(x) = f_g + f_u = \frac{f(x) + f(-x)}{2} + \frac{f(x) - f(-x)}{2} \]
  und
  \[ \int_{-a}^a f_g \cdot \obar{f_u} \diffop \lambda = 0, \]
  weil $f_g \cdot \obar{f_u}$ ungerade ist.
\end{exmp}

Eine Familie $M = \{ e_\alpha : \alpha \in A \}$ in einem Hilbertraum heißt ein
\emph{Orthonormalsystem}, wenn $e_\alpha \perp e_\beta$ für $\alpha \ne \beta$
und $\| e_\alpha \| = 1$. Ein totales Orthonormalsystem heißt
\emph{Orthonormalbasis}.

\begin{thm}
  \begin{enumerate}[a)]
  \item Jedes Orthonormalsystem ist linear unabhängig.
  \item Jede Orthonormalbasis ist ein maximales Orthonormalsystem.
  \item Im Hilbertraum ist jedes maximale Orthonormalsystem eine
    Orthonormalbasis.
  \end{enumerate}
\end{thm}

\begin{proof}
  Aufgabe.
\end{proof}

\begin{exmp}
  In $\ell^2(\nat)$, $\ell^2(\integer)$. Orthonormalbasis:
  \[ \{ e_n \}_{n=1}^\infty; \qquad e_n(n) = 1; \quad e_n(m) = 0, \quad n \ne
    m. \]
  Zu zeigen: $(x,e_n) = 0$ für alle $n$ $\Rightarrow$ $x = 0$.
\end{exmp}

\begin{aufg}
  Sei $X = L^2(0,1)$.
  \begin{enumerate}[a)]
  \item $M = \{e_n : n \in \integer \}$, wobei $e_n(x) = e^{2 \pi i n x}$, $x
    \in [0,1]$.

    Zu zeigen: $M$ ist Orthonormalbasis.
  \item $c_n(x) = \sqrt{2} \cos (2 \pi n x)$, $s_n(x) = \sqrt{2} \cos (2 \pi n
    x)$.

    Zu zeigen: $\{ c_n : n \in \nat_0 \} \cup \{ s_n : n \in \nat \}$ ist
    Orthonormalbasis.
  \end{enumerate}
\end{aufg}

Hinweis: Satz von Weierstraß für trig. Polynomräume.

\clearpage

\begin{thm} %% O.11
  Sei $X$ ein Prähilbertraum\footnotemark, $\{ e_{\alpha} : \alpha \in A \}$ ein
  Orthonormalsystem in $X$.
  \begin{enumerate}[a)]
  \item Ist $\{ \alpha_n \}$ eine Folge paarweise verschiedener Elemente aus $A$
    und $\{c_n\}$ eine Folge aus $\mathbb{K}$, so gilt
    \begin{enumerate}[(i)]
    \item Ist $\sum_n c_n e_{\alpha_n}$ konvergent bzw.
      \[ \left\{ \sum_{n=1}^m c_n e_{\alpha_n} \right\} \]
      eine Cauchy-Folge, so ist $\{ c_n \} \in \ell_2$.
    \item isr $\{c_n\} \in \ell_2$, so ist
      \[ \left\{ \sum_{n=1}^m c_n e_{\alpha_n} \right\} \]
      eine Cauchy-Folge.
    \end{enumerate}
  \item Ist $y = \sum_n c_n e_{\alpha_n}$, so gilt $c_n = (e_{\alpha_n}, y)$ und
    \[ \| y \|^2 = \sum_n |c_n|^2, \qquad
      (y,x) = \sum_n (y, e_{\alpha_n}) \cdot (e_{\alpha_n}, x)\]
    für alle $x \in X$.
  \end{enumerate}
\end{thm}
\footnotetext{Ein Raum mit positiv definitem Skalarprodukt, der möglicherweise
  nicht vollständig ist.}

\begin{thm}[Entwicklungspunktsatz] %% O.12
  \begin{enumerate}[a)]
  \item Ist $\{ e_\alpha : \alpha \in A\}$ ein Orthonormalsystem im
    Prähilbertraum $X$, so gilt für alle $x \in X$ die \emph{Bessel'sche
      Ungleichung}
    \[ \sum_{\alpha \in A} |(e_{\alpha},x)|^2 \le \| x \|^2, \]
    wobei in der Summe über höchstens abzählbar viele Terme $\ne 0$ sind.
  \item Ein Orthonormalsystem $\{ e_\alpha : \alpha \in A \}$ in $X$ ist genau
    dann eine Orthonormalbasis, wenn die \emph{Parseval'sche Gleichung}
    \[ \sum_{\alpha \in A} |(e_\alpha,x)|^2 = \| x \|^2 \]
    gilt. Es ist dann
    \[ x = \sum_{\alpha \in A} (e_\alpha, x) \cdot e_\alpha \]
    (Summe abzählbar).
  \item ist $\{ e_\alpha : \alpha \in A \}$ ein Orthonormalsystem im Hilbertraum
    $X$, so ist
    \[ \sum_{\alpha \in A} (e_\alpha, x) \cdot e_\alpha \]
    die orthogonale Projektion von $x$ auf den abgeschlossenen Teilraum
    $\obar{L \{ e_\alpha : \alpha \in A\}}$.
  \end{enumerate}
\end{thm}

\begin{proof}
Zu a): Ist $y = \sum_{n=1}^m c_n \cdot e_{\alpha_n} \in L\{ e_\alpha : \alpha
\in A \}$, so gilt
\begin{align*}
  \| x - y \|^2
  &= \| x \|^2 - 2 \Re \sum_{n=1}^m c_n (x, e_{\alpha_n})
    + \sum_{n=1}^m | c_n |^2 \\
  &= \| x \|^2 - \sum_{n=1}^n |(e_{\alpha_n}, x)^2
    + \sum_{n=1}^m |c_n - (e_{\alpha_n}, x) |^2 \tag{$*$}\\
  &\ge \| x \|^2 - \sum_{n=1}^m | (e_\alpha, x) |^2.
\end{align*}
$\| x - y \|$ also minimal genau dann, wenn $c_n = (e_{\alpha_n}, x)$ (bei
festen $\alpha_1, \ldots, \alpha_n$). Dann gilt
\[ \| x - y \|^2 = \| x \|^2 - \sum_{n=1}^m | (e_{\alpha_n}, x) |^2,\]
also folgt
\[ \sum_{n=1}^m |(e_{\alpha_n}, x)|^2 \le \| x \|^2. \]
Da dies für beliebige endliche Familien $\{\alpha_1, \ldots, \alpha_n\}$ aus $A$
gilt, folgt die Behauptung.

Zu b): Ist $\{ e_\alpha : \alpha \in A \}$ eine Orthonormalbasis, also
$\obar{L\{e_\alpha : \alpha \in A\}} = X$, dann existieren für alle $\eps > 0$
eine Familie $\{ \alpha_1, \ldots, \alpha_m \} \subset A$ und $c_1, \ldots, c_m
\in \mathbb{K}$ mit
\[ \| x - y \| < \eps \]
für $y = \sum_{n=1}^m c_n e_n$ und damit folgt mit $(*)$
\[ \| x \| - \sum_{n=1}^m |(e_{\alpha_n},x)|^2 \le \| x - y \|^2 \le \eps^2, \]
also
\[ \sum_{n=1}^m |(e_{\alpha_n},x)^2| \ge \| x \|^2 - \eps^2. \]
Mit der Bessel'schen Ungleichung folgt damit die Parseval'sche Gleichung.

Ist $\{ \alpha_1, \alpha_2, \ldots \}$ die Menge der $\alpha \in A$ mit
$(e_\alpha, x) \ne 0$ und
\[ y_m := \sum_{n=1}^m (e_{\alpha_n}, x) e_{\alpha_n}, \]
so folgt mit $(*)$ und der Parseval'schen Gleichung
\[ \| x - y_m \|^2 = \|x\|^2 - \sum_{n=1}^m |(e_{\alpha_m},x)|^2 \xrightarrow{m
    \to \infty} 0, \]
das heißt
\[ x = \lim_{m \to \infty} y_m = \sum_{n=1}^\infty (e_{\alpha_n},x)
  e_{\alpha_n} = \sum_{\alpha \in A} (e_{\alpha}, x) e_\alpha. \]

Zu c): Sei
\[ y := \sum_{\alpha \in A} (e_\alpha, x) e_\alpha = \sum_{n=1}^\infty
  (e_{\alpha_n}, x) e_{\alpha_n}. \]
Mit $(*)$ folgt für beliebige $c_1, \ldots, c_m \in \mathbb{K}$
\begin{align*}
  \| x - y \|^2
  &= \| x \|^2 - \sum_{\alpha \in A} |(e_\alpha,x)|^2 \\
  &\le \| x \|^2 - \sum_{n=1}^m |(e_{\alpha_n},x)|^2 \\
  &\le \left\| x - \sum_{n=1}^m c_n e_{\alpha_n} \right\|^2.
\end{align*}
Es gilt $\sum_{n=1}^m c_n e_{\alpha_n} \in \obar{L \{ e_\alpha : \alpha =
  \alpha_1, \ldots, \alpha_m\}}$. Also
\begin{align*}
  \| x - y \|
  &= \inf \left\{ \| x - z \| : z \in L\{e_\alpha\} \right\} \\
  &= \inf \left\{ \| y - z \| : z \in \obar{L\{e_\alpha\}} \right\}.
\end{align*}
Das heißt $y$ ist die orthogonale Projektion von $x$ auf $\obar{L\{e_\alpha\}}$.
\end{proof}

Einfaches Beispiel: Sei $X = \real^2$, $e_1 = (1,0)$, $e_2 = (0,1)$ ist eine
Orthonormalbasis. Die orthogonale Projektion von $x = (2,2) \in X$ auf $\real
e_2$ ist $(0,2)$.

\begin{thm}[Gram-Schmidt'sche Orthogonalisierung] %% O.13
  Sei $X$ ein Prähilbertraum, $F = \{ x_n : n \in \nat \}$ oder $F = \{ x_n : n
  = 1, \ldots, m \}$.
  Dann existiert ein Orthonormalsystem $M = \{ e_n \}$ mit $L(F) = L(M)$.

  Ist $F$ linear unabhängig, dann kann $M$ so gewählt werden, dass
  \[ L(\{x_1, \ldots, x_n\}) = L( \{e_1, \ldots, e_n\}) \]
  für alle $n$. In diesem Fall lassen sich die $e_n$ in der Form
  \[ e_n = \sum_{j=1}^n c_{n,j} x_j \]
  schreiben. Mit der zusätzlichen Forderung $c_{n,n} > 0$ werden sie eindeutig
  bestimmt. 
\end{thm}

\begin{proof}
  Aufgabe.

  Hinweis: O.B.d.A. $F$ linear unabhängig.
  \[ e_1 := \frac{x_1}{\| x_1 \|}, \qquad
    e_{n+1} := \frac{x_{n+1} + P_n x_{n+1}}{ \| x_{n+1} - P_n x_{n+1} \|}, \]
wobei $P_n x$ die orthogonale Projektion von $x$ auf $L\{ e_1, \ldots e_n\}$
ist. $x_{n+1} - P_n x_{n+1} \ne 0$, weil $F$ linear unabhängig ist.
\end{proof}

\clearpage

\begin{thm}[Existenz von Orthormalbasen] % O.14
  \begin{enumerate}[a)]
  \item Jeder separable Prähilbertraum besitzt eine endliche oder abzählbar
    unendliche Orthonormalbasis.
  \item Jeder Hilbertraum besitzt eine Orthonormalbasis.
  \end{enumerate}
\end{thm}

\begin{proof}
  a): Sei $X$ separabel, dann gibt es eine höchstens abzählbare totale Menge
  $\{x_n\}$. Wir dürfen annehmen, dass $\{x_n\}$ linear unabhängig ist.
  Anwendung von O.13 liefert eine orthonormale Basis der gleichen Mächtigkeit.

  b): Sei $\mM$ die Menge aller Orthonormalsysteme in $X$. $\mM$ ist bezüglich
  der Inklusion ``$\subseteq$'' halbgeordnet.

  \emph{Lemma von Zorn:} Besitzt in einer halbgeordneten Menge jede Teilmenge
  eine obere Schranke, so existiert (mindestens) ein maximales Element. Ist
  $\mN$ eine geordnete Teilmenge von $\mM$, so ist die Vereinigung aller $N \in
  \mN$ eine obere Schranke für alle $M \in \mM$.

  $M$ ist ein Orthonormalsystem: Für $e_1, e_2 \in M$ mit $e_1 \ne e_2$
  existieren $N_1, N_2 \in \mN$ mit $e_j \in N_j$. Wegen $N_1 \subset N_2$ oder
  $N_2 \subset N_1$ gilt $e_1, e_2 \in N_2$ oder $e_1, e_2 \in N_1$. Also ist
  $e_1 \perp e_2$.

  Nach dem Lemma von Zorn existiert ein maximales Element $M_{\max} \in \mM$,
  dieses ist eine orthonormale Basis: Wäre $\obar{L(M_{\max})} \ne X$, so gäbe
  es ein $e \in \obar{L(M_{\max})}^\perp$ mit $\| e \| = 1$, das heißt $M_{\max}
  \cup \{e\}$ wäre ein Orthonormalsystem. Widerspruch zur Maximalität!
\end{proof}

\begin{folg} % O.15
  \begin{enumerate}[a)]
  \item In einem separablen Prähilbertraum kann jedes endliche Orthonormalsystem
    zur einer Orthonormalbasis ergänzt werden.
  \item Im Hilbertraum kann jedes Orthonormalsystem zu einer Orthonormalbasis
    ergänzt werden.
  \end{enumerate}
\end{folg}

\begin{proof}
  a): Zu dem endlichen Orthonormalsystem nimmt man eine abzählbare, dichte
  Teilmenge dazu. Man streicht eventuelle linear abhängige Elemente heraus und
  wendet das Gram-Schmidt'sche Verfahren an.

  b): Sei $M_1$ das gegebene Orthonormalsystem und $M_2$ eine Orthonormalbasis
  von $M_1^\perp$. Dann ist $M_1 \cup M_2$ eine Orthonormalbasis von $X$.
\end{proof}

\begin{thm} % O.16
  Alle Orthonormalbasen in einem Prähilbertraum haben die gleiche Mächtigkeit.
\end{thm}

\begin{proof}
  Aufgabe.
\end{proof}

Dieser Satz erlaubt es, die \emph{Hilbertraumdimension} zu definieren: Die
Mächtigkeit einer Orthonormalbasis.

\begin{rmrk*}
  Es gibt Hilberträume mit beliebiger Dimension. Ist $A$ eine Menge mit
  vorgegebener Mächtigkeit, so hat $\ell^2(A)$ (alle quadratisch summierbaren
  Funktionen auf dieser Menge) die Dimension $|A|$.

  Orthonormalbasis: $\{ \ind_{\{a\}} : a \in A \}$.

  $\ell^2(A) = \{ f : A \to \complex \}$ mit $\sum_{a \in A} |f(a)|^2 < \infty$.
  Skalarprodukt:
  \[ (f,g) = \sum_{a \in A} f(a) \cdot \obar{g(a)}. \]
  Es gilt $(\ind_{\{a\}}, \ind_{\{b\}}) = 0$ für $a \ne b$ und $\| \ind_{\{a\}} \|
  = 1$.
\end{rmrk*}

\subsubsection*{Beispiel für die Gram-Schmit-Orthogonalisierung}
Sei $X$ der lineare Raum aller stetigen Funktionen auf $[0,1]$. Das
Skalarprodukt sei
\[ (f,g) = \int_0^1 f(x) \obar{g(x)} \diffop x \]
oder allgemeiner
\[ (f,g) = \int_0^1 f(x) \obar{g(x)} \cdot p(x) \diffop x, \]
wobei $p(x) \ge 0$ stetig.

Die Polynome $1, x, x^2, x^3, \ldots$ sind linear unabhängig.

Aus $c_0 + c_1 x + \ldots + c_n x^n = 0$ für alle $x \in [0,1]$ folgt, dass $c_j
= 0$ für alle $j$. Anwendung von Gram-Schmidt liefert orthogonale Polynome
bezüglich $p$.